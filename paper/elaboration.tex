% \section{Elaboration and Constraint Generation}
% \label{sec:elaboration}

% \begin{figure}

\begin{codeml}
$\elabExpr(CT, \A, \ralloc, \env, e)$ = 
  match $e$ with
  | $\C{new} \fgjN(\bar{e})$ $\longrightarrow$ 
    let $\fbN$ = $\templateTy(\fgjN)$ in
    let $C_1$ = $\typeOk({\A},{\fbN})$ in
    let $C_2$ = match $\fgjN$ with $\RgnZ\inang{T}$ $\longrightarrow$ $\top$
          | _ $\longrightarrow$ $\{\isvalid{\A.\phicx}{\allocRgn(\fbN)=\ralloc}\}$ in
    let ($\_:\taubar$) = $\fields(\fbN)$
    let $(\bar{e'}:\bar{\tau'}, C_3)$ = $\elabExpr(\A,\ralloc,\env,\bar{e})$ in
    let $C_4$ = $\subtypeOk({\A},{\bar{\tau'}},{\taubar})$ in
      ($\C{new} \fbN(\bar{e'}) : \fbN$,$\bigcup_{i=1}^4 C_i$)
  | $e_a(\bar{e})$ $\longrightarrow$ 
    let ($e_a':\inang{\rhoalloc\rhobar\,|\,\phi}\taubar \xrightarrow{\rgn} \tau$,$C_1$) = 
                $\elabExpr(\A,\ralloc,\env,e_a)$ in
    let $\bar{\rho'}$ = $\bar{\fresh_\rho()}$ in
    let $C_2$ = $\{\bar{\rho'} \in \A.\rhoenv\}$ in
    let $\substFn$ = $[\bar{\rho'}/\rhobar][\ralloc/\rhoalloc]$ in
    let $C_3$ = $\{\isvalid{\A.\phicx}{\substFn(\phi)}\}$ in
%*   %let $(C_3,C_4)$ = $(\typeOk(\substFn(\taubar)), \typeOk(\substFn(\tau)))$ in 
*)   let $(\bar{e'}:\bar{\tau'}, C_4)$ = $\elabExpr(\A,\ralloc,\env,\bar{e})$ in
    let $C_5$ = $\subtypeOk({\A},{\bar{\tau'}},{\bar{\substFn(\tau)}})$ in
      ($e_a'\inang{\ralloc\bar{\rho'}}(\bar{e'}):\substFn(\tau)$,$\bigcup_{i=1}^5 C_i$)
  | $\letregion{\rgn}{e_a}$ $\longrightarrow$
    let $\rgn'$ = $\fresh_{\rgn}()$ in 
    let $(\rhoenv,\aenv,\phicx)$ = $\A$ in
    let $\A'$ = ($\rhoenv \cup \{\rgn'\}, \aenv, \phicx \conj (\rhoenv \outlives \rgn')$) in
    let ($e_a':\tau$,$C_1$) = $\elabExpr(\A',\rgn',\env,[\rgn'/\rgn]e_a)$ in
      ($\letregion{\rgn'}{e_a'}:\tau$,$C_1$)
  | $\open{e_a}{\rgn}{y}{e_b}$ $\longrightarrow$ 
    let ($e_a':\RgnZ\inang{T}\inang{\rho}$,$C_1$) = 
                $\elabExpr(\A,\ralloc,\env,e_a)$ in
    let $\rgn'$ = $\fresh_{\rgn}()$ in 
    let $(\rhoenv,\aenv,\phicx)$ = $\A$ in
    let ($\A'$,$\env'$) = ($(\rhoenv \cup \{\rgn'\},\aenv,\phicx)$,$\env[y\mapsto T@\rgn']$) in
    let ($e_b':\tau$,$C_2$) = $\elabExpr(\A',\rgn',\env',[\rgn'/\rgn]e_a)$ in
      ($\open{e_a'}{\rgn'}{y}{e_b'} : \tau$, $C_1 \cup C_2$)
  | _ $\longrightarrow$ ...
\end{codeml}

\caption{Constraint generation for expressions in $\absof{\FB}$}
\label{fig:fb-elabexpr}
\end{figure}

% \newcommand{\hdOf}[2]{\C{class}\; #1\angAlpha\inang{\rhoalloc\rhobar \,|\, #2} \extends \fbN}
\begin{figure}

\begin{codeml}
$\elabMeth(CT, B, \tau \; m\inang{\rhoalloc_m\rhobarm \,|\, \varphi_m} (\taubar \; \xbar)\{\C{return} e;\})$ = 
  let $\hdOf{B}{\varphi}\{\bar{\tau^f}\,\xbar;\;k\;\bar{d}\}$ = $CT(B)$ in
  let ($\rhoenv$,$\aenv$,$\phicx$) as $\A$ = 
            $(\{\rhoalloc,\rhobar,\rhoalloc_m,\rhobarm\},\bar{\tyvar} \extends \bar{\fgjN},\varphi \conj \varphi_m)$ in
  let $C_1$ = $\{\tywf{\rhoenv}{\varphi_m}\}$ in
  let $\env$ = $\cdot[\thisZ \mapsto B\inang{\bar{\tyvar}}\inang{\rhoalloc\rhobar}][\xbar \mapsto \taubar]$ in
  let ($e':\tau'$,$C_2$) = $\elabExpr (\A,\rhoalloc_m,\env,e)$ in
  let $C_3$ = $\subtypeOk({\A},{\tau'},{\tau})$ in
    ($\tau \; m\inang{\rhoalloc_m\rhobarm \,|\, \varphi_m} (\taubar \;
    \xbar)\{\C{return} e';\}$, $C_1 \cup C_2 \cup C_3$)
\end{codeml}

\begin{codeml}
$\elabClass(CT,B)$ = 
  let $\hdOf{B}{\varphi}\{\bar{\tau^f}\,\xbar;\;k\;\bar{d}\}$ = $CT(B)$ in
  let ($\rhoenv$,$\aenv$,$\phicx$) as $\A$ = $(\{\rhoalloc,\rhobar\},\bar{\tyvar} \extends \bar{\fgjN},\varphi)$ in
  let $C_1$ = $\tywf{\rhoenv}{\varphi}$ in
  let ($C_2$,$C_3$) = ($\typeOk({\A},{\fbN})$,$\typeOk({\A},{\bar{\tau^f}})$) in
  let $C_4$ = $\{\isvalid{\phicx}{\allocRgn(\bar{\tau^f}) \outlives \rhoalloc \conj \allocRgn(\fbN) = \rhoalloc}\}$ in
  let ($k'$,$C_5$) = $\elabCons(B,k)$ in
  let ($\bar{d'}$,$C_6$) = $\elabMeth(B,\bar{d})$ in
    ($\hdOf{B}{\varphi}\{\bar{\tau^f}\,\xbar;\;k'\;\bar{d'}\}$, $\bigcup_{i=1}^6 C_i$)
\end{codeml}
%
% \begin{codeml}
% $\elabClassTable(CT)$ = 
%   let ($CT'$,$C$) = (ref $\cdot$, ref $\emptyset$) in
%   let _ = foreach $B \in dom(CT)$ do
%             let ($D_B$,$C_B$) = $\elabClass(CT,B)$ in
%               $CT'$ := $CT'[B \mapsto D_B]$;
%               $C$ := $C \cup C_B$
%            done in
%   let ($\substFn_\rho$,$\substFn_\varphi$) = $\solve(C)$ in
%     $\substFn_\varphi(\substFn_\rho(CT')) $
%       
% \end{codeml}

\caption{Method, class and class table elaboration}
\label{fig:fb-elabmeth}
\end{figure}


%Elaborating $\absof{\FB}$ expressions to $\FB$ expressions involves
%(a). replacing core types in variable declarations and \C{new}
%expressions with fresh region type templates, and (b). explicitly
%instantiating region parameters of methods with fresh region variables
%in method calls and function applications. This elaboration is
%performed with respect to the polymorphic type templates of classes
%and methods computed as per \S\ref{sec:fb-templatization}. 

We now present an integrated algorithm that processes all the methods
in a given program, elaborating them (by introducing region variables
to stand for actual region parameters) and simultaneously generating
the set of constraints that must be solved.

Function $\elabExpr$, shown in Fig.~\ref{fig:fb-elabexpr}, performs
this elaboration for a subset of expressions in $\absof{\FB}$, whose
corresponding $\FB$ expressions have been ascribed static semantics in
Fig.~\ref{fig:fb-staticsem}. $\elabExpr$ is defined under the same
context as the expression typing judgment in
Fig.~\ref{fig:fb-staticsem} with symbols $\A$,$\ralloc$, and $\env$
retaining their meaning. The function traverses expressions in a
syntax-directed manner of a type checker, introducing fresh region
type templates for unknown region types, while generating constraints
over region and predicate variables.
Note that $\elabExpr$ returns the type of the subexpression,
which is used to generate constraints for the expression. Functions
$\typeOk$ and $\subtypeOk$ (definitions not shown) used by $\elabExpr$
implement type well-formedness and subtype judgments from
Fig.~\ref{fig:fb-staticsem}, respectively.

Functions $\elabMeth$ and $\elabClass$ shown in
Fig.~\ref{fig:fb-elabmeth} lift expression elaboration to method and
class definitions, respectively. Both functions first build a context
($\A$) containing a set ($ \rhoenv$) of region variables denoting
regions that are currently live, a map ($\aenv$) mapping type
variables to their bounds, and a constraint formula ($\phicx$)
capturing constraints over live region variables. We use predicate
variables ($\varphi$ and $\varphi_m$) to capture constraints over
variables in $\rhoenv$ denoting the fact that such constraints are yet
to be inferred.

Function $\elabMeth$ elaborates a method definition of class $B$. It
calls $\elabExpr$ with the context $\A$, its allocation context
parameter ($\rhoallocm$), and a type environment ($\env$) that
contains region type bindings for all the arguments of the method,
including the implicit $\C{this}$ argument. The type 
returned by $\elabExpr$ for the method body is checked against its
expected type (derived from the type of the method)
generating more constraints. The function then returns the elaborated
method definition and the set of constraints.

$\elabClass$ elaborates the definition of a class $B$. It relies on
$\elabCons$\footnote{The definition of $\elabCons$ is straightforward,
hence not shown.} and $\elabMeth$ functions to elaborate $B$'s
constructor ($k$) and method definitions ($\bar{d}$), respectively. To
the set of constraints returned by these functions, $\elabClass$ adds
constraints generated by checking the well-formedness of the type
templates of its superclass and fields, and also a new constraint
capturing a couple of safety conditions: first, the allocation regions
of objects referred by the instance variables should outlive the
allocation region of the instance itself, and second, the allocation
regions of a class type and its superclass type must be the same.

Function $\elabClassTable$ (Fig.~\ref{fig:fb-elabmeth}) elaborates
every definition in the class table $CT$, while accumulating
constraints.

