\begin{theorem}
\emph{(\textbf{Progress})}
\label{thm:fb-progress}
$\forall e, \tau, \rhoenv, \rhomap, \phicx, \rgn$, if $\frv(e)
\subseteq dom(\rhomap)$ and $\tywf{\Delta}{\phicx}$ and
$\hastyp{\emptyA,\cdot}{e}{\tau}$, then one of the following holds:\\
  \begin{smathpar}
  \begin{array}{rl}
    (i) & \exists (e',\rhomap').\;\redstoo{(e,\rhomap)}{(e',\rhomap')}\\
    (ii) & \valuee(e)\\
    (iii) & \redstoo{(e,\rhomap)}{\invalidexn}\\
  \end{array}
  \end{smathpar}
\end{theorem}
\begin{proof}
Intros $e$. Induction on $e$. For every subexpressions $e_0$, inductive
hypothesis gives us the following:
\begin{smathpar}
\begin{array}{cr}
  \hspace*{-1in}\forall (\tau_0, \rhoenv_0, \rhomap_0, \phicx_0, \rgn_0). 
    (\tywf{\rhoenv_0}{\phicx_0}) \conj
    (\frv(e_0)\subseteq dom(\rhomap_0)) \conj
    \hastyp{(\rhoenv_0,\cdot,\phi_0), \cdot}{e_0}{\tau_0} \Rightarrow& IH1\\
       (\exists(e_0',\rhomap_0'). \redstoo{(e_0,\rhomap_0)}
                {(e_0',\rhomap_0')}) \disj (\valuee(e_0)) 
       \disj (\redstoo{(e_0,\rhomap_0)}{\invalidexn})& \\
\end{array}
\end{smathpar}
Cases from the induction:
\begin{itemize}
  \item Cases ($e = \unitval$ and $e = x$): proof trivial.

  \item Case ($e = e_0.f_i$): Intros. Hypothesis:
  \begin{smathpar}
  \begin{array}{cr}
    \frv(e)\subseteq dom(\rhomap) & H2\\
%   \rgn \in \rhoenv & H2\\
    \hastyp{\emptyA, \cdot}{e}{\tau} & H5\\
  \end{array}
  \end{smathpar}
  Inverting $H5$:
  \begin{smathpar}
  \begin{array}{cr}
    \hastyp{\emptyA, \cdot}{e_0}{\tau'} & H7\\
    \bar{f} :\taubar = \fields(\bound_{\cdot}(\tau')) & H9\\
  \end{array}
  \end{smathpar}
  Applying $H7$ in $IH1$, we have three cases:
  \begin{itemize}
    \item SCase ($e_0$ takes a step): Hypotheses:
    \begin{smathpar}
    \begin{array}{cr}
      \redstoo{(e_0,\rhomap)}{(e_0',\rhomap_0')} & H11\\
    \end{array}
    \end{smathpar}
    Therefore $(e_0.f_i,\rhomap)$ takes a step to $(e_0'.f_i,\rhomap_0')$ under $\rhoenv$.
    \item SCase ($e_0$ is a value): Since $e_0$ has type $\tau'$ and $\bound_{\cdot}$ is defined for
    $\tau'$, it follows that $e_0$ is $\C{new} \fbN(\vbar)$. From $H7$:
    \begin{smathpar}
    \begin{array}{cr}
      \hastyp{\emptyA, \cdot}{\C{new} \fbN(\vbar)}{\tau'} & H14\\
    \end{array}
    \end{smathpar}
    Inverting $H14$: 
    \begin{smathpar}
    \begin{array}{cr}
      \tywf{\emptyA}{N} & H16\\
    \end{array}
    \end{smathpar}
    Inverting $H16$:
    \begin{smathpar}
    \begin{array}{cr}
      \allocRgn(N) \in \rhoenv & H18\\
    \end{array}
    \end{smathpar}
    From $H9$, $H18$, we know that $(e_0.f_i,\rhomap)$ takes a step to
    $(v_i,\rhomap)$
    \item SCase ($e_0$ raises $\invalidexn$): $e_0.f_i$ also raises
    $\invalidexn$.
    \end{itemize}

  \item Case ($e = \letregion{\rgn_0}{e_0}$): Intros. Hypothesis:
  \begin{smathpar}
  \begin{array}{cr}
    \frv(e)\subseteq dom(\rhomap) & H2\\
    \hastyp{\emptyA, \cdot}{e}{\tau} & H5\\
  \end{array}
  \end{smathpar}
  Inverting $H5$:
  \begin{smathpar}
  \begin{array}{cr}
    \rgn_0 \notin \rhoenv & H7\\
    \hastyp{\emptyADelcupPhicap{\rgn_0}{\rgn_0}, \rgn_0, \cdot}{e_0}{\tau} & H9\\
  \end{array}
  \end{smathpar}
  From $H9$ and $IH1$, we have three cases:
  \begin{itemize}
    \item SCase ($e_0$ takes a step). Hypotheses:
    \begin{smathpar}
    \begin{array}{cr}
      \redsto{\rhoenv \cup \{\rgn_0\}}{(e_0,\rhomap)}{(e_0',\rhomap_0')} & H11\\
    \end{array}
    \end{smathpar}
    From $H7$ and $H11$, $\redstoo{(e,\rhomap)}{(\letregion{\rgn_0}{e_0'},\rhomap_0')}$.
    \item SCase ($e_0$ is a value $v_0$): From $H7$, $\redstoo{(e,\rhomap)}{(v_0,\rhomap)}$
    \item SCase ($e_0$ raises $\invalidexn$): $e$ raises $\invalidexn$ too.
  \end{itemize}
  
  \item Case ($e = \open{e_a}{\rgn_0}{y}{e_b}$): Intros. Hypotheses:
  \begin{smathpar}
  \begin{array}{cr}
    \tywf{\Delta}{\phicx} & H1\\
    \frv(e)\subseteq dom(\rhomap) & H2\\
    \hastyp{\emptyA, \cdot}{e}{\tau} & H5\\
%   \forall (\tau, \rhoenv, \rhomap, \rgn). \rgn \in \rhoenv \conj
%     \hastyp{\emptyA, \cdot}{e_0}{\tau} \;
%     \Rightarrow \; \exists(e',\rhomap'). \redstoo{(e_0,\rhomap)}
%                     {(e',\rhomap')} & IH1\\
  \end{array}
  \end{smathpar}
  $H2$ and the definition of $\frv$ imply:
  \begin{smathpar}
  \begin{array}{cr}
    \frv(e_a)\subseteq dom(\rhomap) & H3\\
  \end{array}
  \end{smathpar}
  Inverting $H5$:
  \begin{smathpar}
  \begin{array}{cr}
    \hastyp{\emptyA, \cdot}{e_a}{\RgnZ\inang{T}\inang{\toprgn}} & H7\\
    \rgn_0 \notin \rhoenv & H9\\
    \hastyp{\emptyADelcup{\rgn_0},[y \mapsto T@\rgn_0]}{e_b}{\tau} & H11\\
  \end{array}
  \end{smathpar}
  We have three cases. First case deals with $(e_a,\rhomap)$ taking a step to $(e_a',\rhomap_a')$
  under $\rhoenv$. Under this context, $(e,\rhomap)$ takes a step to
  $(\open{e_a'}{\rgn_0}{y}{e_b},\rhomap_a')$. So, there is progress.  Second case deals with $e_a$
  raising $\invalidexn$. In this case, execution of $e$ also raises $\invalidexn$. So, we again have
  progress. Third case deals with $e_a$ being a value $\C{new}\;\fbN(\vbar)$. Inverting $H7$, we
  have to consider two possible derivations: one from the generic type rule for values of any type,
  and another from the type rule tailor-made for $\RgnZ$ values. The first rule does not apply
  because $\fields(\RgnZ\inang{T} \inang{\toprgn})$ is undefined. The only rule that applies is the
  special type rule for $\RgnZ$ values. Hence, $e_a$ is $\C{new}\; \RgnZ\inang{T}\inang{\rgn_i}(v)$, where:
  \begin{smathpar}
  \begin{array}{cr}
    \rgn_i \notin \rhoenv & H12\\
    \hastyp{(\{\rgn_i\},\cdot,true), \cdot}{v}{T@\rgn_i} & H14\\
%   \hastyp{\emptyA,
%   \cdot}{\lambdaexp{\rgn}{\rhoalloc}{}{v}}{\inang{\rhoalloc}\unitZ \xrightarrow{\rgn} T@\rhoalloc} & H14\\
  \end{array}
  \end{smathpar}
%  $H13$ follows from the fact that for $\C{new} \RgnZ\inang{T}\inang{\rgn_i}(...)$ to be a value,
%  $\rgn_i$ cannot be $\toprgn$. i
% Inversion on $H14$:
% \begin{smathpar}
% \begin{array}{cr}
%   \rhoalloc \notin \rhoenv & H15\\
%   \hastyp{(dom(\rhomap), \rhoenv \cup \{\rhoalloc\},\cdot,true),\rhoalloc, \cdot}{v}{T@\rhoalloc} & H16\\
% \end{array}
% \end{smathpar}
  Since $e_a$ is $\RgnZ\inang{T}\inang{\rgn_i}(v)$, $H3$ implies: 
  \begin{smathpar}
  \begin{array}{cr}
    \rgn_i \in dom(\rhomap) & H13\\
  \end{array}
  \end{smathpar}
  Renaming the region variable in $H14$:
  \begin{smathpar}
  \begin{array}{cr}
    \hastyp{(\{\rgn_0\},\cdot,true), \cdot}{[\rgn_0/\rgn_i]\,v}{T@\rgn_0} & H15\\
  \end{array}
  \end{smathpar}
  $H9$ and $H1$ tell us that it is safe to strengthen the context in
  $H16$ to the following:
  \begin{smathpar}
  \begin{array}{cr}
    \hastyp{\emptyADelcup{\rgn_0},\cdot}{[\rgn_0/\rgn_i]\,v}{T@\rgn_0} & H16\\
  \end{array}
  \end{smathpar}
  Applying substitution lemma (Lemma~\ref{thm:fb-substitution}) on $H11$ and
  $H16$ gives us:
  \begin{smathpar}
  \begin{array}{cr}
    \hastyp{\emptyADelcup{\rgn_0},\cdot}{[[\rgn_0/\rgn_i]\,v/y]\,e_b}{\tau} & H19\\
  \end{array}
  \end{smathpar}
% From $H9$, $H13$, $H16$, and Lemma~\ref{thm:fb-renaming}:
% \begin{smathpar}
% \begin{array}{cr}
%   \hastyp{\emptyADelcup{\rgn_0},\rgn_0, \cdot}{[\rgn_0/\rgn_i]v}{T@\rgn_0} & H18\\
% \end{array}
% \end{smathpar}
% From $H11$, $H18$ and Lemma~\ref{thm:fb-substitution}:
% \begin{smathpar}
% \begin{array}{cr}
%   \hastyp{\emptyADelcup{\rgn_0},\rgn_0,\cdot}{[[\rgn_0/\rhoalloc]v/y]e_b}{\tau} & H19\\
% \end{array}
% \end{smathpar}
  Since $\rgn_i \in dom(\rhomap)$ (from $H13$), we have three cases:
  \begin{itemize}
    \item SCase ($\rhomap(\rgn_i) \neq \XFERRED$ and $e_b$ is not a value): By inductive hypothesis,
    $([[\rgn_0/\rgn_i]v/y]e_b,\rhomap[\rho \mapsto \OPEN]$ can either (a). take a step to
    $(e_b',\rhomap')$ under $\rhoenv \cup \{\rgn_0\}$, or (b). $e_b$ evaluates to $\invalidexn$.  In
    the first case, $(e,\rhomap)$ itself evaluates to: 
    \begin{smathpar}
    \begin{array}{cr}
      (\open {(\C{new} \; \RgnZ\inang{T}\inang{\rgn_i}(v))}
             {\rgn_0}{y}{e_b'}, \rhomap')
    \end{array}
    \end{smathpar}
    In the second case, the evaluation of $e$ also raises $\invalidexn$.

    \item SCase($\rhomap(\rgn_i) \neq \XFERRED$ and $e_b$ is a value $v_b$): Trivially,
    $\redstoo{(e,\rhomap)}{(v_b,\rhomap)}$.

    \item SCase($\rhomap(\rgn_i) = \XFERRED$): $e$ raises $\invalidexn$.
  \end{itemize}
 
  \item (e = $e_a.m\inang{\rbar}(\ebar)$): Intros. Hypotheses:
  \begin{smathpar}
  \begin{array}{cr}
%   \rgn \in \rhoenv & H2\\
    \tywf{\rhoenv}{\phicx} & H1\\
    \frv(e)\subseteq dom(\rhomap) & H2\\
    \hastyp{\emptyA, \cdot}{e}{\tau} & H4\\
  \end{array}
  \end{smathpar}
  From $H2$ and the fact that $e_a$ and $e_i$ (forall $i$) are
  subexpressions of $e$, we have:
  \begin{smathpar}
  \begin{array}{cr}
    \frv(e_a)\subseteq dom(\rhomap) & H3\\
    \frv(e_i)\subseteq dom(\rhomap) & H5\\
  \end{array}
  \end{smathpar}
  By inversion on $H4$:
  \begin{smathpar}
  \begin{array}{cr}
    \hastyp{\emptyA,\cdot}{e_a}{\tau_a} & H6\\
%   \ralloc = \rgn & H7\\
    \rbar \in \rhoenv & H8\\
    \mtype(m,\bound_{\cdot}(\tau_a)) = \inang{\rhobar\,|\,\phi}\bar{\tau^1} \rightarrow \tau^2 & H10\\
    \hastyp{\emptyA,\cdot}{\ebar}{[\rbar/\rhobar]\bar{\tau^1}} & H11\\
    \isvalid{\phicx}{[\rbar/\rhobar]\phi}
  \end{array}
  \end{smathpar}
  Three cases: 
  \begin{itemize}
    \item SCase ($e_a$ isn't a value): From $H1$, $H3$, $H6$ and $IH$,
    we know that either (a). $e_a$ can take a step, or (b) $e_a$
    reduces to $\invalidexn$. In first case, $e$ can also take a step,
    and in second case, $e$ also reduces to $\invalidexn$.

    \item SCase ($e_a = v_a$, but $\exists i$ such that  $e_i$ isn't a
    value): From $H1$, $H5$, $H11$ and $IH$, we know that
    $(e_i,\rhomap)$ can either (a). take a step to $(e_i',\rhomap')$
    under $\rhoenv$, or (b).  $e_i$ reduces to $\invalidexn$. In the
    first case, we have
    $\redstoo{(v_a.m\inang{\rbar}(...,e_i,...),\rhomap)}
    {(v_a.m\inang{\rbar}(...,e_i',...),\rhomap')}$.

    \item SCase ($e_a = v_a$ and $\forall i.\,e_i = v_i$): $H10$ says
    that bound for $\tau_a$ is defined under empty $\aenv$. This is
    possible only if $\tau_a = \fbN$ and $v_a = \C{new}\;
    \fbN(\bar{v'})$. Furthemore, $\fbN$ cannot be of form
    $\RgnZ\inang{T}\inang{\toprgn}$ because, $\mtype$ isn't defined
    for $\RgnZ$ (in short, inverting $H10$ gives us $\tau_a = \fbN$,
    for some $N \neq \RgnZ\inang{T}\inang{\toprgn}$). Using these
    facts, and inverting $H6$, we get $\tywf{\emptyA}{\fbN}$.
    Inverting it again:
    \begin{smathpar}
    \begin{array}{cr}
      \allocRgn(\fbN) \in \rhoenv & H13\\
    \end{array}
    \end{smathpar}
    Now, since $\mtype(m,\fbN)$ is defined if and only if $\mbody(m,\fbN)$ is defined, we know that:
    \begin{smathpar}
    \begin{array}{cr}
      \mbody(m,\fbN) = \rhobar.\xbar.\,e_m & H14\\
    \end{array}
    \end{smathpar}
    From $H13$ and $H14$, we know that $\redstoo{((v_a.m\inang{\rbar}(\bar{v}),\rhomap)}
      {([\bar{v}/\xbar][\C{new}\;\fbN(\bar{v'})/\thisZ][\rbar/\rhobar]e_m, \rhomap)}$
  \end{itemize}

  \item Case ($e = e_a\inang{\rbar}(\ebar)$). Proof closely follows the proof for
  $e_a.m\inang{\rbar}(\ebar)$. The only difference is that when $e_a$ evaluates to a lambda
  $\lambdaexp{\rgn}{\rhobar \,|\, \phi}{\taubar \; \xbar}{e_b}$, we need a proof that
  $\rgn \in \rhoenv$. This can be obtained by inverting the type judgment for the lambda.

  \item Case($e = \C{new}\; \fbN(\ebar)$): Intros. Hypotheses:
  \begin{smathpar}
  \begin{array}{cr}
    \tywf{\rhoenv}{\phicx} & H1\\
    \frv(e)\subseteq dom(\rhomap) & H2\\
    \hastyp{\emptyA, \cdot}{\C{new}\; \fbN(\ebar)}{\tau} & H4\\
  \end{array}
  \end{smathpar}
  Inverting $H4$ leads to two cases:
  \begin{itemize}
    \item SCase ($\shape(\fbN) \neq \RgnZ\inang{T}$): Hypotheses:
    \begin{smathpar}
    \begin{array}{cr}
      \tywf{\emptyA}{\fbN} & H5\\
      \fields(\fbN) = \bar{f}:\taubar & H7\\
      \hastyp{\emptyA,\cdot}{\ebar}{\taubar} & H8\\
    \end{array}
    \end{smathpar}
    Inverting $H5$:
    \begin{smathpar}
    \begin{array}{cr}
      \allocRgn(\fbN) \in \rhoenv & H10\\
    \end{array}
    \end{smathpar}
    Three cases:
    \begin{itemize}
      \item SSCase ($\exists i$ such that $e_i$ takes a step): In this case, $e$ also takes a step.
      \item SSCase ($\exists i$ such that $e_i$ reduces to $\invalidexn$). In this case, $e$ also
      reduces to $\invalidexn$.
      \item SSCase ($\forall i$ $e_i$ is a value $v_i$): In this case, $e$ is also a value
      $\C{new}\;\fbN(\vbar)$.
    \end{itemize}
    
    \item SCase ($\shape(\fbN) = \RgnZ\inang{T}$ and $e =
    \RgnZ\inang{T}\inang{\rgn_i}(e_0)$): From $H4$:
    \begin{smathpar}
    \begin{array}{cr}
      \hastyp{\emptyA, \cdot}{\C{new}\; \RgnZ\inang{T}\inang{\rgn_i}(e_0)}{\tau} & H4\\
    \end{array}
    \end{smathpar}
    Inversion on $H4$ gives two cases:
    \begin{itemize}
      \item SSCase($\rgn_i = \toprgn$): In this case, $e_0 = \lambdaexp{\rgn}{\rho}{}{e_1}$. 
      In this case, $\redstoo{(\C{new}\; \RgnZ\inang{T}\inang{\toprgn}(e_0),\rhomap)} {(\C{new}\;
      \RgnZ\inang{T}\inang{\rgn_j}([\rgn_j/\rho]e_1), \rhomap[\rgn_j \mapsto \CLOSED])}$, where
      $\rgn_j$ is a new region identifier such that $\rgn_j \notin
      \rhoset \cup dom(\rhomap)$
%     Hypotheses:
%     \begin{smathpar}
%     \begin{array}{cr}
%       \fgjtywf{\cdot}{T} & H12\\
%       \rgn_i = \toprgn & H13\\
%       \hastyp{\vacantA{\rhoalloc},\rhoalloc,\cdot}{e_1} {T@\rhoalloc}& H14\\
%     \end{array}
%     \end{smathpar}
%     From $H14$ and $IH1$, $e_1$ can either take a step, or raise $\invalidexn$, or is a value. In the first
%     two cases, $e$ takes a step, $e$ raises $\invalidexn$, respectively. In the third case, $e_0$
%     is a value $v_0$. By inversion on $H14$, we know that $v_0 =
%     \lambdaexp{\rgn}{\rhoalloc}{}{e_1}$.  In this case, $\redstoo{(\C{new}\;
%     \RgnZ\inang{T}\inang{\toprgn}(v_0),\rhomap)} {(\C{new}\;
%     \RgnZ\inang{T}\inang{\rgn_j}([\rgn_j/\rhoalloc]e_1), \rhomap[\rgn_j \mapsto \CLOSED])}$, where
%     $\rgn_j$ is fresh ($\rgn_j \notin \rhoenv \cup dom(\rhomap)$).

      \item SSCase ($\rgn_i \neq \toprgn$): Hypotheses:
      \begin{smathpar}
      \begin{array}{cr}
        \rgn_i \in dom(\rhomap) & H16\\
        \rgn_i \notin \rhoenv \cup \{\toprgn\} & H18\\
        \hastyp{\emptyADelcup{\rgn_i},\cdot} {e_0}{T@\rgn_i} & H20\\
      \end{array}
      \end{smathpar}
      $H16$ comes from $H2$. Applying $IH1$ using $H1$, $H2$, $H18$ and $H20$, we have three cases:
      \begin{itemize}
        \item SSSCase ($e_0$ is a value $v_0$): In this case, $e = \RgnZ\inang{T}\inang{\rgn_i}(v_0)$ is also
        a value.
        \item SSSCase ($e_0$ raises $\invalidexn$): In this case, $e$ also raises $\invalidexn$.
        \item SSSCase ($\redsto{\rhoenv \cup \{\rgn_i\}}{(e_0,\rhomap)} {(e_0',\rhomap')}$): In this
        case, $\redstoo{(\RgnZ\inang{T}\inang{\rgn_i}(e_0),\rhomap)}
        {(\RgnZ\inang{T}\inang{\rgn_i}(e_0'), \rhomap')}$
      \end{itemize}
    \end{itemize}
  \end{itemize}
%   \begin{smathpar}
%   \begin{array}{cr}
%     \rhoalloc \notin \rhoenv & H17\\
%     \hastyp{(dom(\rhomap), \rhoenv \cup \{\rhoalloc\},\cdot,true),\rgn,\cdot}
%           {e_1}{T@\rhoalloc} & H19\\
%   \end{array}
%   \end{smathpar}
%   Consider a fresh $\rgn_j$:
%   \begin{smathpar}
%   \begin{array}{cr}
%     \rgn_j \notin \rhoenv \cup dom(\rhomap) & H20\\
%   \end{array}
%   \end{smathpar}
%   From $H17$, $H19$, $H20$, and Lemma~\ref{thm:fb-renaming}, we have:
%   \begin{smathpar}
%   \begin{array}{cr}
%     \hastyp{(dom(\rhomap), \rhoenv \cup \{\rhoalloc\},\cdot,true),\rgn,\cdot}
%           {e_1}{T@\rhoalloc} & H19\\
%   \end{array}
%   \end{smathpar}
    
  \item Case ($e = e_0.\transfer\inang{\rgn}()$): Intros. Hypotheses:
  \begin{smathpar}
  \begin{array}{cr}
    \tywf{\rhoenv}{\phicx} & H1\\
    \frv(e)\subseteq dom(\rhomap) & H2\\
    \hastyp{\emptyA, \cdot}{e_0.\transfer\inang{\ralloc}()}{\tau} & H4\\
  \end{array}
  \end{smathpar}
  By inversion on $H4$:
  \begin{smathpar}
  \begin{array}{cr}
    \hastyp{\emptyA,\cdot}{e_0}{\RgnZ\inang{T}\inang{\toprgn}} & H6\\
  \end{array}
  \end{smathpar}
  Now, if $e_0$ can take a step, so can $e$, hence there is progress. Else, if $e_0$ raises an
  exception, so does $e$. The only non-trivial case is when $e_0$ is a value. But, only values of
  type $\RgnZ\inang{T}\inang{\toprgn}$ is $\C{new}\;\RgnZ\inang{T}\inang{\rgn_i}(...)$, where
  $\rgn_i \neq \toprgn$. From $H2$, we get:
  \begin{smathpar}
  \begin{array}{cr}
    \rgn_i \in dom(\rhomap) & H8\\
  \end{array}
  \end{smathpar}
  We have two cases:
  \begin{itemize}
    \item ($\rhomap(\rgn_i) \neq \OPEN$): In this case,
    $\redstoo{(e,\rhomap)}{(\unitval,\rhomap[\rgn_i \mapsto \XFERRED])}$.

    \item ($\rhomap(\rgn_i) = \OPEN$): In this case, evaluation of $e$ raises $\invalidexn$.
  \end{itemize}

  \item Case ($e$ is a lambda abstraction): $e$ is already a value.

  \item Case ($e = e_1;\,e_2$): Proof trivial.
% End of proof cases.
\end{itemize}

\qed
\end{proof}

