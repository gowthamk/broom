\section{Overview of \name}
\label{sec:overview}

Stripped of its region-specific constructs, the core language of \name
is a familar subset of \csharp that includes features such as
class-based inheritance, dynamic dispatch, parametric polymorphism,
and lambdas.  We therefore elide a discussion of the core language,
focusing instead on the concepts specific to region-based memory
management in \name.

\subsection{Stack Regions and Outlives Relation}

Simplest of the region-specific constructs in \name is a \C{letregion}
lexical block that creates a new memory region adressable via its
static identifier. The semantics of the \C{letregion} block is similar
to Tofte and Talpin~\cite{ttpopl94}'s \C{letregion} expression; the
newly introduced region is live only inside the block, where its
identifier is in scope for \C{new} expressions to make allocations
inside the region. For example, in the following code, object \C{a} is
allocated in the region with identifier \C{R0} introduced by the
\C{letregion}:
\begin{center}
\begin{codejava}
  letregion R0 {
    Object a = new@R0 Object();
  }
\end{codejava}
\end{center}
The \C{letregion} blocks can be nested, leading to a stack of regions
that are deallocated in the reverse order in which they are allocated.
Following ~\cite{cyclonepldi02}, we therefore call regions introduced
by \C{letregion} expressions as \emph{stack regions}. The stack
discipline induces \emph{outlives} relationship among regions created
by nested \C{letregion}s, where the region introduced by the outer
\C{letregion} is guaranteed to outlive the one introduced by the inner
\C{letregion}. It is therefore safe to refer to an object allocated in
outer region from the inner region, but the converse is not true. For
example, consider the following code (assume that class \C{A} has
field \C{x} of type \C{Object}):
\begin{center}
\begin{codejava}
  letregion R0 {
    A a0 = new@R0 A();
    letregion R1 {
      A a1 = new@R1 A();
      a1.x = new@R0 Object();
      a0.x = new@R1 Object();
      ...
    }
    ...
  }
\end{codejava}
\end{center}
The code creates two stack regions with identifiers \C{R0} and \C{R1},
where the region \C{R0} outlives the region \C{R1} (denoted as $\C{R0}
\outlives \C{R1}$).  Objects \C{a0} and \C{a1} are allocated in regions
\C{R0} and \C{R1}, respectively. The first assignment statement
assigns to \C{a1.x} an object allocated in outer region (\C{R0}). This
assignment is safe as \C{a1.x} refers to a longer living object, hence
is guaranteed to be a valid reference throughout the lifetime of
\C{a1}.  In contrast, the second assignment is unsafe, as it assigns
to \C{a0.x} an object, whose lifetime is shorter than the lifetime of
\C{a0}, making it unsafe to dereference \C{a0.x} outside the inner
block. 
% Unsafe assignments can also happen indirectly via a function call..
Preempting such unsafe assignments is the \emph{raison d'etre} of the
region type system.

\paragraph{A Note on region identifiers} In \name, every \C{letregion}
expression introduces a new stack region with a static identifier that
uniquely identifies the region within its scope. At runtime, a
\C{letregion} expression could be evaluated multiple times if, for
instance, it is wrapped inside a \C{while} loop or a recursive method,
introducing multiple stack regions with the same static identifier.
Which among these regions the identifier is bound to under a given
runtime context is then determined by the scope of the identifier.
Furthermore, any proposition involving static region identifiers is
true if and only if the proposition is true in \emph{all} possible
runtime contexts containing bindings for those identifiers. For
instance, consider the following example:
%\begin{minipage}{1.55in}
\begin{codejava}
for (int i=0; i<=10; i++) {
  letregion R0 {
    ...
    letregion R1 {
      A a1 = new@R1 A();
    }
  }
}
\end{codejava}
%\end{minipage}
%\begin{minipage}{1.4in}
%\begin{codejava}
%void m(int i) {
%  letregion R0 {
%    A a = new@R0 A();
%  }
%  this.m(++i);
%}
%...
%this.m(1);
%\end{codejava}
%\end{minipage}
When \C{i=2}, the \C{new} expression refers to the \C{R1} (and the
corresponding stack region) introduced in the 2nd iteration of the
loop, as it is the only \C{R1} in scope. Furthermore, the proposition
\emph{\C{a1} refers to an object of type \C{A} contained in region
\C{R1}} (denoted succinctly as $\C{a1}:\C{A@R1}$) is true because it
is true for all stack regions identified by \C{R1} at runtime.
Likewise, the propostion $\C{R0} \outlives \C{R1}$ is true because it
is true in every runtime context that binds \C{R0} and \C{R1} to
regions .

Note that it is possible for multiple stack regions introduced by
evaluating a \C{letregion} multiple times in a recursive definition to
be in outlives relationship. For instance, in the following example:
\begin{codejava}
void m(A a) {
  letregion R0 {
    B b = new@R0 B();
    b.x = a;
    this.m(new@R0 A());
  }
}
\end{codejava}
stack regions identified by \C{R0} introduced in successive recursive
calls are in outlives relationship. Consequently, the assignment to
\C{b.x} should be considered safe. Region type system has to allow for
this possibility.

\subsection{Allocation Context}
\label{sec:alloc-ctxt}

As evident from the examples shown above, the suffix \C{@R} in
\C{new@R} instruct the \C{new} keyword to allocate the newly created
object in the region labeled \C{R}. However, \name does not require
the allocation region to be specified following a \C{new}. In the
absence of explicit specification, \name allocates the newly created
object in its \emph{allocation context}, which is the region
introduced by the closest enclosing \C{letregion} block. Besides being
a syntactic convenience, the facility to elide allocation region
specification is imperative to interface region-oblivious standard
library code with region-aware application code. For instance, the
following \name code, which makes explicit use of regions, calls
the standard library's iterator methods, which are region-oblivious:
\begin{center}
\begin{codejava}
  int findMin(List<int> l) {
    assert(!l.isEmpty());
    int min = l.get(0);
    letregion R0 {
      ListIterator<int> iter = l.listIterator();
      while(iter.hasNext()) {
        int n = iter.getNext();
        if (n<min) min = n;
      }
    }
    return min;
  }
\end{codejava}
\end{center}
Due to the absence of new region creations, and explicit region
annotations with \C{new} expressions, all new objects created by a
standard library method are allocated in the allocation context for
the method call. In the above example, the allocation context for the
call to \C{List} library's \C{listIterator} method is the region
\C{R1}. Therefore, all objects created by the method, including the
iterator object, are allocated in \C{R0}. This fact should be captured
appropriately by the region type of \C{listIterator} method if the
type system were to prove the safety of the above code.

\subsection{Boxed and Unboxed Values}

In \name, like in \csharp, values of some primitive types, such as
integers and bools, are unboxed, whereas others, such as strings are
arrays, are boxed. Unlike unboxed values, boxed values are objects,
hence passed by reference. It is therefore important to be mindful
of allocation regions for strings and arrays, but not so much for
integers and bools. For instance, in the following code, the
assignment to integer variable \C{x} is safe, but the assignment to
the string variable \C{y} is unsafe, as the allocation region (\C{R0})
of the string constant has lesser lifetime than \C{y}.
\begin{center}
\begin{codejava}
  int x;
  string y;
  letregion R0 {
    x = 2;
    y = "Hello, World!";
  }
  ...
\end{codejava}
\end{center}

\subsection{Dynamic (Transferable) Regions}

\name's stack regions have lexically scoped lifetimes, hence are not
suitable for applications that need to transfer data between
autonomous actors, such as \naiad dataflow operators. To support these
applications, it is imperative to support regions whose lifetimes
transcend the boundaries of lexical blocks, functions, or even
programs. Regions that fit this description in \name are called
\emph{dynamic regions}, or \emph{transferable regions}. 

Transferable regions, like stack regions, are memory regions
Transferable regions are first class values of \name; they are
considered objects of \C{Region} class, and, like objects of other
classes, they are created using the \C{new} keyword, can be passed as
arguments, and returned from methods. Like a stack region, a
transferable region facilitates arbitrary object allocations, but
unlike a stack region, it has a well-defined \emph{root} object
denoting the root of the data structure being transferred. In a
typical dataflow computation, an upstream actor (e.g., a \C{SELECT}
operator) constructs a transferable region, sets its root to the data
structure supposed to contain intermediate results, populates the data
structure with intermediate results, and then transfers it to a
downstream actor (e.g., a \C{COUNT} operator), which performs further
processing. Since a transferable region escapes the lifetime of the
sender, there must be no references from inside of the transferable
region to objects allocated in other memory regions of the sender.
Such references, if exist, may become invalid references in the
context of recipient's address space, jeopardizing memory safety.
\name relies on its region type system to prevent such unsafe
references from being created.

\begin{figure}
\begin{codejava}
  public class Region<T> {
    public Region(Func<void,T> mkRoot);
    public void free();
    public void transfer();
  }
\end{codejava}
\caption{The type signature of the \C{Region} class}
\label{fig:region-class}
\end{figure}
The type signature of \C{Region} class is described in
Fig.~\ref{fig:region-class}. The class is parametric over the type of
the root object (i.e., the \emph{root type}) of the transferable
region it creates. The \C{Region} constructor expects a \C{Func}
object - a higher-order function argument that returns an object of
root type.  The idea is to execute this function with the newly
created transferable region as its allocation context. The object thus
created inside the transferable region will serve as its root object.
For instance, the following code creates a transferable region
(\C{rgn}) and sets its root to the string object returned by its
\C{Func} argument (a lambda expression):
\begin{codejava}
  Region<string> rgn = new Region<string>
                        (() => "Hello");
\end{codejava}
Among the methods of the \C{Region} class, \C{free} deallocates
region's memory, while \C{transfer} transfers the region to a
downstream actor as determined by the run-time. The precise semantics
of \C{transfer} are unimportant in the context of the region type
system; it suffices to understand \C{transfer} as an alias of
\C{free}.

A transferable region is said to be in \emph{closed} state when
created. \name provides \C{openalloc} lexical construct to \emph{open}
a transferable region as an allocation context, and to access its root
object:
\begin{codejava}
  Region<string> rgn = new Region<string>
                        (() => "Hello");
  openalloc rgn withroot str {
    print str; //prints "Hello"
    str = str + " World!"; 
  }
\end{codejava}
In the above code, the \C{openalloc} lexical block opens \C{rgn} for
allocation, and sets the newly introduced local variable (\C{str}) to
refer to its root object.  The assignment statement within the block
updates the root. Since \C{rgn} is the allocation context within the
block, the new string is allocated inside \C{rgn}, hence the
assignment is safe. In some cases, it is desirable to open a
transferable region only to read its contents, but not to perform any
allocations. For this purpose, \name provides \C{open} lexical
construct that, like \C{openalloc}, opens a transferable region and
names its root object, but unlike \C{openalloc}, does not make it an
allocation context. A transferable region can be \C{openalloc}'d or
\C{open}'d many times, before it is eventually \C{free}'d or
\C{transfer}'d.

The design decisions reflected above, \emph{viz.}, to ascribe a root
object to every transferable region, to make use of a higher-order
argument to initialize the root object, and to require that a region
be explictly opened in a lexical block for allocations, were made to
reflect the message semantics of transferable regions, to promote
uniform style of programming with different kinds of regions, and to
precisely state and enforce the safety guarantees offered by \name's
region type system (\S~\ref{sec:type-system}).

\begin{figure}[t!]
\begin{numcodejava}
class SelectVertex<TIn extends Cloneable, TOut> {
  Func<TIn, TOut> mapFn;
  Func<TIn, bool> whereFn;
  SelectVertex(Func<TIn, TOut> mapFn, 
               Func<TIn, bool> whereFn) {
    this.mapFn = mapFn;
    this.whereFn = whereFn;
  }
  void onReceive(Region<List<TIn>> inRgn) {
    Func<void,List<TOut>> f = 
        () => new List<TOut>();
    Region<List<TOut>> outRgn =
        new Region<List<TOut>>(f);
    open inRgn withroot inMsg {
      letregion R0 {
        ListIterator<TIn> iter = 
            inMsg.listIterator();
        while (iter.hasNext()) {
          TIn inp = iter.getNext();
          if (!this.whereFn(inp)) continue;
          openalloc outRgn withroot outMsg {
            TOut outp = this.mapFn(inp.clone());
            outMsg.add(outp);
          }
        }
      }
    }
    inRgn.free();
    outRgn.transfer();
  }
}
\end{numcodejava}
\caption{\C{SELECT} dataflow operator written in \name}
\label{fig:naiad-ex}
\end{figure}


Fig~\ref{fig:naiad-ex} shows \C{SelectVertex} class, which is a
straightforward implementation of the \C{SELECT} dataflow operator.
\C{SelectVertex} implements \C{onReceive} method, which is called by
the run-time when a message from an upstream actor is ready to be
delivered. A message is usually a sequence of tuples (e.g., key-value
pairs) wrapped inside a transferable region (\C{inRgn}).
\C{SelectVertex} opens its input region \C{inRgn} (line 16), iterates
through the sequence, filtering the input tuples that meet the
selection criterion of the \C{whereFn} (line 19), and applying the
\C{mapFn} to map the selected input tuples to output (line 22).
Observe that the iterator is allocated in the temporary (stack) region
\C{R0} that lives just long enough.  Functions \C{mapFn} and
\C{whereFn} are defined by the users of the dataflow system, hence
they are called \emph{user-defined functions}.  Unlike dataflow system
builders, users are oblivious of regions, hence user-defined functions
are expected to not contain explicit region annotations. While
\C{whereFn} defines the selection criterion that the user is
interested in (e.g: \C{age $\ge$ 18}), \C{mapFn} can either project
attributes of its input tuple amounting to type \C{TOut}, or can
create and return a new \C{TOut} object. Note that, unlike \C{inRgn},
the output transferable region (\C{outRgn}) was opened for allocation
and the \C{mapFn} is applied under this allocation context. Also note
that the input tuple is cloned under this allocation context.
Consequently, the result (\C{outp}) of applying the \C{mapFn} (line
22), which is subsequently added to \C{outMsg}, is guaranteed to be in
the \C{outRgn} regardless of the semantics of \C{mapFn}. On the other
hand, if the input tuple is not cloned (i.e., \C{inp} instead of
\C{inp.clone()} on line 22) and the \C{mapFn} simply projects its
input, then \C{outp} refers to an object inside \C{inp}, which no
longer exists when the \C{outRgn} is transferred to a downstream
actor. Therefore, if \C{mapFn} is applied over \C{inp} instead of
\C{inp.clone()}, then it better be a function that creates and returns
a new \C{TOut} object. Under such scenario, the region type of
\C{SelectVertex.mapFn} should be precise enough to prevent it from
being assigned a function (line 6) that simply returns its input.
Fortunately, \name's region type system (\S~\ref{sec:type-system}) is
capable of capturing such nuances in the type of
\C{SelectVertex.mapFn}. Furthermore, the type can be automatically
inferred by \name's region type inference
(\S~\ref{sec:type-inference}), which can peform the above non-trivial
reasoning on behalf of the programmer.

