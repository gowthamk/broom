\begin{lemma}
\emph{(\textbf{Substitution Preserves Typing})}
\label{thm:fb-substitution}
$\forall e, z, \tau_1, \tau_2, \rhoenv, \phicx$, if $\hastyp{\emptyA,
[z \mapsto \tau_1]}{e}{\tau_2}$ and $\hastyp{\emptyA, \cdot}{v}{\tau_1}$, then 
$\hastyp{\emptyA, \cdot}{[v/z]e}{\tau_2}$
\end{lemma}
\begin{proof}
Intros $e$. Induction on $e$. For every subexpression $e_0$, inductive
hypothesis is the following:
\begin{smathpar}
\begin{array}{cr}
  \forall (z, \tau_1, \tau_2, \rhoset, \rhoenv, \phicx).~ \hastyp{\emptyA,
  [z \mapsto \tau_1]}{e_0}{\tau_2} \spc \conj \spc \hastyp{\emptyA, \cdot}{v}{\tau_1} & IH1 \\
  \Rightarrow \; \hastyp{\emptyA, \cdot}{[v/z]e_0}{\tau_2} & \\
\end{array}
\end{smathpar}
In all the inductive cases, we have the following hypotheses:
\begin{smathpar}
\begin{array}{cr}
  \hastyp{\emptyA,[z \mapsto \tau_1]}{e}{\tau_2} & H2\\
  \hastyp{\emptyA, \cdot}{v}{\tau_1} & H4\\
\end{array}
\end{smathpar}
In each case, proof strategy is the same: invert on $H2$, apply $IH1$, and then construct the
proof term for the goal by applying type rules.
% \begin{itemize}
%   \item Proof for cases $\unitval$, $x$, $e_0.f_i$, $\C{new} \fbN(\bar{e})$, $\C{new}\;
%   \RgnZT{\toprgn}$ follow straightforwardly from the inductive hypothesis.
%   \item Case ($\C{new}\; \RgnZT{\toprgn} : \RgnZT{\toprgn}$): 
% \end{itemize}
\qed
\end{proof}

