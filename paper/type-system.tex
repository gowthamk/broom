\section{The Region Type System}
\label{sec:type-system}

\begin{figure*}[t!]
%
\begin{smathpar}
\renewcommand{\arraystretch}{1.2}
\begin{array}{lclcl} 
\multicolumn{5}{c}{
  {\rgn} \in \mathtt{Static \; region \; ids} \qquad
  {\rho} \in \mathtt{Region \; variables} \qquad
  {\tyvar, \tyvarb} \in \mathtt{Type \; variables} \qquad
  {m} \in \mathtt{Method \; names} \qquad
  {x,y,f} \in \mathtt{Variables \; and \; fields} }\\
cn & \in & \M{Class \; names} & \coloneqq & \ObjZ \ALT \RgnZ \ALT A \ALT B\\
     \fgjN & \in & \M{FGJ \; class \; types} & \coloneqq & cn\inang{\tbar}\\
T  & \in & \M{FGJ \; types} & \coloneqq & \tyvar \ALT  \fgjN \ALT \unitZ
     \ALT \bar{T} \rightarrow T \\
\fbN  & \in & \M{Region-annotated \; class \; types} & \coloneqq & 
     cn\inang{\tbar}\inang{\rbar} \\
\tau &\in& \M{types} & \coloneqq & T@\rgn  
      \ALT \fbN \ALT \unitZ 
      \ALT \inang{\rhobar \,|\, \phi}\bar{\tau}
      \xrightarrow{\rgn} \tau \\
C  & \in & \M{Class \; definitions} & \coloneqq & 
     \C{class} \; cn\inang{\bar{\tyvar} \extends \bar{\fgjN}} 
                    \inang{\rhobar \,|\, \phi}\extends \fbN 
                    \{\bar{\tau} \; \bar{f};\; \bar{d}\}\;\\
%k  & \in & \M{Constructors} & \coloneqq & 
%     cn(\bar{\tau} \; \bar{x})\{\C{super}(\bar{x}); \;
%                                \C{this}.\bar{f}\,=\,\bar{x};\}\\
d  & \in & \M{Methods} & \coloneqq & 
     \tau \; m\inang{\rhobar \,|\, \phi} (\taubar \; \xbar)
     \{\C{return}\;e;\}\\
\phi,\phicx &\in& \M{Region\;constraints} & \coloneqq & true 
      \ALT \rho \outlives \rho \ALT \rho = \rho \ALT \phi \conj \phi\\
e  & \in & \M{Expressions} & \coloneqq & \unitval \ALT x \ALT e.f 
     \ALT e.m\inang{\rbar}(\ebar) \ALT \C{new}\;\fbN(\ebar)
     \ALT \lambdaexp{\rgn}{\rhobar \,|\, \phi}
                    {\xbar:\taubar} {e}
           \ALT e\inang{\rbar}(\bar{e})\\
   & & & & \ALT \letexp{x}{e}{e} \ALT \letregion{\rho}{e} 
           \ALT \open{x}{\rgn}{y}{e} \\
\end{array}
\end{smathpar}

\caption{\fbname: Syntax}
\label{fig:fb-syntax}
\end{figure*}

\begin{figure*}[t]
%
\begin{minipage}{2.25in}
\begin{smathpar}
\begin{array}{lcl}
  allocRgn(A\inang{\ralloc\rbar}\inang{\tbar}) & = & \ralloc\\
  allocRgn(\inang{\rhoalloc\rhobar \,|\, \phi}\bar{\tau^1}
      \xrightarrow{\ralloc} \tau^2) & = & \ralloc\\
  shape(A\inang{\rhoalloc\rhobar}\inang{\tbar}) & = & A\inang{\tbar}\\
  bound_{\aenv}(\tyvar@\ralloc) & = & \aenv(\tyvar)@\ralloc\\
  bound_{\aenv}(\fbN) & = & \fbN\\
\end{array}
\end{smathpar}
\end{minipage}
%
\begin{minipage}{1.8in}
\begin{smathpar}
\begin{array}{c}
\renewcommand*{\arraystretch}{1.2}
\RULE
  {
    \\
    B \in \{\ObjZ,\RgnZ\}
  }
  {
    fields(B\inang{\ralloc\rbar}\inang{\tbar}) \;=\; \bullet
  }
\end{array}
\end{smathpar}
\end{minipage}
%
\begin{minipage}{3in}
\begin{smathpar}
\begin{array}{c}
\renewcommand*{\arraystretch}{1.2}
\RULE
  {
    CT(B) = \headerOf{B}\{\bar{\tau^f}\;\bar{f};\,...\}\\
    \substFn = [\rbar/\rhobar, \ralloc/\rhoalloc, \tbar/\bar{\tyvar}] \qquad 
    fields(\substFn(\fbN)) = \bar{g}:\bar{\tau^g}
  }
  {
    fields(B\inang{\ralloc\rbar}\inang{\tbar}) \;=\;
      \bar{g}:\bar{\tau^g},\,\bar{f}:\substFn(\bar{\tau^f})
  }
\end{array}
\end{smathpar}
\end{minipage}
%
\bigskip

\begin{minipage}{3.5in}
\begin{smathpar}
\begin{array}{lcl}
  ctype(\ObjZ\inang{\rgn}) & = & \bullet \\
% ctype(\RgnZ\inang{\rgn}\inang{T}) & = & \inang{\rhoalloc}
%   {\unitZ}\rightarrow{T@\rhoalloc}\\
  ctype(B\inang{\ralloc\rbar}\inang{\tbar}) & = & 
    fields(B\inang{\ralloc\rbar}\inang{\tbar})\\
  mtype(\C{transfer}, \RgnZ\inang{\rgn}\inang{T}) & = & 
    \inang{\rhoalloc} {\unitZ}\rightarrow{\unitZ}\\
  mtype(\C{free}, \RgnZ\inang{\rgn}\inang{T}) & = & 
    \inang{\rhoalloc} {\unitZ}\rightarrow{\unitZ}\\
\end{array}
\end{smathpar}
\end{minipage}
%
\begin{minipage}{3in}
\begin{smathpar}
\begin{array}{c}
\renewcommand*{\arraystretch}{1.2}
\RULE
  {
    CT(B) = \headerOf{B}\{\bar{\tau^f}\;\bar{f};\,k\;\bar{d}\}\\
    m \notin \bar{d} \qquad 
    \substFn = [\rbar/\rhobar, \ralloc/\rhoalloc, \tbar/\bar{\tyvar}]
  }
  {
    mtype (m,B\inang{\ralloc\rbar}\inang{\tbar}) \;=\;
    mtype (m, \substFn(\fbN))
  }
\end{array}
\end{smathpar}
\end{minipage}
%
\bigskip

\begin{minipage}{3.25in}
\begin{smathpar}
\begin{array}{c}
\renewcommand*{\arraystretch}{1.2}
\RULE
  {
    CT(B) = \headerOf{B}\{\bar{\tau^f}\;\bar{f};\,k\;\bar{d}\}\\
    \tau^2 \; m\mang (\bar{\tau^1}\;\bar{x})\{...\} \in \bar{d} \qquad
    \substFn = [\rbar/\rhobar, \ralloc/\rhoalloc, \tbar/\bar{\tyvar}]
  }
  {
    mtype (m,B\inang{\ralloc\rbar}\inang{\tbar}) \;=\;
    \substFn(\mang\bar{\tau^1} \rightarrow \tau^2)
  }
\end{array}
\end{smathpar}
\end{minipage}
%
\begin{minipage}{3.5in}
\begin{smathpar}
\begin{array}{c}
\renewcommand*{\arraystretch}{1.2}
\RULE
  {

    \substFn = \subst{\bar{\rho_2}}{\bar{\rho_1}}
               \subst{\rhoalloc_2}{\rhoalloc_1} \spc
    mtype(m,\fbN) = \inang{\rhoalloc_1\bar{\rho_1},|\, \phi_1}\bar{\tau^{11}} 
                      \rightarrow \tau^{12} \spc \texttt{implies}\\
    \isvalid{\A.\phicx}{\phi_2 \Leftrightarrow \substFn(\phi_1)} 
        \spc \texttt{and} \spc
    \bar{\tau^{21}} = \substFn(\bar{\tau^{11}}) \spc \texttt{and} \spc
    \subtyp{\A}{\tau^{22}} {\substFn(\tau^{12})}
    %\substFn = [\rbar/\rhobar, \ralloc/\rhoalloc, \tbar/\bar{\tyvar}]
  }
  {
    override(\A,\fbN,\inang{\rhoalloc_2\bar{\rho_2},|\, \phi_1}
              \bar{\tau^{21}} \rightarrow \tau^{22})
  }
\end{array}
\end{smathpar}
\end{minipage}
%
\bigskip

\begin{minipage}{5in}
\begin{smathpar}
\begin{array}{c}
  \rhoset,\rhoenv \in 2^{\rho} \qquad
  \aenv \in \tyvar \rightarrow \fgjN \qquad
  \A = (\subtypcx)\\
\end{array}
\end{smathpar}
\end{minipage}
%

\caption{\fbname: Auxiliary Definitions}
\label{fig:fb-auxdef}
\end{figure*}

\begin{figure*}[t]
%
\textbf{Auxiliary Definitions} \\
\begin{minipage}{1.75in}
\begin{smathpar}
\begin{array}{lcl}
  allocRgn(A\inang{\rhoalloc\rhobar}\inang{\tbar}) & = & \rhoalloc\\
  bound_{\Delta}(\alpha) & = & \Delta(\alpha)\\
  bound_{\Delta}(N) & = & N\\
\end{array}
\end{smathpar}
\end{minipage}
%
\begin{minipage}{1.8in}
\begin{smathpar}
\begin{array}{c}
\renewcommand*{\arraystretch}{1.2}
\RULE
  {
    \\
    B \in \{\ObjZ,\RgnZ\}
  }
  {
    fields(B\inang{\ralloc\rbar}\inang{\tbar}) \;=\; \bullet
  }
\end{array}
\end{smathpar}
\end{minipage}
%
\begin{minipage}{3in}
\begin{smathpar}
\begin{array}{c}
\renewcommand*{\arraystretch}{1.2}
\RULE
  {
    CT(B) = \headerOf{B}\{\bar{\tau^f}\;\bar{f};\,...\}\\
    \substFn = [\rbar/\rhobar, \ralloc/\rhoalloc, \tbar/\bar{\alpha}] \qquad 
    fields(\substFn(N)) = \bar{\tau^g}\;\bar{g}
  }
  {
    fields(B\inang{\ralloc\rbar}\inang{\tbar}) \;=\;
      \bar{\tau^g}\,\bar{g},\,\substFn(\bar{\tau^f})\,\bar{f}
  }
\end{array}
\end{smathpar}
\end{minipage}
%
\bigskip

\begin{minipage}{3.75in}
\begin{smathpar}
\begin{array}{c}
\renewcommand*{\arraystretch}{1.2}
\RULE
  {
    CT(B) = \headerOf{B}\{\bar{\tau^f}\;\bar{f};\,k\;\bar{d}\}\\
    \tau^2 \; m\mang (\bar{\tau^1}\;\bar{x})\{...\} \in \bar{d} \qquad
    \substFn = [\rbar/\rhobar, \ralloc/\rhoalloc, \tbar/\bar{\alpha}]
  }
  {
    mtype (m,B\inang{\ralloc\rbar}\inang{\tbar}) \;=\;
    \substFn(\mang\bar{\tau^1} \rightarrow \tau^2)
  }
\end{array}
\end{smathpar}
\end{minipage}
%
\begin{minipage}{3in}
\begin{smathpar}
\begin{array}{c}
\renewcommand*{\arraystretch}{1.2}
\RULE
  {
    CT(B) = \headerOf{B}\{\bar{\tau^f}\;\bar{f};\,k\;\bar{d}\}\\
    m \notin \bar{d} \qquad 
    \substFn = [\rbar/\rhobar, \ralloc/\rhoalloc, \tbar/\bar{\alpha}]
  }
  {
    mtype (m,B\inang{\ralloc\rbar}\inang{\tbar}) \;=\;
    mtype (m, \substFn(N))
  }
\end{array}
\end{smathpar}
\end{minipage}
%
\bigskip
\begin{smathpar}
\A \;=\; (\subtypcx)
\end{smathpar}
%
\bigskip

\textbf{Subtyping}  \; \fbox
  {\(\subtyp{\A}{\tau_1}{\tau_2}\)}\\

%
\begin{minipage}{1in}
\begin{smathpar}
\begin{array}{c}
\renewcommand*{\arraystretch}{1.2}
\RULE
  {
    \\
  }
  {
    \subtyp{\A}{\tau}{\tau}
  }
\end{array}
\end{smathpar}
\end{minipage}
%
\begin{minipage}{1.2in}
\begin{smathpar}
\begin{array}{c}
\renewcommand*{\arraystretch}{1.2}
\RULE
  {
    \\
  }
  {
    \subtyp{\A}{\tyvar @\rho}{\aenv(\tyvar) @\rho}
  }
\end{array}
\end{smathpar}
\end{minipage}
%
\begin{minipage}{2in}
\begin{smathpar}
\begin{array}{c}
\renewcommand*{\arraystretch}{1.2}
\RULE
  {
    \subtyp{\A}{\tau_1}{\tau_2}\qquad
    \subtyp{\A}{\tau_2}{\tau_3}
  }
  {
    \subtyp{\A}{\tau_1}{\tau_3}
  }
\end{array}
\end{smathpar}
\end{minipage}
%
\begin{minipage}{1.5in}
\begin{smathpar}
\begin{array}{c}
\renewcommand*{\arraystretch}{1.2}
\RULE
  {
    \\
  }
  {
    \subtyp{\A}{\exists\rho.\tau}{\exists\rho'.[\rho'/\rho]\tau}
  }
\end{array}
\end{smathpar}
\end{minipage}
%
\bigskip
\begin{minipage}{1.5in}
\begin{smathpar}
\begin{array}{c}
\renewcommand*{\arraystretch}{1.2}
\RULE
  { }
  { }
\end{array}
\end{smathpar}
\end{minipage}
%
\begin{minipage}{3.5in}
\begin{smathpar}
\begin{array}{c}
\renewcommand*{\arraystretch}{1.2}
\RULE
  {
    CT(B) = \headerOf{B}\{\bar{\tau^f}\;\bar{f};\,k\;\bar{d}\}\\
    \tywf{\A}{B\inang{\pi^a\bar{\pi}\inang{\taubar}}}\qquad
    \substFn = [\rbar/\rhobar, \ralloc/\rhoalloc, \tbar/\bar{\tyvar}] \qquad 
    \tywf{\A}{\substFn(\fbN)}
    \
  }
  {
    \subtyp{\A}{B\inang{\pi^a\bar{\pi}\inang{\taubar}}}{\substFn(\fbN)}
  }
\end{array}
\end{smathpar}
\end{minipage}
%

%
\bigskip

\textbf{Well-formedness}  \; \fbox
  {\(\tywf{\A}{\tau}, \spc 
     \tywf{\A}{\phi}\)}\\

%
\begin{minipage}{1.25in}
\begin{smathpar}
\begin{array}{c}
\renewcommand*{\arraystretch}{1.2}
\RULE
  {
    \\
    \\
    \rgn \in \A.\rhoenv
  }
  {
    \tywf{\A}{\ObjZ\inang{\rgn}}
  }
\end{array}
\end{smathpar}
\end{minipage}
% %
% \begin{minipage}{1.5in}
% \begin{smathpar}
% \begin{array}{c}
% \renewcommand*{\arraystretch}{1.2}
% \RULE
%   {
%     \rgn \in \A.\rhoenv\\
%     \fgjtywf{\A.\aenv}{\RgnZ\inang{T}}
%   }
%   {
%     \tywf{\A}{\RgnZ\inang{\rgn}\inang{T}}
%   }
% \end{array}
% \end{smathpar}
% \end{minipage}
% %
% \begin{minipage}{1.5in}
% \begin{smathpar}
% \begin{array}{c}
% \renewcommand*{\arraystretch}{1.2}
% \RULE
%   {
%     \fgjtywf{\A.\aenv}{\RgnZ\inang{T}}
%   }
%   {
%     \tywf{\A}{\exists\rho.\RgnZ\inang{\rho}\inang{T}}
%   }
% \end{array}
% \end{smathpar}
% \end{minipage}
%
\begin{minipage}{2.75in}
\begin{smathpar}
\begin{array}{c}
\renewcommand*{\arraystretch}{1.2}
\RULE
  {
    \rgn \in \A.\rhoenv \spc
    \{\rhoalloc,\rhobar\} \notin \A.\rhoenv \\
    \A' = (\A.\rhoset, \A.\rhoenv \cup \{\rhoalloc,\rhobar\}, 
           \A.\aenv, \A.\phicx \conj \phi) \\
    \tywf{\A'.\rhoenv}{\phi}\spc 
    \tywf{\A'}{\bar{\tau^1}} \spc
    \tywf{\A'}{\tau^2}
  }
  {
    \tywf{\A}{\inang{\rhoalloc\rhobar \,|\, \phi}
              \bar{\tau^1} \xrightarrow{\rgn} \tau^2}
  }
\end{array}
\end{smathpar}
\end{minipage}
%
\begin{minipage}{1.5in}
\begin{smathpar}
\begin{array}{c}
\renewcommand*{\arraystretch}{1.2}
\RULE
  { 
    \\
    \\
    \fgjtywf{\A.\aenv}{T}
  }
  {
    \tywf{\A}{\RgnZ\inang{T}\inang{\rgn}}
  }
\end{array}
\end{smathpar}
\end{minipage}
%
\begin{minipage}{1in}
\begin{smathpar}
\begin{array}{c}
\renewcommand*{\arraystretch}{1.2}
\RULE
  {
    \\
    \\
    \rho_0,\rho_1 \in \rhoenv
  }
  {
    \tywf{\rhoenv}{\rho_0 \outlives \rho_1}
  }
\end{array}
\end{smathpar}
\end{minipage}
%

%
\begin{minipage}{3.5in}
\begin{smathpar}
\begin{array}{c}
\renewcommand*{\arraystretch}{1.2}
\RULE
  {
    CT(B) = \headerOf{B}\{...\}\spc
    \ralloc,\rbar \in \A.\rhoenv \\
    \fgjtywf{\A.\aenv}{B\inang{\tbar}}\spc
    \substFn = [\rbar/\rhobar, \ralloc/\rhoalloc, \tbar/\bar{\tyvar}] \spc
    \isvalid{\A.\phicx}{\substFn(\phi)}
  }
  {
    \tywf{\A}{B\inang{\ralloc\rbar}\inang{\tbar}}
  }
\end{array}
\end{smathpar}
\end{minipage}
%
\begin{minipage}{1.65in}
\begin{smathpar}
\begin{array}{c}
\renewcommand*{\arraystretch}{1.2}
\RULE
  {
    \fgjtywf{\A.\aenv}{T}\spc
    \ralloc \in \A.\rhoenv\\
    \fgjsubtyp{\A.\aenv}{T}{\ObjZ}\spc
  }
  {
    \tywf{\A}{T@\ralloc}
  }
\end{array}
\end{smathpar}
\end{minipage}
%
\begin{minipage}{0.75in}
\begin{smathpar}
\begin{array}{c}
\renewcommand*{\arraystretch}{1.2}
\RULE
  {
    \tywf{\rhoenv}{\phi_0} \\ \tywf{\rhoenv}{\phi_1}
  }
  {
    \tywf{\rhoenv}{\phi_0 \wedge \phi_1}
  }
\end{array}
\end{smathpar}
\end{minipage}

%
\bigskip
\caption{Static semantics of {\sc Featherweight} \name}
\label{fig:fb-syntax}
\end{figure*}

/Users/gowtham/git/broom/fullversion/broom/paper/fb-morewfrules.tex

We build on the Featherweight Generic Java (FGJ)~\cite{fgj} formalism
to formalize \name and its region type system.
Fig~\ref{fig:fb-syntax} describes the syntax of the formal language,
which we call \fbname (\FB). The language is seeded with \ObjZ and
\RgnZ classes. More classes can be defined using the \C{class}
keyword. Class types in \FB are region-annotated variants of class
types in FGJ (also called \emph{core types}). This correspondence is
reflected in the $T@\rgn$ syntax of the region type of an object,
whose core type is $T$, and which is contained in the region $\rgn$.
We let $\rgn$ range over static identifiers of regions in \FB. Note
that all \FB objects are boxed values, hence their \FB types are
always region-annotated. The only unboxed value in \FB is $\unitval$
of type \unitZ.  

\FB class definitions can be understood as extensions of FGJ class
definitions with region annotations and region parameters. More
precisely, a \fbname definition of a class is obtained from its
Featherweight Generic Java definition by (a). lifting all core types
in the body of the class (i.e, fields, constructors and methods) to
their region-annotated versions, and (b). adding region parameters to
its header. Note that region parameterization is independent of its
parameterization over core types; they are not conflated to make the
class definition parameteric over region types. We let $a, b$ range
over (core) type parameters and $\rho$ range over region parameters of
a class. The following definition of \C{Pair} class is an example of
type and region parametric definition in \FB (The symbol $\extends$
should be read \emph{extends}):
\begin{codejava}[mathescape=true]
class Pair<$\rho^a, \rho_1, \rho_2 \,|\, \rho_1 \outlives \rho^a 
                      \conj \rho_2 \outlives \rho^a$>
          <a $\extends$ Object, b $\extends$ Object> {
  a@$\rho_1$ fst; 
  b@$\rho_2$ snd;
  Pair(a@$\rho_1$ fst, b@$\rho_2$ snd) {
    super(); 
    this.fst = fst; 
    this.snd = snd;
  }
  a@$\rho_1$ getFst() {
    return this.fst;
  }
}
\end{codejava}
Every class definition in \FB is necessarily parameterized over the
allocation region of its objects. As a rule, allocation region
parameter (denoted $\rho^a$) is the first region parameter of a class.
Besides the allocation region of its objects, \C{Pair} class is also
parametric over the regions its first and second elements are
allocated in. References between objects allocated in different
regions are only allowed if the referred object is guaranteed to
outlive the referring object. In case of \C{Pair} class, this means
that allocation regions ($\rho_1$ and $\rho_2$) of both objects that
make up the pair must outlive the allocation region ($\rho^a$) of the
\C{Pair} object. Such conditions over region parameters of a class
need to be recorded in its header as region constraints ($\phi$) in
order for the class to be judged well-formed by the type system
(Fig.~\ref{fig:fb-morewfrules}). 

To construct objects of the \C{Pair} class, its type and region
parameters need to be instantiated with core types ($T$) and concrete
region identifiers ($\rgn$), respectively. Region instantiation has to
satisfy the stated constraints over region parameters. For example,
the following code snippet appropriately instantiates region
parameters of the \C{Pair} class to construct a \C{Pair} of
{\ObjZ}{\!}s, each allocated in different regions:
\begin{codejava}
letregion $\rgn_0$ {
  Object<$\rgn_0$> fst = new Object<$\rgn_0$>();
  letregion $\rgn_1$ {
    Object<$\rgn_1$> snd = new Object<$\rgn_1$>();
    Pair<Object,Object><$\rgn_1$,$\rgn_0$,$\rgn_1$> p = 
       new Pair<Object,Object><$\rgn_1$,$\rgn_0$,$\rgn_1$>(fst,snd);
  }
}
\end{codejava}
Since $\rgn_0 \outlives \rgn_1 \conj \rgn_1 \outlives \rgn_1$, the
region constraints of the \C{Pair} class are satisfied. Observe that
the region type of pair object \C{p} conveys the fact that (a). it is
allocated in region $\rgn_1$, and (b). it holds references to
object(s) allocated in region $\rgn_0$. On the other hand, if we
choose to allocate the \C{fst} object also in $\rgn_1$, the region
type of \C{p} would be
\C{Pair<\ObjZ,\ObjZ><$\rgn_1$,$\rgn_1$,$\rgn_1$>}, which can be
written simply as \C{Pair<\ObjZ,\ObjZ>@$\rgn_1$}. In general, the
\C{@} notation in a region type of an object \C{x} highlights that
\C{x}, and all the objects reachable from \C{x} via references are
allocated in a single region. We say that \C{x} is \emph{contained} in
the region. The objects of $\ObjZ$ class, by the virtue of
containing no references to other objects, are always contained in
their allocation region. Hence the type $\ObjZ@\rgn$ is equivalent to
the type $\ObjZ\inang{\rgn}$.
%%An object \emph{allocated} in a region $\rgn$ contains its spine in
%%$\rgn$, but can refer to objects allocated in other regions.  On the
%%other hand,

A method definitions ($d$) in \fbname can be region polymorphic with
respect to (a). its allocation context (Sec.~\ref{sec:alloc-ctxt}),
and (b). the regions occuring in the region types of its arguments.
Region parameters on the methods, like those on classes, are
accompanied by constraints ($\phi$) capturing the conditions that the
parameters need to satisfy for the method to be considered well-formed
(Fig.~\ref{fig:fb-morewfrules}). Allocation context ($\rho^a$ or
$\rho^a_m$) is the first and inevitable region parameter of every
method in \FB. If a method is not polymorphic with respect to its
allocation context (for example, if its allocation context needs to be
same as the allocation region of its class), then the monomorphism
needs to be encoded as an explicit equality constraint in $\phi$.  

\fbname supports first-class functions via lambda expressions and
arrow types. A lambda expression defines a region-polymorphic
multi-argument function closure parameterized over function's
allocation context parameter. The arrow type of a function closure is
prenex-quantified over its constrained region arguments
($\inang{\rho^a\rhobar \,|\, \phi}$). Since closures can escape the
context in which they are created, it is important to keep track of
the regions where closures are allocated in order to prevent unsafe
dereferences. The $\rgn^a$ annotation above the arrow in the arrow
type denotes the allocation region of the corresponding closure. Note
that it is important to distinguish between the allocation context
argument of a function and the allocation region of its closure. For
instance, in the following example:
\begin{codejava}
letregion $\rgn$ {
  let f = $\lambda{\inang{\rho^a}}$()$\,\Rightarrow\,$new Object$\inang{\rho^a}$() 
  in f
}
\end{codejava}
The type of \C{f} is $\inang{\rho^a}\unitZ \xrightarrow{\rgn}
\ObjZ\inang{\rho^a}$, coveying that (a). \C{f}'s closure is allocated
in $\rgn$, and (b). when executed under an allocation context
$\rho^a$, the closure returns an object allocated in $\rho^a$.

