\section{The Core Language}
\label{sec:type-system}

\begin{figure*}[t!]
%
%\begin{minipage}
\begin{smathpar}
\renewcommand{\arraystretch}{1.2}
\begin{array}{lr} 
\multicolumn{2}{c}{
  {\rgn} \in \mathtt{Static \; region \; ids} \qquad
  {\rho} \in \mathtt{Region \; variables} \qquad
  {\tyvar, \tyvarb} \in \mathtt{Type \; variables} \qquad
  {m} \in \mathtt{Method \; names} \qquad
  {x,y,f} \in \mathtt{Variables \; and \; fields} }\\
\begin{array}{lclcl}
  cn & \in & \M{Class \; names} & \coloneqq & \ObjZ \ALT \RgnZ \ALT A \ALT B\\
  T  & \in & \M{FGJ \; types} & \coloneqq & \tyvar \ALT  \fgjN \ALT \unitZ
       \ALT \bar{T} \rightarrow T \\
  D  & \in & \M{Classes} & \coloneqq & 
       \C{class} \; cn\inang{\bar{\tyvar} \extends \bar{\fgjN}} 
                      \inang{\rhoalloc \rhobar \,|\, \phi}\extends \fbN 
                      \{\bar{\tau} \; \bar{f};\; k\; \bar{d}\}\;\\
  k  & \in & \M{Constructors} & \coloneqq & 
       cn(\bar{\tau} \; \bar{x})\{\C{super}(\bar{x}); \;
                                  \C{this}.\bar{f}\,=\,\bar{x};\}\\
  d  & \in & \M{Methods} & \coloneqq & 
       \tau \; m\inang{\rhoalloc \rhobar \,|\, \phi} (\taubar \; \xbar)
       \{\C{return}\;e;\}\\
\end{array}
\begin{array}{lclcl}
  \fgjN & \in & \M{FGJ \; class \; types} & \coloneqq & cn\inang{\tbar}\\
  \fbN  & \in & \M{FB \; class \; types} & \coloneqq & 
       cn\inang{\tbar}\inang{\ralloc \rbar} \\
  \tau &\in& \M{types} & \coloneqq & T@\rgn  
        \ALT \fbN \ALT \unitZ \\
       &   & & & \ALT \inang{\rhoalloc\rhobar \,|\, \phi}\bar{\tau}
        \xrightarrow{\rgn} \tau \\
  \phi,\Phi &\in& \M{Constraints} & \coloneqq & true 
        \ALT \rho \outlives \rho \\
        & & & & \ALT \rho = \rho \ALT \phi \conj \phi\\
  % v  & \in & \M{Value\;expressions} & \coloneqq & x \ALT \C{new} \; \fbN(\vbar)\\
\end{array}\\
\begin{array}{lclcl}
e  & \in & \M{Expressions} & \coloneqq & \unitval \ALT x \ALT e.f 
     \ALT e.m\inang{\ralloc \rbar}(\ebar) \ALT \C{new}\;\fbN(\ebar) 
     \ALT \lambdaexp{\ralloc}{\rhoalloc\rhobar \,|\, \phi}{\xbar:\taubar} {e}
     \ALT e\inang{\ralloc\rbar}(\bar{e})\\
%    \ALT (\tau)\, e\\
%    \ALT e\,;\,e  \\
   & & & & \ALT \letexp{x}{e}{e}
           \ALT \letregion{\rho}{e} 
           \ALT \open{e}{\rgn}{y}{e} \\
%   & & & & \ALT \unpackexp{\rgn}{x}{e}{e} \ALT e\,;\,e\\
% s  & \in & \M{Statements} & \coloneqq & \tau\;x\;=\;e \ALT x\;=\;e
%      \ALT e.f\;=\;e 
%      \ALT \C{open}\;x\;\C{withroot}\;y\;\{s\} \\
%     & & & & \ALT \C{openalloc}\;x\;\C{withroot}\;y\;\{s\} 
%      \ALT (\rho,\tau\;x)\;=\;\C{unpack}\;e \ALT s\,;\,s\\
\end{array}
\end{array}
\end{smathpar}

\caption{\fbname: Syntax}
\label{fig:fb-syntax}
\end{figure*}


\subsection{Syntax}
\label{sec:fb-syntax}

We build on the Featherweight Generic Java (FGJ)~\cite{fgj} formalism
to formalize \name and its region type system. Our development reuses
notations from~\cite{fgj}, and relies on several of its definitions,
such as the definition of type well-formedness for the core
(region-free) language. We refrain from reproducing them here, and
instead encourage the reader to refer to~\cite{fgj,techrep}.

Fig~\ref{fig:fb-syntax} describes the syntax of our formal language,
which we call \fbname (\FB). Class types in \FB are region-annotated
variants of class types in FGJ (also called \emph{core types}). This
correspondence is reflected in the $T@\rgn$ syntax of a region type,
which is the simplest form of a region type describing an object of
core type $T$ contained in a region $\rgn$. We let $\rgn$ range over
static identifiers of regions in \FB. The only unboxed value in \FB is
$\unitval$ of type \unitZ. Rest are objects.

\FB extends FGJ's expression language with assignments, local variable
declarations (via \C{let} expressions), region constructs
(\C{letregion} and \C{open}), lambda abstraction
($\lambdaexp{...}{\xbar : \taubar}{e}$) and application
($e\inang{...}(\bar{e})$) expressions. Assignments and region
constructs, which are statements in \name, are treated as effectful
expressions in \FB. Lambda abstraction and application expressions are
to define and apply anonymous functions. Modulo the angle braces
(whose presence will be justified later), these expressions are
uncurried variants of the corresponding expressions from System F.
Since FGJ formalism does not include higher-order functions, we extend
the syntactic class (\C{T}) of FGJ types with an (uncurried) arrow
type.

Class definitions in \FB extend FGJ's with region parameters
($\rho$). Informally, region variables ($\rho$) are to region
identifiers ($\rgn$) as type variables ($a$) are to core types ($T$);
they can be instantiated with region identifiers, as needed. A class
in \FB is necessarily parameterized over the allocation region of its
objects. We use $\rho^a$ to denote this parameter, and often refer to
it as \emph{class}'s allocation region, although it is really the
allocation region of its objects. An Object in \FB can contain fields
referring to objects allocated in regions ($\rhobar$) other than its
own allocation region ($\rhoalloc$), provided that the former outlive
the later (i.e., $\rhobar \outlives \rho$). In such cases, the
definition of object's class needs to be parametric over allocation
regions of its fields (i.e., their classes). Furthermore, the
constraint that such regions must outlive the allocation region of the
class needs to be made explicit in the definition. For instance, a
\C{Pair} object allocated in a region ${\rhoalloc}$ can have its first
component coming from ${\rho_1}$ and second from $\rho_2$, where all
the $\rho$'s are distinct, and $\rho_1, \rho_2$ outlive
$\rhoalloc$. The corresponding definition of the \C{Pair} class is as
follows(The symbol $\extends$ should be read \emph{extends}):  
\begin{codejava}[mathescape=true]
class Pair<a $\extends$ Object, b $\extends$ Object>
          <$\rho^a, \rho_1, \rho_2 \,|\, \rho_1 \outlives \rho^a \conj \rho_2 \outlives \rho^a$> $\extends$ Object<$\rho^a$>
           {
  a@$\rho_1$ fst; 
  b@$\rho_2$ snd;
  Pair(a@$\rho_1$ fst, b@$\rho_2$ snd) {
    super(); 
    this.fst = fst; 
    this.snd = snd;
  }
  a@$\rho_1$ getFst() {
    return this.fst;
  }
}
\end{codejava}
To construct objects of the \C{Pair} class, its type and region
parameters need to be instantiated with core types ($T$) and concrete
region identifiers ($\rgn$), respectively. For example:
\begin{codejava}
letregion $\rgn_0$ {
  Object<$\rgn_0$> snd = new Object<$\rgn_0$>();
  letregion $\rgn_1$ {
    Object<$\rgn_1$> fst = new Object<$\rgn_1$>();
    let p = new Pair<Object,Object><$\rgn_1$,$\rgn_0$,$\rgn_1$>
                  (fst,snd);
  }
}
\end{codejava}
In the above code, the instantiation of $\rhoalloc$ and $\rho_1$ with
$\rgn_0$, and $\rho_2$ with $\rgn_1$ is allowed because (a) $\rgn_0$
and $\rgn_1$ are live during the instantation, and (b). $\rgn_0
\outlives \rgn_1$ and $\rgn_1 \outlives \rgn_1$ (since outlives is
reflexive). Observe that the region type of \C{p} conveys
the fact that (a). it is allocated in region $\rgn_1$, and (b). it
holds references to objects allocated in region $\rgn_0$ and $\rgn_1$.
In contrast, if we choose to allocate the \C{snd} object also in
$\rgn_1$, the region type of \C{p} would be
\C{Pair<\ObjZ,\ObjZ><$\rgn_1$,$\rgn_1$,$\rgn_1$>}, which we abbreviate
as \C{Pair<\ObjZ,\ObjZ>@$\rgn_1$}\footnote{In general, we treat
$B\inang{\tbar}@\rgn$ as an abbreviation of
$B\inang{\tbar}\inang{\rbar}$}. The $\ObjZ$ class in \FB has no
fields, hence its region type is of form $\ObjZ@\rgn$.
% In general, the \C{@} notation in a
% region type of an object \C{x} highlights that
% \C{x}, and all the objects reachable from \C{x} via references are
% allocated in a single region. We say that \C{x} is \emph{contained} in
% the region. The $\ObjZ$ class in \FB contains no references to other
% objects, hence its objects are always contained in their allocation
% region. Their region type is $\ObjZ\inang{\rgn}$ (or equivalently,
% $\ObjZ@\rgn$), for some region $\rgn$.

% Note that when a generic class definition in FGJ is lifted to a
% region-polymorphic definition in \FB, it nonetheless remains generic
% with respect to core types. We say that the class is now a
% region-polymorphic generic class. Region parameterization of a class
% in \FB is independent of its parameterization over core types; they
% are not conflated to make the class parametric over region types. For
% instance, following is the region-polymorphic definition of a generic
% \C{Pair} class (The symbol $\extends$ should be read \emph{extends}):

% When objects of a class are allocated in a region $\rgn$, it means
% that the class's constructor is run with $\rgn$ as its allocation
% context. Every class definition in \FB is necessarily polymorphic with
% respect to the allocation region of its objects, i.e., the allocation
% context of its constructor. We adopt a convention that requires the
% allocation region parameter (denoted $\rho^a$) to be the first region
% parameter of a class.  Besides the allocation region of its objects,
% \C{Pair} class is also parametric over the regions its first and
% second elements are allocated in. References between objects allocated
% in different regions are only allowed if the referred object is
% guaranteed to outlive the referring object. In case of \C{Pair} class,
% this means that allocation regions ($\rho_1$ and $\rho_2$) of both
% objects that make up the pair must outlive the allocation region
% ($\rho^a$) of the \C{Pair} object. Such conditions over region
% parameters of a class need to be recorded in its header as region
% constraints ($\phi$) in order for the class to be judged well-formed
% by the type system (Fig.~\ref{fig:fb-morewfrules}). 

%%An object \emph{allocated} in a region $\rgn$ contains its spine in
%%$\rgn$, but can refer to objects allocated in other regions.  On the
%%other hand,

% \FB's $\RgnZ$ objects, like $\ObjZ$ objects, have region type of form
% $\RgnZ\inang{\rgn}$. However, unlike the $\rgn$ in
% $\ObjZ\inang{\rgn}$, $\rgn$ in $\RgnZ\inang{\rgn}$ cannot be any
% region. Recall that $\RgnZ$ objects have special semantics in \name -
% they act as handlers to transferable regions.  Constructing a new
% $\RgnZ$ object entails the creation of a new transferable region, and
% it is in this region that the new object is allocated in. It follows
% that $\rgn$ in $\RgnZ\inang{\rgn}$ should be the static identifier of
% the new transferable region. But, static identifiers for transferable
% regions are introduced only when such regions are opened for
% allocation via \C{open} expression. What, then, should be the region
% type of a \C{new Region} expression?

% \FB resolves this problem by existentially quantifying the allocation
% region of $\RgnZ$ objects when they are created. In other words, the
% type of \C{new Region} expression in \FB is
% $\exists\rho.\RgnZ\inang{\rho}$. Existential quantification in the
% type captures the fact that there now exists a transferable region
% containing the newly constructed $\RgnZ$ object. Elimination of
% existential quantification is facilitated by the \C{open} expression,
% which opens the transferable region and assigns it an identifier
% ($\rgn$).  Within the scope of \C{open}, the transferable region is
% identified with $\rgn$, allowing its handler to instantiate the
% existentially bound region variable with $\rgn$, and assume the type
% of $\RgnZ\inang{\rgn}$.

Like classes, methods can also exhibit qualified region polymorphism.
A method definition in \FB is necessarily polymorphic over its
allocation context (Sec.~\ref{sec:alloc-ctxt}), and optionally
polymorphic with respect to the regions containing its arguments.
Region parameters, like those on classes, are qualified with
constraints ($\phi$).
% Region parameters on the methods, like those on classes, are
% accompanied by constraints ($\phi$) capturing the conditions that the
% parameters need to satisfy for the method to be considered well-formed
% (Fig.~\ref{fig:fb-morewfrules}). Allocation context (usually $\rho^a$
% or $\rho^a_m$) is the first and inevitable region parameter of every
% method in \FB. 
Note that if a method is not intended to be polymorphic with respect
to its allocation context (for example, if its allocation context
needs to be same as the allocation region of its \emph{this}
argument), then the required monomorphism can be captured as an
equality constraint in $\phi$.  

% Like methods, functions can also be region-polymorphic. 
% A lambda expression defines a region-polymorphic
% multi-argument function closure parameterized over function's
% allocation context parameter. 
The angle braces in the lambda expression ($\inang{\rho^a\rhobar \,|\,
\phi}$) serve the same purpose as they do in a method definition - to
capture region parameters along with their constraints. Region
parameters also appear in the arrow type of the lambda expression 
($\inang{\rhoalloc\rhobar \,|\, \phi}\bar{\tau} \xrightarrow{\rgn}
\tau$) at the prenex position, similar to ML type schemes. However,
unlike ML, we don't distinguish between types and type schemes; any of
the $\tau$'s in the arrow type can themselves be region-parametric
arrow types. Like with methods, region parameters on functions can be
instantiated when they are applied.
% In this respect, our region type system is more like
% System F's type system, which admits higher-rank parametric
% polymorphism. Like System F, \FB provides a region instantiation
% expression ($e\inang{\ralloc\rbar}$), and a region generalization
% expression ($\Lambdaexp{\rhoalloc\rhobar \,|\, \phi}{e}$) to
% instantiate and generalize region variables. 

A lambda expression creates a closure, which can escape the
context in which it is created. It is therefore important to keep track of
the region in which a closure is allocated in order to avoid unsafe
dereferences. The $\rgn$ annotation above the arrow in the arrow
type denotes the allocation region of the corresponding closure. Note
that it is important to distinguish between the allocation context
argument ($\rhoalloc$) of a function and the allocation region
($\rgn$) of its closure. In \name, the later corresponds to the region where
a \C{Func} object is allocated, while the former corresponds to the
region where it is applied ($e\inang{\ralloc\rbar}(\bar{e})$). 
% For instance, in the following example:
% \begin{codejava}
% letregion $\rgn$ {
%   let f = $\lambda{\inang{\rho^a}}$().$\,$new Object$\inang{\rho^a}$() 
%   in f
% }
% \end{codejava}
% The type of \C{f} is $\inang{\rho^a}\unitZ \xrightarrow{\rgn}
% \ObjZ\inang{\rho^a}$, coveying that (a). \C{f}'s closure is allocated
% in $\rgn$, and (b). when executed under an allocation context
% $\rho^a$, the closure returns an object allocated in $\rho^a$.

\begin{figure*}[t!]
%
\textbf{Auxiliary Definitions}\\
\begin{figure*}[t]
%
\begin{minipage}{2.25in}
\begin{smathpar}
\begin{array}{lcl}
  allocRgn(A\inang{\ralloc\rbar}\inang{\tbar}) & = & \ralloc\\
  allocRgn(\inang{\rhoalloc\rhobar \,|\, \phi}\bar{\tau^1}
      \xrightarrow{\ralloc} \tau^2) & = & \ralloc\\
  shape(A\inang{\rhoalloc\rhobar}\inang{\tbar}) & = & A\inang{\tbar}\\
  bound_{\aenv}(\tyvar@\rgn) & = & \aenv(\tyvar)@\rgn\\
  bound_{\aenv}(\fbN) & = & \fbN\\
\end{array}
\end{smathpar}
\end{minipage}
%
\begin{minipage}{1.8in}
\begin{smathpar}
\begin{array}{c}
\renewcommand*{\arraystretch}{1.2}
\RULE
  {
    \\
    B \in \{\ObjZ,\RgnZ\}
  }
  {
    fields(B\inang{\ralloc\rbar}\inang{\tbar}) \;=\; \bullet
  }
\end{array}
\end{smathpar}
\end{minipage}
%
\begin{minipage}{3in}
\begin{smathpar}
\begin{array}{c}
\renewcommand*{\arraystretch}{1.2}
\RULE
  {
    CT(B) = \headerOf{B}\{\bar{\tau^f}\;\bar{f};\,...\}\\
    \substFn = [\rbar/\rhobar, \ralloc/\rhoalloc, \tbar/\bar{\tyvar}] \qquad 
    fields(\substFn(\fbN)) = \bar{g}:\bar{\tau^g}
  }
  {
    fields(B\inang{\ralloc\rbar}\inang{\tbar}) \;=\;
      \bar{g}:\bar{\tau^g},\,\bar{f}:\substFn(\bar{\tau^f})
  }
\end{array}
\end{smathpar}
\end{minipage}
%
\bigskip

\begin{minipage}{3.5in}
\begin{smathpar}
\begin{array}{lcl}
  ctype(\ObjZ\inang{\rgn}) & = & \bullet \\
% ctype(\RgnZ\inang{\rgn}\inang{T}) & = & \inang{\rhoalloc}
%   {\unitZ}\rightarrow{T@\rhoalloc}\\
  ctype(B\inang{\ralloc\rbar}\inang{\tbar}) & = & 
    fields(B\inang{\ralloc\rbar}\inang{\tbar})\\
  mtype(\C{transfer}, \exists\rho.\RgnZ\inang{\rho}\inang{T}) & = & 
    \inang{\rhoalloc} {\unitZ}\rightarrow{\unitZ}\\
  mtype(\C{free}, \exists\rho.\RgnZ\inang{\rho}\inang{T}) & = & 
    \inang{\rhoalloc} {\unitZ}\rightarrow{\unitZ}\\
\end{array}
\end{smathpar}
\end{minipage}
%
\begin{minipage}{3in}
\begin{smathpar}
\begin{array}{c}
\renewcommand*{\arraystretch}{1.2}
\RULE
  {
    CT(B) = \headerOf{B}\{\bar{\tau^f}\;\bar{f};\,k\;\bar{d}\}\\
    m \notin \bar{d} \qquad 
    \substFn = [\rbar/\rhobar, \ralloc/\rhoalloc, \tbar/\bar{\tyvar}]
  }
  {
    mtype (m,B\inang{\ralloc\rbar}\inang{\tbar}) \;=\;
    mtype (m, \substFn(\fbN))
  }
\end{array}
\end{smathpar}
\end{minipage}
%
\bigskip

\begin{minipage}{3.25in}
\begin{smathpar}
\begin{array}{c}
\renewcommand*{\arraystretch}{1.2}
\RULE
  {
    CT(B) = \headerOf{B}\{\bar{\tau^f}\;\bar{f};\,k\;\bar{d}\}\\
    \tau^2 \; m\mang (\bar{\tau^1}\;\bar{x})\{...\} \in \bar{d} \qquad
    \substFn = [\rbar/\rhobar, \ralloc/\rhoalloc, \tbar/\bar{\tyvar}]
  }
  {
    mtype (m,B\inang{\ralloc\rbar}\inang{\tbar}) \;=\;
    \substFn(\mang\bar{\tau^1} \rightarrow \tau^2)
  }
\end{array}
\end{smathpar}
\end{minipage}
%
\begin{minipage}{3.5in}
\begin{smathpar}
\begin{array}{c}
\renewcommand*{\arraystretch}{1.2}
\RULE
  {

    \substFn = \subst{\bar{\rho_2}}{\bar{\rho_1}}
               \subst{\rhoalloc_2}{\rhoalloc_1} \spc
    mtype(m,\fbN) = \inang{\rhoalloc_1\bar{\rho_1},|\, \phi_1}\bar{\tau^{11}} 
                      \rightarrow \tau^{12} \spc \texttt{implies}\\
    \isvalid{\A.\phicx}{\phi_2 \Leftrightarrow \substFn(\phi_1)} 
        \spc \texttt{and} \spc
    \bar{\tau^{21}} = \substFn(\bar{\tau^{11}}) \spc \texttt{and} \spc
    \subtyp{\A}{\tau^{22}} {\substFn(\tau^{12})}
    %\substFn = [\rbar/\rhobar, \ralloc/\rhoalloc, \tbar/\bar{\tyvar}]
  }
  {
    override(\A,\fbN,\inang{\rhoalloc_2\bar{\rho_2},|\, \phi_1}
              \bar{\tau^{21}} \rightarrow \tau^{22})
  }
\end{array}
\end{smathpar}
\end{minipage}
%
\bigskip

\begin{minipage}{5in}
\begin{smathpar}
\begin{array}{c}
  \rhoset,\rhoenv \in 2^{\rho} \qquad
  \aenv \in \tyvar \rightarrow \fgjN \qquad
  \A = (\subtypcx)\\
\end{array}
\end{smathpar}
\end{minipage}
%

\caption{\fbname: Auxiliary Definitions}
\label{fig:fb-auxdef}
\end{figure*}

%

\vspace*{0.1in}
%
\textbf{Type, and Type Constraint Well-formedness}  \; \fbox
  {\(\tywf{\A}{\tau}, \spc 
     \tywf{\rhoenv}{\phi}\)}\\
%
\begin{minipage}{1.25in}
\begin{smathpar}
\begin{array}{c}
\renewcommand*{\arraystretch}{1.2}
\RULE
  {
    \\
    \\
    \rgn \in \A.\rhoenv
  }
  {
    \tywf{\A}{\ObjZ\inang{\rgn}}
  }
\end{array}
\end{smathpar}
\end{minipage}
% %
% \begin{minipage}{1.5in}
% \begin{smathpar}
% \begin{array}{c}
% \renewcommand*{\arraystretch}{1.2}
% \RULE
%   {
%     \rgn \in \A.\rhoenv\\
%     \fgjtywf{\A.\aenv}{\RgnZ\inang{T}}
%   }
%   {
%     \tywf{\A}{\RgnZ\inang{\rgn}\inang{T}}
%   }
% \end{array}
% \end{smathpar}
% \end{minipage}
% %
% \begin{minipage}{1.5in}
% \begin{smathpar}
% \begin{array}{c}
% \renewcommand*{\arraystretch}{1.2}
% \RULE
%   {
%     \fgjtywf{\A.\aenv}{\RgnZ\inang{T}}
%   }
%   {
%     \tywf{\A}{\exists\rho.\RgnZ\inang{\rho}\inang{T}}
%   }
% \end{array}
% \end{smathpar}
% \end{minipage}
%
\begin{minipage}{2.75in}
\begin{smathpar}
\begin{array}{c}
\renewcommand*{\arraystretch}{1.2}
\RULE
  {
    \rgn \in \rhoenv \spc
    \rhobar \notin \A.\rhoenv \\
    \rhoenv' = \rhoenv \cup \{\rhobar\} \spc
    \A' = (\rhoenv', \aenv, \phicx \conj \phi) \\
    \tywf{\rhoenv'}{\phi}\spc 
    \tywf{\A'}{\bar{\tau^1}} \spc
    \tywf{\A'}{\tau^2}
  }
  {
    \tywf{(\rhoenv,\aenv,\phicx)}{\inang{\rhobar \,|\, \phi}
              \bar{\tau^1} \xrightarrow{\rgn} \tau^2}
  }
\end{array}
\end{smathpar}
\end{minipage}
%
\begin{minipage}{1.5in}
\begin{smathpar}
\begin{array}{c}
\renewcommand*{\arraystretch}{1.2}
\RULE
  { 
    \\
    \\
    \fgjtywf{\A.\aenv}{T}
  }
  {
    \tywf{\A}{\RgnZ\inang{T}\inang{\toprgn}}
  }
\end{array}
\end{smathpar}
\end{minipage}
%
\begin{minipage}{1in}
\begin{smathpar}
\begin{array}{c}
\renewcommand*{\arraystretch}{1.2}
\RULE
  {
    \\
    \\
    \rho_0,\rho_1 \in \rhoenv
  }
  {
    \tywf{\rhoenv}{\rho_0 \outlives \rho_1}
  }
\end{array}
\end{smathpar}
\end{minipage}
%

%
\begin{minipage}{3.5in}
\begin{smathpar}
\begin{array}{c}
\renewcommand*{\arraystretch}{1.2}
\RULE
  {
    CT(B) = \headerOf{B}\{...\}\\
    \rbar \in \rhoenv \spc
    \fgjtywf{\aenv}{B\inang{\tbar}}\spc
%   \substFn = [\rbar/\rhobar, \tbar/\bar{\tyvar}] \spc
    \isvalid{\phicx}{[\rbar/\rhobar](\phi)}
  }
  {
    \tywf{(\rhoenv,\aenv,\phicx)}{B\inang{\rbar}\inang{\tbar}}
  }
\end{array}
\end{smathpar}
\end{minipage}
%
\begin{minipage}{1.65in}
\begin{smathpar}
\begin{array}{c}
\renewcommand*{\arraystretch}{1.2}
\RULE
  {
    \fgjtywf{\A.\aenv}{T}\spc
    \rgn \in \A.\rhoenv\\
    \fgjsubtyp{\A.\aenv}{T}{\ObjZ}\spc
  }
  {
    \tywf{\A}{T@\rgn}
  }
\end{array}
\end{smathpar}
\end{minipage}
%
\begin{minipage}{0.75in}
\begin{smathpar}
\begin{array}{c}
\renewcommand*{\arraystretch}{1.2}
\RULE
  {
    \tywf{\rhoenv}{\phi_0} \\ \tywf{\rhoenv}{\phi_1}
  }
  {
    \tywf{\rhoenv}{\phi_0 \wedge \phi_1}
  }
\end{array}
\end{smathpar}
\end{minipage}


% \vspace*{0.1in}
% %
% \textbf{Subtyping}  \; \fbox
%   {\(\subtyp{\A}{\tau_1}{\tau_2}\)}\\
% %
\begin{minipage}{1.2in}
\begin{smathpar}
\begin{array}{c}
\renewcommand*{\arraystretch}{1.2}
  \subtyp{\A}{\tau}{\tau} \qquad
  \subtyp{\A}{\tyvar @\rho}{\aenv(\tyvar) @\rho}\qquad
% \subtyp{\A}{\RgnZ\inang{\rgn}}{\RgnZ\inang{\toprgn}}\qquad
% \subtyp{\A}{\RgnZ\inang{\toprgn}}{\RgnZ\inang{\rgn}}
\end{array}
\end{smathpar}
\end{minipage}
%

\begin{minipage}{1in}
\begin{smathpar}
\begin{array}{c}
\renewcommand*{\arraystretch}{1.2}
\RULE
  {
    \subtyp{\A}{\tau_1}{\tau_2}\\
    \subtyp{\A}{\tau_2}{\tau_3}
  }
  {
    \subtyp{\A}{\tau_1}{\tau_3}
  }
\end{array}
\end{smathpar}
\end{minipage}
% Existential type subtyping is redundant.
%%\begin{minipage}{1.5in}
%%\begin{smathpar}
%%\begin{array}{c}
%%\renewcommand*{\arraystretch}{1.2}
%%\RULE
%%  {
%%    \rho' \;\texttt{not free in}\; \tau
%%  }
%%  {
%%    \subtyp{\A}{\exists\rho.\tau}{\exists\rho'.[\rho'/\rho]\tau}
%%  }
%%\end{array}
%%\end{smathpar}
%%\end{minipage}
%
\begin{minipage}{2.55in}
\begin{smathpar}
\begin{array}{c}
\renewcommand*{\arraystretch}{1.2}
\RULE
  {
    \\
    CT(B) = \headerOf{B}\{...\}
%   \tywf{\A}{B\inang{\tbar}\inang{\pi^a\bar{\pi}}}\spc
%   \substFn = [\rbar/\rhobar, \ralloc/\rhoalloc, \tbar/\bar{\tyvar}]
  }
  {
    \subtyp{\A}{B\inang{\tbar}\inang{\bar{\pi}}}
        {[\rbar/\rhobar, \tbar/\bar{\tyvar}](\fbN)}
  }
\end{array}
\end{smathpar}
\end{minipage}
%
\begin{minipage}{2.75in}
\begin{smathpar}
\begin{array}{c}
\renewcommand*{\arraystretch}{1.2}
\RULE
  {
    \isvalid{\A.\phicx}{\phi_1 \Rightarrow \phi_2} \\
    \subtyp{\A}{\bar{\tau^{11}}}{\bar{\tau^{21}}} \spc
    \subtyp{\A}{\tau^{22}}{\tau^{12}}
  }
  {
    \subtyp{\A}
      {\inang{\rhobar \,|\, \phi_2}\bar{\tau^{21}}
          \xrightarrow{\rgn} \tau^{22}}
      {\inang{\rhobar \,|\, \phi_1}\bar{\tau^{11}}
          \xrightarrow{\rgn} \tau^{12}}
  }
\end{array}
\end{smathpar}
\end{minipage}


% %

\vspace*{0.1in}
\textbf{Expression Typing}  \; \fbox
  {\(\hastyp{\exptycx{\ralloc}{\env}}{e}{\tau}\)}\\
%
\begin{minipage}{1in}
\begin{smathpar}
\begin{array}{l}
\renewcommand*{\arraystretch}{1.2}
\hastyp{\exptycx{\ralloc}{\env}}{\unitval}{\unitZ}\\
\hastyp{\exptycx{\ralloc}{\env}}{x}{\env(\tau)}
% \RULE
%   {
%     \\
%     \env(x) = \tau
%   }
%   {
%     \hastyp{\exptycx{\ralloc}{\env}}{x}{\tau}
%   }
\end{array}
\end{smathpar}
\end{minipage}
%
% \begin{minipage}{1.2in}
% \begin{smathpar}
% \begin{array}{c}
% \renewcommand*{\arraystretch}{1.2}
% \RULE
%   {
%     \\
%     \\
%   }
%   {
%     \hastyp{\exptycx{\ralloc}{\env}}{\unitval}{\unitZ}
%   }
% \end{array}
% \end{smathpar}
% \end{minipage}
%
\begin{minipage}{2in}
\begin{smathpar}
\begin{array}{c}
\renewcommand*{\arraystretch}{1.2}
\RULE
  {
    \hastyp{\exptycx{\ralloc}{\env}}{x}{\tau'}\\
    f:\tau \,\in\, fields(bound_{\A.\aenv}(\tau'))
  }
  {
    \hastyp{\exptycx{\ralloc}{\env}}{e.f}{\tau}
  }
\end{array}
\end{smathpar}
\end{minipage}
%
% --- PACK expressio removed ---
% \begin{minipage}{1.75in}
% \begin{smathpar}
% \begin{array}{c}
% \renewcommand*{\arraystretch}{1.2}
% \RULE
%   {
%     \tywf{\A}{\exists\rho.\tau}\\
%     \hastyp{\exptycx{\ralloc}{\env}}{e}{[\rho_0/\rho]\tau}
%   }
%   {
%     \hastyp{\exptycx{\ralloc}{\env}}
%            {\C{pack} \; (\rho_0,e) \; \C{as} \; \exists\rho.\tau}
%            {\exists\rho.\tau}
%   }
% \end{array}
% \end{smathpar}
% \end{minipage}
%
\bigskip

%
\begin{minipage}{2.1in}
\begin{smathpar}
\begin{array}{c}
\renewcommand*{\arraystretch}{1.2}
\RULE
  {
    \tywf{\exptycx{\ralloc}{\env}}{\fbN} \spc
    ctype(\fbN) = \taubar \\
    \isvalid{\A.\phicx}{\rgn \outlives \ralloc}\spc
    allocRgn(\fbN) = \rgn \\
    \hastyp{\exptycx{\ralloc}{\env}}{\bar{e}}{\bar{\tau'}} \spc
    \subtyp{\A}{\bar{\tau'}}{\taubar}
  }
  {
    \hastyp{\exptycx{\ralloc}{\env}}{\C{new}\; \fbN(\bar{e})}{\fbN}
  }
\end{array}
\end{smathpar}
\end{minipage}
% %
% --- No special treatment of NEW REGION ---
% \begin{minipage}{2.1in}
% \begin{smathpar}
% \begin{array}{c}
% \renewcommand*{\arraystretch}{1.2}
% \RULE
%   {
%     \fgjtywf{\A.\aenv}{\RgnZ\inang{T}}\\
%     \hastyp{\exptycx{\ralloc}{\env}}{e}{\tau} \\
%     \subtyp{\A}{\tau}{\inang{\rhoalloc}
%     {\unitZ}\xrightarrow{\rgn}{T@\rhoalloc}}
%   }
%   {
%     {\exptycx{\ralloc}{\env}}\,\vdash\,
%           {\C{new}\; \RgnZ\inang{\rgn}\inang{T}(e)}\\
%            \hspace*{0.5in}{:\exists\rho.\RgnZ\inang{\rho}\inang{T}}
%   }
% \end{array}
% \end{smathpar}
% \end{minipage}
%
\begin{minipage}{2.75in}
\begin{smathpar}
\begin{array}{c}
\renewcommand*{\arraystretch}{1.2}
\RULE
  {
    \hastyp{\exptycx{\ralloc}{\env}}{e_0}{\tau} \spc
    \rbar \in \A.\rhoenv \\
    mtype(m,bound_{\A.\aenv}(\tau)) = \inang{\rhoalloc\rhobar \,|\, 
        \phi}\bar{\tau^1}\rightarrow{\tau^2} \\
    \substFn = [\rbar/\rhobar, \ralloc/\rhoalloc] \spc
%   \tywf{\A}{\substFn(\bar{\tau^1})} \spc
%   \tywf{\A}{\substFn(\tau^2)} \\
    \hastyp{\exptycx{\ralloc}{\env}}{\bar{e}}{\bar{\tau'}} \spc
    \subtyp{\A}{\bar{\tau'}}{\substFn(\bar{\tau^1})} \spc
    \isvalid{\A.\phicx}{\substFn(\phi)}
  }
  {
    \hastyp{\exptycx{\ralloc}{\env}}{e_0.m\inang{\ralloc\rbar}(\bar{e})} 
           {\substFn(\tau^2)}
  }
\end{array}
\end{smathpar}
\end{minipage}
%
\bigskip

%
\begin{minipage}{3.5in}
\begin{smathpar}
\begin{array}{c}
\renewcommand*{\arraystretch}{1.2}
\RULE
  {
%   \A = (\subtypcx)\spc
    \rhoalloc,\rhobar \notin \A.\rhoenv \spc
%   \rhoenv' = \rhoenv \cup \{\rhoalloc,\rhobar\}\spc
    \A' = (\A.\rhoenv \cup \{\rhoalloc,\rhobar\}, \A.\aenv, 
          \A.\phicx \conj \phi)\\
    \tywf{\A'.\rhoenv}{\phi}\spc
    \tywf{\A'}{\bar{\tau^1}}\spc
    \hastyp{\A',\rhoalloc,\env[\xbar \mapsto \bar{\tau^1}]}{e}{\tau^2}
  }
  {
    \hastyp{\exptycx{\ralloc}{\env}}
           {\lambdaexp{\rhoalloc\rhobar \,|\, \phi}
                      {\xbar:\bar{\tau^1}}{e}}
           {\inang{\rhoalloc\rhobar \,|\, \phi}
            \bar{\tau^1} \xrightarrow{\ralloc} \tau^2}
  }
\end{array}
\end{smathpar}
\end{minipage}
%
\begin{minipage}{3in}
\begin{smathpar}
\begin{array}{c}
\renewcommand*{\arraystretch}{1.2}
\RULE
  {
    \A = (\subtypcx) \spc
    \rgn \notin \rhoenv \spc
    \phi = \rhoenv \outlives \rgn\\
    \A' = (\rhoenv \cup \{\rgn\}, \aenv, \phicx \conj \phi)\spc
    \hastyp{\A',\rgn,\env}{e}{\unitZ}
  }
  {
    \hastyp{\exptycx{\ralloc}{\env}}{\letregion{\rgn}{e}}{\unitZ}
  }
\end{array}
\end{smathpar}
\end{minipage}
%
\bigskip

%
\begin{minipage}{3in}
\begin{smathpar}
\begin{array}{c}
\renewcommand*{\arraystretch}{1.2}
\RULE
  {
   \hastyp{\exptycx{\ralloc}{\env}}{x}
            {\RgnZ\inang{T}\inang{\rgn'}}\\
    \A = (\subtypcx) \spc
    \rgn \notin \rhoenv \spc
    \A' = (\rhoenv \cup \{\rgn\},\aenv,\phicx) \\
    \env' =  \env[y\mapsto T@\rgn]\spc
    \hastyp{\A',\ralloc,\env'}{e}{\unitZ} 
  }
  {
    \hastyp{\exptycx{\ralloc}{\env}}{\open{x}{\rgn}{y}{e}}
            {\unitZ}
  }
\end{array}
\end{smathpar}
\end{minipage}
% % --- No OPENALLOC ---
% \begin{minipage}{3.25in}
% \begin{smathpar}
% \begin{array}{c}
% \renewcommand*{\arraystretch}{1.2}
% \RULE
%   {
%     \A = (\subtypcx) \spc
%     \A' = (\rhoset,\rhoenv \cup \{\rgn\},\aenv,\phicx) \\
%     \hastyp{\exptycx{\ralloc}{\env}}{x}{\RgnZ\inang{\rgn}\inang{T}}\spc
%     \hastyp{\A',\rgn,\env[y\mapsto T@\rgn]}{e}{\unitZ}
%   }
%   {
%     \hastyp{\exptycx{\ralloc}{\env}}{\openalloc{x}{y}{e}}
%             {\unitZ}
%   }
% \end{array}
% \end{smathpar}
% \end{minipage}
%
% % --- BIG LAMBDA not needed ---
% \begin{minipage}{3.75in}
% \begin{smathpar}
% \begin{array}{c}
% \renewcommand*{\arraystretch}{1.2}
% \RULE
%   {
%     \rhoalloc,\rhobar \notin \A.\rhoenv \spc
% %   \rhoenv' = \rhoenv \cup \{\rhoalloc,\rhobar\}\\
%     \A' = (\A.\rhoenv \cup \{\rhoalloc,\rhobar\}, 
%             \A.\aenv, \A.\phicx \conj \phi)\\
%     \tywf{\A'.\rhoenv}{\phi}\spc
%     \hastyp{\A',\ralloc,\env}{e}{\bar{\tau^1} \xrightarrow
%             {\rgn} \tau^2}
%   }
%   {
%     \hastyp{\exptycx{\ralloc}{\env}}
%            {\Lambdaexp{\rhoalloc\rhobar \,|\, \phi}{e}}
%            {\inang{\rhoalloc\rhobar \,|\, \phi}
%             \bar{\tau^1} \xrightarrow{\rgn} \tau^2}
%   }
% \end{array}
% \end{smathpar}
% \end{minipage}
%
\begin{minipage}{3.2in}
\begin{smathpar}
\begin{array}{c}
\renewcommand*{\arraystretch}{1.2}
\RULE
  {
    \rbar \in \A.\rhoenv \spc
    \hastyp{\exptycx{\ralloc}{\env}}{e}
        {\inang{\rhoalloc\rhobar \,|\, \phi}
            \bar{\tau^1} \xrightarrow{\rgn} \tau^2}\\
    \substFn = \subst{\rbar}{\rhobar}
               \subst{\ralloc}{\rhoalloc} \spc
    \isvalid{\A.\phicx}{\substFn(\phi)} \spc
    \hastyp{\exptycx{\ralloc}{\env}}{\bar{e}}
        {\bar{\tau}}\spc
    \subtyp{\A}{\taubar}{\bar{\tau^1}}
  }
  {
    \hastyp{\A,\ralloc,\env}{e\inang{\ralloc\rbar}(\bar{e})}
           {\substFn(\tau^2)}
  }
\end{array}
\end{smathpar}
\end{minipage}
%
\bigskip

% % --- APPLICATION coalesced with instantiation ---
% \begin{minipage}{2in}
% \begin{smathpar}
% \begin{array}{c}
% \renewcommand*{\arraystretch}{1.2}
% \RULE
%   {
%     \hastyp{\exptycx{\ralloc}{\env}}{e_0}
%         {\bar{\tau^1} \xrightarrow{\rgn} \tau^2}\\
%     \hastyp{\exptycx{\ralloc}{\env}}{\bar{e}}
%         {\bar{\tau}}\spc
%     \subtyp{\A}{\taubar}{\bar{\tau^1}}
%   }
%   {
%     \hastyp{\A,\ralloc,\env}{e_0(\bar{e})}
%            {\tau^2}
%   }
% \end{array}
% \end{smathpar}
% \end{minipage}
%
\begin{minipage}{2in}
\begin{smathpar}
\begin{array}{c}
\renewcommand*{\arraystretch}{1.2}
\RULE
  {
    \hastyp{\exptycx{\ralloc}{\env}}{e_1}{\tau_1}\\
    \hastyp{\exptycx{\ralloc}{\env[x\mapsto\tau_1]}}{e_2}{\tau_2}\\
  }
  {
    \hastyp{\exptycx{\ralloc}{\env}}{\letexp{x}{e_1}{e_2}}{\tau_2}
  }
\end{array}
\end{smathpar}
\end{minipage}
%
\begin{minipage}{2in}
\begin{smathpar}
\begin{array}{c}
\renewcommand*{\arraystretch}{1.2}
\RULE
  {
    e_1 \in \{x,\,e.f\}\spc
    \hastyp{\exptycx{\ralloc}{\env}}{e_1}{\tau_1}\\
    \hastyp{\exptycx{\ralloc}{\env}}{e_2}{\tau_2}\spc
    \subtyp{\A}{\tau_2}{\tau_1}
  }
  {
    \hastyp{\exptycx{\ralloc}{\env}}{e_1\,:=\,e_2}{\unitZ}
  }
\end{array}
\end{smathpar}
\end{minipage}
%
% \begin{minipage}{1.1in}
% \begin{smathpar}
% \begin{array}{c}
% % --- UPCAST removed ---
% \renewcommand*{\arraystretch}{1.2}
% \RULE
%   {
%     \hastyp{\exptycx{\ralloc}{\env}}{e}{\tau_1}\\
%     \subtyp{\A}{\tau_1}{\tau}
%   }
%   {
%     \hastyp{\exptycx{\ralloc}{\env}}{(\tau)\,e}{\tau}
%   }
% \end{array}
% \end{smathpar}
% \end{minipage}
% % --- No UNPACK expression ---
% \begin{minipage}{3.25in}
% \begin{smathpar}
% \begin{array}{c}
% \renewcommand*{\arraystretch}{1.2}
% \RULE
%   {
%     \hastyp{\exptycx{\ralloc}{\env}}{e_1}{\exists \rho.\tau}\spc
%     \A = (\subtypcx) \spc
%     \rgn \notin \rhoset \\
%     \A' = (\rhoset \cup \{\rgn\},\rhoenv,\aenv,\phicx) \spc
%     \hastyp{\A',\ralloc,\env[x \mapsto \subst{\rgn}{\rho}\tau]}
%            {e_2}{\tau_2}\spc
%   }
%   {
%     \hastyp{\exptycx{\ralloc}{\env}}{\unpackexp{\rgn}{x}{e_1}{e_2}}
%             {\tau_2}
%   }
% \end{array}
% \end{smathpar}
% \end{minipage}
%
\begin{minipage}{1.5in}
\begin{smathpar}
\begin{array}{c}
\renewcommand*{\arraystretch}{1.2}
\RULE
  {
    \hastyp{\exptycx{\ralloc}{\env}}{e_1}{\unitZ}\\
    \hastyp{\exptycx{\ralloc}{\env}}{e_2}{\tau}\spc
  }
  {
    \hastyp{\exptycx{\ralloc}{\env}}{e_1\,;\,e_2}{\tau}
  }
\end{array}
\end{smathpar}
\end{minipage}
%

%
%%\textbf{Statement Semantics}  \; \fbox
%%  {\(\stmtsem{\ralloc}{s}{\stmtsemcx}{\stmtsemcxp}\)}\\
%%%
\begin{minipage}{2.2in}
\begin{smathpar}
\begin{array}{c}
\renewcommand*{\arraystretch}{1.2}
\RULE
  {
    \\
    \hastyp{\exptycx{\ralloc}{\env}}{e}{\tau'}\\
    \tywf{\A}{\tau}\spc
    \subtyp{\A}{\tau'}{\tau}
  }
  {
    \stmtsem{\ralloc}{\tau\;x\;=\;e}{\stmtsemcx}{\A,\,\env[x\mapsto\tau]}
  }
\end{array}
\end{smathpar}
\end{minipage}
%
\begin{minipage}{2.1in}
\begin{smathpar}
\begin{array}{c}
\renewcommand*{\arraystretch}{1.2}
\RULE
  {
    \\
    e_1 \in \{x,\,e.f\}\spc
    \hastyp{\exptycx{\ralloc}{\env}}{e_1}{\tau_1}\\
    \hastyp{\exptycx{\ralloc}{\env}}{e_2}{\tau_2}\spc
    \subtyp{\A}{\tau_2}{\tau_1}
  }
  {
    \stmtsem{\ralloc}{e_1\;=\;e_2}{\stmtsemcx}{\stmtsemcx}
  }
\end{array}
\end{smathpar}
\end{minipage}
%
\begin{minipage}{2.2in}
\begin{smathpar}
\begin{array}{c}
\renewcommand*{\arraystretch}{1.2}
\RULE
  {
    \A = (\subtypcx) \spc
    \rgn \notin \rhoset \\
    \rhoset' = \rhoset \cup \{\rgn\} \spc
    \rhoenv' = \rhoenv \cup \{\rgn\} \spc
    \phicx' = \phicx \conj \rhoenv \outlives \rgn \\
    \A' = (\rhoset',\rhoenv',\aenv,\phicx') \spc
    \stmtsem{\rgn}{s}{\A',\,\env}{\A'',\,\env'}
  }
  {
    \stmtsem{\ralloc}{\C{letregion}\;\rgn\;\{s\}}{\stmtsemcx}{\stmtsemcx}
  }
\end{array}
\end{smathpar}
\end{minipage}
%
\bigskip

%
\begin{minipage}{3.5in}
\begin{smathpar}
\begin{array}{c}
\renewcommand*{\arraystretch}{1.2}
\RULE
  {
    \hastyp{\exptycx{\ralloc}{\env}}{x}{\RgnZ\inang{\rgn}\inang{T}}\\
    \A = (\subtypcx) \spc
    \rhoenv' = \rhoenv \cup \{\rgn\} \spc
    \A' = (\rhoset,\rhoenv',\aenv,\phicx) \\
    \env' = \env[y\mapsto T@\rgn]\spc
    \stmtsem{\ralloc}{s} {\A',\,\env'}{\A'',\,\env''}
  }
  {
    \stmtsem{\ralloc}{\C{open}\;x\;\C{withroot}\;y\;\{s\}}
            {\stmtsemcx}{\stmtsemcx}
  }
\end{array}
\end{smathpar}
\end{minipage}
%
\begin{minipage}{3.5in}
\begin{smathpar}
\begin{array}{c}
\renewcommand*{\arraystretch}{1.2}
\RULE
  {
    \hastyp{\exptycx{\ralloc}{\env}}{x}{\RgnZ\inang{\rgn}\inang{T}}\\
    \A = (\subtypcx) \spc
    \rhoenv' = \rhoenv \cup \{\rgn\} \spc
    \A' = (\rhoset,\rhoenv',\aenv,\phicx) \\
    \env' = \env[y\mapsto T@\rgn]\spc
    \stmtsem{\rgn}{s} {\A',\,\env'}{\A'',\,\env''}
  }
  {
    \stmtsem{\ralloc}{\C{openalloc}\;x\;\C{withroot}\;y\;\{s\}}
            {\stmtsemcx}{\stmtsemcx}
  }
\end{array}
\end{smathpar}
\end{minipage}
%
\bigskip

%
\begin{minipage}{3.75in}
\begin{smathpar}
\begin{array}{c}
\renewcommand*{\arraystretch}{1.2}
\RULE
  {
    \hastyp{\exptycx{\ralloc}{\env}}{e}{\exists \rho.\tau}\spc
    \A = (\subtypcx) \spc
    \rho' \notin \rhoset \\
    \A' = (\rhoset \cup \{\rho'\},\rhoenv,\aenv,\phicx) \spc
    \tau' = [\rho'/\rho]\tau \spc
    \env' = \env[x\mapsto\tau']
  }
  {
    \stmtsem{\ralloc}{(\rho',\tau'\;x)\;=\;\C{unpack}\;e }
            {\stmtsemcx}{\stmtsemcxp}
  }
\end{array}
\end{smathpar}
\end{minipage}
%
\begin{minipage}{2.5in}
\begin{smathpar}
\begin{array}{c}
\renewcommand*{\arraystretch}{1.2}
\RULE
  {
    \stmtsem{\ralloc}{s_1}
            {\stmtsemcx}{\stmtsemcxp} \spc
    \stmtsem{\ralloc}{s_2}
            {\stmtsemcxp}{\stmtsemcxpp}
  }
  {
    \stmtsem{\ralloc}{s_1\,;\,s_2}
            {\stmtsemcx}{\stmtsemcxpp}
  }
\end{array}
\end{smathpar}
\end{minipage}
%


\caption{\fbname: Static Semantics}
\label{fig:fb-staticsem}
\end{figure*}


%% Trash
%%\env = [\xbar \mapsto \bar{\tau^B},
%%        \thisZ \mapsto B\inang{\rhoalloc\rhobar}\inang{\bar{\tyvar}}]\spc
%%\hastyp{\exptycx{\rhoalloc}{\env}}{\bar{u}}
%%       {\bar{\tau^{u}}}\spc
%%\hastyp{\exptycx{\rhoalloc}{\env}}{\bar{v}}
%%       {\bar{\tau^{v}}}\spc

\subsection{Types and Well-formedness}

Static semantics of \fbname is defined by the rules to establish
well-formedness of region types and region constraints, check
subtyping between a pair of region types, and type-check expressions
against region types. Fig.~\ref{fig:fb-staticsem} contains an
illustrative subset of such rules\footnote{The full formal development
can be found in our technical report~\cite{techrep}}, along with some
auxiliary definitions. The judgments defined by these rules make use
of a context ($\A$), which is a tuple of:
\begin{itemize}
\item A set ($\rhoenv \in 2^\rgn$) of static identifiers of regions
that are estimated to be live,
\item A finite map ($\aenv \in \tyvar \mapsto \fgjN$) of type
variables to their bounds\footnote{A bound of a type variable
($\tyvar$) in FGJ~\cite{fgj} is the class ($\fgjN$) that type variable
was declared to extend.  The map $\tyvar \mapsto \fgjN$ is denoted
$\Delta$ in FGJ formalism}, and
\item A constraint formula ($\phicx$) that captures the outlives
constraints on regions in $\rhoenv$.
\end{itemize}
The context for the expression typing judgment also includes:
\begin{itemize}
\item A type environment ($\env \in x \mapsto \tau$) that contains the
type bindings for variables in scope, and 
\item The static identifier ($\rgn^a$) of the allocation context for
the expression is being type-checked.  
\end{itemize}

\begin{figure}

\begin{codejava}
class Region<a $\extends$ Object><${\rhoalloc}$> $\extends$ Object<$\rhoalloc$>{
  <${\rho}$>unit$\xrightarrow{\rhoalloc}$a@$\rho$ thunk;
  Region(<${\rho}$>unit$\xrightarrow{\rhoalloc}$a@$\rho$ thunk) {
    super(); 
    this.thunk = thunk;
  }
  unit transfer$\inang{\rho}$(unit u) {
    return u;
  }
}
\end{codejava}

\caption{The stub used for $\RgnZ$ class}
\label{fig:region-stub}
\end{figure}

\noindent The well-formedness judgment on region types
($\tywf{\A}{\tau}$) makes use of the well-formedness and subtyping
judgments on core types \footnote{We use a double-piped turnstile
($\Vdash$) for judgments in FGJ~\cite{fgj}, and a simple turnstile
($\vdash$) for those in \FB.}. All judgments are implicitly
parameterized on a class table ($CT \in cn \mapsto D$) that maps class
names to their definitions in \FB. For judgments in FGJ, we use a
class table that maps class names to their region-erased definitions.
We denote the region-erased class table as ($\absof{CT}$). The precise
semantics of region erasure ($\absof{\cdot}$) are defined
in~\cite{techrep}, but it can be understood as an operation that
erases region annotations on types. The class table is assumed to bind
the name $\RgnZ$ to the stub shown in Fig.~\ref{fig:region-stub}.
This mostly eliminates the need to treat $\RgnZ$ objects, which double
up as transferable region handlers, as a special case in semantics.

The well-formedness rule for region instantiation in class types
($B\inang{\tbar}\inang{\ralloc\rbar}$) requires that the region-erased
(core) type be well-formed, instantiated regions be live, and that
they satisfy the constraints ($\phi$) imposed by the class on its
region parameters. The later is enforced by checking the validity of
$\phi$, with actual region arguments substituted\footnote{The notation
$[a/b](e)$ stands for "$a$ is substituted for $b$ in $e$"} for formal
region parameters, under the conditions ($\A.\phicx$) guaranteed by
the context. The semantics of this sequent is straightforward, and
follows directly from the properties of outlives and equality
relations. For any well-formed core type $T$, $T@\rgn$ is a
well-formed region type if $\rgn$ is a valid region. Well-formedness
of region constraints requires that the regions referred by the
constraints be live.

The subtype relation between region types of objects
($B\inang{\ralloc\rbar}\inang{\tbar}$) follows directly from the
subclass relation between their class types. Note that, in general, we
don't lift outlives relation between regions to subtype relation
between objects (of same core type) allocated in those regions. Such
lifting is unsound for the same reason as covariant/contravariant
subtyping between mutable references is unsound. However, we make a
special case for $\RgnZ$ objects: we allow $\RgnZ$ objects allocated
in any region to be considered as objects allocated in an eternal
region ($\toprgn$) that outlives every other region. This lets $\RgnZ$
objects escape the context they are created in (as required by the
\C{transfer} operation), and also facilitates data structures of
regions created in different contexts (for e.g., a list of $\RgnZ$
objects, whose root objects are of (core) type $T$, can be typed
$\C{List}\inang{\RgnZ\inang{T}}\inang{\rgn,\toprgn}$, where $\rgn$ is
the region where the spine of the list is allocated). As described in
\S\ref{sec:memory-safety}, runtime checks are used to prevent the
unsound subtyping rule on region objects from effecting the memory safety.
% \begin{smathpar}
% \begin{array}{c}
% \renewcommand*{\arraystretch}{1.1}
% \RULE
%   {
%     \isvalid{\A.\phicx}{\rgn_1 \outlives \rgn_2}
%   }
%   {
%     \subtyp{\A}{T@{\rgn_1}}{T@{\rgn_2}}
%   }
% \end{array}
% \end{smathpar}

The actual implementations of region-based memory management for \name
may choose to realize the $\toprgn$ as a garbage-collected region, but
this is not required. Our operational semantics
(\S\ref{sec:fb-opsem}), for example, does not even associate a region
with $\toprgn$, instead opting to allocate $\RgnZ$ objects in the same
transferable regions that they represent. Our type safety result
nonetheless guarantees memory safety in this implementation.

% Such a rule is unsound for it allows an object in a region ($\rgn_1$)
% to be treated as an object in a shorter living region ($\rgn_2
% \underlives \rgn_1$), thus allowing it to refer to shorter living
% objects in $\rgn_2$. The type system should prevent such unsafe
% references by restricting subtype relation to only those region types
% that are invariant in their allocation region. For instance, assuming
% that class $A$ is a subclass of $\ObjZ$ (as per FGJ), the subtype
% relation $A@\rgn <: \ObjZ@\rgn$ should be allowed, but not the
% relation $A@\rgn <: \ObjZ@\rgn'$, for any $\rgn' \neq \rgn$. \FB's
% type system enforces this not by restricting the subtype rule, but by
% constraining the subclass relation between region-annotated classes.
% In particular, it lets a class $A\inang{\rhoalloc}$ to be declared to
% extend $\ObjZ\inang{\rhoalloc}$, but never to extend
% $\ObjZ\inang{\rhoalloc_1}$, where $\rhoalloc_1 \neq \rhoalloc$. The
% well-formedness rule on class definitions (not shown) captures this
% restriction.

As expected, subtyping between arrow types is contravariant in
argument types and covariant in return types. It is also contravariant
in the region constraints that the region parameters have to satisfy.
For example, the type $\inang{\rhoalloc\rho} A@\rho \xrightarrow{\rgn}
\unitZ$ is a subtype of $\inang{\rhoalloc\rho \,|\, \rho =  \rhoalloc}
A@\rho \xrightarrow{\rgn} \unitZ$, for the former's precondition is
weaker than the later's. Finally, the subtyping is invariant with
respect to the allocation region ($\rgn$) of closures. 
% This subtype rule assumes that both arrow types have same set of
% region parameters.  However, this is not restrictive as (a). it is
% possible to add more region parameters to a function type and declare
% them to be equal to existing ones (as the arrow type above does), and
% (b). it is possible to rename region parameters by composing
% instantiation and generalization (big-lambda) expressions together.

The type rule for \C{letregion} expression requires that the static
identifier introduced by the expression be unique under the current
context (i.e., $\rgn \notin \rhoenv$). This condition is needed in
order to prevent the new region from incorrectly assuming any existing
outlives relationships on an eponymus region. Provided this is
satisfied, the expression ($e$) under \C{letregion} is then
type-checked against $\unitZ$ assuming that the new region is live
($\rgn \in \rhoenv$) and that it is outlived by all existing live
regions ($\rhoenv \outlives \rgn$).

Like the rule for \C{letregion}, the type rule for the \C{open}
expression requires that the static identifier ($\rgn$) introduced to
refer to the transferable region being opened be unique. However,
unlike the \C{letregion} rule, the rule for \C{open} does not
introduce any outlives relationship between the newly opened region
and any pre-existing region while checking the type of the expression
($e$) under \C{open}. This effectively preempts the possibility of any
references from objects inside the transferable region to those
outside. Environment ($\env$) is extended with binding for the type of
root object while type-checking $e$. 

The type rule for lamdba expression requires that its region
parameters ($\rhoalloc\rhobar$) be unique under the current context,
and that the constraints ($\phi$) over region parameters be
well-formed. The lambda-bound expression ($e$) is checked under an
extended type environment containing bindings for function's
arguments, assuming that region parameters are live, and that declared
constraints over region parameters hold. Note that the closure is
always allocated in the current allocation context ($\rgn^a$). This
prevents the closure from escaping the context in which it is created,
thus trivially ensuring the safety of any dereferences inside the
closure.

The type rule for the \C{new} expression is a straightforward
adaptation of corresponding FGJ rule to the the region type setting.

% The type rule for assignment expression ($e_1 := e_2$) ensures that a
% (local or instance) variable is only assigned objects of either the
% same type or a subtype. Recall that subtyping between region types is
% invariant with respect to the allocation regions of their objects. It
% follows that if a variable is declared to refer to objects in a region
% $\rgn$, then it is only ever assigned objects that allocated in
% $\rgn$. This prevents, for example, a temporary object allocated
% inside a stack region within a method from escaping the region's scope
% via an assignment to an instance variable.

\subsection{Operational Semantics and Type Safety}
\label{sec:fb-opsem}

Operational semantics of \fbname defines an evaluation relation
($\redsto$) over its expression language. In particular, the
evaluation gets stuck if an expression attempts to dereference a
pointer into a region that has either been deallocated or transferred.
On the other hand, if expression attempts to commit an operation on a
transferable region handler (a $\RgnZ$ object) that is not sanctioned
by the state transition discipline in Fig.~\ref{fig:region-fsm}, then
it is reduced to a value called $\invalidexn$. The accompanying
technical report~\cite{techrep} contains full operational semantics of
\fbname. 

To help state the type safety theorem, we extend \FB's expressions
with $\invalidexn$, and define the syntactic class of values:
\begin{smathpar}
\begin{array}{lclcl}
v & \in & \mathtt{values} & \coloneqq & \C{new}\; \fbN(\bar{v}) \ALT
\lambdaexp{\ralloc}{\rhoalloc\rhobar\,|\,\phi}{\taubar \; \xbar}{e}\\
\end{array}
\end{smathpar}
Type safety theorem is now stated thus:
\begin{theorem}
\emph{(\textbf{Type Safety})}
\label{thm:core-safety}
Let $\A_{\emptyset} = (\emptyset,\cdot,true)$ denote an empty context.
If $\hastyp{\A_{\emptyset}, \toprgn, \cdot}{e}{\tau}$, then either $e$
is a value, or $e \redsto \invalidexn$, or there exists an $e'$ such
that $e \redsto e'$ and $\hastyp{\A_{\emptyset}, \toprgn,
\cdot}{e'}{\tau}$.
\end{theorem}

