\section{The Core Language}
\label{sec:type-system}

\begin{figure*}[t!]
%
%\begin{minipage}
\begin{smathpar}
\renewcommand{\arraystretch}{1.2}
\begin{array}{lr} 
\multicolumn{2}{c}{
  {\rgn} \in \mathtt{Static \; region \; ids} \qquad
  {\rho} \in \mathtt{Region \; variables} \qquad
  {\tyvar, \tyvarb} \in \mathtt{Type \; variables} \qquad
  {m} \in \mathtt{Method \; names} \qquad
  {x,y,f} \in \mathtt{Variables \; and \; fields} }\\
\begin{array}{lclcl}
  cn & \in & \M{Class \; names} & \coloneqq & \ObjZ \ALT \RgnZ \ALT A \ALT B\\
  T  & \in & \M{FGJ \; types} & \coloneqq & \tyvar \ALT  \fgjN \ALT \unitZ
       \ALT \bar{T} \rightarrow T \\
  D  & \in & \M{Classes} & \coloneqq & 
       \C{class} \; cn\inang{\bar{\tyvar} \extends \bar{\fgjN}} 
                      \inang{\rhoalloc \rhobar \,|\, \phi}\extends \fbN 
                      \{\bar{\tau} \; \bar{f};\; k\; \bar{d}\}\;\\
  k  & \in & \M{Constructors} & \coloneqq & 
       cn(\bar{\tau} \; \bar{x})\{\C{super}(\bar{x}); \;
                                  \C{this}.\bar{f}\,=\,\bar{x};\}\\
  d  & \in & \M{Methods} & \coloneqq & 
       \tau \; m\inang{\rhoalloc \rhobar \,|\, \phi} (\taubar \; \xbar)
       \{\C{return}\;e;\}\\
\end{array}
\begin{array}{lclcl}
  \fgjN & \in & \M{FGJ \; class \; types} & \coloneqq & cn\inang{\tbar}\\
  \fbN  & \in & \M{FB \; class \; types} & \coloneqq & 
       cn\inang{\tbar}\inang{\ralloc \rbar} \\
  \tau &\in& \M{types} & \coloneqq & T@\rgn  
        \ALT \fbN \ALT \unitZ \\
       &   & & & \ALT \inang{\rhoalloc\rhobar \,|\, \phi}\bar{\tau}
        \xrightarrow{\rgn} \tau \\
  \phi,\Phi &\in& \M{Constraints} & \coloneqq & true 
        \ALT \rho \outlives \rho \\
        & & & & \ALT \rho = \rho \ALT \phi \conj \phi\\
  % v  & \in & \M{Value\;expressions} & \coloneqq & x \ALT \C{new} \; \fbN(\vbar)\\
\end{array}\\
\begin{array}{lclcl}
e  & \in & \M{Expressions} & \coloneqq & \unitval \ALT x \ALT e.f 
     \ALT e.m\inang{\ralloc \rbar}(\ebar) \ALT \C{new}\;\fbN(\ebar) 
     \ALT \lambdaexp{\ralloc}{\rhoalloc\rhobar \,|\, \phi}{\xbar:\taubar} {e}
     \ALT e\inang{\ralloc\rbar}(\bar{e})\\
%    \ALT (\tau)\, e\\
%    \ALT e\,;\,e  \\
   & & & & \ALT \letexp{x}{e}{e}
           \ALT \letregion{\rho}{e} 
           \ALT \open{e}{\rgn}{y}{e} \\
%   & & & & \ALT \unpackexp{\rgn}{x}{e}{e} \ALT e\,;\,e\\
% s  & \in & \M{Statements} & \coloneqq & \tau\;x\;=\;e \ALT x\;=\;e
%      \ALT e.f\;=\;e 
%      \ALT \C{open}\;x\;\C{withroot}\;y\;\{s\} \\
%     & & & & \ALT \C{openalloc}\;x\;\C{withroot}\;y\;\{s\} 
%      \ALT (\rho,\tau\;x)\;=\;\C{unpack}\;e \ALT s\,;\,s\\
\end{array}
\end{array}
\end{smathpar}

\caption{\fbname: Syntax}
\label{fig:fb-syntax}
\end{figure*}

\begin{figure*}[t]
%
\begin{minipage}{2.25in}
\begin{smathpar}
\begin{array}{lcl}
  allocRgn(A\inang{\ralloc\rbar}\inang{\tbar}) & = & \ralloc\\
  allocRgn(\inang{\rhoalloc\rhobar \,|\, \phi}\bar{\tau^1}
      \xrightarrow{\ralloc} \tau^2) & = & \ralloc\\
  shape(A\inang{\rhoalloc\rhobar}\inang{\tbar}) & = & A\inang{\tbar}\\
  bound_{\aenv}(\tyvar@\rgn) & = & \aenv(\tyvar)@\rgn\\
  bound_{\aenv}(\fbN) & = & \fbN\\
\end{array}
\end{smathpar}
\end{minipage}
%
\begin{minipage}{1.8in}
\begin{smathpar}
\begin{array}{c}
\renewcommand*{\arraystretch}{1.2}
\RULE
  {
    \\
    B \in \{\ObjZ,\RgnZ\}
  }
  {
    fields(B\inang{\ralloc\rbar}\inang{\tbar}) \;=\; \bullet
  }
\end{array}
\end{smathpar}
\end{minipage}
%
\begin{minipage}{3in}
\begin{smathpar}
\begin{array}{c}
\renewcommand*{\arraystretch}{1.2}
\RULE
  {
    CT(B) = \headerOf{B}\{\bar{\tau^f}\;\bar{f};\,...\}\\
    \substFn = [\rbar/\rhobar, \ralloc/\rhoalloc, \tbar/\bar{\tyvar}] \qquad 
    fields(\substFn(\fbN)) = \bar{g}:\bar{\tau^g}
  }
  {
    fields(B\inang{\ralloc\rbar}\inang{\tbar}) \;=\;
      \bar{g}:\bar{\tau^g},\,\bar{f}:\substFn(\bar{\tau^f})
  }
\end{array}
\end{smathpar}
\end{minipage}
%
\bigskip

\begin{minipage}{3.5in}
\begin{smathpar}
\begin{array}{lcl}
  ctype(\ObjZ\inang{\rgn}) & = & \bullet \\
% ctype(\RgnZ\inang{\rgn}\inang{T}) & = & \inang{\rhoalloc}
%   {\unitZ}\rightarrow{T@\rhoalloc}\\
  ctype(B\inang{\ralloc\rbar}\inang{\tbar}) & = & 
    fields(B\inang{\ralloc\rbar}\inang{\tbar})\\
  mtype(\C{transfer}, \exists\rho.\RgnZ\inang{\rho}\inang{T}) & = & 
    \inang{\rhoalloc} {\unitZ}\rightarrow{\unitZ}\\
  mtype(\C{free}, \exists\rho.\RgnZ\inang{\rho}\inang{T}) & = & 
    \inang{\rhoalloc} {\unitZ}\rightarrow{\unitZ}\\
\end{array}
\end{smathpar}
\end{minipage}
%
\begin{minipage}{3in}
\begin{smathpar}
\begin{array}{c}
\renewcommand*{\arraystretch}{1.2}
\RULE
  {
    CT(B) = \headerOf{B}\{\bar{\tau^f}\;\bar{f};\,k\;\bar{d}\}\\
    m \notin \bar{d} \qquad 
    \substFn = [\rbar/\rhobar, \ralloc/\rhoalloc, \tbar/\bar{\tyvar}]
  }
  {
    mtype (m,B\inang{\ralloc\rbar}\inang{\tbar}) \;=\;
    mtype (m, \substFn(\fbN))
  }
\end{array}
\end{smathpar}
\end{minipage}
%
\bigskip

\begin{minipage}{3.25in}
\begin{smathpar}
\begin{array}{c}
\renewcommand*{\arraystretch}{1.2}
\RULE
  {
    CT(B) = \headerOf{B}\{\bar{\tau^f}\;\bar{f};\,k\;\bar{d}\}\\
    \tau^2 \; m\mang (\bar{\tau^1}\;\bar{x})\{...\} \in \bar{d} \qquad
    \substFn = [\rbar/\rhobar, \ralloc/\rhoalloc, \tbar/\bar{\tyvar}]
  }
  {
    mtype (m,B\inang{\ralloc\rbar}\inang{\tbar}) \;=\;
    \substFn(\mang\bar{\tau^1} \rightarrow \tau^2)
  }
\end{array}
\end{smathpar}
\end{minipage}
%
\begin{minipage}{3.5in}
\begin{smathpar}
\begin{array}{c}
\renewcommand*{\arraystretch}{1.2}
\RULE
  {

    \substFn = \subst{\bar{\rho_2}}{\bar{\rho_1}}
               \subst{\rhoalloc_2}{\rhoalloc_1} \spc
    mtype(m,\fbN) = \inang{\rhoalloc_1\bar{\rho_1},|\, \phi_1}\bar{\tau^{11}} 
                      \rightarrow \tau^{12} \spc \texttt{implies}\\
    \isvalid{\A.\phicx}{\phi_2 \Leftrightarrow \substFn(\phi_1)} 
        \spc \texttt{and} \spc
    \bar{\tau^{21}} = \substFn(\bar{\tau^{11}}) \spc \texttt{and} \spc
    \subtyp{\A}{\tau^{22}} {\substFn(\tau^{12})}
    %\substFn = [\rbar/\rhobar, \ralloc/\rhoalloc, \tbar/\bar{\tyvar}]
  }
  {
    override(\A,\fbN,\inang{\rhoalloc_2\bar{\rho_2},|\, \phi_1}
              \bar{\tau^{21}} \rightarrow \tau^{22})
  }
\end{array}
\end{smathpar}
\end{minipage}
%
\bigskip

\begin{minipage}{5in}
\begin{smathpar}
\begin{array}{c}
  \rhoset,\rhoenv \in 2^{\rho} \qquad
  \aenv \in \tyvar \rightarrow \fgjN \qquad
  \A = (\subtypcx)\\
\end{array}
\end{smathpar}
\end{minipage}
%

\caption{\fbname: Auxiliary Definitions}
\label{fig:fb-auxdef}
\end{figure*}

\begin{figure*}[t!]
%
\textbf{Auxiliary Definitions}\\
\begin{figure*}[t]
%
\begin{minipage}{2.25in}
\begin{smathpar}
\begin{array}{lcl}
  allocRgn(A\inang{\ralloc\rbar}\inang{\tbar}) & = & \ralloc\\
  allocRgn(\inang{\rhoalloc\rhobar \,|\, \phi}\bar{\tau^1}
      \xrightarrow{\ralloc} \tau^2) & = & \ralloc\\
  shape(A\inang{\rhoalloc\rhobar}\inang{\tbar}) & = & A\inang{\tbar}\\
  bound_{\aenv}(\tyvar@\rgn) & = & \aenv(\tyvar)@\rgn\\
  bound_{\aenv}(\fbN) & = & \fbN\\
\end{array}
\end{smathpar}
\end{minipage}
%
\begin{minipage}{1.8in}
\begin{smathpar}
\begin{array}{c}
\renewcommand*{\arraystretch}{1.2}
\RULE
  {
    \\
    B \in \{\ObjZ,\RgnZ\}
  }
  {
    fields(B\inang{\ralloc\rbar}\inang{\tbar}) \;=\; \bullet
  }
\end{array}
\end{smathpar}
\end{minipage}
%
\begin{minipage}{3in}
\begin{smathpar}
\begin{array}{c}
\renewcommand*{\arraystretch}{1.2}
\RULE
  {
    CT(B) = \headerOf{B}\{\bar{\tau^f}\;\bar{f};\,...\}\\
    \substFn = [\rbar/\rhobar, \ralloc/\rhoalloc, \tbar/\bar{\tyvar}] \qquad 
    fields(\substFn(\fbN)) = \bar{g}:\bar{\tau^g}
  }
  {
    fields(B\inang{\ralloc\rbar}\inang{\tbar}) \;=\;
      \bar{g}:\bar{\tau^g},\,\bar{f}:\substFn(\bar{\tau^f})
  }
\end{array}
\end{smathpar}
\end{minipage}
%
\bigskip

\begin{minipage}{3.5in}
\begin{smathpar}
\begin{array}{lcl}
  ctype(\ObjZ\inang{\rgn}) & = & \bullet \\
% ctype(\RgnZ\inang{\rgn}\inang{T}) & = & \inang{\rhoalloc}
%   {\unitZ}\rightarrow{T@\rhoalloc}\\
  ctype(B\inang{\ralloc\rbar}\inang{\tbar}) & = & 
    fields(B\inang{\ralloc\rbar}\inang{\tbar})\\
  mtype(\C{transfer}, \exists\rho.\RgnZ\inang{\rho}\inang{T}) & = & 
    \inang{\rhoalloc} {\unitZ}\rightarrow{\unitZ}\\
  mtype(\C{free}, \exists\rho.\RgnZ\inang{\rho}\inang{T}) & = & 
    \inang{\rhoalloc} {\unitZ}\rightarrow{\unitZ}\\
\end{array}
\end{smathpar}
\end{minipage}
%
\begin{minipage}{3in}
\begin{smathpar}
\begin{array}{c}
\renewcommand*{\arraystretch}{1.2}
\RULE
  {
    CT(B) = \headerOf{B}\{\bar{\tau^f}\;\bar{f};\,k\;\bar{d}\}\\
    m \notin \bar{d} \qquad 
    \substFn = [\rbar/\rhobar, \ralloc/\rhoalloc, \tbar/\bar{\tyvar}]
  }
  {
    mtype (m,B\inang{\ralloc\rbar}\inang{\tbar}) \;=\;
    mtype (m, \substFn(\fbN))
  }
\end{array}
\end{smathpar}
\end{minipage}
%
\bigskip

\begin{minipage}{3.25in}
\begin{smathpar}
\begin{array}{c}
\renewcommand*{\arraystretch}{1.2}
\RULE
  {
    CT(B) = \headerOf{B}\{\bar{\tau^f}\;\bar{f};\,k\;\bar{d}\}\\
    \tau^2 \; m\mang (\bar{\tau^1}\;\bar{x})\{...\} \in \bar{d} \qquad
    \substFn = [\rbar/\rhobar, \ralloc/\rhoalloc, \tbar/\bar{\tyvar}]
  }
  {
    mtype (m,B\inang{\ralloc\rbar}\inang{\tbar}) \;=\;
    \substFn(\mang\bar{\tau^1} \rightarrow \tau^2)
  }
\end{array}
\end{smathpar}
\end{minipage}
%
\begin{minipage}{3.5in}
\begin{smathpar}
\begin{array}{c}
\renewcommand*{\arraystretch}{1.2}
\RULE
  {

    \substFn = \subst{\bar{\rho_2}}{\bar{\rho_1}}
               \subst{\rhoalloc_2}{\rhoalloc_1} \spc
    mtype(m,\fbN) = \inang{\rhoalloc_1\bar{\rho_1},|\, \phi_1}\bar{\tau^{11}} 
                      \rightarrow \tau^{12} \spc \texttt{implies}\\
    \isvalid{\A.\phicx}{\phi_2 \Leftrightarrow \substFn(\phi_1)} 
        \spc \texttt{and} \spc
    \bar{\tau^{21}} = \substFn(\bar{\tau^{11}}) \spc \texttt{and} \spc
    \subtyp{\A}{\tau^{22}} {\substFn(\tau^{12})}
    %\substFn = [\rbar/\rhobar, \ralloc/\rhoalloc, \tbar/\bar{\tyvar}]
  }
  {
    override(\A,\fbN,\inang{\rhoalloc_2\bar{\rho_2},|\, \phi_1}
              \bar{\tau^{21}} \rightarrow \tau^{22})
  }
\end{array}
\end{smathpar}
\end{minipage}
%
\bigskip

\begin{minipage}{5in}
\begin{smathpar}
\begin{array}{c}
  \rhoset,\rhoenv \in 2^{\rho} \qquad
  \aenv \in \tyvar \rightarrow \fgjN \qquad
  \A = (\subtypcx)\\
\end{array}
\end{smathpar}
\end{minipage}
%

\caption{\fbname: Auxiliary Definitions}
\label{fig:fb-auxdef}
\end{figure*}

%

\vspace*{0.1in}
%
\textbf{Type, and Type Constraint Well-formedness}  \; \fbox
  {\(\tywf{\A}{\tau}, \spc 
     \tywf{\rhoenv}{\phi}\)}\\
%
\begin{minipage}{1.25in}
\begin{smathpar}
\begin{array}{c}
\renewcommand*{\arraystretch}{1.2}
\RULE
  {
    \\
    \\
    \rgn \in \A.\rhoenv
  }
  {
    \tywf{\A}{\ObjZ\inang{\rgn}}
  }
\end{array}
\end{smathpar}
\end{minipage}
% %
% \begin{minipage}{1.5in}
% \begin{smathpar}
% \begin{array}{c}
% \renewcommand*{\arraystretch}{1.2}
% \RULE
%   {
%     \rgn \in \A.\rhoenv\\
%     \fgjtywf{\A.\aenv}{\RgnZ\inang{T}}
%   }
%   {
%     \tywf{\A}{\RgnZ\inang{\rgn}\inang{T}}
%   }
% \end{array}
% \end{smathpar}
% \end{minipage}
% %
% \begin{minipage}{1.5in}
% \begin{smathpar}
% \begin{array}{c}
% \renewcommand*{\arraystretch}{1.2}
% \RULE
%   {
%     \fgjtywf{\A.\aenv}{\RgnZ\inang{T}}
%   }
%   {
%     \tywf{\A}{\exists\rho.\RgnZ\inang{\rho}\inang{T}}
%   }
% \end{array}
% \end{smathpar}
% \end{minipage}
%
\begin{minipage}{2.75in}
\begin{smathpar}
\begin{array}{c}
\renewcommand*{\arraystretch}{1.2}
\RULE
  {
    \rgn \in \rhoenv \spc
    \rhobar \notin \A.\rhoenv \\
    \rhoenv' = \rhoenv \cup \{\rhobar\} \spc
    \A' = (\rhoenv', \aenv, \phicx \conj \phi) \\
    \tywf{\rhoenv'}{\phi}\spc 
    \tywf{\A'}{\bar{\tau^1}} \spc
    \tywf{\A'}{\tau^2}
  }
  {
    \tywf{(\rhoenv,\aenv,\phicx)}{\inang{\rhobar \,|\, \phi}
              \bar{\tau^1} \xrightarrow{\rgn} \tau^2}
  }
\end{array}
\end{smathpar}
\end{minipage}
%
\begin{minipage}{1.5in}
\begin{smathpar}
\begin{array}{c}
\renewcommand*{\arraystretch}{1.2}
\RULE
  { 
    \\
    \\
    \fgjtywf{\A.\aenv}{T}
  }
  {
    \tywf{\A}{\RgnZ\inang{T}\inang{\toprgn}}
  }
\end{array}
\end{smathpar}
\end{minipage}
%
\begin{minipage}{1in}
\begin{smathpar}
\begin{array}{c}
\renewcommand*{\arraystretch}{1.2}
\RULE
  {
    \\
    \\
    \rho_0,\rho_1 \in \rhoenv
  }
  {
    \tywf{\rhoenv}{\rho_0 \outlives \rho_1}
  }
\end{array}
\end{smathpar}
\end{minipage}
%

%
\begin{minipage}{3.5in}
\begin{smathpar}
\begin{array}{c}
\renewcommand*{\arraystretch}{1.2}
\RULE
  {
    CT(B) = \headerOf{B}\{...\}\\
    \rbar \in \rhoenv \spc
    \fgjtywf{\aenv}{B\inang{\tbar}}\spc
%   \substFn = [\rbar/\rhobar, \tbar/\bar{\tyvar}] \spc
    \isvalid{\phicx}{[\rbar/\rhobar](\phi)}
  }
  {
    \tywf{(\rhoenv,\aenv,\phicx)}{B\inang{\rbar}\inang{\tbar}}
  }
\end{array}
\end{smathpar}
\end{minipage}
%
\begin{minipage}{1.65in}
\begin{smathpar}
\begin{array}{c}
\renewcommand*{\arraystretch}{1.2}
\RULE
  {
    \fgjtywf{\A.\aenv}{T}\spc
    \rgn \in \A.\rhoenv\\
    \fgjsubtyp{\A.\aenv}{T}{\ObjZ}\spc
  }
  {
    \tywf{\A}{T@\rgn}
  }
\end{array}
\end{smathpar}
\end{minipage}
%
\begin{minipage}{0.75in}
\begin{smathpar}
\begin{array}{c}
\renewcommand*{\arraystretch}{1.2}
\RULE
  {
    \tywf{\rhoenv}{\phi_0} \\ \tywf{\rhoenv}{\phi_1}
  }
  {
    \tywf{\rhoenv}{\phi_0 \wedge \phi_1}
  }
\end{array}
\end{smathpar}
\end{minipage}


% \vspace*{0.1in}
% %
% \textbf{Subtyping}  \; \fbox
%   {\(\subtyp{\A}{\tau_1}{\tau_2}\)}\\
% %
\begin{minipage}{1.2in}
\begin{smathpar}
\begin{array}{c}
\renewcommand*{\arraystretch}{1.2}
  \subtyp{\A}{\tau}{\tau} \qquad
  \subtyp{\A}{\tyvar @\rho}{\aenv(\tyvar) @\rho}\qquad
% \subtyp{\A}{\RgnZ\inang{\rgn}}{\RgnZ\inang{\toprgn}}\qquad
% \subtyp{\A}{\RgnZ\inang{\toprgn}}{\RgnZ\inang{\rgn}}
\end{array}
\end{smathpar}
\end{minipage}
%

\begin{minipage}{1in}
\begin{smathpar}
\begin{array}{c}
\renewcommand*{\arraystretch}{1.2}
\RULE
  {
    \subtyp{\A}{\tau_1}{\tau_2}\\
    \subtyp{\A}{\tau_2}{\tau_3}
  }
  {
    \subtyp{\A}{\tau_1}{\tau_3}
  }
\end{array}
\end{smathpar}
\end{minipage}
% Existential type subtyping is redundant.
%%\begin{minipage}{1.5in}
%%\begin{smathpar}
%%\begin{array}{c}
%%\renewcommand*{\arraystretch}{1.2}
%%\RULE
%%  {
%%    \rho' \;\texttt{not free in}\; \tau
%%  }
%%  {
%%    \subtyp{\A}{\exists\rho.\tau}{\exists\rho'.[\rho'/\rho]\tau}
%%  }
%%\end{array}
%%\end{smathpar}
%%\end{minipage}
%
\begin{minipage}{2.55in}
\begin{smathpar}
\begin{array}{c}
\renewcommand*{\arraystretch}{1.2}
\RULE
  {
    \\
    CT(B) = \headerOf{B}\{...\}
%   \tywf{\A}{B\inang{\tbar}\inang{\pi^a\bar{\pi}}}\spc
%   \substFn = [\rbar/\rhobar, \ralloc/\rhoalloc, \tbar/\bar{\tyvar}]
  }
  {
    \subtyp{\A}{B\inang{\tbar}\inang{\bar{\pi}}}
        {[\rbar/\rhobar, \tbar/\bar{\tyvar}](\fbN)}
  }
\end{array}
\end{smathpar}
\end{minipage}
%
\begin{minipage}{2.75in}
\begin{smathpar}
\begin{array}{c}
\renewcommand*{\arraystretch}{1.2}
\RULE
  {
    \isvalid{\A.\phicx}{\phi_1 \Rightarrow \phi_2} \\
    \subtyp{\A}{\bar{\tau^{11}}}{\bar{\tau^{21}}} \spc
    \subtyp{\A}{\tau^{22}}{\tau^{12}}
  }
  {
    \subtyp{\A}
      {\inang{\rhobar \,|\, \phi_2}\bar{\tau^{21}}
          \xrightarrow{\rgn} \tau^{22}}
      {\inang{\rhobar \,|\, \phi_1}\bar{\tau^{11}}
          \xrightarrow{\rgn} \tau^{12}}
  }
\end{array}
\end{smathpar}
\end{minipage}


% %

\vspace*{0.1in}
\textbf{Expression Typing}  \; \fbox
  {\(\hastyp{\exptycx{\ralloc}{\env}}{e}{\tau}\)}\\
%
\begin{minipage}{1in}
\begin{smathpar}
\begin{array}{l}
\renewcommand*{\arraystretch}{1.2}
\hastyp{\exptycx{\ralloc}{\env}}{\unitval}{\unitZ}\\
\hastyp{\exptycx{\ralloc}{\env}}{x}{\env(\tau)}
% \RULE
%   {
%     \\
%     \env(x) = \tau
%   }
%   {
%     \hastyp{\exptycx{\ralloc}{\env}}{x}{\tau}
%   }
\end{array}
\end{smathpar}
\end{minipage}
%
% \begin{minipage}{1.2in}
% \begin{smathpar}
% \begin{array}{c}
% \renewcommand*{\arraystretch}{1.2}
% \RULE
%   {
%     \\
%     \\
%   }
%   {
%     \hastyp{\exptycx{\ralloc}{\env}}{\unitval}{\unitZ}
%   }
% \end{array}
% \end{smathpar}
% \end{minipage}
%
\begin{minipage}{2in}
\begin{smathpar}
\begin{array}{c}
\renewcommand*{\arraystretch}{1.2}
\RULE
  {
    \hastyp{\exptycx{\ralloc}{\env}}{x}{\tau'}\\
    f:\tau \,\in\, fields(bound_{\A.\aenv}(\tau'))
  }
  {
    \hastyp{\exptycx{\ralloc}{\env}}{e.f}{\tau}
  }
\end{array}
\end{smathpar}
\end{minipage}
%
% --- PACK expressio removed ---
% \begin{minipage}{1.75in}
% \begin{smathpar}
% \begin{array}{c}
% \renewcommand*{\arraystretch}{1.2}
% \RULE
%   {
%     \tywf{\A}{\exists\rho.\tau}\\
%     \hastyp{\exptycx{\ralloc}{\env}}{e}{[\rho_0/\rho]\tau}
%   }
%   {
%     \hastyp{\exptycx{\ralloc}{\env}}
%            {\C{pack} \; (\rho_0,e) \; \C{as} \; \exists\rho.\tau}
%            {\exists\rho.\tau}
%   }
% \end{array}
% \end{smathpar}
% \end{minipage}
%
\bigskip

%
\begin{minipage}{2.1in}
\begin{smathpar}
\begin{array}{c}
\renewcommand*{\arraystretch}{1.2}
\RULE
  {
    \tywf{\exptycx{\ralloc}{\env}}{\fbN} \spc
    ctype(\fbN) = \taubar \\
    \isvalid{\A.\phicx}{\rgn \outlives \ralloc}\spc
    allocRgn(\fbN) = \rgn \\
    \hastyp{\exptycx{\ralloc}{\env}}{\bar{e}}{\bar{\tau'}} \spc
    \subtyp{\A}{\bar{\tau'}}{\taubar}
  }
  {
    \hastyp{\exptycx{\ralloc}{\env}}{\C{new}\; \fbN(\bar{e})}{\fbN}
  }
\end{array}
\end{smathpar}
\end{minipage}
% %
% --- No special treatment of NEW REGION ---
% \begin{minipage}{2.1in}
% \begin{smathpar}
% \begin{array}{c}
% \renewcommand*{\arraystretch}{1.2}
% \RULE
%   {
%     \fgjtywf{\A.\aenv}{\RgnZ\inang{T}}\\
%     \hastyp{\exptycx{\ralloc}{\env}}{e}{\tau} \\
%     \subtyp{\A}{\tau}{\inang{\rhoalloc}
%     {\unitZ}\xrightarrow{\rgn}{T@\rhoalloc}}
%   }
%   {
%     {\exptycx{\ralloc}{\env}}\,\vdash\,
%           {\C{new}\; \RgnZ\inang{\rgn}\inang{T}(e)}\\
%            \hspace*{0.5in}{:\exists\rho.\RgnZ\inang{\rho}\inang{T}}
%   }
% \end{array}
% \end{smathpar}
% \end{minipage}
%
\begin{minipage}{2.75in}
\begin{smathpar}
\begin{array}{c}
\renewcommand*{\arraystretch}{1.2}
\RULE
  {
    \hastyp{\exptycx{\ralloc}{\env}}{e_0}{\tau} \spc
    \rbar \in \A.\rhoenv \\
    mtype(m,bound_{\A.\aenv}(\tau)) = \inang{\rhoalloc\rhobar \,|\, 
        \phi}\bar{\tau^1}\rightarrow{\tau^2} \\
    \substFn = [\rbar/\rhobar, \ralloc/\rhoalloc] \spc
%   \tywf{\A}{\substFn(\bar{\tau^1})} \spc
%   \tywf{\A}{\substFn(\tau^2)} \\
    \hastyp{\exptycx{\ralloc}{\env}}{\bar{e}}{\bar{\tau'}} \spc
    \subtyp{\A}{\bar{\tau'}}{\substFn(\bar{\tau^1})} \spc
    \isvalid{\A.\phicx}{\substFn(\phi)}
  }
  {
    \hastyp{\exptycx{\ralloc}{\env}}{e_0.m\inang{\ralloc\rbar}(\bar{e})} 
           {\substFn(\tau^2)}
  }
\end{array}
\end{smathpar}
\end{minipage}
%
\bigskip

%
\begin{minipage}{3.5in}
\begin{smathpar}
\begin{array}{c}
\renewcommand*{\arraystretch}{1.2}
\RULE
  {
%   \A = (\subtypcx)\spc
    \rhoalloc,\rhobar \notin \A.\rhoenv \spc
%   \rhoenv' = \rhoenv \cup \{\rhoalloc,\rhobar\}\spc
    \A' = (\A.\rhoenv \cup \{\rhoalloc,\rhobar\}, \A.\aenv, 
          \A.\phicx \conj \phi)\\
    \tywf{\A'.\rhoenv}{\phi}\spc
    \tywf{\A'}{\bar{\tau^1}}\spc
    \hastyp{\A',\rhoalloc,\env[\xbar \mapsto \bar{\tau^1}]}{e}{\tau^2}
  }
  {
    \hastyp{\exptycx{\ralloc}{\env}}
           {\lambdaexp{\rhoalloc\rhobar \,|\, \phi}
                      {\xbar:\bar{\tau^1}}{e}}
           {\inang{\rhoalloc\rhobar \,|\, \phi}
            \bar{\tau^1} \xrightarrow{\ralloc} \tau^2}
  }
\end{array}
\end{smathpar}
\end{minipage}
%
\begin{minipage}{3in}
\begin{smathpar}
\begin{array}{c}
\renewcommand*{\arraystretch}{1.2}
\RULE
  {
    \A = (\subtypcx) \spc
    \rgn \notin \rhoenv \spc
    \phi = \rhoenv \outlives \rgn\\
    \A' = (\rhoenv \cup \{\rgn\}, \aenv, \phicx \conj \phi)\spc
    \hastyp{\A',\rgn,\env}{e}{\unitZ}
  }
  {
    \hastyp{\exptycx{\ralloc}{\env}}{\letregion{\rgn}{e}}{\unitZ}
  }
\end{array}
\end{smathpar}
\end{minipage}
%
\bigskip

%
\begin{minipage}{3in}
\begin{smathpar}
\begin{array}{c}
\renewcommand*{\arraystretch}{1.2}
\RULE
  {
   \hastyp{\exptycx{\ralloc}{\env}}{x}
            {\RgnZ\inang{T}\inang{\rgn'}}\\
    \A = (\subtypcx) \spc
    \rgn \notin \rhoenv \spc
    \A' = (\rhoenv \cup \{\rgn\},\aenv,\phicx) \\
    \env' =  \env[y\mapsto T@\rgn]\spc
    \hastyp{\A',\ralloc,\env'}{e}{\unitZ} 
  }
  {
    \hastyp{\exptycx{\ralloc}{\env}}{\open{x}{\rgn}{y}{e}}
            {\unitZ}
  }
\end{array}
\end{smathpar}
\end{minipage}
% % --- No OPENALLOC ---
% \begin{minipage}{3.25in}
% \begin{smathpar}
% \begin{array}{c}
% \renewcommand*{\arraystretch}{1.2}
% \RULE
%   {
%     \A = (\subtypcx) \spc
%     \A' = (\rhoset,\rhoenv \cup \{\rgn\},\aenv,\phicx) \\
%     \hastyp{\exptycx{\ralloc}{\env}}{x}{\RgnZ\inang{\rgn}\inang{T}}\spc
%     \hastyp{\A',\rgn,\env[y\mapsto T@\rgn]}{e}{\unitZ}
%   }
%   {
%     \hastyp{\exptycx{\ralloc}{\env}}{\openalloc{x}{y}{e}}
%             {\unitZ}
%   }
% \end{array}
% \end{smathpar}
% \end{minipage}
%
% % --- BIG LAMBDA not needed ---
% \begin{minipage}{3.75in}
% \begin{smathpar}
% \begin{array}{c}
% \renewcommand*{\arraystretch}{1.2}
% \RULE
%   {
%     \rhoalloc,\rhobar \notin \A.\rhoenv \spc
% %   \rhoenv' = \rhoenv \cup \{\rhoalloc,\rhobar\}\\
%     \A' = (\A.\rhoenv \cup \{\rhoalloc,\rhobar\}, 
%             \A.\aenv, \A.\phicx \conj \phi)\\
%     \tywf{\A'.\rhoenv}{\phi}\spc
%     \hastyp{\A',\ralloc,\env}{e}{\bar{\tau^1} \xrightarrow
%             {\rgn} \tau^2}
%   }
%   {
%     \hastyp{\exptycx{\ralloc}{\env}}
%            {\Lambdaexp{\rhoalloc\rhobar \,|\, \phi}{e}}
%            {\inang{\rhoalloc\rhobar \,|\, \phi}
%             \bar{\tau^1} \xrightarrow{\rgn} \tau^2}
%   }
% \end{array}
% \end{smathpar}
% \end{minipage}
%
\begin{minipage}{3.2in}
\begin{smathpar}
\begin{array}{c}
\renewcommand*{\arraystretch}{1.2}
\RULE
  {
    \rbar \in \A.\rhoenv \spc
    \hastyp{\exptycx{\ralloc}{\env}}{e}
        {\inang{\rhoalloc\rhobar \,|\, \phi}
            \bar{\tau^1} \xrightarrow{\rgn} \tau^2}\\
    \substFn = \subst{\rbar}{\rhobar}
               \subst{\ralloc}{\rhoalloc} \spc
    \isvalid{\A.\phicx}{\substFn(\phi)} \spc
    \hastyp{\exptycx{\ralloc}{\env}}{\bar{e}}
        {\bar{\tau}}\spc
    \subtyp{\A}{\taubar}{\bar{\tau^1}}
  }
  {
    \hastyp{\A,\ralloc,\env}{e\inang{\ralloc\rbar}(\bar{e})}
           {\substFn(\tau^2)}
  }
\end{array}
\end{smathpar}
\end{minipage}
%
\bigskip

% % --- APPLICATION coalesced with instantiation ---
% \begin{minipage}{2in}
% \begin{smathpar}
% \begin{array}{c}
% \renewcommand*{\arraystretch}{1.2}
% \RULE
%   {
%     \hastyp{\exptycx{\ralloc}{\env}}{e_0}
%         {\bar{\tau^1} \xrightarrow{\rgn} \tau^2}\\
%     \hastyp{\exptycx{\ralloc}{\env}}{\bar{e}}
%         {\bar{\tau}}\spc
%     \subtyp{\A}{\taubar}{\bar{\tau^1}}
%   }
%   {
%     \hastyp{\A,\ralloc,\env}{e_0(\bar{e})}
%            {\tau^2}
%   }
% \end{array}
% \end{smathpar}
% \end{minipage}
%
\begin{minipage}{2in}
\begin{smathpar}
\begin{array}{c}
\renewcommand*{\arraystretch}{1.2}
\RULE
  {
    \hastyp{\exptycx{\ralloc}{\env}}{e_1}{\tau_1}\\
    \hastyp{\exptycx{\ralloc}{\env[x\mapsto\tau_1]}}{e_2}{\tau_2}\\
  }
  {
    \hastyp{\exptycx{\ralloc}{\env}}{\letexp{x}{e_1}{e_2}}{\tau_2}
  }
\end{array}
\end{smathpar}
\end{minipage}
%
\begin{minipage}{2in}
\begin{smathpar}
\begin{array}{c}
\renewcommand*{\arraystretch}{1.2}
\RULE
  {
    e_1 \in \{x,\,e.f\}\spc
    \hastyp{\exptycx{\ralloc}{\env}}{e_1}{\tau_1}\\
    \hastyp{\exptycx{\ralloc}{\env}}{e_2}{\tau_2}\spc
    \subtyp{\A}{\tau_2}{\tau_1}
  }
  {
    \hastyp{\exptycx{\ralloc}{\env}}{e_1\,:=\,e_2}{\unitZ}
  }
\end{array}
\end{smathpar}
\end{minipage}
%
% \begin{minipage}{1.1in}
% \begin{smathpar}
% \begin{array}{c}
% % --- UPCAST removed ---
% \renewcommand*{\arraystretch}{1.2}
% \RULE
%   {
%     \hastyp{\exptycx{\ralloc}{\env}}{e}{\tau_1}\\
%     \subtyp{\A}{\tau_1}{\tau}
%   }
%   {
%     \hastyp{\exptycx{\ralloc}{\env}}{(\tau)\,e}{\tau}
%   }
% \end{array}
% \end{smathpar}
% \end{minipage}
% % --- No UNPACK expression ---
% \begin{minipage}{3.25in}
% \begin{smathpar}
% \begin{array}{c}
% \renewcommand*{\arraystretch}{1.2}
% \RULE
%   {
%     \hastyp{\exptycx{\ralloc}{\env}}{e_1}{\exists \rho.\tau}\spc
%     \A = (\subtypcx) \spc
%     \rgn \notin \rhoset \\
%     \A' = (\rhoset \cup \{\rgn\},\rhoenv,\aenv,\phicx) \spc
%     \hastyp{\A',\ralloc,\env[x \mapsto \subst{\rgn}{\rho}\tau]}
%            {e_2}{\tau_2}\spc
%   }
%   {
%     \hastyp{\exptycx{\ralloc}{\env}}{\unpackexp{\rgn}{x}{e_1}{e_2}}
%             {\tau_2}
%   }
% \end{array}
% \end{smathpar}
% \end{minipage}
%
\begin{minipage}{1.5in}
\begin{smathpar}
\begin{array}{c}
\renewcommand*{\arraystretch}{1.2}
\RULE
  {
    \hastyp{\exptycx{\ralloc}{\env}}{e_1}{\unitZ}\\
    \hastyp{\exptycx{\ralloc}{\env}}{e_2}{\tau}\spc
  }
  {
    \hastyp{\exptycx{\ralloc}{\env}}{e_1\,;\,e_2}{\tau}
  }
\end{array}
\end{smathpar}
\end{minipage}
%

%
%%\textbf{Statement Semantics}  \; \fbox
%%  {\(\stmtsem{\ralloc}{s}{\stmtsemcx}{\stmtsemcxp}\)}\\
%%%
\begin{minipage}{2.2in}
\begin{smathpar}
\begin{array}{c}
\renewcommand*{\arraystretch}{1.2}
\RULE
  {
    \\
    \hastyp{\exptycx{\ralloc}{\env}}{e}{\tau'}\\
    \tywf{\A}{\tau}\spc
    \subtyp{\A}{\tau'}{\tau}
  }
  {
    \stmtsem{\ralloc}{\tau\;x\;=\;e}{\stmtsemcx}{\A,\,\env[x\mapsto\tau]}
  }
\end{array}
\end{smathpar}
\end{minipage}
%
\begin{minipage}{2.1in}
\begin{smathpar}
\begin{array}{c}
\renewcommand*{\arraystretch}{1.2}
\RULE
  {
    \\
    e_1 \in \{x,\,e.f\}\spc
    \hastyp{\exptycx{\ralloc}{\env}}{e_1}{\tau_1}\\
    \hastyp{\exptycx{\ralloc}{\env}}{e_2}{\tau_2}\spc
    \subtyp{\A}{\tau_2}{\tau_1}
  }
  {
    \stmtsem{\ralloc}{e_1\;=\;e_2}{\stmtsemcx}{\stmtsemcx}
  }
\end{array}
\end{smathpar}
\end{minipage}
%
\begin{minipage}{2.2in}
\begin{smathpar}
\begin{array}{c}
\renewcommand*{\arraystretch}{1.2}
\RULE
  {
    \A = (\subtypcx) \spc
    \rgn \notin \rhoset \\
    \rhoset' = \rhoset \cup \{\rgn\} \spc
    \rhoenv' = \rhoenv \cup \{\rgn\} \spc
    \phicx' = \phicx \conj \rhoenv \outlives \rgn \\
    \A' = (\rhoset',\rhoenv',\aenv,\phicx') \spc
    \stmtsem{\rgn}{s}{\A',\,\env}{\A'',\,\env'}
  }
  {
    \stmtsem{\ralloc}{\C{letregion}\;\rgn\;\{s\}}{\stmtsemcx}{\stmtsemcx}
  }
\end{array}
\end{smathpar}
\end{minipage}
%
\bigskip

%
\begin{minipage}{3.5in}
\begin{smathpar}
\begin{array}{c}
\renewcommand*{\arraystretch}{1.2}
\RULE
  {
    \hastyp{\exptycx{\ralloc}{\env}}{x}{\RgnZ\inang{\rgn}\inang{T}}\\
    \A = (\subtypcx) \spc
    \rhoenv' = \rhoenv \cup \{\rgn\} \spc
    \A' = (\rhoset,\rhoenv',\aenv,\phicx) \\
    \env' = \env[y\mapsto T@\rgn]\spc
    \stmtsem{\ralloc}{s} {\A',\,\env'}{\A'',\,\env''}
  }
  {
    \stmtsem{\ralloc}{\C{open}\;x\;\C{withroot}\;y\;\{s\}}
            {\stmtsemcx}{\stmtsemcx}
  }
\end{array}
\end{smathpar}
\end{minipage}
%
\begin{minipage}{3.5in}
\begin{smathpar}
\begin{array}{c}
\renewcommand*{\arraystretch}{1.2}
\RULE
  {
    \hastyp{\exptycx{\ralloc}{\env}}{x}{\RgnZ\inang{\rgn}\inang{T}}\\
    \A = (\subtypcx) \spc
    \rhoenv' = \rhoenv \cup \{\rgn\} \spc
    \A' = (\rhoset,\rhoenv',\aenv,\phicx) \\
    \env' = \env[y\mapsto T@\rgn]\spc
    \stmtsem{\rgn}{s} {\A',\,\env'}{\A'',\,\env''}
  }
  {
    \stmtsem{\ralloc}{\C{openalloc}\;x\;\C{withroot}\;y\;\{s\}}
            {\stmtsemcx}{\stmtsemcx}
  }
\end{array}
\end{smathpar}
\end{minipage}
%
\bigskip

%
\begin{minipage}{3.75in}
\begin{smathpar}
\begin{array}{c}
\renewcommand*{\arraystretch}{1.2}
\RULE
  {
    \hastyp{\exptycx{\ralloc}{\env}}{e}{\exists \rho.\tau}\spc
    \A = (\subtypcx) \spc
    \rho' \notin \rhoset \\
    \A' = (\rhoset \cup \{\rho'\},\rhoenv,\aenv,\phicx) \spc
    \tau' = [\rho'/\rho]\tau \spc
    \env' = \env[x\mapsto\tau']
  }
  {
    \stmtsem{\ralloc}{(\rho',\tau'\;x)\;=\;\C{unpack}\;e }
            {\stmtsemcx}{\stmtsemcxp}
  }
\end{array}
\end{smathpar}
\end{minipage}
%
\begin{minipage}{2.5in}
\begin{smathpar}
\begin{array}{c}
\renewcommand*{\arraystretch}{1.2}
\RULE
  {
    \stmtsem{\ralloc}{s_1}
            {\stmtsemcx}{\stmtsemcxp} \spc
    \stmtsem{\ralloc}{s_2}
            {\stmtsemcxp}{\stmtsemcxpp}
  }
  {
    \stmtsem{\ralloc}{s_1\,;\,s_2}
            {\stmtsemcx}{\stmtsemcxpp}
  }
\end{array}
\end{smathpar}
\end{minipage}
%


\caption{\fbname: Static Semantics}
\label{fig:fb-staticsem}
\end{figure*}


%% Trash
%%\env = [\xbar \mapsto \bar{\tau^B},
%%        \thisZ \mapsto B\inang{\rhoalloc\rhobar}\inang{\bar{\tyvar}}]\spc
%%\hastyp{\exptycx{\rhoalloc}{\env}}{\bar{u}}
%%       {\bar{\tau^{u}}}\spc
%%\hastyp{\exptycx{\rhoalloc}{\env}}{\bar{v}}
%%       {\bar{\tau^{v}}}\spc

/Users/gowtham/git/broom/fullversion/broom/paper/fb-morewfrules.tex

\subsection{Syntax}
\label{sec:fb-syntax}

We build on the Featherweight Generic Java (FGJ)~\cite{fgj} formalism
to formalize \name and its region type system. Our development reuses
notations from~\cite{fgj}, and relies on several of its definitions,
such as the definition of type well-formedness for the core
(region-free) language. Due to space constraints, we are unable to
reproduce them here, and instead encourage the reader to refer
to~\cite{fgj}.

Fig~\ref{fig:fb-syntax} describes the syntax of our formal language,
which we call \fbname (\FB). The language is seeded with \ObjZ and
\RgnZ classes. More classes can be defined using the \C{class}
keyword. Class types in \FB are region-annotated variants of class
types in FGJ (also called \emph{core types}). This correspondence is
reflected in the $T@\rgn$ syntax of a region type, which is the
simplest form of a region type describing an object of core type $T$
contained in a region $\rgn$. We let $\rgn$ range over static
identifiers of regions in \FB. Note that all \FB objects are boxed
values, hence their \FB types are always region-annotated. The only
unboxed value in \FB is $\unitval$ of type \unitZ.

The expression language of FGJ is impoverished, lacking, among others,
an assignment expression. While it is possible to encode
assignments to instance variables (object fields) in FGJ by creating a
new object with its fields set to new values, such encoding is not
semantics preserving in case of \name, as the new object could be
allocated in a region different from the allocation region of the
original object. To avoid such complications, we extend the expression
language with assignments and local variable declarations (via \C{let}
expressions). Instead of distinguishing between statements (e.g.,
assignments) and expressions (e.g., method calls) in \FB, we consider
statements as expressions of \unitZ type.  Statements that introduce
lexical blocks in \name, such as \C{letregion} and \C{open} are, are
converted to their expression form using the \C{in} keyword. We let
$\rgn$ range over static region identifiers introduced by these
expressions.

The expression language has been equipped with lambda abstraction
($\lambdaexp{...}{\xbar : \taubar}{e}$) and application ($e(\bar{e})$)
expressions to define and apply anonymous functions. These expressions
are uncurried variants of the corresponding expressions from System F.
Note that FGJ formalism does not include higher-order functions, hence
the syntactic class (\C{T}) of FGJ types needs to be extended with an
(uncurried) arrow type.

Informally, \FB class definitions are extensions of FGJ class
definitions with regions. A class definition in FGJ can be lifted to
its \FB definition by (a). lifting all core types in the body of the
class (i.e, fields, constructors and methods) to their
region-annotated versions, (b). parameterizing the class over
region variables that occur free in the region-annotated body of the
class, and (c). recording any outlives constraints on the region
parameters of the class. The later two steps effectively make the
class region polymorphic (cf. \S\ref{sec:alloc-ctxt}: qualified region
polymorphism). Note that in \FB, regions, and their static identifiers
($\rgn$) are confined to the expression language. Consequently, the
region annotated types of class fields, constructors and methods do
not refer to identifiers of concrete regions, but rather their place
holders called \emph{region variables} ($\rho$). Informally, region
variables ($\rho$) are to region identifiers ($\rgn$) as type
variables ($a$) are to core types ($T$). 

Note that when a generic class definition in FGJ is lifted to a
region-polymorphic definition in \FB, it nonetheless remains generic
with respect to core types. We say that the class is now a
region-polymorphic generic class. Region parameterization of a class
in \FB is independent of its parameterization over core types; they
are not conflated to make the class parametric over region types. For
instance, following is the region-polymorphic definition of a generic
\C{Pair} class (The symbol $\extends$ should be read \emph{extends}):
\begin{codejava}[mathescape=true]
class Pair<$\rho^a, \rho_1, \rho_2 \,|\, \rho_1 \outlives \rho^a 
                      \conj \rho_2 \outlives \rho^a$>
          <a $\extends$ Object, b $\extends$ Object> {
  a@$\rho_1$ fst; 
  b@$\rho_2$ snd;
  Pair(a@$\rho_1$ fst, b@$\rho_2$ snd) {
    super(); 
    this.fst = fst; 
    this.snd = snd;
  }
  a@$\rho_1$ getFst() {
    return this.fst;
  }
}
\end{codejava}
When objects of a class are allocated in a region $\rgn$, it means
that the class's constructor is run with $\rgn$ as its allocation
context. Every class definition in \FB is necessarily polymorphic with
respect to the allocation region of its objects, i.e., the allocation
context of its constructor. We adopt a convention that requires the
allocation region parameter (denoted $\rho^a$) to be the first region
parameter of a class.  Besides the allocation region of its objects,
\C{Pair} class is also parametric over the regions its first and
second elements are allocated in. References between objects allocated
in different regions are only allowed if the referred object is
guaranteed to outlive the referring object. In case of \C{Pair} class,
this means that allocation regions ($\rho_1$ and $\rho_2$) of both
objects that make up the pair must outlive the allocation region
($\rho^a$) of the \C{Pair} object. Such conditions over region
parameters of a class need to be recorded in its header as region
constraints ($\phi$) in order for the class to be judged well-formed
by the type system (Fig.~\ref{fig:fb-morewfrules}). 

To construct objects of the \C{Pair} class, its type and region
parameters need to be instantiated with core types ($T$) and concrete
region identifiers ($\rgn$), respectively. Region instantiation has to
satisfy the stated constraints over region parameters. For example,
the following code snippet appropriately instantiates region
parameters of the \C{Pair} class to construct a \C{Pair} of
{\ObjZ}{\!}s, each allocated in different regions:
\begin{codejava}
letregion $\rgn_0$ {
  Object<$\rgn_0$> fst = new Object<$\rgn_0$>();
  letregion $\rgn_1$ {
    Object<$\rgn_1$> snd = new Object<$\rgn_1$>();
    let p = new Pair<Object,Object><$\rgn_1$,$\rgn_0$,$\rgn_1$>
                  (fst,snd);
  }
}
\end{codejava}
Since $\rgn_0 \outlives \rgn_1 \conj \rgn_1 \outlives \rgn_1$, the
region constraints of the \C{Pair} class are satisfied, hence the
instantiation is valid. Observe that the region type of \C{p} conveys
the fact that (a). it is allocated in region $\rgn_1$, and (b). it
holds references to objects allocated in region $\rgn_0$ and $\rgn_1$.
In contrast, if we choose to allocate the \C{fst} object also in
$\rgn_1$, the region type of \C{p} would be
\C{Pair<\ObjZ,\ObjZ><$\rgn_1$,$\rgn_1$,$\rgn_1$>}, which can be
written simply as \C{Pair<\ObjZ,\ObjZ>@$\rgn_1$}. In general, the
\C{@} notation in a region type of an object \C{x} highlights that
\C{x}, and all the objects reachable from \C{x} via references are
allocated in a single region. We say that \C{x} is \emph{contained} in
the region. The $\ObjZ$ class in \FB contains no references to other
objects, hence its objects are always contained in their allocation
region. Their region type is $\ObjZ\inang{\rgn}$ (or equivalently,
$\ObjZ@\rgn$), for some region $\rgn$.
%%An object \emph{allocated} in a region $\rgn$ contains its spine in
%%$\rgn$, but can refer to objects allocated in other regions.  On the
%%other hand,

\FB's $\RgnZ$ objects, like $\ObjZ$ objects, have region type of form
$\RgnZ\inang{\rgn}$. However, unlike the $\rgn$ in
$\ObjZ\inang{\rgn}$, $\rgn$ in $\RgnZ\inang{\rgn}$ cannot be any
region. Recall that $\RgnZ$ objects have special semantics in \name -
they act as handlers to transferable regions.  Constructing a new
$\RgnZ$ object entails the creation of a new transferable region, and
it is in this region that the new object is allocated in. It follows
that $\rgn$ in $\RgnZ\inang{\rgn}$ should be the static identifier of
the new transferable region. But, static identifiers for transferable
regions are introduced only when such regions are opened for
allocation via \C{open} expression. What, then, should be the region
type of a \C{new Region} expression?

\FB resolves this problem by existentially quantifying the allocation
region of $\RgnZ$ objects when they are created. In other words, the
type of \C{new Region} expression in \FB is
$\exists\rho.\RgnZ\inang{\rho}$. Existential quantification in the
type captures the fact that there now exists a transferable region
containing the newly constructed $\RgnZ$ object. Elimination of
existential quantification is facilitated by the \C{open} expression,
which opens the transferable region and assigns it an identifier
($\rgn$).  Within the scope of \C{open}, the transferable region is
identified with $\rgn$, allowing its handler to instantiate the
existentially bound region variable with $\rgn$, and assume the type
of $\RgnZ\inang{\rgn}$.

A method definition ($d$) can be region polymorphic with respect to
(a). its allocation context (Sec.~\ref{sec:alloc-ctxt}), and (b). the
regions occuring in the region types of its arguments.  Region
parameters on the methods, like those on classes, are accompanied by
constraints ($\phi$) capturing the conditions that the parameters need
to satisfy for the method to be considered well-formed
(Fig.~\ref{fig:fb-morewfrules}). Allocation context (usually $\rho^a$
or $\rho^a_m$) is the first and inevitable region parameter of every
method in \FB. If a method is not intended to be polymorphic with
respect to its allocation context (for example, if its allocation
context needs to be same as the allocation region of its object), then
the monomorphism needs to be encoded as an explicit equality
constraint in $\phi$.  

Like methods, anonymous functions can also be region-polymorphic. 
% A lambda expression defines a region-polymorphic
% multi-argument function closure parameterized over function's
% allocation context parameter. 
The angle braces in the lambda expression ($\inang{\rho^a\rhobar \,|\,
\phi}$) serve the same purpose as they do in a method definition - to
capture region parameters along with their constraints. Region
parameters also appear in the arrow type of the lambda expression 
($\inang{\rhoalloc\rhobar \,|\, \phi}\bar{\tau} \xrightarrow{\rgn} 
\tau$) at the prenex position, similar to ML type schemes. However,
unlike ML, we don't distinguish between region types and region type
schemes; any of the $\tau$'s in the arrow type can themselves be
region-parametric arrow types. In this respect, our region type system
is more like System F's type system, which admits higher-rank
parametric polymorphism. Like System F, \FB provides a region
instantiation expression ($e\inang{\ralloc\rbar}$), and a region
generalization expression ($\Lambdaexp{\rhoalloc\rhobar \,|\,
\phi}{e}$) to instantiate and generalize region variables. 

Note that a lambda expression creates a closure, which can escape the
context in which it is created. It is therefore important to keep track of
the region in which a closure is allocated in order to avoid unsafe
dereferences. The $\rgn$ annotation above the arrow in the arrow
type denotes the allocation region of the corresponding closure. Note
that it is important to distinguish between the allocation context
argument ($\rhoalloc$) of a function and the allocation region
($\rgn$) of its closure. In \name, the later corresponds to the region where
a \C{Func} object is allocated, while the former corresponds to the
region where it is applied. 
% For instance, in the following example:
% \begin{codejava}
% letregion $\rgn$ {
%   let f = $\lambda{\inang{\rho^a}}$().$\,$new Object$\inang{\rho^a}$() 
%   in f
% }
% \end{codejava}
% The type of \C{f} is $\inang{\rho^a}\unitZ \xrightarrow{\rgn}
% \ObjZ\inang{\rho^a}$, coveying that (a). \C{f}'s closure is allocated
% in $\rgn$, and (b). when executed under an allocation context
% $\rho^a$, the closure returns an object allocated in $\rho^a$.

\subsection{Static Semantics}

\newcommand{\redstoo}[2]{\redsto{\rhoenv}{#1}{#2}}
\newcommand{\redstocup}[3]{\redsto{\rhoenv \cup \{#1\}}{#2}{#3}}
\newcommand{\anobjty}[0]{B\inang{\tbar}\inang{\ralloc\rbar}}
\newcommand{\anobj}[0]{\C{new} \; \anobjty(\bar{v})}
\begin{figure*}[t!]

%
\fbox {\(\redstoo{(e,\rhomap)}{(e',\rhomap')}\)}\\

%
\begin{minipage}{2.7in}
\begin{smathpar}
\begin{array}{c}
\renewcommand*{\arraystretch}{1.2}
\RULE
  {
    \allocRgn(\fbN) \in \rhoenv \spc
    \fields(\fbN) = \taubar\;\bar{f}
  }
  {
    \redstoo{((\C{new} \; \fbN(\vbar)).f_i,\rhomap)}{(v_i,\rhomap)}
  }
\end{array}
\end{smathpar}
\end{minipage}
%
% \begin{minipage}{3in}
% \begin{smathpar}
% \begin{array}{c}
% \renewcommand*{\arraystretch}{1.2}
% \RULE
%   {
%     \ralloc \in \rhoenv \spc
%     \redstoo{(e_i,\rhomap)}{(e_i',\rhomap')}
%   }
%   {
%     \redstoo{(\C{new} \; \fbN(...,e_i,...),\rhomap)}
%             {(\C{new} \; \fbN(...,e_i',...),\rhomap')}
%   }
% \end{array}
% \end{smathpar}
% \end{minipage}
%
\begin{minipage}{3.2in}
\begin{smathpar}
\begin{array}{c}
\renewcommand*{\arraystretch}{1.2}
\RULE
  {
    \rgn \notin \rhoenv \spc
    \redsto{\rhoenv \cup \{\rgn\}}{(e,\rhomap)}{(e',\rhomap')}
  }
  {
    \redstoo{(\letregion{\rgn}{e},\rhomap)}{(\letregion{\rgn}{e'},\rhomap')}
  }
\end{array}
\end{smathpar}
\end{minipage}
%

%
\begin{minipage}{3in}
\begin{smathpar}
\begin{array}{c}
\renewcommand*{\arraystretch}{1.2}
\RULE
  {
    \rgn \notin \rhoenv
  }
  {
    \redstoo{(\letregion{\rgn}{v},\rhomap)}{(v,\rhomap)}
  }
\end{array}
\end{smathpar}
\end{minipage}
%
\begin{minipage}{3.3in}
\begin{smathpar}
\begin{array}{c}
\renewcommand*{\arraystretch}{1.2}
\RULE
  {
    \fgjN = \RgnZ\inang{T} \spc
    \ralloc \in \rhoenv \spc
    \rgn \notin dom(\rhomap) \cup \rhoenv \spc
    \rhomap' = \rhomap[\rho \mapsto \CLOSED]
  }
  {
    \redstoo{(\C{new} \; \fgjN\inang{\toprgn}
                (\lambdaexp{\ralloc}{\rhoalloc}{}{e}),\rhomap)}
            {(\C{new} \; \fgjN\inang{\rgn}
                ([\rgn/\rhoalloc]e),\rhomap')}
  }
\end{array}
\end{smathpar}
\end{minipage}
%

%
\begin{minipage}{3.6in}
\begin{smathpar}
\begin{array}{c}
\renewcommand*{\arraystretch}{1.2}
\RULE
  {
    \fgjN = \RgnZ\inang{T} \spc
    \rgn \in dom(\rhomap) \spc
    \redstocup{\rgn}{(e,\rhomap)}{(e',\rhomap')}
  }
  {
    \redstoo{(\C{new} \; \fgjN\inang{\rgn}
                (e),\rhomap)}
            {(\C{new} \; \fgjN\inang{\rgn}
                (e'),\rhomap')}
  }
\end{array}
\end{smathpar}
\end{minipage}
%
\begin{minipage}{2.5in}
\begin{smathpar}
\begin{array}{c}
\renewcommand*{\arraystretch}{1.2}
\RULE
  {
    v_a = \C{new} \; \RgnZ\inang{T}\inang{\rgn}(v_r) \spc
    \rhomap(\rgn) \neq \XFERRED \spc
    \rgn_0 \notin \rhoenv \spc
  }
  {
    \redstoo{(\open{v_a}{\rgn_0}{x}{v_b},\rhomap)} {(v_b,\rhomap)}
  }
\end{array}
\end{smathpar}
\end{minipage}
%

%
\begin{minipage}{4in}
\begin{smathpar}
\begin{array}{c}
\renewcommand*{\arraystretch}{1.2}
\RULE
  {
    v_a = \C{new} \; \RgnZ\inang{T}\inang{\rgn}(v_r) \spc
    \rhomap(\rgn) \neq \XFERRED \spc
    \rgn_0 \notin \rhoenv \\
    \redstocup{\rgn_0}{([[\rgn_0/\rgn]v_r/x]e_b,
      \rhomap[\rgn \mapsto \OPEN])}{(e_b',\rhomap')} \spc
    \rhomap'' = \rhomap'[\rgn \mapsto \rhomap(\rgn)]
  }
  {
    \redstoo{(\open{v_a}{\rgn_0}{x}{e_b},\rhomap)} 
            {(\open{v_a}{\rgn_0}{x}{e_b'},\rhomap'')}
  }
\end{array}
\end{smathpar}
\end{minipage}
%
%
\begin{minipage}{3in}
\begin{smathpar}
\begin{array}{c}
\renewcommand*{\arraystretch}{1.2}
\RULE
  {
    \\
    v_a = \C{new} \; \RgnZ\inang{T}\inang{\rgn}(v_r) \spc
    \rhomap(\rgn) = \XFERRED \spc
%   \rgn_0 \notin \rhoenv \\
%   \redstocup{\rgn_0}{([[\rgn_0/\rgn]v_r/x]e_b,
%     \rhomap[\rgn \mapsto \OPEN])}{(e_b',\rhomap')}
  }
  {
    \redstoo{(\open{v_a}{\rgn_0}{x}{e_b},\rhomap)} 
            {\invalidexn}
  }
\end{array}
\end{smathpar}
\end{minipage}
%

%
\begin{minipage}{3.85in}
\begin{smathpar}
\begin{array}{c}
\renewcommand*{\arraystretch}{1.2}
\RULE
  {
    \allocRgn(\fbN),\ralloc \in \rhoenv \spc
    \mbody(m\inang{\ralloc \rbar},\fbN) = \bar{x}.e 
%   \redstoo{(, \rhomap)}{(e',\rhomap')}
  }
  {
    \redstoo{((\C{new}\;\fbN(\bar{v})).m\inang{\ralloc \rbar}
                      (\bar{v'}),\rhomap)}
            {([\bar{v'}/\xbar][\C{new} \; \fbN(\bar{v})/\thisZ]\,e,\rhomap)}
  }
\end{array}
\end{smathpar}
\end{minipage}
%
\begin{minipage}{3.25in}
\begin{smathpar}
\begin{array}{c}
\renewcommand*{\arraystretch}{1.2}
\RULE
  {
%   \ralloc \in \rhoenv \spc
    \fbN = \RgnZ\inang{T}\inang{\rgn}\spc
%   \redstoo{(, \rhomap)}{(e',\rhomap')}
    \rhomap(\rgn) \neq \OPEN \spc
    \rhomap' = \rhomap[\rgn \mapsto \XFERRED]
  }
  {
    \redstoo{((\C{new}\;\fbN(v)).\transfer\inang{\ralloc}
                      (),\rhomap)}
            {(\unitval,\rhomap')}
  }
\end{array}
\end{smathpar}
\end{minipage}
%

%
\begin{minipage}{2.5in}
\begin{smathpar}
\begin{array}{c}
\renewcommand*{\arraystretch}{1.2}
\RULE
  {
    \fbN = \RgnZ\inang{T}\inang{\rgn}\spc
%   \ralloc \in \rhoenv \spc
%   \redstoo{(, \rhomap)}{(e',\rhomap')}
    \rhomap(\rgn) = \OPEN \spc
  }
  {
    \redstoo{((\C{new}\;\fbN(v)).\transfer\inang{\ralloc}
                      (),\rhomap)}
            {\invalidexn}
  }
\end{array}
\end{smathpar}
\end{minipage}
%
\begin{minipage}{2.5in}
\begin{smathpar}
\begin{array}{c}
\renewcommand*{\arraystretch}{1.2}
\RULE
  {
    v_a = \lambdaexp{\rgn_a}{\rhoalloc\rhobar}
                        {\taubar \; \xbar}{e} \spc
    \rgn_a,\ralloc \in \rhoenv \spc 
  }
  {
    \redstoo{(v_a\inang{\ralloc\rbar}(\bar{v}) ,\rhomap)}
            {([\bar{v}/\xbar][\rbar/\rhobar][\ralloc/\rhoalloc]\,e,\rhomap)}
  }
\end{array}
\end{smathpar}
\end{minipage}
%
% %
% \begin{minipage}{1.8in}
% \begin{smathpar}
% \begin{array}{c}
% \renewcommand*{\arraystretch}{1.2}
% \RULE
%   {
%     \redstoo{(e_1,\rhomap)}{(e_1',\rhomap')}
%   }
%   {
%     \redstoo{(e_1;\,e_2,\rhomap)}
%             {(e_1';\,e_2,\rhomap')}
%   }
% \end{array}
% \end{smathpar}
% \end{minipage}
% %

\begin{minipage}{1.8in}
\begin{smathpar}
\begin{array}{c}
\renewcommand*{\arraystretch}{1.2}
\RULE
  {
    
  }
  {
    \redstoo{(\unitval;\,e_2,\rhomap)}
            {(e_2,\rhomap)}
  }
\end{array}
\end{smathpar}
\end{minipage}
% 
\begin{minipage}{1.8in}
\begin{smathpar}
\begin{array}{c}
\renewcommand*{\arraystretch}{1.2}
\RULE
  {
    \redstoo{(e,\rhomap)}{(e',\rhomap')}
  }
  {
    \redstoo{(E\lbrack e \rbrack, \rhomap)}
            {(E\lbrack e' \rbrack, \rhomap')}
  }
\end{array}
\end{smathpar}
\end{minipage}
%
\begin{minipage}{1.2in}
\begin{smathpar}
\begin{array}{c}
\renewcommand*{\arraystretch}{1.2}
\RULE
  {
    \redstoo{(e,\rhomap)}{\invalidexn}
  }
  {
    \redstoo{(E\lbrack e \rbrack, \rhomap)}
            {\invalidexn}
  }
\end{array}
\end{smathpar}
\end{minipage}

%
\begin{minipage}{2in}
\begin{smathpar}
\begin{array}{c}
\renewcommand*{\arraystretch}{1.2}
\RULE
  {
    \\
    \rgn \notin \rhoenv \spc
    \redsto{\rhoenv \cup \{\rgn\}}{(e,\rhomap)}{\invalidexn}
  }
  {
    \redstoo{(\letregion{\rgn}{e},\rhomap)}{\invalidexn}
  }
\end{array}
\end{smathpar}
\end{minipage}
%
\begin{minipage}{2in}
\begin{smathpar}
\begin{array}{c}
\renewcommand*{\arraystretch}{1.2}
\RULE
  {
    \fgjN = \RgnZ\inang{T} \spc
    \rgn \in dom(\rhomap) \\
    \redstocup{\rgn}{(e,\rhomap)}{\invalidexn}
  }
  {
    \redstoo{(\C{new} \; \fgjN\inang{\rgn}
                (e),\rhomap)}
            {\invalidexn}
  }
\end{array}
\end{smathpar}
\end{minipage}
%
\begin{minipage}{2.5in}
\begin{smathpar}
\begin{array}{c}
\renewcommand*{\arraystretch}{1.2}
\RULE
  {
    v_a = \C{new} \; \RgnZ\inang{T}\inang{\rgn}(v_r) \spc
    \rhomap(\rgn) \neq \XFERRED \spc
    \rgn_0 \notin \rhoenv \\
    \redstocup{\rgn_0}{([[\rgn_0/\rgn]v_r/x]e_b,
      \rhomap[\rgn \mapsto \OPEN])}{\invalidexn}
  }
  {
    \redstoo{(\open{v_a}{\rgn_0}{x}{e_b},\rhomap)} 
            {\invalidexn}
  }
\end{array}
\end{smathpar}
\end{minipage}
%

%
\bigskip

\textbf{Evaluation Context} \fbox {\(E\)}\\
\begin{smathpar}
\begin{array}{lcl}
E & \coloneqq & \bullet \ALT (\bullet).f \ALT \bullet.m\inang{\ralloc\rbar}(\ebar) \ALT
      v.m\inang{\ralloc\rbar}(...,\bullet,...) \ALT \C{new}\; \fbN(...,\bullet,...) \ALT
      \C{new} \; \RgnZ\inang{T}\inang{\toprgn}(\bullet) \ALT \bullet\inang{\ralloc\rbar}(\ebar) \\
  &  & \ALT v\inang{\ralloc\rbar}(...,\bullet,...) \ALT \bullet;\,e \ALT \open{\bullet}{\rgn}{y}{e}
\end{array}
\end{smathpar}

\caption{\fbname: Operational Semantics}
\label{fig:fb-opsem}
\end{figure*}


\onecolumn
\begin{lemma}
\emph{(\textbf{Substitution Preserves Typing})}
\label{thm:fb-substitution}
$\forall e, z, \tau_1, \tau_2, \rhoset, \rhoenv, \env, \phicx$, if $\hastyp{\emptyA,\rgn,
\env[z \mapsto \tau_1]}{e}{\tau_2}$ and $\hastyp{\emptyA,\rgn, \env}{v}{\tau_1}$, then 
$\hastyp{\emptyA,\rgn, \env}{[v/z]e}{\tau_2}$
\end{lemma}
\begin{proof}
Intros $e$. Induction on $e$. For every subexpression $e_0$, inductive hypothesis says the
following:
\begin{smathpar}
\begin{array}{cr}
  \forall (z, \tau_1, \tau_2, \rhoset, \rhoenv, \env, \phicx).\spc \hastyp{\emptyA,\rgn,
  \env[z \mapsto \tau_1]}{e_0}{\tau_2} \spc \conj \spc \hastyp{\emptyA,\rgn, \env}{v}{\tau_1} & IH1 \\
  \Rightarrow \; \hastyp{\emptyA,\rgn, \env}{[v/z]e_0}{\tau_2} & \\
\end{array}
\end{smathpar}
In all the inductive cases, we have the following hypotheses:
\begin{smathpar}
\begin{array}{cr}
  \hastyp{\emptyA,\rgn,\env[z \mapsto \tau_1]}{e}{\tau_2} & H2\\
  \hastyp{\emptyA,\rgn, \env}{v}{\tau_1} & H4\\
\end{array}
\end{smathpar}
In each case, proof strategy is the same: invert on $H2$, apply $IH1$, and then construct the
proof term for the goal by applying type rules.
% \begin{itemize}
%   \item Proof for cases $\unitval$, $x$, $e_0.f_i$, $\C{new} \fbN(\bar{e})$, $\C{new}\;
%   \RgnZT{\toprgn}$ follow straightforwardly from the inductive hypothesis.
%   \item Case ($\C{new}\; \RgnZT{\toprgn} : \RgnZT{\toprgn}$): 
% \end{itemize}
\qed
\end{proof}


\begin{lemma}
\emph{(\textbf{Weakening the context})}
\label{thm:fb-tywf}
$\forall v, \tau, \rhoenv, \rhoset, \phi, \rgn, \rgn_0$, such that $\value(v)$, $\rgn
\in \rhoenv$ and $\rgn_0 \notin \rhoenv$, if $\hastyp{(\rhoset, \rhoenv \cup
\{\rgn_0\}, \cdot, \phi \conj (\rhoenv \outlives \rgn_0)),\rgn_0,\cdot}{v}{\tau}$
and $\tywf{(\rhoset, \rhoenv, \cdot, \phi)}{\tau}$, then $\hastyp{(\rhoset,
\rhoenv, \cdot, \phi),\rgn,\cdot}{v}{\tau}$.
\end{lemma}
\nobreak

\begin{lemma}
\emph{(\textbf{Region Renaming Preserves Typing})}
\label{thm:fb-renaming}
$\forall v, \rhoset, T, \rhoenv, \rho, \rgn$, if $\rho,\rgn \notin
\rhoenv$ and $\hastyp{(\rhoset,\rhoenv \cup \{\rho\},\cdot,true),
\rho, \cdot}{v}{\tau}$, then $\hastyp{(\rhoset,\rhoenv \cup \{\rgn\}, 
\cdot,true),\rgn, \cdot}{[\rgn/\rho]v}{[\rgn/\rho]\tau}$
\end{lemma}

\begin{theorem}
\emph{(\textbf{Progress})}
\label{thm:fb-progress}
$\forall e, \tau, \rhoenv, \rhomap, \phicx$, if $\frv(e)
\subseteq dom(\rhomap)$ and $\tywf{\Delta}{\phicx}$ and
$\hastyp{\emptyA,\cdot}{e}{\tau}$, then one of the following holds:\\
  \begin{smathpar}
  \begin{array}{rl}
    (i) & \exists (e',\rhomap').\;\redstoo{(e,\rhomap)}{(e',\rhomap')}\\
    (ii) & \valuee(e)\\
    (iii) & \redstoo{(e,\rhomap)}{\invalidexn}\\
  \end{array}
  \end{smathpar}
\end{theorem}
\begin{proof}
Intros $e$. Induction on $e$. For every subexpressions $e_0$, inductive
hypothesis gives us the following:
\begin{smathpar}
\begin{array}{cr}
  \hspace*{-1in}\forall (\tau_0, \rhoenv_0, \rhomap_0, \phicx_0, \rgn_0). 
    (\tywf{\rhoenv_0}{\phicx_0}) \conj
    (\frv(e_0)\subseteq dom(\rhomap_0)) \conj
    \hastyp{(\rhoenv_0,\cdot,\phi_0), \cdot}{e_0}{\tau_0} \Rightarrow& IH1\\
       (\exists(e_0',\rhomap_0'). \redstoo{(e_0,\rhomap_0)}
                {(e_0',\rhomap_0')}) \disj (\valuee(e_0)) 
       \disj (\redstoo{(e_0,\rhomap_0)}{\invalidexn})& \\
\end{array}
\end{smathpar}
Cases from the induction:
\begin{itemize}
  \item Cases ($e = \unitval$ and $e = x$): proof trivial.

  \item Case ($e = e_0.f_i$): Intros. Hypothesis:
  \begin{smathpar}
  \begin{array}{cr}
    \frv(e)\subseteq dom(\rhomap) & H2\\
%   \rgn \in \rhoenv & H2\\
    \hastyp{\emptyA, \cdot}{e}{\tau} & H5\\
  \end{array}
  \end{smathpar}
  Inverting $H5$:
  \begin{smathpar}
  \begin{array}{cr}
    \hastyp{\emptyA, \cdot}{e_0}{\tau'} & H7\\
    \bar{f} :\taubar = \fields(\bound_{\cdot}(\tau')) & H9\\
  \end{array}
  \end{smathpar}
  Applying $H7$ in $IH1$, we have three cases:
  \begin{itemize}
    \item SCase ($e_0$ takes a step): Hypotheses:
    \begin{smathpar}
    \begin{array}{cr}
      \redstoo{(e_0,\rhomap)}{(e_0',\rhomap_0')} & H11\\
    \end{array}
    \end{smathpar}
    Therefore $(e_0.f_i,\rhomap)$ takes a step to $(e_0'.f_i,\rhomap_0')$ under $\rhoenv$.
    \item SCase ($e_0$ is a value): Since $e_0$ has type $\tau'$ and $\bound_{\cdot}$ is defined for
    $\tau'$, it follows that $e_0$ is $\C{new} \fbN(\vbar)$. From $H7$:
    \begin{smathpar}
    \begin{array}{cr}
      \hastyp{\emptyA, \cdot}{\C{new} \fbN(\vbar)}{\tau'} & H14\\
    \end{array}
    \end{smathpar}
    Inverting $H14$: 
    \begin{smathpar}
    \begin{array}{cr}
      \tywf{\emptyA}{N} & H16\\
    \end{array}
    \end{smathpar}
    Inverting $H16$:
    \begin{smathpar}
    \begin{array}{cr}
      \allocRgn(N) \in \rhoenv & H18\\
    \end{array}
    \end{smathpar}
    From $H9$, $H18$, we know that $(e_0.f_i,\rhomap)$ takes a step to
    $(v_i,\rhomap)$
    \item SCase ($e_0$ raises $\invalidexn$): $e_0.f_i$ also raises
    $\invalidexn$.
    \end{itemize}

  \item Case ($e = \letregion{\rgn_0}{e_0}$): Intros. Hypothesis:
  \begin{smathpar}
  \begin{array}{cr}
    \frv(e)\subseteq dom(\rhomap) & H2\\
    \hastyp{\emptyA, \cdot}{e}{\tau} & H5\\
  \end{array}
  \end{smathpar}
  Inverting $H5$:
  \begin{smathpar}
  \begin{array}{cr}
    \rgn_0 \notin \rhoenv & H7\\
    \hastyp{\emptyADelcupPhicap{\rgn_0}{\rgn_0}, \rgn_0, \cdot}{e_0}{\tau} & H9\\
  \end{array}
  \end{smathpar}
  From $H9$ and $IH1$, we have three cases:
  \begin{itemize}
    \item SCase ($e_0$ takes a step). Hypotheses:
    \begin{smathpar}
    \begin{array}{cr}
      \redsto{\rhoenv \cup \{\rgn_0\}}{(e_0,\rhomap)}{(e_0',\rhomap_0')} & H11\\
    \end{array}
    \end{smathpar}
    From $H7$ and $H11$, $\redstoo{(e,\rhomap)}{(\letregion{\rgn_0}{e_0'},\rhomap_0')}$.
    \item SCase ($e_0$ is a value $v_0$): From $H7$, $\redstoo{(e,\rhomap)}{(v_0,\rhomap)}$
    \item SCase ($e_0$ raises $\invalidexn$): $e$ raises $\invalidexn$ too.
  \end{itemize}
  
  \item Case ($e = \open{e_a}{\rgn_0}{y}{e_b}$): Intros. Hypotheses:
  \begin{smathpar}
  \begin{array}{cr}
    \tywf{\Delta}{\phicx} & H1\\
    \frv(e)\subseteq dom(\rhomap) & H2\\
    \hastyp{\emptyA, \cdot}{e}{\tau} & H5\\
%   \forall (\tau, \rhoenv, \rhomap, \rgn). \rgn \in \rhoenv \conj
%     \hastyp{\emptyA, \cdot}{e_0}{\tau} \;
%     \Rightarrow \; \exists(e',\rhomap'). \redstoo{(e_0,\rhomap)}
%                     {(e',\rhomap')} & IH1\\
  \end{array}
  \end{smathpar}
  $H2$ and the definition of $\frv$ imply:
  \begin{smathpar}
  \begin{array}{cr}
    \frv(e_a)\subseteq dom(\rhomap) & H3\\
  \end{array}
  \end{smathpar}
  Inverting $H5$:
  \begin{smathpar}
  \begin{array}{cr}
    \hastyp{\emptyA, \cdot}{e_a}{\RgnZ\inang{T}\inang{\toprgn}} & H7\\
    \rgn_0 \notin \rhoenv & H9\\
    \hastyp{\emptyADelcup{\rgn_0},[y \mapsto T@\rgn_0]}{e_b}{\tau} & H11\\
  \end{array}
  \end{smathpar}
  We have three cases. First case deals with $(e_a,\rhomap)$ taking a step to $(e_a',\rhomap_a')$
  under $\rhoenv$. Under this context, $(e,\rhomap)$ takes a step to
  $(\open{e_a'}{\rgn_0}{y}{e_b},\rhomap_a')$. So, there is progress.  Second case deals with $e_a$
  raising $\invalidexn$. In this case, execution of $e$ also raises $\invalidexn$. So, we again have
  progress. Third case deals with $e_a$ being a value $\C{new}\;\fbN(\vbar)$. Inverting $H7$, we
  have to consider two possible derivations: one from the generic type rule for values of any type,
  and another from the type rule tailor-made for $\RgnZ$ values. The first rule does not apply
  because $\fields(\RgnZ\inang{T} \inang{\toprgn})$ is undefined. The only rule that applies is the
  special type rule for $\RgnZ$ values. Hence, $e_a$ is $\C{new}\; \RgnZ\inang{T}\inang{\rgn_i}(v)$, where:
  \begin{smathpar}
  \begin{array}{cr}
    \rgn_i \notin \rhoenv & H12\\
    \hastyp{(\{\rgn_i\},\cdot,true), \cdot}{v}{T@\rgn_i} & H14\\
%   \hastyp{\emptyA,
%   \cdot}{\lambdaexp{\rgn}{\rhoalloc}{}{v}}{\inang{\rhoalloc}\unitZ \xrightarrow{\rgn} T@\rhoalloc} & H14\\
  \end{array}
  \end{smathpar}
%  $H13$ follows from the fact that for $\C{new} \RgnZ\inang{T}\inang{\rgn_i}(...)$ to be a value,
%  $\rgn_i$ cannot be $\toprgn$. i
% Inversion on $H14$:
% \begin{smathpar}
% \begin{array}{cr}
%   \rhoalloc \notin \rhoenv & H15\\
%   \hastyp{(dom(\rhomap), \rhoenv \cup \{\rhoalloc\},\cdot,true),\rhoalloc, \cdot}{v}{T@\rhoalloc} & H16\\
% \end{array}
% \end{smathpar}
  Since $e_a$ is $\RgnZ\inang{T}\inang{\rgn_i}(v)$, $H3$ implies: 
  \begin{smathpar}
  \begin{array}{cr}
    \rgn_i \in dom(\rhomap) & H13\\
  \end{array}
  \end{smathpar}
  Renaming the region variable in $H14$:
  \begin{smathpar}
  \begin{array}{cr}
    \hastyp{(\{\rgn_0\},\cdot,true), \cdot}{[\rgn_0/\rgn_i]\,v}{T@\rgn_0} & H15\\
  \end{array}
  \end{smathpar}
  $H9$ and $H1$ tell us that it is safe to strengthen the context in
  $H16$ to the following:
  \begin{smathpar}
  \begin{array}{cr}
    \hastyp{\emptyADelcup{\rgn_0},\cdot}{[\rgn_0/\rgn_i]\,v}{T@\rgn_0} & H16\\
  \end{array}
  \end{smathpar}
  Applying substitution lemma (Lemma~\ref{thm:fb-substitution}) on $H11$ and
  $H16$ gives us:
  \begin{smathpar}
  \begin{array}{cr}
    \hastyp{\emptyADelcup{\rgn_0},\cdot}{[[\rgn_0/\rgn_i]\,v/y]\,e_b}{\tau} & H19\\
  \end{array}
  \end{smathpar}
% From $H9$, $H13$, $H16$, and Lemma~\ref{thm:fb-renaming}:
% \begin{smathpar}
% \begin{array}{cr}
%   \hastyp{\emptyADelcup{\rgn_0},\rgn_0, \cdot}{[\rgn_0/\rgn_i]v}{T@\rgn_0} & H18\\
% \end{array}
% \end{smathpar}
% From $H11$, $H18$ and Lemma~\ref{thm:fb-substitution}:
% \begin{smathpar}
% \begin{array}{cr}
%   \hastyp{\emptyADelcup{\rgn_0},\rgn_0,\cdot}{[[\rgn_0/\rhoalloc]v/y]e_b}{\tau} & H19\\
% \end{array}
% \end{smathpar}
  Since $\rgn_i \in dom(\rhomap)$ (from $H13$), we have three cases:
  \begin{itemize}
    \item SCase ($\rhomap(\rgn_i) \neq \XFERRED$ and $e_b$ is not a value): By inductive hypothesis,
    $([[\rgn_0/\rgn_i]v/y]e_b,\rhomap[\rho \mapsto \OPEN]$ can either (a). take a step to
    $(e_b',\rhomap')$ under $\rhoenv \cup \{\rgn_0\}$, or (b). $e_b$ evaluates to $\invalidexn$.  In
    the first case, $(e,\rhomap)$ itself evaluates to: 
    \begin{smathpar}
    \begin{array}{cr}
      (\open {(\C{new} \; \RgnZ\inang{T}\inang{\rgn_i}(v))}
             {\rgn_0}{y}{e_b'}, \rhomap')
    \end{array}
    \end{smathpar}
    In the second case, the evaluation of $e$ also raises $\invalidexn$.

    \item SCase($\rhomap(\rgn_i) \neq \XFERRED$ and $e_b$ is a value $v_b$): Trivially,
    $\redstoo{(e,\rhomap)}{(v_b,\rhomap)}$.

    \item SCase($\rhomap(\rgn_i) = \XFERRED$): $e$ raises $\invalidexn$.
  \end{itemize}
 
  \item (e = $e_a.m\inang{\rbar}(\ebar)$): Intros. Hypotheses:
  \begin{smathpar}
  \begin{array}{cr}
%   \rgn \in \rhoenv & H2\\
    \tywf{\rhoenv}{\phicx} & H1\\
    \frv(e)\subseteq dom(\rhomap) & H2\\
    \hastyp{\emptyA, \cdot}{e}{\tau} & H4\\
  \end{array}
  \end{smathpar}
  From $H2$ and the fact that $e_a$ and $e_i$ (forall $i$) are
  subexpressions of $e$, we have:
  \begin{smathpar}
  \begin{array}{cr}
    \frv(e_a)\subseteq dom(\rhomap) & H3\\
    \frv(e_i)\subseteq dom(\rhomap) & H5\\
  \end{array}
  \end{smathpar}
  By inversion on $H4$:
  \begin{smathpar}
  \begin{array}{cr}
    \hastyp{\emptyA,\cdot}{e_a}{\tau_a} & H6\\
%   \ralloc = \rgn & H7\\
    \rbar \in \rhoenv & H8\\
    \mtype(m,\bound_{\cdot}(\tau_a)) = \inang{\rhobar\,|\,\phi}\bar{\tau^1} \rightarrow \tau^2 & H10\\
    \hastyp{\emptyA,\cdot}{\ebar}{[\rbar/\rhobar]\bar{\tau^1}} & H11\\
    \isvalid{\phicx}{[\rbar/\rhobar]\phi}
  \end{array}
  \end{smathpar}
  Three cases: 
  \begin{itemize}
    \item SCase ($e_a$ isn't a value): From $H1$, $H3$, $H6$ and $IH$,
    we know that either (a). $e_a$ can take a step, or (b) $e_a$
    reduces to $\invalidexn$. In first case, $e$ can also take a step,
    and in second case, $e$ also reduces to $\invalidexn$.

    \item SCase ($e_a = v_a$, but $\exists i$ such that  $e_i$ isn't a
    value): From $H1$, $H5$, $H11$ and $IH$, we know that
    $(e_i,\rhomap)$ can either (a). take a step to $(e_i',\rhomap')$
    under $\rhoenv$, or (b).  $e_i$ reduces to $\invalidexn$. In the
    first case, we have
    $\redstoo{(v_a.m\inang{\rbar}(...,e_i,...),\rhomap)}
    {(v_a.m\inang{\rbar}(...,e_i',...),\rhomap')}$.

    \item SCase ($e_a = v_a$ and $\forall i.\,e_i = v_i$): $H10$ says
    that bound for $\tau_a$ is defined under empty $\aenv$. This is
    possible only if $\tau_a = \fbN$ and $v_a = \C{new}\;
    \fbN(\bar{v'})$. Furthemore, $\fbN$ cannot be of form
    $\RgnZ\inang{T}\inang{\toprgn}$ because, $\mtype$ isn't defined
    for $\RgnZ$ (in short, inverting $H10$ gives us $\tau_a = \fbN$,
    for some $N \neq \RgnZ\inang{T}\inang{\toprgn}$). Using these
    facts, and inverting $H6$, we get $\tywf{\emptyA}{\fbN}$.
    Inverting it again:
    \begin{smathpar}
    \begin{array}{cr}
      \allocRgn(\fbN) \in \rhoenv & H13\\
    \end{array}
    \end{smathpar}
    Now, since $\mtype(m,\fbN)$ is defined if and only if $\mbody(m,\fbN)$ is defined, we know that:
    \begin{smathpar}
    \begin{array}{cr}
      \mbody(m,\fbN) = \rhobar.\xbar.\,e_m & H14\\
    \end{array}
    \end{smathpar}
    From $H13$ and $H14$, we know that $\redstoo{((v_a.m\inang{\rbar}(\bar{v}),\rhomap)}
      {([\bar{v}/\xbar][\C{new}\;\fbN(\bar{v'})/\thisZ][\rbar/\rhobar]e_m, \rhomap)}$
  \end{itemize}

  \item Case ($e = e_a\inang{\rbar}(\ebar)$). Proof closely follows the proof for
  $e_a.m\inang{\rbar}(\ebar)$. The only difference is that when $e_a$ evaluates to a lambda
  $\lambdaexp{\rgn}{\rhobar \,|\, \phi}{\taubar \; \xbar}{e_b}$, we need a proof that
  $\rgn \in \rhoenv$. This can be obtained by inverting the type judgment for the lambda.

  \item Case($e = \C{new}\; \fbN(\ebar)$): Intros. Hypotheses:
  \begin{smathpar}
  \begin{array}{cr}
    \tywf{\rhoenv}{\phicx} & H1\\
    \frv(e)\subseteq dom(\rhomap) & H2\\
    \hastyp{\emptyA, \cdot}{\C{new}\; \fbN(\ebar)}{\tau} & H4\\
  \end{array}
  \end{smathpar}
  Inverting $H4$ leads to two cases:
  \begin{itemize}
    \item SCase ($\shape(\fbN) \neq \RgnZ\inang{T}$): Hypotheses:
    \begin{smathpar}
    \begin{array}{cr}
      \tywf{\emptyA}{\fbN} & H5\\
      \fields(\fbN) = \bar{f}:\taubar & H7\\
      \hastyp{\emptyA,\cdot}{\ebar}{\taubar} & H8\\
    \end{array}
    \end{smathpar}
    Inverting $H5$:
    \begin{smathpar}
    \begin{array}{cr}
      \allocRgn(\fbN) \in \rhoenv & H10\\
    \end{array}
    \end{smathpar}
    Three cases:
    \begin{itemize}
      \item SSCase ($\exists i$ such that $e_i$ takes a step): In this case, $e$ also takes a step.
      \item SSCase ($\exists i$ such that $e_i$ reduces to $\invalidexn$). In this case, $e$ also
      reduces to $\invalidexn$.
      \item SSCase ($\forall i$ $e_i$ is a value $v_i$): In this case, $e$ is also a value
      $\C{new}\;\fbN(\vbar)$.
    \end{itemize}
    
    \item SCase ($\shape(\fbN) = \RgnZ\inang{T}$ and $e =
    \RgnZ\inang{T}\inang{\rgn_i}(e_0)$): From $H4$:
    \begin{smathpar}
    \begin{array}{cr}
      \hastyp{\emptyA, \cdot}{\C{new}\; \RgnZ\inang{T}\inang{\rgn_i}(e_0)}{\tau} & H4\\
    \end{array}
    \end{smathpar}
    Inversion on $H4$ gives two cases:
    \begin{itemize}
      \item SSCase($\rgn_i = \toprgn$): In this case, $e_0 = \lambdaexp{\rgn}{\rho}{}{e_1}$. 
      In this case, $\redstoo{(\C{new}\; \RgnZ\inang{T}\inang{\toprgn}(e_0),\rhomap)} {(\C{new}\;
      \RgnZ\inang{T}\inang{\rgn_j}([\rgn_j/\rho]e_1), \rhomap[\rgn_j \mapsto \CLOSED])}$, where
      $\rgn_j$ is a new region identifier such that $\rgn_j \notin
      \rhoset \cup dom(\rhomap)$
%     Hypotheses:
%     \begin{smathpar}
%     \begin{array}{cr}
%       \fgjtywf{\cdot}{T} & H12\\
%       \rgn_i = \toprgn & H13\\
%       \hastyp{\vacantA{\rhoalloc},\rhoalloc,\cdot}{e_1} {T@\rhoalloc}& H14\\
%     \end{array}
%     \end{smathpar}
%     From $H14$ and $IH1$, $e_1$ can either take a step, or raise $\invalidexn$, or is a value. In the first
%     two cases, $e$ takes a step, $e$ raises $\invalidexn$, respectively. In the third case, $e_0$
%     is a value $v_0$. By inversion on $H14$, we know that $v_0 =
%     \lambdaexp{\rgn}{\rhoalloc}{}{e_1}$.  In this case, $\redstoo{(\C{new}\;
%     \RgnZ\inang{T}\inang{\toprgn}(v_0),\rhomap)} {(\C{new}\;
%     \RgnZ\inang{T}\inang{\rgn_j}([\rgn_j/\rhoalloc]e_1), \rhomap[\rgn_j \mapsto \CLOSED])}$, where
%     $\rgn_j$ is fresh ($\rgn_j \notin \rhoenv \cup dom(\rhomap)$).

      \item SSCase ($\rgn_i \neq \toprgn$): Hypotheses:
      \begin{smathpar}
      \begin{array}{cr}
        \rgn_i \in dom(\rhomap) & H16\\
        \rgn_i \notin \rhoenv \cup \{\toprgn\} & H18\\
        \hastyp{\emptyADelcup{\rgn_i},\cdot} {e_0}{T@\rgn_i} & H20\\
      \end{array}
      \end{smathpar}
      $H16$ comes from $H2$. Applying $IH1$ using $H1$, $H2$, $H18$ and $H20$, we have three cases:
      \begin{itemize}
        \item SSSCase ($e_0$ is a value $v_0$): In this case, $e = \RgnZ\inang{T}\inang{\rgn_i}(v_0)$ is also
        a value.
        \item SSSCase ($e_0$ raises $\invalidexn$): In this case, $e$ also raises $\invalidexn$.
        \item SSSCase ($\redsto{\rhoenv \cup \{\rgn_i\}}{(e_0,\rhomap)} {(e_0',\rhomap')}$): In this
        case, $\redstoo{(\RgnZ\inang{T}\inang{\rgn_i}(e_0),\rhomap)}
        {(\RgnZ\inang{T}\inang{\rgn_i}(e_0'), \rhomap')}$
      \end{itemize}
    \end{itemize}
  \end{itemize}
%   \begin{smathpar}
%   \begin{array}{cr}
%     \rhoalloc \notin \rhoenv & H17\\
%     \hastyp{(dom(\rhomap), \rhoenv \cup \{\rhoalloc\},\cdot,true),\rgn,\cdot}
%           {e_1}{T@\rhoalloc} & H19\\
%   \end{array}
%   \end{smathpar}
%   Consider a fresh $\rgn_j$:
%   \begin{smathpar}
%   \begin{array}{cr}
%     \rgn_j \notin \rhoenv \cup dom(\rhomap) & H20\\
%   \end{array}
%   \end{smathpar}
%   From $H17$, $H19$, $H20$, and Lemma~\ref{thm:fb-renaming}, we have:
%   \begin{smathpar}
%   \begin{array}{cr}
%     \hastyp{(dom(\rhomap), \rhoenv \cup \{\rhoalloc\},\cdot,true),\rgn,\cdot}
%           {e_1}{T@\rhoalloc} & H19\\
%   \end{array}
%   \end{smathpar}
    
  \item Case ($e = e_0.\transfer\inang{\rgn}()$): Intros. Hypotheses:
  \begin{smathpar}
  \begin{array}{cr}
    \tywf{\rhoenv}{\phicx} & H1\\
    \frv(e)\subseteq dom(\rhomap) & H2\\
    \hastyp{\emptyA, \cdot}{e_0.\transfer\inang{\ralloc}()}{\tau} & H4\\
  \end{array}
  \end{smathpar}
  By inversion on $H4$:
  \begin{smathpar}
  \begin{array}{cr}
    \hastyp{\emptyA,\cdot}{e_0}{\RgnZ\inang{T}\inang{\toprgn}} & H6\\
  \end{array}
  \end{smathpar}
  Now, if $e_0$ can take a step, so can $e$, hence there is progress. Else, if $e_0$ raises an
  exception, so does $e$. The only non-trivial case is when $e_0$ is a value. But, only values of
  type $\RgnZ\inang{T}\inang{\toprgn}$ is $\C{new}\;\RgnZ\inang{T}\inang{\rgn_i}(...)$, where
  $\rgn_i \neq \toprgn$. From $H2$, we get:
  \begin{smathpar}
  \begin{array}{cr}
    \rgn_i \in dom(\rhomap) & H8\\
  \end{array}
  \end{smathpar}
  We have two cases:
  \begin{itemize}
    \item ($\rhomap(\rgn_i) \neq \OPEN$): In this case,
    $\redstoo{(e,\rhomap)}{(\unitval,\rhomap[\rgn_i \mapsto \XFERRED])}$.

    \item ($\rhomap(\rgn_i) = \OPEN$): In this case, evaluation of $e$ raises $\invalidexn$.
  \end{itemize}

  \item Case ($e$ is a lambda abstraction): $e$ is already a value.

  \item Case ($e = e_1;\,e_2$): Proof trivial.
% End of proof cases.
\end{itemize}

\qed
\end{proof}


\begin{theorem}
\emph{(\textbf{Preservation})}
\label{thm:fb-preservation}
$\forall e, \tau, \rhoenv, \rhomap, \phicx, \rgn$, such that $\rgn \in
\rhoenv$, if $\hastyp{\emptyA,\rgn, \cdot}{e}{\tau}$ and
$\redstoo{(e,\rhomap)}{(e',\rhomap')}$, then 
$\hastyp{\emptyASigp,\rgn, \cdot}{e'}{\tau}$.
\end{theorem}
\begin{proof}
Intros $e$. Induction on $e$. For every subexpressions $e_0$, inductive
hypothesis gives us the following:
\begin{smathpar}
\begin{array}{cr}
  \forall (\tau_0, \rhoenv_0, \rhomap_0, \phicx_0, \rgn_0). \spc 
  (\rgn_0 \in \rhoenv_0)
  \;\conj\; (\hastyp{(dom(\rhomap_0),\rhoenv_0,\cdot,\phicx_0),\rgn_0,
    \cdot}{e_0}{\tau_0}) \;\conj\; (\redstoo{(e_0,\rhomap_0)}
                {(e_0',\rhomap_0')}) & IH1\\
     \Rightarrow \hastyp{(dom(\rhomap_0'),\rhoenv_0,\cdot,\phicx_0),\rgn_0,
    \cdot}{e_0'}{\tau_0} & \\
\end{array}
\end{smathpar}
Cases from induction
\begin{itemize}
\item Case ($e = \unitval$ or $e = x$): Proof is trivial.
\item Case ($e = e_0.f_i$): Intros. Hypothesis:
  \begin{smathpar}
  \begin{array}{cr}
    \rgn \in \rhoenv & H2\\
    \hastyp{\emptyA,\rgn, \cdot}{e}{\tau} & H4\\
    \redstoo{(e,\rhomap)}{(e',\rhomap')} & H6\\
  \end{array}
  \end{smathpar}
  Inverting $H4$:
  \begin{smathpar}
  \begin{array}{cr}
    \hastyp{\emptyA,\rgn, \cdot}{e_0}{\tau'} & H7\\
    \bar{f} :\taubar = \fields(\bound_{\cdot}(\tau')) & H9\\
  \end{array}
  \end{smathpar}
  Inverting $H6$, we get two cases:
  \begin{itemize}
    \item SCase ($\redstoo{(e_0,\rhomap)}{(e_0',\rhomap')}$): In this
    case, $\redstoo{(e_0.f_i,\rhomap)}{(e_0'.f_i,\rhomap')}$. $H7$ and $IH1$ gives:
    \begin{smathpar}
    \begin{array}{cr}
      \hastyp{\emptyASigp,\rgn, \cdot}{e_0'}{\tau'} & H11\\
    \end{array}
    \end{smathpar}
    Proof follows from $H11$ and $H9$.

    \item SCase ($e_0$ is a value $\C{new} \; \fbN(\vbar)$):
    Hypotheses:
    \begin{smathpar}
    \begin{array}{cr}
      \allocRgn(N) \in \rhoenv & H13\\
      \redstoo{(e,\rhomap)}{(v_i,\rhomap)} & H15\\
    \end{array}
    \end{smathpar}
    \end{itemize}
    We need to prove that $\hastyp{(\emptyA,\rgn,\cdot)}{v_i}{\tau_i}$. 
    From $H7$, since $e_0 = \C{new} \; \fbN(\vbar)$:
    \begin{smathpar}
    \begin{array}{cr}
      \hastyp{\emptyA,\rgn, \cdot}{\C{new} \;\fbN(\vbar)}{\tau'} & H16\\
    \end{array}
    \end{smathpar}
    Inverting $H16$ and using $H9$ gives us the proof.

  \item Case ($e = \letregion{\rgn_0}{e_0}$): Intros. Hypothesis:
  \begin{smathpar}
  \begin{array}{cr}
    \rgn \in \rhoenv & H2\\
    \hastyp{\emptyA,\rgn, \cdot}{e}{\tau} & H4\\
    \redstoo{(e,\rhomap)}{(e',\rhomap')} & H6\\
  \end{array}
  \end{smathpar}
  Inverting $H4$:
  \begin{smathpar}
  \begin{array}{cr}
    \rgn_0 \notin \rhoenv & H7\\
    \tywf{\emptyA}{\tau} & H8\\
    \hastyp{\emptyADelcupPhicap{\rgn_0}{\rgn_0}, \rgn_0, \cdot}{e_0}{\tau} & H9\\
  \end{array}
  \end{smathpar}
  Inverting $H6$, we get two cases:
  \begin{itemize}
    \item SCase ($\redstocup{\rgn_0}{(e_0,\rhomap)}{(e_0',\rhomap')}$): In this
    case, $\redstoo{(\letregion{\rgn_0}{e_0},\rhomap)} {(\letregion{\rgn_0}{e_0'},\rhomap')}$. 
    $H9$ and $IH1$ gives:
    \begin{smathpar}
    \begin{array}{cr}
      \hastyp{\emptyADelcupPhicap{\rgn_0}{\rgn_0},\rgn,
          \cdot}{e_0'}{\tau} & H11\\
    \end{array}
    \end{smathpar}
    From $H7$ and $H11$, we can conclude that
    $\hastyp{\emptyASigp,\rgn, \cdot}{\letregion{\rgn_0}{e_0'}}{\tau}$.

    \item SCase ($e_0$ is a value $v_0$): In this case, $\redstoo{(\letregion{\rgn_0}{e_0},\rhomap)}
    {(v_0,\rhomap')}$. From $H2$, $H7-9$, and Lemma~\ref{thm:fb-tywf}, we have:
    \begin{smathpar}
    \begin{array}{cr}
      \hastyp{\emptyA, \rgn, \cdot}{v_0}{\tau} & H13\\
    \end{array}
    \end{smathpar}
    Thus, type is preserved. 
  \end{itemize}

  \item Case ($e = \open{e_a}{\rgn_0}{y}{e_b}$): Intros. Hypotheses:
  \begin{smathpar}
  \begin{array}{cr}
    \rgn \in \rhoenv & H2\\
    \hastyp{\emptyA,\rgn, \cdot}{e}{\tau} & H4\\
    \redstoo{(e,\rhomap)}{(e',\rhomap')} & H6\\
%   \forall (\tau, \rhoenv, \rhomap, \rgn). \rgn \in \rhoenv \conj
%     \hastyp{\emptyA,\rgn, \cdot}{e_0}{\tau} \;
%     \Rightarrow \; \exists(e',\rhomap'). \redstoo{(e_0,\rhomap)}
%                     {(e',\rhomap')} & IH1\\
  \end{array}
  \end{smathpar}
  Inverting $H4$:
  \begin{smathpar}
  \begin{array}{cr}
    \hastyp{\emptyA,\rgn, \cdot}{e_a}{\RgnZ\inang{T}\inang{\toprgn}} & H7\\
    \tywf{\emptyA}{\tau} & H8\\
    \rgn_0 \notin \rhoenv & H9\\
    \hastyp{\emptyASigpDelcup{\rgn_0},\rgn_0,[y \mapsto T@\rgn_0]}{e_b}{\tau} & H11\\
  \end{array}
  \end{smathpar}
  Inverting $H6$, we get many cases:
  \begin{itemize}
    \item SCase ($\redstoo{(e_a,\rhomap)}{(e_a',\rhomap')}$): Since the domain
    of $\rhomap$ monotonically increases during the evaluation, we have:
    \begin{smathpar}
    \begin{array}{cr}
      dom(\rhomap) \subseteq dom(\rhomap') & H13\\
    \end{array}
    \end{smathpar}
    Since strenthening the context trivially preserves typing and
    well-formedness, from $H7-11$ and $H13$, we have:
    \begin{smathpar}
    \begin{array}{cr}
      \hastyp{\emptyASigp,\rgn, \cdot}{e_a}{\RgnZ\inang{T}\inang{\toprgn}} & H15\\
      \tywf{\emptyASigp}{\tau} & H17\\
      \rgn_0 \notin \rhoenv & H19\\
      \hastyp{\emptyASigpDelcup{\rgn_0},\rgn_0,[y \mapsto
      T@\rgn_0]}{e_b}{\tau} & H20\\
    \end{array}
    \end{smathpar}
    From $H15-20$, we have $\hastyp{\emptyASigp,\rgn, \cdot}{e}{\tau}$.

    \item SCase ($e_a$ is a value $v_a$, and $e_b$ steps to $e_b'$): Hypotheses:
    \begin{smathpar}
    \begin{array}{cr}
      v_a = \C{new}\; \RgnZ\inang{T}\inang{\rgn_i}(v_r) & H22\\
      \rgn_i \neq \toprgn & H23\\
      \rhomap(\rgn_i) \neq \XFERRED & H24\\
      \rgn_0 \notin \rhoenv & H26\\
      \redstocup{\rgn_0}{([[\rgn_0/\rgn_i]v_r/y]e_b,
          \rhomap[\rgn_i \mapsto \OPEN])} {(e_b',\rhomap')} & H27\\
      \rhomap'' = \rhomap'[\rgn_i \mapsto \rhomap(\rgn_i)] & H29\\
    \end{array}
    \end{smathpar}
    We need to prove that
    $\hastyp{\emptyASigpp,\rgn,\cdot}{\open{v_a}{\rgn_0}{y}{ e_b'}}{\tau}$.  Note that $dom(\rhomap'') = dom(\rhomap')$. Hence, the proof
    obligation is
    $\hastyp{\emptyASigp,\rgn,\cdot}{\open{v_a}{\rgn_0}{y}{e_b'}}{\tau}$.
    First, since the domain of $\rhomap$ monotonically increases during the
    evaluation, we have:
    \begin{smathpar}
    \begin{array}{cr}
      dom(\rhomap) \subseteq dom(\rhomap') & H31\\
    \end{array}
    \end{smathpar}
    Next, since $e_a = v_a$, from $H7$ and $H22$, we have:
    \begin{smathpar}
    \begin{array}{cr}
      \hastyp{\emptyA,\rgn, \cdot}{\C{new}\;
        \RgnZ\inang{T}\inang{\rgn_i}(v_r)}
        {\RgnZ\inang{T}\inang{\toprgn}} & H33\\
    \end{array}
    \end{smathpar}
    Since $H23$, inversion on $H33$ gives:
    \begin{smathpar}
    \begin{array}{cr}
      \hastyp{\emptyADelcup{\rgn_i},\rgn_i,\cdot}{v_r}{T@\rgn_i} & H34\\
      \rgn_i \notin \rhoenv & H35\\
    \end{array}
    \end{smathpar}
    From $H23$, $H26$, $H34$, $H35$, and Lemma~\ref{thm:fb-renaming}, we have:
    \begin{smathpar}
    \begin{array}{cr}
      \hastyp{\emptyADelcup{\rgn_0},\rgn_0,\cdot}{[\rgn_0/\rgn_i]v_r}{T@\rgn_0} & H36\\
    \end{array}
    \end{smathpar}
    From $H11$, $H36$ and Lemma~\ref{thm:fb-substitution}, we get:
    \begin{smathpar}
    \begin{array}{cr}
      \hastyp{\emptyADelcup{\rgn_0},\rgn_0,
        \cdot}{[[\rgn_0/\rgn_i]v_r/y]e_b}{\tau} & H38\\
    \end{array}
    \end{smathpar}
    $H24$ says that $\rgn_i \in dom(\rhomap)$. Hence, from $H38$:
    \begin{smathpar}
    \begin{array}{cr}
      \hastyp{\emptyASigxDelcup{[\rgn_i \mapsto \OPEN]}{\rgn_0},
          \rgn_0, \cdot}{[[\rgn_0/\rgn_i]v_r/y]e_b}{\tau} & H40\\
    \end{array}
    \end{smathpar}
    From $H40$, $H27$ and $IH1$:
    \begin{smathpar}
    \begin{array}{cr}
      \hastyp{\emptyASigpDelcup{\rgn_0} ,\rgn_0, \cdot}{e_b'}{\tau} & H42\\
    \end{array}
    \end{smathpar}
%   From $H35$, $H26$, and $H42$:
%   \begin{smathpar}
%   \begin{array}{cr}
%     \hastyp{(dom(\rhomap'),\rhoenv \cup \{\rgn_i\},\cdot,true),\rgn_i,
%       \cdot}{[\rgn_i/\rgn_0]e_b'}{[\rgn_i/\rgn_0]\tau} & H44\\
%   \end{array}
%   \end{smathpar}
%   From $H8$ and $H26$, we know that $\rgn_0 \notin \frv(\tau)$. Hence:
%   \begin{smathpar}
%   \begin{array}{cr}
%     \hastyp{(dom(\rhomap'),\rhoenv \cup \{\rgn_i\},\cdot,true),\rgn_i,
%       \cdot}{[\rgn_i/\rgn_0]e_b'}{\tau} & H46\\
%   \end{array}
%   \end{smathpar}
    By strengthening the type context:
    \begin{smathpar}
    \begin{array}{cr}
      \hastyp{\emptyASigpDelcup{\rgn_0} ,\rgn_0,
        \cdot[y \mapsto T@\rgn_0]}{e_b'}{\tau} & H48\\
    \end{array}
    \end{smathpar}
    Since $dom(\rhomap) \subseteq dom(\rhomap')$ (from $H31$), we get the
    following by strengthening the context in $H7-11$:
    \begin{smathpar}
    \begin{array}{cr}
      \hastyp{\emptyASigpp,\rgn, \cdot}{v_a}
        {\RgnZ\inang{T}\inang{\toprgn}} & H50\\
      \tywf{\emptyASigpp}{\tau} & H51\\
    \end{array}
    \end{smathpar}
    From $H48$, $H50$, $H51$, we have the required goal:
    \begin{smathpar}
    \begin{array}{cr}
      \hastyp{\emptyASigp,\rgn,\cdot}{\open{v_a}{\rgn_0}{y}{e_b'}}{\tau}
    \end{array}
    \end{smathpar}
    
    \item SCase ($e_a$ is a value $v_a$, and $e_b$ is a value $v_b$): In this
    case, $\redstoo{(\open{v_a}{\rgn_0}{y}{v_b},\rhomap)} {(v_b,\rhomap)}$. 
    From $H11$:
    \begin{smathpar}
    \begin{array}{cr}
      \hastyp{\emptyADelcup{\rgn_0} ,\rgn_0, [y \mapsto T@\rgn_0]}{v_b}{\tau} & H53\\
    \end{array}
    \end{smathpar}
    Since $\valuee(v_b)$, it has not free variables. Consequently:
    \begin{smathpar}
    \begin{array}{cr}
      \hastyp{\emptyADelcup{\rgn_0} ,\rgn_0,
        \cdot}{v_b}{\tau} & H55\\
    \end{array}
    \end{smathpar}
%   Since $dom(\rhomap) \subseteq dom(\rhomap')$:
%   \begin{smathpar}
%   \begin{array}{cr}
%     \hastyp{(dom(\rhomap'),\rhoenv \cup \{\rgn_0\},\cdot,true),\rgn_0,
%       \cdot}{v_b}{\tau} & H57\\
%   \end{array}
%   \end{smathpar}
    From $H2$, $H8$, $H9$, $H55$ and Lemma~\ref{thm:fb-tywf}, we prove the
    required goal:
    \begin{smathpar}
    \begin{array}{cr}
      \hastyp{\emptyA,\rgn, \cdot}{v_b}{\tau} & \\
    \end{array}
    \end{smathpar}
  % End of cases for open   
  \end{itemize}

  \item Case ($e = \C{new}\; \RgnZT{\toprgn}(e_0)$): 
% Preservation follows trivially from the $IH$ when $e_0$ takes a step. The non-trivial case is when
% $e_0$ is a value $v_0$. 
  Hypotheses:
  \begin{smathpar}
  \begin{array}{cr}
    \rgn \in \rhoenv & H2\\
    \hastyp{\emptyA,\rgn, \cdot}{\C{new}\; \RgnZT{\toprgn}(e_0)}{\tau} & H4\\
    \redstoo{(e,\rhomap)}{(e',\rhomap')} & H6\\
  \end{array}
  \end{smathpar}
  Inverting $H4$, we know that $e_0 = \lambdaexp{\rgn}{\rhoalloc}{}{e_1}$, and:
  \begin{smathpar}
  \begin{array}{cr}
    \hastyp{\vacantA{\rhoalloc},\rhoalloc,\cdot}{e_1} {T@\rhoalloc}& H7\\
%   \hastyp{\emptyA,\rgn, \cdot}{v_0}{\inang{\rhoalloc}\unitZ
%       \xrightarrow{\rgn} T@\rhoalloc} & H7\\
    \fgjtywf{\cdot}{T} & H8\\
%   \rgn_0 \notin \rhoenv & H9\\
%   \hastyp{(dom(\rhomap),\rhoenv \cup \{\rgn_0\},\cdot,true),\rgn_0,[y \mapsto
%   T@\rgn_0]}{e_b}{\tau} & H11\\
  \end{array}
  \end{smathpar}
  From $H7$ and Lemma~\ref{thm:fb-renaming}:
  \begin{smathpar}
  \begin{array}{cr}
    \hastyp{\vacantA{\rgn_i},\rgn_i,\cdot}{[\rgn_i/\rhoalloc]e_1} {T@\rgn_i}& H10\\
  \end{array}
  \end{smathpar}
% Inverting $H7$, we know that $v_0 = \lambdaexp{\rgn}{\rhoalloc}{}{e_1}$, and
% we get the following hypotheses:
% \begin{smathpar}
% \begin{array}{cr}
%   \rhoalloc \notin \rhoenv & H13\\
%   \hastyp{\emptyADelcup{\rhoalloc},\rhoalloc,\cdot}{e}{T@\rhoalloc} & H15\\
% \end{array}
% \end{smathpar}
  Inverting $H6$:
  \begin{smathpar}
  \begin{array}{cr}
    \rgn_i \notin dom(\rhomap) \cup \rhoenv & H17\\
    \rhomap' = \rhomap[\rgn_i \mapsto \CLOSED] & H19\\
    \redstoo{(\C{new}\; \RgnZT{\toprgn}(\lambdaexp{\rgn}{\rhoalloc}{}{e_1}))}
      {(\C{new}\; \RgnZT{\rgn_i}([\rgn_i/\rhoalloc]e_1),\rhomap')} & H21\\
  \end{array}
  \end{smathpar}
  Since strengthening the context preserves typing, strengthening the context for type judgment in
  $H10$ gives us the following:
  \begin{smathpar}
  \begin{array}{cr}
    \hastyp{\emptyASigpDelcup{\rgn_i},\rgn_i,\cdot}{[\rgn_i/\rhoalloc]e}
        {T@\rgn_i} & H24\\
  \end{array}
  \end{smathpar}
% From $H13$, $H17$, $H15$ and Lemma~\ref{thm:fb-renaming}, we get:
% \begin{smathpar}
% \begin{array}{cr}
%   \hastyp{\emptyADelcup{\rgn_i},\rgn_i,\cdot}{[\rgn_i/\rhoalloc]e}
%       {T@\rgn_i} & H23\\
% \end{array}
% \end{smathpar}
% Since $dom(\rhomap) \subseteq dom(\rhomap')$, $H23$ is equivalent to:
% \begin{smathpar}
% \begin{array}{cr}
%   \hastyp{\emptyASigpDelcup{\rgn_i},\rgn_i,\cdot}{[\rgn_i/\rhoalloc]e}
%       {T@\rgn_i} & H24\\
% \end{array}
% \end{smathpar}
  $H19$ implies $\rgn_i \in dom(\rhomap')$. This, and $H17$, $H8$, and $H24$ entail the required goal:
  \begin{smathpar}
  \begin{array}{cr}
    \hastyp{\emptyASigp,\rgn, \cdot}{\C{new}\; \RgnZT{\rgn_i}
        ([\rgn_i/\rhoalloc]e_1)}{\RgnZT{\toprgn}} & \\
  \end{array}
  \end{smathpar}

  \item Case ($e = \C{new} \RgnZT{\rgn_i}(e_0)$, where $\rgn_i \neq \toprgn$):Hypotheses:
  \begin{smathpar}
  \begin{array}{cr}
    \rgn \in \rhoenv & H2\\
    \hastyp{\emptyA,\rgn, \cdot}{\C{new}\; \RgnZT{\rgn_i}(e_0)}{\tau} & H4\\
    \redstoo{(\C{new} \RgnZT{\rgn_i}(e_0),\rhomap)}{(e',\rhomap')} & H6\\
  \end{array}
  \end{smathpar}
  Since $e$ isn't a value,  inverting $H6$ tells us that $\C{new}
  \RgnZT{\rgn_i}(e_0)$ takes a step to $\C{new} \RgnZT{\rgn_i}(e_0')$ when $e_0$
  takes a step to $e_0'$.  The proof for this case is similar to the previous
  case; we invert $H4$ and $H6$, apply inductive hypothesis to derive typing
  judgment for $e_0$ under a context containing $\rgn_i$, and finally apply the
  type rule for $\C{new} \RgnZT{\rgn_i}(e_0')$ (where $\rgn_i
  \neq \toprgn$) to prove the preservation.

  \item Case ($e$ is a lambda expression): $e$ is a value, hence cannot take a
  step. Preservation trivially holds.

  \item Case ($e$ is a method/function call, or a \C{let} expression): Proof
  follows directly from the inductive hypothesis, substitution lemma
  (\ref{thm:fb-substitution}) and renaming lemma (\ref{thm:fb-renaming}).

%End of cases for proof
\end{itemize}

\qed
\end{proof}


\begin{theorem}
\emph{(\textbf{Type Safety})}
\label{thm:fb-type-safety}
$\forall e, \tau, \rhoenv, \rhomap$, if $\tywf{\Delta}{\phicx}$ and 
$\frv(e) \subseteq dom(\Sigma)$ and
$\hastyp{\emptyA,\cdot}{e}{\tau}$, then one of the following holds:\\
  \begin{smathpar}
  \begin{array}{rl}
    (i) & \exists (e',\rhomap').~\redstoo{(e,\rhomap)}{(e',\rhomap')}
        \conj \frv(e') \subseteq dom(\Sigma')\\
    (ii) & \valuee(e)\\
    (iii) & \redstoo{(e,\rhomap)}{\invalidexn}\\
%   (i) & e is a value
%   (i) & \exists (e',\rhomap').\;\mathit{such\; that}\; 
%     \redstoo{(e,\rhomap)}{(e',\rhomap')}
%     \;\mathit{and}\; \hastyp{\emptyASigp,\rgn, \cdot}{e}{\tau}\\
%   (ii) & \redstoo{(e,\rhomap)}{\invalidexn}\\
  \end{array}
  \end{smathpar}
\end{theorem}
\begin{proof}
  Directly follows from Theorems~\ref{thm:fb-progress}
  and~\ref{thm:fb-preservation}.
\end{proof}

\twocolumn
