\section{\fbname}
\label{sec:type-system}

The purpose of \name's region type system is to enforce the key
invariant required for memory safety, namely that an object $o_1$ in a
region $R_1$ contains a reference to an object $o_2$ in $R_2$, only if
$R_2$ is guaranteed to outlive $R_1$.  We now formally develop this
intuition via \fbname (\FB), our explicitly typed core language (with
region types) that incorporates the features introduced in the
previous section. \fbname builds on the Featherweight Generic Java
(FGJ)~\cite{fgj} formalism, and reuses notations and various
definitions from~\cite{fgj}, such as the definition of type
well-formedness for the core (region-free) language.  (The language
used in Section~\ref{sec:overview} is essentially a version of FGJ
without region types and with some syntactic sugar.)

% In this section, we formally develop \fbname, a language that
% incorporates the features introduced in the previous section, as well
% as its type system, building on the Featherweight Generic Java
% (FGJ)~\cite{fgj} formalism.  Our development reuses notations and
% various definitions from~\cite{fgj}, such as the definition of type
% well-formedness for the core (region-free) language, which are
% elaborated further in the appendix~\cite{techrep}.

\subsection{Syntax}
\label{sec:fb-syntax}

Fig~\ref{fig:fb-syntax} describes the syntax of \FB.
We refer to the class types of FGJ as \emph{core types}.
%
The following definition of \C{Pair} class in \FB illustrates some of
the key elements of the formal language (the symbol $\extends$ should
be read \emph{extends}, and the symbol $\outlives$ stands for
\emph{outlives}):
\begin{codejava}[mathescape=true]
class Pair<a $\extends$ Object, b $\extends$ Object>
          <$\rho^a, \rho_1, \rho_2 \,|\, \rho_1 \outlives \rho^a \conj \rho_2 \outlives \rho^a$> $\extends$ Object<$\rho^a$> {
  a@$\rho_1$ fst; 
  b@$\rho_2$ snd;
  a@$\rho_1$ getFst() { return this.fst; }
}
\end{codejava}
% \vspace*{-0.15in}
%  Pair(a@$\rho_1$ fst, b@$\rho_2$ snd) {
%    super(); this.fst = fst; this.snd = snd;
%  }
Note that we elide showing constructors since they are uninteresting
from the type system's standpoint; their behavior in \FB is same as in
FGJ. 

A class in \FB is parametric over zero or more type variables (as
in FGJ) as well as one or more region variables $\rho$.
% In the above example, the class \C{Pair} is parameterized over
% $\rho^a, \rho_1, \rho_2$.
We refer to the first region parameter, usually denoted $\rho^a$, as
the \emph{allocation region} of the class: it serves to identify the
region where an instance of the class is allocated.
%
An object in \FB can contain fields referring to objects allocated in regions ($\rhobar$) other than its
own allocation region ($\rhoalloc$), provided that the former outlive
the latter (i.e., $\rhobar \outlives \rhoalloc$). In such case, the
definition of object's class needs to be parametric over allocation
regions of its fields (i.e., their classes). Furthermore, the
constraint that such regions must outlive the allocation region of the
class needs to be made explicit in the definition, as the \C{Pair}
class does in the above definition. We say that the \C{Pair} class
exhibits \emph{constrained region polymorphism}.
%
% For instance, a \C{Pair} object allocated in a region ${\rhoalloc}$ can have its first
% component coming from ${\rho_1}$ and second from $\rho_2$, as long as
% $\rho_1, \rho_2$ outlive $\rhoalloc$, as illustrated by the above definition.

To construct objects of the \C{Pair} class, its type and region
parameters need to be instantiated with core types ($T$) and region
annotations\footnote{Region annotations ($\rgn$) include region
variables ($\rho$) and region identifiers ($\pi$). Region identifiers
are to region variables, as types ($T$) are to type variables ($a$)}
($\rgn$), respectively. For example:
\begin{codejava}
letregion $\pi_0$ in
  let snd = new Object<$\pi_0$>() in
  letregion $\pi_1$ in
    let fst = new Object<$\pi_1$>() in
    let p = new Pair<Object,Object><$\pi_1$,$\pi_1$,$\pi_0$> (fst,snd);
\end{codejava}
%\vspace*{-0.1in}
In the above code, the instantiation of $\rhoalloc$ and $\rho_1$ with
$\pi_1$, and $\rho_2$ with $\pi_0$ is allowed because (a) $\pi_0$
and $\pi_1$ are live during the instantiation, and (b). $\pi_0
\outlives \pi_1$ and $\pi_1 \outlives \pi_1$ (since outlives is
reflexive). Observe that the region type of \C{p} conveys
the fact that (a). it is allocated in region $\pi_1$, and (b). it
holds references to objects allocated in region $\pi_0$ and $\pi_1$.
In contrast, if we choose to allocate the \C{snd} object also in
$\pi_1$, then \C{p} would be contained in $\pi_1$, and its region
type would be \C{Pair<\ObjZ,\ObjZ><$\pi_1$,$\pi_1$,$\pi_1$>}, which
we abbreviate as \C{Pair<\ObjZ,\ObjZ>@$\pi_1$}. In general, we treat
$B\inang{\tbar}@\pi$ as being equivalent to
$B\inang{\tbar}\inang{\rbar}$. Region annotation on type $a$, where
$a$ is a type variable, assumes the form $a@\pi$. If $a$ is
instantiated with \C{Pair<\ObjZ,\ObjZ>}, the result is the type of a
\C{Pair} object contained in  $\pi$. Class fields are not allowed to
be type polymorphic, in keeping with C\#, which constitutes the core
of \name. Separate parameterization over types and regions is needed
in \FB to support region-polymorphic higher-order functions as class
fields. For such higher-order class fields, regions are parameterized
at the (class) field level whereas types are parameterized at the
class level. This allows, for example, a generic class, whose type
parameters are $a$ and $b$, to contain a region-polymorphic function of type
$\inang{\rho_0,\rho_1,\rho_2 \,|\, \rho_1 \outlives \rho_0 \conj
\rho_2 \outlives \rho_0}(a@\rho_1,b@\rho_2) \rightarrow
\C{Pair}\inang{a,b}\inang{\rho_0,\rho_1,\rho_2}$ as its field. Such
region-polymorphic higher-order fields are used frequently by the
dataflow operators, which apply them in the context of various regions
(\emph{e.g.,} see Fig.~\ref{fig:motivating-eg-in-broom}). Keeping type
parameterization seperate from region parameterization allows \FB to
type such examples. This choice also simplifies type inference in the
presence of higher-order functions.


% The region type of $\ObjZ$ class in \FB has no fields, hence its
% region type is of form $\ObjZ@\rgn$. 
%Class definitions in \FB extend FGJ's with region parameters
%($\rho$). Informally, region variables ($\rho$) are to region
%identifiers ($\rgn$) as type variables ($a$) are to core types ($T$);
%they can be instantiated with region identifiers, as needed. A class
%in \FB is necessarily parameterized over the allocation region of its
%objects. We use $\rho^a$ to denote this parameter, and often refer to
%it as \emph{class}'s allocation region, although it is really the
%allocation region of its objects. An Object in \FB can contain fields
%referring to objects allocated in regions ($\rhobar$) other than its
%own allocation region ($\rhoalloc$), provided that the former outlive
%the later (i.e., $\rhobar \outlives \rho$). In such cases, the
%definition of object's class needs to be parametric over allocation
%regions of its fields (i.e., their classes). Furthermore, the
%constraint that such regions must outlive the allocation region of the
%class needs to be made explicit in the definition. For instance, a
%\C{Pair} object allocated in a region ${\rhoalloc}$ can have its first
%component coming from ${\rho_1}$ and second from $\rho_2$, where all
%the $\rho$'s are distinct, and $\rho_1, \rho_2$ outlive
%$\rhoalloc$. T

% Class types in \FB are region-annotated variants of class types in
% FGJ (also called \emph{core types}). This correspondence is
% reflected in the $T@\rgn$ syntax of a region type, which is the
% simplest form of a region type describing an object of core type $T$
% contained in a region $\rgn$. We let $\rgn$ range over static
% identifiers of regions in \FB. The only unboxed value in \FB is
% $\unitval$ of type \unitZ.  Rest are objects.

% \FB extends FGJ's expression language with local variable
% declarations (via \C{let} expressions), region constructs
% (\C{letregion} and \C{open}), lambda abstraction
% ($\lambdaexp{...}{\xbar : \taubar}{e}$) and application
% ($e\inang{...}(\bar{e})$) expressions. Lambda abstraction and
% application expressions are to define and apply anonymous functions.
% Modulo the angle braces (whose presence will be justified later),
% these expressions are uncurried variants of the corresponding
% expressions from System F.  Since FGJ formalism does not include
% higher-order functions, we extend the syntactic class (\C{T}) of FGJ
% types with an (uncurried) arrow type.


% In general, the \C{@} notation in a
% region type of an object \C{x} highlights that
% \C{x}, and all the objects reachable from \C{x} via references are
% allocated in a single region. We say that \C{x} is \emph{contained} in
% the region. The $\ObjZ$ class in \FB contains no references to other
% objects, hence its objects are always contained in their allocation
% region. Their region type is $\ObjZ\inang{\rgn}$ (or equivalently,
% $\ObjZ@\rgn$), for some region $\rgn$.

% Note that when a generic class definition in FGJ is lifted to a
% region-polymorphic definition in \FB, it nonetheless remains generic
% with respect to core types. We say that the class is now a
% region-polymorphic generic class. Region parameterization of a class
% in \FB is independent of its parameterization over core types; they
% are not conflated to make the class parametric over region types. For
% instance, following is the region-polymorphic definition of a generic
% \C{Pair} class (The symbol $\extends$ should be read \emph{extends}):

% When objects of a class are allocated in a region $\rgn$, it means
% that the class's constructor is run with $\rgn$ as its allocation
% context. Every class definition in \FB is necessarily polymorphic with
% respect to the allocation region of its objects, i.e., the
% allocation
% context of its constructor. We adopt a convention that requires the
% allocation region parameter (denoted $\rho^a$) to be the first region
% parameter of a class.  Besides the allocation region of its objects,
% \C{Pair} class is also parametric over the regions its first and
% second elements are allocated in. References between objects allocated
% in different regions are only allowed if the referred object is
% guaranteed to outlive the referring object. In case of \C{Pair} class,
% this means that allocation regions ($\rho_1$ and $\rho_2$) of both
% objects that make up the pair must outlive the allocation region
% ($\rho^a$) of the \C{Pair} object. Such conditions over region
% parameters of a class need to be recorded in its header as region
% constraints ($\phi$) in order for the class to be judged well-formed
% by the type system (Fig.~\ref{fig:fb-morewfrules}). 

%%An object \emph{allocated} in a region $\rgn$ contains its spine in
%%$\rgn$, but can refer to objects allocated in other regions.  On the
%%other hand,

% \FB's $\RgnZ$ objects, like $\ObjZ$ objects, have region type of form
% $\RgnZ\inang{\rgn}$. However, unlike the $\rgn$ in
% $\ObjZ\inang{\rgn}$, $\rgn$ in $\RgnZ\inang{\rgn}$ cannot be any
% region. Recall that $\RgnZ$ objects have special semantics in \name -
% they act as handlers to transferable regions.  Constructing a new
% $\RgnZ$ object entails the creation of a new transferable region, and
% it is in this region that the new object is allocated in. It follows
% that $\rgn$ in $\RgnZ\inang{\rgn}$ should be the static identifier of
% the new transferable region. But, static identifiers for transferable
% regions are introduced only when such regions are opened for
% allocation via \C{open} expression. What, then, should be the region
% type of a \C{new Region} expression?

% \FB resolves this problem by existentially quantifying the allocation
% region of $\RgnZ$ objects when they are created. In other words, the
% type of \C{new Region} expression in \FB is
% $\exists\rho.\RgnZ\inang{\rho}$. Existential quantification in the
% type captures the fact that there now exists a transferable region
% containing the newly constructed $\RgnZ$ object. Elimination of
% existential quantification is facilitated by the \C{open} expression,
% which opens the transferable region and assigns it an identifier
% ($\rgn$).  Within the scope of \C{open}, the transferable region is
% identified with $\rgn$, allowing its handler to instantiate the
% existentially bound region variable with $\rgn$, and assume the type
% of $\RgnZ\inang{\rgn}$.

\begin{figure*}[t]
%
\textbf{Auxiliary Definitions} \\
\begin{minipage}{1.75in}
\begin{smathpar}
\begin{array}{lcl}
  allocRgn(A\inang{\rhoalloc\rhobar}\inang{\tbar}) & = & \rhoalloc\\
  bound_{\Delta}(\alpha) & = & \Delta(\alpha)\\
  bound_{\Delta}(N) & = & N\\
\end{array}
\end{smathpar}
\end{minipage}
%
\begin{minipage}{1.8in}
\begin{smathpar}
\begin{array}{c}
\renewcommand*{\arraystretch}{1.2}
\RULE
  {
    \\
    B \in \{\ObjZ,\RgnZ\}
  }
  {
    fields(B\inang{\ralloc\rbar}\inang{\tbar}) \;=\; \bullet
  }
\end{array}
\end{smathpar}
\end{minipage}
%
\begin{minipage}{3in}
\begin{smathpar}
\begin{array}{c}
\renewcommand*{\arraystretch}{1.2}
\RULE
  {
    CT(B) = \headerOf{B}\{\bar{\tau^f}\;\bar{f};\,...\}\\
    \substFn = [\rbar/\rhobar, \ralloc/\rhoalloc, \tbar/\bar{\alpha}] \qquad 
    fields(\substFn(N)) = \bar{\tau^g}\;\bar{g}
  }
  {
    fields(B\inang{\ralloc\rbar}\inang{\tbar}) \;=\;
      \bar{\tau^g}\,\bar{g},\,\substFn(\bar{\tau^f})\,\bar{f}
  }
\end{array}
\end{smathpar}
\end{minipage}
%
\bigskip

\begin{minipage}{3.75in}
\begin{smathpar}
\begin{array}{c}
\renewcommand*{\arraystretch}{1.2}
\RULE
  {
    CT(B) = \headerOf{B}\{\bar{\tau^f}\;\bar{f};\,k\;\bar{d}\}\\
    \tau^2 \; m\mang (\bar{\tau^1}\;\bar{x})\{...\} \in \bar{d} \qquad
    \substFn = [\rbar/\rhobar, \ralloc/\rhoalloc, \tbar/\bar{\alpha}]
  }
  {
    mtype (m,B\inang{\ralloc\rbar}\inang{\tbar}) \;=\;
    \substFn(\mang\bar{\tau^1} \rightarrow \tau^2)
  }
\end{array}
\end{smathpar}
\end{minipage}
%
\begin{minipage}{3in}
\begin{smathpar}
\begin{array}{c}
\renewcommand*{\arraystretch}{1.2}
\RULE
  {
    CT(B) = \headerOf{B}\{\bar{\tau^f}\;\bar{f};\,k\;\bar{d}\}\\
    m \notin \bar{d} \qquad 
    \substFn = [\rbar/\rhobar, \ralloc/\rhoalloc, \tbar/\bar{\alpha}]
  }
  {
    mtype (m,B\inang{\ralloc\rbar}\inang{\tbar}) \;=\;
    mtype (m, \substFn(N))
  }
\end{array}
\end{smathpar}
\end{minipage}
%
\bigskip
\begin{smathpar}
\A \;=\; (\subtypcx)
\end{smathpar}
%
\bigskip

\textbf{Subtyping}  \; \fbox
  {\(\subtyp{\A}{\tau_1}{\tau_2}\)}\\

%
\begin{minipage}{1in}
\begin{smathpar}
\begin{array}{c}
\renewcommand*{\arraystretch}{1.2}
\RULE
  {
    \\
  }
  {
    \subtyp{\A}{\tau}{\tau}
  }
\end{array}
\end{smathpar}
\end{minipage}
%
\begin{minipage}{1.2in}
\begin{smathpar}
\begin{array}{c}
\renewcommand*{\arraystretch}{1.2}
\RULE
  {
    \\
  }
  {
    \subtyp{\A}{\tyvar @\rho}{\aenv(\tyvar) @\rho}
  }
\end{array}
\end{smathpar}
\end{minipage}
%
\begin{minipage}{2in}
\begin{smathpar}
\begin{array}{c}
\renewcommand*{\arraystretch}{1.2}
\RULE
  {
    \subtyp{\A}{\tau_1}{\tau_2}\qquad
    \subtyp{\A}{\tau_2}{\tau_3}
  }
  {
    \subtyp{\A}{\tau_1}{\tau_3}
  }
\end{array}
\end{smathpar}
\end{minipage}
%
\begin{minipage}{1.5in}
\begin{smathpar}
\begin{array}{c}
\renewcommand*{\arraystretch}{1.2}
\RULE
  {
    \\
  }
  {
    \subtyp{\A}{\exists\rho.\tau}{\exists\rho'.[\rho'/\rho]\tau}
  }
\end{array}
\end{smathpar}
\end{minipage}
%
\bigskip
\begin{minipage}{1.5in}
\begin{smathpar}
\begin{array}{c}
\renewcommand*{\arraystretch}{1.2}
\RULE
  { }
  { }
\end{array}
\end{smathpar}
\end{minipage}
%
\begin{minipage}{3.5in}
\begin{smathpar}
\begin{array}{c}
\renewcommand*{\arraystretch}{1.2}
\RULE
  {
    CT(B) = \headerOf{B}\{\bar{\tau^f}\;\bar{f};\,k\;\bar{d}\}\\
    \tywf{\A}{B\inang{\pi^a\bar{\pi}\inang{\taubar}}}\qquad
    \substFn = [\rbar/\rhobar, \ralloc/\rhoalloc, \tbar/\bar{\tyvar}] \qquad 
    \tywf{\A}{\substFn(\fbN)}
    \
  }
  {
    \subtyp{\A}{B\inang{\pi^a\bar{\pi}\inang{\taubar}}}{\substFn(\fbN)}
  }
\end{array}
\end{smathpar}
\end{minipage}
%

%
\bigskip

\textbf{Well-formedness}  \; \fbox
  {\(\tywf{\A}{\tau}, \spc 
     \tywf{\A}{\phi}\)}\\

%
\begin{minipage}{1.25in}
\begin{smathpar}
\begin{array}{c}
\renewcommand*{\arraystretch}{1.2}
\RULE
  {
    \\
    \\
    \rgn \in \A.\rhoenv
  }
  {
    \tywf{\A}{\ObjZ\inang{\rgn}}
  }
\end{array}
\end{smathpar}
\end{minipage}
% %
% \begin{minipage}{1.5in}
% \begin{smathpar}
% \begin{array}{c}
% \renewcommand*{\arraystretch}{1.2}
% \RULE
%   {
%     \rgn \in \A.\rhoenv\\
%     \fgjtywf{\A.\aenv}{\RgnZ\inang{T}}
%   }
%   {
%     \tywf{\A}{\RgnZ\inang{\rgn}\inang{T}}
%   }
% \end{array}
% \end{smathpar}
% \end{minipage}
% %
% \begin{minipage}{1.5in}
% \begin{smathpar}
% \begin{array}{c}
% \renewcommand*{\arraystretch}{1.2}
% \RULE
%   {
%     \fgjtywf{\A.\aenv}{\RgnZ\inang{T}}
%   }
%   {
%     \tywf{\A}{\exists\rho.\RgnZ\inang{\rho}\inang{T}}
%   }
% \end{array}
% \end{smathpar}
% \end{minipage}
%
\begin{minipage}{2.75in}
\begin{smathpar}
\begin{array}{c}
\renewcommand*{\arraystretch}{1.2}
\RULE
  {
    \rgn \in \A.\rhoenv \spc
    \{\rhoalloc,\rhobar\} \notin \A.\rhoenv \\
    \A' = (\A.\rhoset, \A.\rhoenv \cup \{\rhoalloc,\rhobar\}, 
           \A.\aenv, \A.\phicx \conj \phi) \\
    \tywf{\A'.\rhoenv}{\phi}\spc 
    \tywf{\A'}{\bar{\tau^1}} \spc
    \tywf{\A'}{\tau^2}
  }
  {
    \tywf{\A}{\inang{\rhoalloc\rhobar \,|\, \phi}
              \bar{\tau^1} \xrightarrow{\rgn} \tau^2}
  }
\end{array}
\end{smathpar}
\end{minipage}
%
\begin{minipage}{1.5in}
\begin{smathpar}
\begin{array}{c}
\renewcommand*{\arraystretch}{1.2}
\RULE
  { 
    \\
    \\
    \fgjtywf{\A.\aenv}{T}
  }
  {
    \tywf{\A}{\RgnZ\inang{T}\inang{\rgn}}
  }
\end{array}
\end{smathpar}
\end{minipage}
%
\begin{minipage}{1in}
\begin{smathpar}
\begin{array}{c}
\renewcommand*{\arraystretch}{1.2}
\RULE
  {
    \\
    \\
    \rho_0,\rho_1 \in \rhoenv
  }
  {
    \tywf{\rhoenv}{\rho_0 \outlives \rho_1}
  }
\end{array}
\end{smathpar}
\end{minipage}
%

%
\begin{minipage}{3.5in}
\begin{smathpar}
\begin{array}{c}
\renewcommand*{\arraystretch}{1.2}
\RULE
  {
    CT(B) = \headerOf{B}\{...\}\spc
    \ralloc,\rbar \in \A.\rhoenv \\
    \fgjtywf{\A.\aenv}{B\inang{\tbar}}\spc
    \substFn = [\rbar/\rhobar, \ralloc/\rhoalloc, \tbar/\bar{\tyvar}] \spc
    \isvalid{\A.\phicx}{\substFn(\phi)}
  }
  {
    \tywf{\A}{B\inang{\ralloc\rbar}\inang{\tbar}}
  }
\end{array}
\end{smathpar}
\end{minipage}
%
\begin{minipage}{1.65in}
\begin{smathpar}
\begin{array}{c}
\renewcommand*{\arraystretch}{1.2}
\RULE
  {
    \fgjtywf{\A.\aenv}{T}\spc
    \ralloc \in \A.\rhoenv\\
    \fgjsubtyp{\A.\aenv}{T}{\ObjZ}\spc
  }
  {
    \tywf{\A}{T@\ralloc}
  }
\end{array}
\end{smathpar}
\end{minipage}
%
\begin{minipage}{0.75in}
\begin{smathpar}
\begin{array}{c}
\renewcommand*{\arraystretch}{1.2}
\RULE
  {
    \tywf{\rhoenv}{\phi_0} \\ \tywf{\rhoenv}{\phi_1}
  }
  {
    \tywf{\rhoenv}{\phi_0 \wedge \phi_1}
  }
\end{array}
\end{smathpar}
\end{minipage}

%
\bigskip
\caption{Static semantics of {\sc Featherweight} \name}
\label{fig:fb-syntax}
\end{figure*}


Like classes, methods can also exhibit constrained region
polymorphism.  A method definition in \FB is necessarily polymorphic
over its allocation context (\S~\ref{sec:alloc-ctxt}), and
optionally polymorphic with respect to the regions containing its
arguments.  Region parameters, like those on classes, are qualified
with constraints ($\phi$).
% Region parameters on the methods, like those on classes, are
% accompanied by constraints ($\phi$) capturing the conditions that the
% parameters need to satisfy for the method to be considered well-formed
% (Fig.~\ref{fig:fb-morewfrules}). Allocation context (usually $\rho^a$
% or $\rho^a_m$) is the first and inevitable region parameter of every
% method in \FB. 
If a method is not intended to be polymorphic with respect to its
allocation context (for example, if its allocation context needs to be
same as the allocation region of its \emph{this} argument), then the
required monomorphism can be captured as an equality constraint in
$\phi$.  

% Like methods, functions can also be region-polymorphic. 
% A lambda expression defines a region-polymorphic
% multi-argument function closure parameterized over function's
% allocation context parameter. 
\FB extends FGJ's expression language with a lambda expression and an
application expression ($e\inang{\rbar}(\bar{e})$) to define
and apply functions (lambdas). Functions, like methods, exhibit
constrained region polymorphism, as evident in their arrow region type
($\inang{\rhobar \,|\, \phi}\bar{\tau} \xrightarrow{\rgn}
\tau$).
% at the prenex position, similar to ML type schemes. However,
% unlike ML, we don't distinguish between types and type schemes; any of
% the $\tau$'s in the arrow type can themselves be region-parametric
% arrow types. Like with methods, region parameters on functions can be
% instantiated when they are applied.
% In this respect, our region type system is more like
% System F's type system, which admits higher-rank parametric
% polymorphism. Like System F, \FB provides a region instantiation
% expression ($e\inang{\ralloc\rbar}$), and a region generalization
% expression ($\Lambdaexp{\rhoalloc\rhobar \,|\, \phi}{e}$) to
% instantiate and generalize region variables. 
Since a function closure can escape the context in which it is
created, it is important to keep track of the region in which it is
created in order to avoid unsafe dereferences. The $\rgn$ annotation
above the arrow in the arrow type denotes the allocation region of the
corresponding closure. 
% Note that it is important to distinguish
% between the allocation context argument ($\rhoalloc$) of a function
% and the allocation region ($\rgn$) of its closure. In \name, the later
% corresponds to the region where a \C{Func} object is allocated, while
% the former corresponds to the region where it is applied
% For instance, in the following example:
% \begin{codejava}
% letregion $\rgn$ {
%   let f = $\lambda{\inang{\rho^a}}$().$\,$new Object$\inang{\rho^a}$() 
%   in f
% }
% \end{codejava}
% The type of \C{f} is $\inang{\rho^a}\unitZ \xrightarrow{\rgn}
% \ObjZ\inang{\rho^a}$, coveying that (a). \C{f}'s closure is allocated
% in $\rgn$, and (b). when executed under an allocation context
% $\rho^a$, the closure returns an object allocated in $\rho^a$.

\subsection{Types and Well-formedness}

Well-formedness and typing rules of \fbname establish the conditions
under which a region type is considered well-formed, and an expression
is considered to have a certain region type, respectively.
Fig.~\ref{fig:fb-staticsem} contains an illustrative subset of such
rules\footnote{Full formal development can be found in the
appendix}. The rules refer to a context ($\A$), which is a tuple of:
\begin{itemize}
\item A set ($\rhoenv \in 2^\rgn$) of regions that are estimated to be
live,
\item A finite map ($\aenv \in \tyvar \mapsto \fgjN$) of type
variables to their \emph{bounds}, i.e., classes they are declared to
extend (this artifact is inherited from FGJ), and
\item A constraint formula ($\phicx$) that captures the outlives
constraints on regions in $\rhoenv$.
\end{itemize}
In addition, the context for the expression typing judgment also
includes a type environment ($\env \in x \mapsto \tau$) that contains
the type bindings for variables in scope. Like the judgments in
FGJ~\cite{fgj}, all the judgments defined by the rules in
Fig.~\ref{fig:fb-staticsem} are implicitly parameterized on a class
table ($CT \in cn \mapsto D$) that maps class names to their
definitions in \FB.

\noindent The well-formedness judgment on region types
($\tywf{\A}{\tau}$) makes use of the well-formedness and subtyping
judgments on core types. We use a double-piped turnstile ($\Vdash$)
for judgments in FGJ~\cite{fgj}, and a simple turnstile ($\vdash$) for
those in \FB. The class table ($\absof{CT}$) for FGJ judgments is
derived from \FB's class table ($CT$) by erasing all region
annotations on types, and region arguments in expressions
($\absof{\cdot}$ denotes the region erasure operation). 
% For judgments in FGJ, we use a
% class table that maps class names to their region-erased definitions.
% We denote the region-erased class table as ($\absof{CT}$). The precise
% semantics of region erasure ($\absof{\cdot}$) are defined
% in~\cite{techrep}, but it can be understood as an operation that
% erases region annotations on types. The class table is assumed to bind
% the name $\RgnZ$ to the stub shown in Fig.~\ref{fig:region-stub}.
% This mostly eliminates the need to treat $\RgnZ$ objects, which double
% up as transferable region handlers, as a special case in semantics.
The well-formedness rule for class types
($B\inang{\tbar}\inang{\rbar}$) is responsible for enforcing the
safety property that prevents objects from containing unsafe
references. It does so by insisting that regions $\rbar$ satisfy the
constraints ($\phi$) imposed by the class on its region parameters.
The later is enforced by checking the validity of $\phi$, with actual
region arguments substituted for formal region parameters, under the
conditions ($\phicx$) guaranteed by the context. The semantics of this
sequent is straightforward, and follows directly from the properties
of outlives and equality relations. For any well-formed core type $T$,
$T@\rgn$ is a well-formed region type if $\rgn$ is a valid region. The
type $\RgnZT{\rgn}$ is well-formed only if $\rgn = \toprgn$, where
$\toprgn$ is a special immortal region that outlives every other live
region. This arrangement allows \C{\RgnZ} handlers to be aliased and
referenced freely from objects in various regions, regardless of their
lifetimes. On the flip side, this also opens up the possibility of
references between transferable regions, which become unsafe in
context of the recipient's address space. Fortunately, such references
are explicitly prohibited by the type rule of \C{\RgnZ} objects, as
described below.

% Recall that a class definition in \FB is annotated with constraints
% over its region arguments, which require the regions referred by its
% fields to outlive its own allocation region. The well-formedness rule
% requires that (core) type be well-formed, instantiated regions be
% live, and that they   Well-formedness of region constraints requires
% that the regions referred by the constraints be live.

% The subtype relation between region types of objects
% ($B\inang{\ralloc\rbar}\inang{\tbar}$) follows directly from the
% subclass relation between their class types. Note that, in general, we
% don't lift outlives relation between regions to subtype relation
% between objects (of same core type) allocated in those regions. Such
% lifting is unsound for the same reason as covariant/contravariant
% subtyping between mutable references is unsound. However, we make a
% special case for $\RgnZ$ objects: we allow $\RgnZ$ objects allocated
% in any region to be considered as objects allocated in an eternal
% region ($\toprgn$) that outlives every other region. This lets $\RgnZ$
% objects escape the context they are created in (as required by the
% \C{transfer} operation), and also facilitates data structures of
% regions created in different contexts (for e.g., a list of $\RgnZ$
% objects, whose root objects are of (core) type $T$, can be typed
% $\C{List}\inang{\RgnZ\inang{T}}\inang{\rgn,\toprgn}$, where $\rgn$ is
% the region where the spine of the list is allocated). As described in
% \S\ref{sec:memory-safety}, runtime checks are used to prevent the
% unsound subtyping rule on region objects from effecting the memory safety.
% \begin{smathpar}
% \begin{array}{c}
% \renewcommand*{\arraystretch}{1.1}
% \RULE
%   {
%     \isvalid{\A.\phicx}{\rgn_1 \outlives \rgn_2}
%   }
%   {
%     \subtyp{\A}{T@{\rgn_1}}{T@{\rgn_2}}
%   }
% \end{array}
% \end{smathpar}

%The actual implementations of region-based memory management for \name
%may choose to realize the $\toprgn$ as a garbage-collected region, but
%this is not required. Our operational semantics
%(\S\ref{sec:fb-opsem}), for example, does not even associate a region
%with $\toprgn$, instead opting to allocate $\RgnZ$ objects in the same
%transferable regions that they represent. Our type safety result
%nonetheless guarantees memory safety in this implementation.

% Such a rule is unsound for it allows an object in a region ($\rgn_1$)
% to be treated as an object in a shorter living region ($\rgn_2
% \underlives \rgn_1$), thus allowing it to refer to shorter living
% objects in $\rgn_2$. The type system should prevent such unsafe
% references by restricting subtype relation to only those region types
% that are invariant in their allocation region. For instance, assuming
% that class $A$ is a subclass of $\ObjZ$ (as per FGJ), the subtype
% relation $A@\rgn <: \ObjZ@\rgn$ should be allowed, but not the
% relation $A@\rgn <: \ObjZ@\rgn'$, for any $\rgn' \neq \rgn$. \FB's
% type system enforces this not by restricting the subtype rule, but by
% constraining the subclass relation between region-annotated classes.
% In particular, it lets a class $A\inang{\rhoalloc}$ to be declared to
% extend $\ObjZ\inang{\rhoalloc}$, but never to extend
% $\ObjZ\inang{\rhoalloc_1}$, where $\rhoalloc_1 \neq \rhoalloc$. The
% well-formedness rule on class definitions (not shown) captures this
% restriction.

% As expected, subtyping between arrow types is contravariant in
% argument types and covariant in return types. It is also contravariant
% in the region constraints that the region parameters have to satisfy.
% For example, the type $\inang{\rhoalloc\rho} A@\rho \xrightarrow{\rgn}
% \unitZ$ is a subtype of $\inang{\rhoalloc\rho \,|\, \rho =  \rhoalloc}
% A@\rho \xrightarrow{\rgn} \unitZ$, for the former's precondition is
% weaker than the later's. Finally, the subtyping is invariant with
% respect to the allocation region ($\rgn$) of closures. 
% This subtype rule assumes that both arrow types have same set of
% region parameters.  However, this is not restrictive as (a). it is
% possible to add more region parameters to a function type and declare
% them to be equal to existing ones (as the arrow type above does), and
% (b). it is possible to rename region parameters by composing
% instantiation and generalization (big-lambda) expressions together.

The type rules distinguish between the \C{new} expressions that create
objects of  the \RgnZ class, and \C{new} expressions that create
objects of other classes. The rule for the latter relies on an
auxiliary definition called $\fields$ (undefined for $\RgnZ$ class)
that returns the sequence of type bindings for fields (instance
variables) of a given class type.  Like in FGJ, the names and types of
a constructor's arguments in \FB are same as the names and types of
its class's fields, and the type rule relies on this fact to typecheck
constructor applications.  

The type rule for \C{new \RgnZ} expressions expects the \RgnZ class's
constructor to be called with a nullary function that returns a value
in its allocation context. 
% It enforces this by typechecking the body ($e$) of the function
% under an empty context containing nothing but the allocation context
% of the function. 
This ensures that the value returned by the function stores no
references to objects allocated elsewhere, including the top region
($\toprgn$), thus preventing cross-region references originating from
transferable regions. The body of the function might however create
new regions while execution, but this is not a problem as long as such
regions, and objects allocated in them, don't find their way into the
result of its evaluation.

The type rule for \C{letregion} expression requires that the static
identifier introduced by the expression be unique under the current
context (i.e., $\rgn \notin \rhoenv$). This condition is needed in
order to prevent the new region from incorrectly assuming existing
outlives relationships on an eponymous region. Provided this is
satisfied, the expression ($e$) under \C{letregion} is then
typechecked assuming that the new region is live ($\rgn \in \rhoenv$)
and that it is outlived by all existing live regions ($\rhoenv
\outlives \rgn$). The result of a \C{letregion} expression must have a
type that is well-formed under a context not containing the new
region. This ensures that the value obtained by evaluating a
\C{letregion} expression contains no references to the temporary
objects inside the region.

The rule for \C{open} expression, unlike the rule for \C{letregion},
does not introduce any outlives relationship between the newly opened
region and any pre-existing region while checking the type of the
expression ($e$) under \C{open}. This prevents new objects allocated
inside the transferable region from storing references to those
outside.  Environment ($\env$) is extended with binding for the type
of root object while typechecking $e$. 

The type rule for lambda expression typechecks
% requires that its region
% parameters ($\rhoalloc\rhobar$) be unique under the current context,
% and that the constraints ($\phi$) over region parameters be
% well-formed. 
the lambda-bound expression ($e$) under an extended type environment
containing bindings for function's arguments, assuming that region
parameters are live, and that declared constraints over region
parameters hold. The constraints ($\phi$) are required to be
well-formed under $\rhoenv$ extended with the function's region
parameters ($\rhobar$). The premise $\forall
(\rgn'\in\frv(e)\setminus\{\rhobar\}). ~\isvalid{\phicx}
{\rgn'\outlives\rgn}$ ensures the safety of dereferences inside the
closure by requring that all free region variables ($\frv$) trapped by
the closure outlive the allocation region of the closure.

% The type rule for the \C{new} expression is a straightforward
% adaptation of corresponding FGJ rule to the the region type setting.

% The type rule for assignment expression ($e_1 := e_2$) ensures that a
% (local or instance) variable is only assigned objects of either the
% same type or a subtype. Recall that subtyping between region types is
% invariant with respect to the allocation regions of their objects. It
% follows that if a variable is declared to refer to objects in a region
% $\rgn$, then it is only ever assigned objects that allocated in
% $\rgn$. This prevents, for example, a temporary object allocated
% inside a stack region within a method from escaping the region's scope
% via an assignment to an instance variable.

\subsection{Operational Semantics and Type Safety}
\label{sec:fb-opsem}

Fig.~\ref{fig:fb-opsem-1} and~\ref{fig:fb-opsem-2} in the appendix
defines a small-step operational semantics for \fbname via a
five-place reduction relation:
% \vspace*{-0.15in}
\begin{smathpar}
  \redstoo{\rhoenv}{(e,\mem)}{(e',\mem')}
\end{smathpar}
Operational semantics replaces abstract region annotations ($\pi$ and
$\rho$) with \emph{locations} ($\loc$), where each \emph{location}
abstracts all the memory locations associated with a region (i.e.,
each memory region is associated with a single location). The
typestate of regions is tracked by a finite map ($\mem$) from
locations to typestates (Fig.~\ref{fig:region-fsm}). The reduction
relation should be read as following: given a set ($\rhoenv$) of
regions that are currently live, and an initial typestate map
($\mem$), the expression $e$ reduces to $e'$, while updating $\mem$ to
$\mem'$. The semantics gets ``stuck'' if $e$ attempts to access an
object whose allocation region is not present in $\rhoenv$, or if $e$
tries to \C{open} a transferable region that is not mapped to an
appropriate typestate by $\mem$.  On the other hand, if $e$ attempts
to commit an operation on a $\RgnZ$ object that is not sanctioned by
the transition discipline in Fig.~\ref{fig:region-fsm}, then it raises
an exception value ($\invalidexn$).

Replacing $\pi$'s and $\rho$'s with $\loc$'s at runtime results in a
new class of expressions that are not captured in
Fig.~\ref{fig:fb-syntax}. Appendix contains full syntax that includes
such expressions. To help state the type safety theorem, we define the
syntactic class of (runtime) values:
\begin{smathpar}
\begin{array}{lclcl}
v & \in & \mathtt{values} & \coloneqq & \C{new}\;
    B\inang{\tbar}\inang{\overline{\loc}}(\bar{v}) \ALT
\lambdaexp{\loc}{\rhobar\,|\,\phi}{\taubar \; \xbar}{e} \ALT
\C{new}\; \RgnZT{\toploc\loc}(v)
%  &     & & & \C{new}\; \RgnZT{\rgn}(v)\\
\end{array}
\end{smathpar}
The first two forms are obtained by using locations for region
annotations in $\C{new}$ and lambda expressions. The last form is the
value that $\C{new}\;\RgnZ$ expressions gets evaluated to; the
location $\toploc$ stands for the region $\toprgn$, and $\loc$ is the
location of the newly allocated transferable region.  The following
type safety theorem shows that a well-typed program will never attempt
to dereference an ``invalid'' reference (a reference to an object in a
region that has been transferred or freed):
\begin{theorem}
\emph{(\textbf{Type Safety})}
\label{thm:fb-type-safety}
$\forall e, \tau, \Delta, \mem$, such that $\consistent(\Delta,\mem)$
and $\tywf{\rhoenv}{\phicx}$, if $\hastyp{\emptyA,
\cdot}{e}{\tau}$, then either $e$ is a value, or $e$ raises an
exception ($\redstoo{\Delta}{(e,\mem)}{\invalidexn}$), or there exists an
$e'$ and a $\mem'$ such that $\consistent(\Delta,\mem')$ and
$\redstoo{\Delta}{(e,\mem)}{(e',\mem')}$ and
$\hastyp{\emptyASigp,\cdot}{e'}{\tau}$.
\end{theorem}
The relation $\consistent$ relates $\Delta$ and $\mem$ only if both
make consistent assumptions about liveness of regions. Since $\Delta$
is consistent with both $\mem$ and $\mem'$, the theorem also captures
the key property of operational semantics that no live region is ever
freed or transferred.

% Furthermore, we prove the following theorem about \FB, which, in
% conjunction with the type safety theorem, implies the safety of region
% transfers across address spaces:
% \begin{theorem}
% \emph{(\textbf{Transfer Safety})}
% \label{thm:fb-transfer-safety}
% $\forall$ $v$, $\rhoenv$, $\rhoenv'$, $\mem$, $\mem'$, $\phicx$,
% $\phicx'$, such that $\loc \not\in dom(\mem')$, if $\hastyp{\emptyA, \cdot}{\C{new}\;
% \RgnZT{\toploc\loc}(v)}{\RgnZT{\toprgn}}$, then
% $\hastyp{\rhoenv',\cdot,\phicx')}
% {\\\C{new}\; \RgnZT{\rgn_i}(v)}{\RgnZT{\toprgn}}$
% \end{theorem}
% The above theorem states that if a $\C{new}\; \RgnZT{\rgn_i}(v)$ value
% is well-typed under one context, then it is also well-typed under
% every other context, whose $\mem$ maps $\rgn_i$ to closed
% ($\CLOSED$) state.  Thus, a recipient of a transferable region only
% needs to add a binding for the region to its $\mem$ in order to
% preserve its type safety. 
% Notably, this result could not have been
% established if it were possible for a transferable region to contain
% references to objects outside the region.
