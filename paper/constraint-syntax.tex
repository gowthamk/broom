\begin{figure*}[!ht]
% \begin{smathpar}

\[
{
\renewcommand{\arraystretch}{1.2}
\setlength{\arraycolsep}{7pt}
\begin{array}{rcl} 

\multicolumn{3}{c}{
   \pi \in R \; \mathtt{ (Region \; constants) } \;\;
   \nu \in V \; \mathtt{ (Region \; vars) } \;\;
   \varphi \in P \; \mathtt{ (Predicate \; vars) } \;\;
   \Delta \subseteq R
} \\

\rho \in \mathtt{Region \; Identifiers} & \coloneqq & \pi \ALT \nu \\

\phi \in \mathtt{Region \; Constraint} & \coloneqq & true \ALT \rho \outlives \rho \ALT \phi \conj \phi \\

F \in \mathtt{Substitution} & \coloneqq & \cdot \ALT [\rho/\rho]F \\

\ell \in \mathtt{Antecedent} & \coloneqq & \phi \ALT \varphi \ALT \varphi \conj \phi \\

r \in \mathtt{Consequent} & \coloneqq & \phi \ALT F(\varphi) \\

\mathtt{Constraint} & \coloneqq & \isvalid{\ell}{r} \ALT \nu \in \Delta \ALT \tywf{\Delta}{\varphi} \\

\end{array}
}
\]

% \end{smathpar}
\caption{Syntax of Constraints}
\label{fig:constraint-syntax}
% \vspace*{-0.15in}
\end{figure*}
