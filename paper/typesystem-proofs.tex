\section{Type System: Proofs}

We first define the $\consistent$ relation between $\Delta$ and
$\mem$:

\begin{definition}[\consistent($\Delta$,$\mem$)]
A set $\Delta \in 2^{\rgn}$ of region annotations is said to be
consistent with a map $\mem \in \loc \rightarrow s$ from region
locations to typestates if and only if forall $\loc\in\Delta$,
$\mem(\loc) = \LIVE$.
\end{definition}

Consistency between $\Delta$ and $\mem$ is preserved by the reduction
relation:

\begin{lemma}[\textbf{consistency preservation}]
\label{lem:consistency}
$\forall(e,\Delta,\mem,\phicx,\rgn,\tau)$. if $\consistent(\Delta,\mem)$
and $\tywf{\Delta}{\phicx}$ %and $\rgn \in \Delta$ 
and $\hastyp{(\Delta,\cdot,\phicx),\cdot,\rgn}{e}{\tau}$ 
and $\redstoo{\Delta}{(e,\mem)}{(e',\mem')}$, then
$\consistent(\Delta,\mem')$.
\end{lemma}
\begin{proof}
Proof is by induction on $\redstoo{\Delta}{(e,\mem)}{(e',\mem')}$.
For the rules where $\mem'=\mem$, proof is trivial. For the rules
where $\mem'$ is a result of executing a subexpression under
$\Delta$, proof follows from the inductive hypothesis. Remaining rules
are discussed below:
% \begin{smathpar}
% \begin{array}{cl}
%   \forall(\Delta,\mem,\phicx,\tau).~\consistent(\Delta,\mem)
%   ~\conj ~\tywf{\Delta}{\phicx}
%   ~\conj ~\hastyp{(\Delta,\cdot,\phicx),\cdot}{e_0}{\tau} & \\
%   \hspace*{1in}
%   ~\redstoo{\Delta}{(e_0,\mem)}{(e_0',\mem')}
%   ~\Rightarrow~ \consistent(\Delta,\mem') & IH\\
% \end{array}
% \end{smathpar}
\begin{itemize}
  \item Rule $\rulelabel{LetRegionBegin}$: All live regions in $\mem$
  are also live in $\mem'$. Proof follows
  \item Rule $\rulelabel{LetRegion}$: By inductive hypothesis, $\mem'$
  is consistent with $\Delta$. Hence, it is consistent
% is consistent with $\Delta \cup \{\loc\}$. Hence, it is consistent
% with $\Delta$.
  \item Rule $\rulelabel{LetRegionEnd}$: Here, $\loc$ is set to
  $\XFERRED$ in $\mem'$. However, since $e$ is the \C{letd}
  expression, and the type rule for the \C{letd} expression gives us
  $\loc \not\in \Delta$. Hence $\Delta$ is consistent with $\mem'$.
  \item Rule $\rulelabel{NewRegion}$: $\mem'$ extends $\mem$ with a
  new binding. Consistency is trivially preserved.
% \item Rule \#2 of $\rulelabel{NewRegion}$ : Since $\mem(\loc)=\USED$
% and $\consistent(\Delta,\mem)$, $\loc \not\in \Delta$. The proof
% follows from inductive hypothesis.
  \item Rule \#2 of $\rulelabel{NewRegion}$ : Since $\consistent(\Delta,
  \mem)$, we have $\consistent(\Delta \cup \{\loc\}, \mem[\loc\mapsto 
  \LIVE])$. By inductive hypothesis, $\consistent(\Delta \cup \{\loc\}, 
  \mem')$. Hence, $\consistent(\Delta, \mem')$.
  \item Rule $\rulelabel{Open}$: As $\mem'=\mem[\loc\mapsto\LIVE]$,
  hence $\Delta$ remains consistent with $\mem'$. Here, we also take
  note of the fact that if the result expression (\C{opened}) is
  tagged with a typestate of $\USED$, then $\loc\notin \Delta$.
  \item Rule $\rulelabel{Opened}$: Inductive hypothesis guarantees
  that $\Delta$ is consistent with $\mem'$.
% that $\Delta\cup\{\loc\}$ is consistent with $\mem'$. Hence,
% $\Delta$ is consistent with $\mem'$. 
  We also take note of the fact that the typestate tagged with the
  $\C{opened}$ expression remains invariant during reduction.
  \item Rule $\rulelabel{OpenEnd}$: $\mem'=\mem[\loc\mapsto s]$, where
  $s$ is the typestate tagged with the \C{opened} expression. As clear
  from the $\rulelabel{Open}$ rule, $s$ can be either $\USED$ or
  $\LIVE$. The type rule for \C{opened} tells us that if $s$ is
  $\USED$ then $\loc\notin\Delta$. Hence, in either case $\Delta$
  remains consistent with $\mem$.
  \item Rules $\rulelabel{Transfer}$: changes binding for a non-live
  location $\loc$. Hence, consistency is preserved.
\end{itemize} 
\qed
\end{proof}

\begin{lemma}[value substitution preserves typing]
\label{lem:substitution}
$\forall(e,z,\tau_1,\tau_2,\Delta,\phicx)$, if $\hastyp{(\Delta,\cdot,
\phicx),\cdot[x\mapsto\tau_1]}{e}{\tau_2}$ and $\hastyp{(\Delta,\cdot,
\phicx),\cdot}{v}{\tau_1}$, then $\hastyp{(\Delta,\cdot,
\phicx),\cdot} {[v/x]e}{\tau_2}$.
\end{lemma}
\begin{proof}
The proof is by induction on typing derivation and follows on the
lines of similar proof for FGJ.
\qed
\end{proof}

\begin{lemma}[progress]
\label{lem:progress}
$\forall e, \tau, \mem, \rhomap, \phicx, \rgn$, if $\consistent(\Delta,\mem)$ and 
$\tywf{\Delta}{\phicx}$ and
$\hastyp{\emptyA,\cdot, \rgn}{e}{\tau}$, then one of the following holds:\\
  \begin{smathpar}
  \begin{array}{rl}
    (i) & \exists (e',\rhomap').\;\redstoo{\Delta}{(e,\rhomap)}{(e',\rhomap')}\\
    (ii) & \valuee(e)\\
    (iii) & \redstoo{\Delta}{(e,\rhomap)}{\invalidexn}\\
  \end{array}
  \end{smathpar}
\end{lemma}
\begin{proof}
Proof is by induction on the typing derivation of $e:\tau$. Most cases
follow from the inductive hypothesis (IH), which claims that if a
subexpression has a typing derivation, then it can make progress. We
will consider cases where all subexpressions are values, but the
expression itself is not a value.
\begin{itemize}
  \item $B\inang{\tbar}\inang{\locbar}(\vbar).f_i$ case: This expression has
  a type only if $\tywf{\emptyA}{B\inang{\tbar}\inang{\locbar}}$, which is
  possible only if $\locbar \in \Delta$. Since $\Delta$ is consistent
  with $\mem$, we have $\mem(\locbar)=\LIVE$. Hence $e$ can make progress.

  \item $\C{let} \; x = v \; \C{in} \; e$ case: Always takes step via
  \rulelabel{LetExp}.

  \item $\C{new}\; \RgnZT{\toprgn}(v)$ case: takes a step by
  \rulelabel{NewRegion} rule to $\C{new}\; \RgnZT{\toploc\loc}
  (v\inang{\loc}())$.

  \item $\C{new}\; \RgnZT{\toploc\loc}(e)$ case: takes a step by
  \rulelabel{NewRegion} rule \#2 to $\C{new}\; \RgnZT{\toploc\loc}(e')$.

  \item Method call case
  ($v_0.m\inang{\loc'\overline{\loc'}}(\overline{v'})$): From the type
  rule, since $\bound$ is defined for the type of $v_0$, we infer that
  the type is of form $A\inang{\tbar}\inang{\loc\locbar}$ and $v_0 =
  A\inang{\tbar}\inang{\loc\locbar}(\vbar)$.  Since $\mtype$ is
  defined, $\mbody$ is also defined. Well-formedness requirement on
  $A\inang{\tbar}\inang{\loc\locbar}$ gives $\loc \in \Delta$. Since
  $\consistent(\Delta,\mem)$, we have $\mem(\loc)=\LIVE$. Hence the
  execution takes a step by $\rulelabel{MethodInv}$.

  \item Lambda application case ($e = (\lambdaexp{\loc'}{\rho\rhobar \,|\, 
  \phi}{\xbar:\bar{\tau^1}}{e_0})\inang{\rgn\rbar}(\vbar)$): The type
  rule gives $\loc',\rgn,\rbar \in \Delta$. Since
  $\consistent(\Delta,\mem)$ and $dom(\mem)$ only contains locations,
  we have $\rgn=\loc$ and $\rbar=\locbar$, for some locations $\loc,
  \locbar$. Since $\loc' \in \Delta$, $\consistent(\Delta,\mem)$ also
  gives $\mem(\loc')=\LIVE$. Hence, $e$ takes step by
  \rulelabel{FnApply}.

  \item \C{letregion} case: takes a step by \rulelabel{LetRegionBegin}
  rule.

  \item \C{open} case: takes a step by \rulee{Open} rule, or throws an
  exception by \rulee{OpenTransferred} rule, depending on the typestate
  of the region location.

  \item \C{letd} case: takes a step via \rulee{LetRegion} rule.  If
  subexpression is a value, then \C{letd} takes a step via
  \rulee{LetRegionEnd} rule.

  \item \C{opened} case: takes a step via \rulee{Opened} rule.  If
  subexpression is a value, then \C{letd} takes a step via
  \rulee{OpenEnd} rule.

  \item \C{transfer} case: Either takes a step via \rulee{Transfer}
  rule, or throws an exception via \rulee{TransferOpened}, depending
  on the typestate of the region.

  \item Subtype case: The premise is that
  $\hastyp{(\Delta,\cdot,\phicx),\cdot,\rgn}{e}{\tau'}$ and 
  $\subtyp{(\Delta,\cdot,\phicx)}{\tau'}{\tau}$. Inductive hypothesis
  from $\hastyp{(\Delta,\cdot,\phicx),\cdot,\rgn}{e}{\tau'}$ directly
  gives the proof.

\end{itemize}
\qed
\end{proof}

\begin{lemma}[preservation]
\label{lem:preservation}
$\forall e, \tau, \Delta, \mem, \rgn$, such that $\consistent(\Delta,\mem)$
and $\tywf{\rhoenv}{\phicx}$, if $\hastyp{\emptyA,
\cdot,\rgn}{e}{\tau}$, and $\redstoo{\Delta}{(e,\mem)}{(e',\mem')}$, then 
$\hastyp{\emptyASigp,\cdot,\rgn}{e'}{\tau}$.
\end{lemma}
\begin{proof}
  Proof is by induction on the reduction step. Most cases follow from
  the inductive hypothesis, which asserts that if a subexpression
  takes a step, it preserves its type under the same $\Delta$ and
  $\phicx$. We consider interesting cases below:
  \begin{itemize}
    \item \rulee{FieldAccess} case: $\C{new}\; A\inang{\tbar}
    \inang{\loc\locbar}(\vbar).f_i$ takes a step to $v_i$. The field
    access expression has a type of $i^{th}$ field returned by
    $\fields$ definition, and so does $v_i$, if $\C{new}\;
    A\inang{\tbar} \inang{\loc\locbar}(\vbar)$ has to be typeable.
    Hence the type is preserved.

    \item \rulee{MethodInv} case ($e = (\C{new}\; A\inang{\tbar}
    \inang{\loc\locbar}(\vbar)).m\inang{\loc'\overline{\loc'}}
    (\overline{v'})$): Let $\mtype(A\inang{\tbar}\inang{\loc\locbar}) 
    \,=\, \inang{\rho\rhobar \,|\, \phi}\overline{\tau^1}\rightarrow 
    \tau^2$. Hence $\tau = \tau^2$. Method invocation type rule tells
    us $r = \loc'$. Further, from the method well-formedness
    condition, method body ($e_0$) has a type $\tau^2$ under
    $(\{\loc,\locbar,\rho,\rhobar\},\cdot,\phi),\cdot[\xbar \mapsto
    \overline{\tau^1}], \rho$. Also,
    $\tywf{\loc,\locbar,\rho,\rhobar}{\phi}$. Since $\loc' \neq \rho$
    and $\overline{\loc'} \neq \rhobar$ (locations are never equal to
    region variables), we have:
    \begin{center}
    $\hastyp{(\{\loc,\locbar,\loc',\overline{\loc'}\},\cdot,
      [\loc'\overline{\loc'}/\rho\rhobar]\phi),\cdot[\xbar
      \mapsto \overline{\tau^1}],\loc'}{e_0}{\tau^2}$ and
    $\tywf{\{\loc,\locbar,\loc',\overline{\loc'}\}}
          {[\loc'\overline{\loc'}/\rho\rhobar]\phi}$
    \end{center}
    Since $\{\{\loc,\locbar,\loc',\overline{\loc'}\} \subseteq \Delta$, and
    $\phicx \vdash [\loc'\overline{\loc'}/\rho\rhobar]\phi$, we can strengthen
    the context and derive:
    \begin{center}
      $\hastyp{(\Delta,\cdot,\phicx),\cdot[\xbar \mapsto
      \overline{\tau^1}], \loc'}{e_0}{\tau^2}$
    \end{center}
    Applying the substitution lemma (Lemma~\ref{lem:substitution})
    gives:
    \begin{center}
      $\hastyp{(\Delta,\cdot,\phicx),\cdot,
        \loc'}{[\overline{v'}/\xbar]e_0}{\tau^2}$
    \end{center}
    Since $e' \,=\, [\overline{v'}/\xbar]e_0$ and $\tau = \tau_2$, we
    have the required:
    \begin{center}
      $\hastyp{(\Delta,\cdot,\phicx),\cdot, \loc'}{e'}{\tau}$
    \end{center}
    It is important to note that if $\rho$, the allocation context
    parameter of $m$, was not instantiated with the current allocation
    context $\loc'$, the proof would have failed.
    \item \rulee{FnApply} case: proceeds on the similar lines as
    \rulee{MethodInv} case (an exception is that the typing context
    for the method body already includes $\Delta$).

    \item \rulee{LetRegionBegin} case ($e = \letregion{\pi}{e'}$):
    Steps to $\letd{\loc}{[\loc/\pi]e'}$.
    $\loc$ is the new location introduced. Since $\loc \notin
    dom(\mem)$, it follows that $\loc \notin \Delta$. Rest of the
    premises required to apply the \C{letd} type rule on result
    expression are obtained from the \C{letregion} type rule of the
    initial expression ($e$), by substituing $\loc$ for $\pi$.

    \item \rulee{LetRegionEnd} case ($e = \letd{\loc}{v}$): Steps to
    $v$. Inverting the typing derivation of $e$ yeilds the premise
    that $v$ is well-typed under the context $(\Delta \cup
    \{\loc\},\cdot,\phicx \conj \Delta \outlives \loc), \cdot, \loc$,
    and that the type is well-formed under the smaller context
    $(\Delta,\cdot,\phicx)$. Let the type be $\tau$. We now have to
    prove that $v$ is well-typed ($\tau$) under the typing context
    $(\Delta,\cdot,\phicx), \cdot, \loc'$. We carry out this proof by
    induction on the structure of value $v$:
    \begin{itemize}
      \item When $v$ is an object value (i.e., $\C{new}\;
      B\inang{\tbar}\inang{\locbar}(\vbar)$): We know that the type of
      $\vbar$ is determined by
      $\fields(B\inang{\tbar}\inang{\locbar})$. Since
      $B\inang{\tbar}\inang{\locbar}$ is well-formed under
      $(\Delta,\cdot,\phicx)$ (premise), and $\loc \notin \Delta$, we
      know that types of $\vbar$ are well-formed under
      $(\Delta,\cdot,\phicx)$ (see class well-formedness rule). Hence,
      by IH, they retain their type under the typing context
      $(\Delta,\cdot,\phicx), \cdot, \loc'$. Using the type rule of
      \C{new}, we derive that $\C{new}\;
      B\inang{\tbar}\inang{\locbar}(\vbar)$ has the type $\tau$ under
      $(\Delta,\cdot,\phicx), \cdot, \loc'$.

      \item When $v$ is a region handler (i.e.,
      $\RgnZT{\toploc\loc''}(v')$): $v'$ has type $T@\loc''$ under
      $(\Delta \cup \{\loc,\loc''\},\cdot,\phicx \conj \Delta
      \outlives \loc), \cdot, \loc$, and the type is well-formed under
      $(\Delta \cup \{\loc,\loc''\},\cdot,\phicx \conj \Delta
      \outlives \loc)$. But, $T@\loc''$ is a type that is well-formed
      under the minimal context of $(\{\loc''\},\cdot,true)$. Since
      the context can be strengthened without affecting
      well-formedness, the type is also well-formed under $(\Delta
      \cup \{\loc''\},\cdot,\phicx)$. By IH, $v'$ retains it's type
      ($T@\loc''$) under $(\Delta \cup \{\loc''\},\cdot,\phicx)$. By
      applying IH, we derive $\hastyp{(\Delta \cup
      \{\loc''\},\cdot,\phicx), \cdot,
      \loc'}{\RgnZT{\toploc\loc''}(v')}{\RgnZT{\toprgn}}$.

      \item When $v$ is a function closure (i.e.,
      $\lambdaexp{\loc''}{\rho\rhobar\,|\,\phi}{\xbar:\taubar}{e_0}$).
      By inversion on the typing deriviation of $v$, we know that
      $\loc'' = \loc$ since a function closure is always allocated in
      the current allocation context (i.e., the youngest region).
      Hence it's type ($\tau$) is $\inang{\rho\rhobar \,|\,
      \phi}\taubar\xrightarrow{\loc}\tau'$, for some $\tau'$. Since
      $\loc \notin \Delta$, $\tau$ can not be well-formed under
      $(\Delta, \cdot, \phicx)$ - a contradiction. In other words, a
      function closure can never escape a region's lexical scope.
    \end{itemize}

  \item \rulee{Open} case ($e \,=\, \open{v_a}{y}{\pi}{e_b}$): We know
  that $v_a \,=\, \RgnZT{\toploc\loc}(v)$, for some location $\loc$
  and value $v$. By inversion on the typing derivation:
  \begin{smathpar}
  \begin{array}{l}
    \pi \notin \Delta \\
    \hastyp{(\Delta \cup \{\pi\},\cdot,\phicx), [y\mapsto T@\pi],
    \pi}{e_b}{\tau}\\
    \tywf{(\Delta, \cdot, \phicx)}{\tau}\\
    \hastyp{(\Delta,\cdot,\phicx),\cdot,\rgn}{v_a} {\RgnZT{\toprgn}}\\
  \end{array}
  \end{smathpar}
  Since $\pi \notin \Delta$ and $\tywf{\Delta}{\phicx}$ and
  $\tywf{(\Delta, \cdot, \phicx)}{\tau}$ (i.e., $\pi$ does not occur
  free in $\phicx$ and $\tau$):
  \begin{smathpar}
  \begin{array}{l}
    \hastyp{(\Delta \cup \{\loc\},\cdot,\phicx), [y\mapsto T@\loc],
    \loc}{[\loc/\pi]e_b}{\tau}\\
  \end{array}
  \end{smathpar}
  Since $\hastyp{(\Delta \cup \{\loc\},\cdot,\phicx), \cdot, \loc}
  {v}{T@\loc}$, by substitution lemma:
  \begin{smathpar}
  \begin{array}{l}
    \hastyp{(\Delta \cup \{\loc\},\cdot,\phicx), \cdot, \loc}
      {[v/y][\loc/\pi]e_b}{\tau}\\
  \end{array}
  \end{smathpar}
  We know that the reduced expression ($e'$) is $\opened{l}{s} 
  {[v/y][\loc/\pi]e_b}$, where $s = \mem(\loc)$. Moreover, if 
  $\loc \in \Delta$, then since $\consistent(\Delta,\mem)$, we have
  $\mem(\loc)=\LIVE$. Therefore, by applying the type rule for
  \C{opened} expression, we derive:
  \begin{smathpar}
  \begin{array}{l}
    \hastyp{(\Delta,\cdot,\phicx), \cdot, \rgn} 
      {\opened{\loc}{s}{[v/y][\loc/\pi]e_b}}{\tau}\\
  \end{array}
  \end{smathpar}
  Hence, $e'$ has the same type as $e$.

  \item \rulee{OpenEnd} case: The proof is very similar to the
  \C{letd} case.
  \qed
  \end{itemize}
\end{proof}

\begin{proof}[\textbf{Theorem~\ref{thm:fb-type-safety}}]
Follows from Lemmas~\ref{lem:consistency},~\ref{lem:progress},
and~\ref{lem:preservation}.
\qed
\end{proof}
