\section{Type System: Proofs}

We first define the $\consistent$ relation between $\Delta$ and
$\mem$:

\begin{definition}[\consistent($\Delta$,$\mem$)]
A set $\Delta \in 2^{\rgn}$ of region annotations is said to be
consistent with a map $\mem \in \loc \rightarrow s$ from region
locations to typestates if and only if forall $\loc\in\Delta$,
$\mem(\loc) = \LIVE$.
\end{definition}

Consistency between $\Delta$ and $\mem$ is preserved by the reduction
relation:

\begin{lemma}[\textbf{consistency preservation}]
\label{lem:consistency}
$\forall(e,\Delta,\mem,\phicx,\tau)$. if $\consistent(\Delta,\mem)$
and $\tywf{\Delta}{\phicx}$ and
$\hastyp{(\Delta,\cdot,\phicx),\cdot}{e}{\tau}$ and
$\redstoo{\Delta}{(e,\mem)}{(e',\mem')}$, then
$\consistent(\Delta,\mem')$.
\end{lemma}
\begin{proof}
Proof is by induction on $\redstoo{\Delta}{(e,\mem)}{(e',\mem')}$.
For the rules where $\mem'=\mem$, proof is trivial. For the rules
where $\mem'$ is a result of executing a subexpression under
$\Delta$, proof follows from the inductive hypothesis. Remaining rules
are discussed below:
% \begin{smathpar}
% \begin{array}{cl}
%   \forall(\Delta,\mem,\phicx,\tau).~\consistent(\Delta,\mem)
%   ~\conj ~\tywf{\Delta}{\phicx}
%   ~\conj ~\hastyp{(\Delta,\cdot,\phicx),\cdot}{e_0}{\tau} & \\
%   \hspace*{1in}
%   ~\redstoo{\Delta}{(e_0,\mem)}{(e_0',\mem')}
%   ~\Rightarrow~ \consistent(\Delta,\mem') & IH\\
% \end{array}
% \end{smathpar}
\begin{itemize}
  \item Rule $\rulelabel{LetRegionBegin}$: All live regions in $\mem$
  are also live in $\mem'$. Proof follows
  \item Rule $\rulelabel{LetRegion}$: By inductive hypothesis, $\mem'$
  is consistent with $\Delta \cup \{\loc\}$. Hence, it is consistent
  with $\Delta$.
  \item Rule $\rulelabel{LetRegionEnd}$: Here, $\loc$ is set to
  $\XFERRED$ in $\mem'$. However, since $e$ is the \C{letd}
  expression, and the type rule for the \C{letd} expression gives us
  $\loc \not\in \Delta$. Hence $\Delta$ is consisten with $\mem'$.
  \item Rule $\rulelabel{NewRegion}$: $\mem'$ extends $\mem$ with a
  new binding. Consistency is trivially preserved.
  \item Rule \#2 of $\rulelabel{NewRegion}$ : Since $\mem(\loc)=\USED$
  and $\consistent(\Delta,\mem)$, $\loc \not\in \Delta$. The proof
  follows from inductive hypothesis.
  \item Rule $\rulelabel{Open}$: As $\mem'=\mem[\loc\mapsto\LIVE]$,
  hence $\Delta$ remains consistent with $\mem'$. Here, we also take
  note of the fact that if the result expression (\C{opened}) is
  tagged with a typestate of $\USED$, then $\loc\notin \Delta$.
  \item Rule $\rulelabel{Opened}$: Inductive hypothesis guarantees
  that $\Delta\cup\{\loc\}$ is consistent with $\mem'$. Hence,
  $\Delta$ is consistent with $\mem'$. We also take note of the fact
  that the typestate tagged with the $\C{opened}$ expression remains
  invariant during reduction.
  \item Rule $\rulelabel{OpenEnd}$: $\mem'=\mem[\loc\mapsto s]$, where
  $s$ is the typestate tagged with the \C{opened} expression. As clear
  from the $\rulelabel{Open}$ rule, $s$ can be either $\USED$ or
  $\LIVE$, and if $s$ is $\USED$ then $\loc\notin\Delta$. Hence, in
  either case $\Delta$ remains consistent with $\mem$.
  \item Rules $\rulelabel{Transfer}$: changes binding for a non-live
  location $\loc$. Hence, consistency is preserved.
\end{itemize} 
\qed
\end{proof}

\begin{lemma}[value substitution preserves typing]
\label{lem:substitution}
$\forall(e,z,\tau_1,\tau_2,\Delta,\phicx)$, if $\hastyp{(\Delta,\cdot,
\phicx),\cdot[x\mapsto\tau_1]}{e}{\tau_2}$ and $\hastyp{(\Delta,\cdot,
\phicx),\cdot}{v}{\tau_1}$, then $\hastyp{(\Delta,\cdot,
\phicx),\cdot} {[v/x]e}{\tau_2}$.
\end{lemma}
\begin{proof}
The proof is by induction on typing derivation and follows on the
lines of similar proof for FGJ.
\qed
\end{proof}


% \begin{lemma}[weakening]
% $\forall(v,\tau,\Delta,\Delta_0,\phicx,\phicx_0)$, 
% if 
% $\hastyp{(\Delta \cup \Delta_0,\cdot, \phicx \conj
% \phicx_0),\cdot}{v}{\tau}$
% and
% $\tywf{(\Delta,\cdot,\phicx)}{\tau}$
% then 
% $\hastyp{(\Delta,\cdot, \phicx),\cdot} {v}{\tau}$.
% \end{lemma}
% \begin{proof}
% The proof is by induction on $\hastyp{(\Delta \cup \Delta_0,\cdot, \phicx \conj
% \phicx_0),\cdot}{v}{\tau}$. Since $v$ is a value, we have few cases:

% \begin{itemize}
%   \item Case ($v = \C{new}\; \fbN_0(\vbar)$ and $\tau = \fbN_0$): By
%   inverting 
% \end{itemize}

% \end{proof}

\begin{lemma}[progress]
\label{lem:progress}
$\forall e, \tau, \mem, \rhomap, \phicx$, if $\consistent(\Delta,\mem)$ and 
$\tywf{\Delta}{\phicx}$ and
$\hastyp{\emptyA,\cdot}{e}{\tau}$, then one of the following holds:\\
  \begin{smathpar}
  \begin{array}{rl}
    (i) & \exists (e',\rhomap').\;\redstoo{\Delta}{(e,\rhomap)}{(e',\rhomap')}\\
    (ii) & \valuee(e)\\
    (iii) & \redstoo{\Delta}{(e,\rhomap)}{\invalidexn}\\
  \end{array}
  \end{smathpar}
\end{lemma}
\begin{proof}
Proof is by induction on the typing derivation of $e:\tau$. Most cases
follow from the inductive hypothesis (IH), which claims that if a
subexpression has a typing derivation, then it can make progress. We
will consider cases where all subexpressions are values, but the
expression itself is not a value.
\begin{itemize}
  \item $B\inang{\tbar}{\locbar}(\vbar).f_i$ case: This expression has
  a type only if $\tywf{\emptyA}{B\inang{\tbar}\inang{\locbar}}$, which is
  possible only if $\locbar \in \Delta$. Hence $e$ can make progress.

  \item $\C{let} \; x = v \; \C{in} \; e$ case: Always takes step via
  \rulelabel{LetExp}.

  \item $\C{new}\; \RgnZT{\toprgn}(v)$ case: takes a step by
  \rulelabel{NewRegion} rule to $\C{new}\; \RgnZT{\toploc\loc}
  (v\inang{\loc}())$.

  \item Method call case: Since $\mtype$ is defined, $\mbody$ is also
  defined, and the execution takes a step by $\rulelabel{MethodInv}$.

  \item Lambda application case: takes step by \rulelabel{FnApply}

  \item \C{letregion} case: takes a step by \rulelabel{LetRegionBegin}
  rule.

  \item \C{open} case: takes a step by \rulee{Open} rule, or throws an
  exception by \rulee{OpenTransferred} rule, depending on the typestate
  of the region location.

  \item \C{letd} case: The subexpression is typed under $\Delta \cup
  \{\loc\}$, hence the subexpression takes a step under $\Delta \cup
  \{\loc\}$. This allows \C{letd} expression take a step via
  \rulee{LetRegion} rule. If subexpression is a value, then \C{letd}
  takes a step via \rulee{LetRegionEnd} rule.

  \item \C{opened} case: similar to \C{letd} case.
\end{itemize}
\qed
\end{proof}

\begin{lemma}[preservation]
\label{lem:preservation}
$\forall e, \tau, \Delta, \mem$, such that $\consistent(\Delta,\mem)$
and $\tywf{\rhoenv}{\phicx}$, if $\hastyp{\emptyA,
\cdot}{e}{\tau}$, and $\redstoo{\Delta}{(e,\mem)}{(e',\mem')}$, then 
$\hastyp{\emptyASigp,\cdot}{e'}{\tau}$.
\end{lemma}
\begin{proof}
  Proof is by induction on the reduction step. Most cases follow from
  the inductive hypothesis, which asserts that if a subexpression
  takes a step, it preserves its type under the same $\Delta$ and
  $\phicx$. We consider interesting cases below:
  \begin{itemize}
    \item \rulee{FieldAccess} case: $\C{new}\; A\inang{\tbar}
    \inang{\loc\locbar}(\vbar).f_i$ takes a step to $v_i$. The field
    access expression has a type of $i^{th}$ field returned by
    $\fields$ definition, and so does $v_i$, if ${new}\;
    A\inang{\tbar} \inang{\loc\locbar}(\vbar)$ has to be typable.
    Hence the type is preserved.

    \item \rulee{MethodInv} case: From the method invocation type rule
    and method well-formedness condition, method body ($e'$) has a type
    $\tau^2$ under $(\{\loc,\locbar,\rhobar\},\cdot,\phi),\cdot[\xbar
    \mapsto \overline{\tau^1}]$. Also,
    $\tywf{\loc,\locbar,\rhobar}{\phi}$. Since $\overline{\loc'} \neq \rhobar$
    (locations are never equal to region variables), we have:
    \begin{center}
    $\hastyp{(\{\loc,\locbar,\overline{\loc'}\},\cdot,[\overline{\loc'}/\rhobar]\phi),\cdot[\xbar
    \mapsto \overline{\tau^1}]}{e'}{\tau^2}$ and
    $\tywf{\{\loc,\locbar,\overline{\loc'}\}}{[\overline{\loc'}/\rhobar]\phi}$
    \end{center}
    Since $\{\{\loc,\locbar,\overline{\loc'}\} \subseteq \Delta$, and
    $\phicx \vdash [\overline{\loc'}/\rhobar]\phi$, we can strengthen
    the context and derive:
    \begin{center}
      $\hastyp{(\Delta,\cdot,\phicx),\cdot[\xbar \mapsto \overline{\tau^1}]}{e'}{\tau^2}$
    \end{center}
    Applying the substitution lemma (Lemma~\ref{lem:substitution}) and
    inductive hypothesis gives the proof.

    \item \rulee{FnApply} case: proceeds on the similar lines as
    \rulee{MethodInv} case, except that no strengthening is needed;
    function body is typed under a context that includes $\Delta$.

    \item \rulee{LetRegionBegin} case: Since $\loc \notin dom(\mem)$,
    it follows that $\loc \notin \Delta$. Rest of the premises
    required to apply the \C{letd} type rule on result expression are
    obtained from the \C{letregion} type rule of the initial
    expression ($e$), by substituing $\loc$ for $\pi$.

    \item \rulee{LetRegionEnd} case: $e$ is a \C{letd} expression
    which reduces to a value $v$. Invering the typing derivation of
    $e$ yeilds the premise that $v$ is well-typed under the context
    $(\Delta \cup \{\loc\},\cdot,\phicx \conj \Delta \outlives \loc)$.
    However, we have to prove that $v$ is well-typed under the smaller
    context $(\Delta,\cdot,\phicx)$. We carry out this proof by
    induction on the structure of value $v$:
    \begin{itemize}
      \item If $v$ is an object value (e.g.,
      $B\inang{\tbar}\inang{\locbar}(\vbar)$), IH yeilds the
      well-typedness of $\vbar$ under $(\Delta,\cdot,\phicx)$, and the
      premise 
      $\tywf{(\Delta,\cdot,\phicx)} {B\inang{\tbar}\inang{\locbar}}$,
      obtained by inverting the type judgment for \C{letd}, yeilds the
      well-typedness of whole value.

      \item A region handler value ($\RgnZT{\toploc\loc'}(v)$) is
      well-typed under any context.

      \item If $v$ is a function closure, then it is well-tuped under
      $(\Delta,\cdot,\phicx)$ only if it doesn't trap any references
      to $\loc$, the \C{letd} location. Since closure's type needs to
      be well-formed under $(\Delta,\cdot,\phicx)$, its allocation
      region belongs to $\Delta$, hence outlives $\loc$. Since the
      closure typing rule requires all free region variables of the
      closure body outlive the allocation region of closure, it
      follows that all free region variables strictly outlive $\loc$,
      hence they belong to $\Delta$. Hence the function closure is
      well-typed outside \C{letd}.
    \end{itemize}

  \item \rulee{OpenEnd} case: The proof is similar to the \C{letd}
  case.
  \qed
  \end{itemize}
\end{proof}

\begin{proof}[\textbf{Theorem~\ref{thm:fb-type-safety}}]
Follows from Lemmas~\ref{lem:consistency},~\ref{lem:progress},
and~\ref{lem:preservation}.
\qed
\end{proof}
