%% ************ MACROS *************
\newcommand{\hdOf}[2]{\C{class}\; #1\angAlpha\inang{\rhobar \,|\, #2} \extends \fbN}

% mathpartir's inferrule: seems to have problems with linebreaks
\newcommand{\genconstraint}[2]{\inferrule*{#1}{#2}}
\newcommand{\gcarrow}{\leadsto}
\newcommand{\gctransforms}{\models}

\newcommand{\gcrule}[2]{%
\begin{smathpar}\begin{array}{c}%
\renewcommand*{\arraystretch}{1.2}%
\RULE {#1} {#2}%
\end{array}\end{smathpar}%
}

\newcommand{\lgcrule}[3]{%
\begin{smathpar}\begin{array}{c}%
\renewcommand*{\arraystretch}{1.2}%
[\rulelabel{#1}] \spc \RULE {#2} {#3}%
\end{array}\end{smathpar}%
}

\newcommand{\lgcfact}[2]{%
\begin{smathpar}\begin{array}{c}%
\renewcommand*{\arraystretch}{1.2}%
[\rulelabel{#1}] \spc {#2} %
\end{array}\end{smathpar}%
}

\newcommand{\minigcrule}[3]{%
\begin{minipage}{#1}\begin{smathpar}\begin{array}{c}%
\renewcommand*{\arraystretch}{1.2}%
\RULE {#2} {#3}%
\end{array}\end{smathpar}\end{minipage}%
}

\newcommand{\subtypesym}{<:}
\newcommand{\typeok}[3]{{#1\,\vdash\,#2 \; \texttt{ok} \, \lhd #3}}
\newcommand{\exprok}[4]{{#1} \, \vdash \, {#2} : {#3} \, \lhd {#4}}
\newcommand{\subtypeok}[4]{{#1} \, \vdash \, {#2}  \subtypesym {#3} \, \lhd {#4}} 
\newcommand{\stdcontext}{\A,\env}

%% ************ END MACROS *************

\begin{figure*}[t!]

%%%%%%%%%%% Header Box %%%%%%%%%%%
\fbox{  \( \exprok{\stdcontext}{e}{\tau}{C} \)}
\\


%%%%%%%%%%% () and x %%%%%%%%%%%
\lgcfact{UNIT}{\exprok{\stdcontext}{\unitval}{\unitZ}{\{\}}}

\lgcfact{VAR}{\exprok{\stdcontext}{x}{\env(\tau)}{\{\}}}

% \begin{minipage}{1.2in}
% \begin{smathpar}
% \begin{array}{l}
% \renewcommand*{\arraystretch}{1.2}
% \exprok{\stdcontext}{\unitval}{\unitZ}{\{\}} \\
% \exprok{\stdcontext}{x}{\env(\tau)}{\{\}}
% \end{array}
% \end{smathpar}
% \end{minipage}

%%%%%%%%%%% FIELD-ACCESS: e.f %%%%%%%%%%%
\lgcrule{FIELD-ACCESS}
  {
    \exprok{\stdcontext}{e}{\tau'}{C} \spc
    \bar{f}:\taubar \,=\, \fields(\bound_{\A.\aenv}(\tau'))
  }
  {
    \exprok{\stdcontext}{e.f_i}{\tau_i}{C}
  }

%%%%%%%%%%% NEW %%%%%%%%%%%
  \lgcrule{NEW}
  {
    \typeok {\A} {\fbN} {C_1} \spc
    \exprok {\stdcontext} {\bar{e}} {\bar{\tau'}} {C_2} \spc
    C_3 = \{ \isvalid{\A.\phicx}{\allocRgn(\fbN)=\ralloc} \}
    \\
    \fields(\fbN) = \bar{f} : \taubar \spc
    \subtypeok {\A} {\bar{\tau'}} {\bar{\tau}} {C_4}
  }
  {
    \exprok {\stdcontext}   {\C{new} \fbN(\bar{e})} {\fbN} {\cup_{i=1}^4 C_i}
  }

%%%%%%%%%%% LET %%%%%%%%%%%
  \lgcrule{LET}
  {
    \exprok{\stdcontext}{e_1}{\tau_1}{C_1} \spc
    \exprok{\exptycx{\ralloc}{\env[x\mapsto\tau_1]}}{e_2}{\tau_2}{C_2} \\
  }
  {
    \exprok{\stdcontext}{\letexp{x}{e_1}{e_2}}{\tau_2}{C_1 \cup C_2}
  }

%%%%%%%%%%% NEW-REGION %%%%%%%%%%%
  \lgcrule{NEW-REGION}
  {
    \typeok {\A.\aenv} {T} {C_1} \spc
    \rgn \in \A.\rhoenv
    \\
    \exprok{(\{\rho\},\A.\aenv,true),{\cdot}}{e}{T@\rho}{C_2}
  }
  {
    \exprok{\stdcontext}{\C{new}\;
    \RgnZ\inang{T}\inang{\toprgn}(\lambdaexp{\rgn}{\rho}{}{e})}{\fgjN\inang{\toprgn}}{C_1 \cup C_2}
  }

%%%%%%%%%%% METHOD-INV %%%%%%%%%%%
  \lgcrule{METHOD-INV}
  {
    \exprok {\stdcontext} {e_0} {\tau} {C_1} \spc
    \rbar \in \A.\rhoenv \\
    \mtype(m,\bound_{\A.\aenv}(\tau)) = \inang{\rhobar \,|\, 
        \phi}\bar{\tau^1}\rightarrow{\tau^2} \\
%   \substFn = [\rbar/\rhobar] \\
	\typeok {\A} {\inang{\rhobar \,|\,\phi}\bar{\tau^1}\rightarrow{\tau^2}} {C_2}
    \spc
    \exprok {\stdcontext} {\bar{e}} {[\rbar/\rhobar](\bar{\tau^1})} {C_3}
    \\
%   \subtyp{\A}{\bar{\tau'}}{\substFn(\bar{\tau^1})} \spc
    \isvalid{\A.\phicx}{[\rbar/\rhobar](\phi)}
  }
  {
    \exprok {\stdcontext} {e_0.m\inang{\rbar}(\bar{e})} 
       {[\rbar/\rhobar](\tau^2)} {C_1 \cup C_2 \cup C_3}
  }

%%%%%%%%%%% LET-REGION %%%%%%%%%%%

\lgcrule{LET-REGION}{
\A= (\rhoenv,\aenv,\phicx)  \spc
\rgn \notin \rhoenv \spc
\A' = (\rhoenv \cup \{\rgn\}, \aenv, \phicx \conj (\rhoenv \outlives \rgn))
\\
\exprok{\A',\rgn,\env} {e_a} {\tau} {C_1} \spc
\typeok {\A} {\tau} {C_2}
}{
\exprok{\stdcontext} {\letregion{\rgn}{e_a}} {\tau} {(C_1 \cup C_2)}
}

\caption{Constraint generation}
\label{fig:constraint-gen}
\end{figure*}

\begin{figure*}[t!]

%%%%%%%%%%% LAMBDA %%%%%%%%%%%

\lgcrule{LAMBDA}
  {
    \rgn \in \A.\rhoenv \spc
    \rhobar \notin \A.\rhoenv
    \spc
%   \rhoenv' = \rhoenv \cup \{\rhoalloc,\rhobar\}\spc
    \A' = (\A.\rhoenv \cup \{\rhobar\}, \A.\aenv, 
          \A.\phicx \conj \phi)\spc
    \tywf{\A'.\rhoenv}{\phi}
    \\
    \typeok {\A'} {\bar{\tau^1}} {C} \spc
    \typeok {\A'} {\tau^2} {C} \spc
    \exprok {\A',\env[\xbar \mapsto \bar{\tau^1}]} {e} {\tau^2} {C}
  }
  {
    \exprok {\stdcontext}
           {\lambdaexp{\rgn}{\rhobar \,|\, \phi} {\xbar:\bar{\tau^1}}{e}}
           {\inang{\rhobar \,|\, \phi} \bar{\tau^1} \xrightarrow{\rgn} \tau^2}
	   {C}
  }

%%%%%%%%%%% OPEN-REGION %%%%%%%%%%%
\lgcrule{OPEN}{
\exprok {\stdcontext} {e_a} {\RgnZ\inang{T}\inang{\rho}} {C_1} \spc
\A = (\rhoenv,\aenv,\phicx) \spc
\rgn \notin \rhoenv
\\
(\A',\env') = ((\rhoenv \cup \{\rgn\},\aenv,\phicx),\env[y\mapsto T@\rgn] \spc
\exprok {\A',\rgn,\env'} {e_b} {\tau} {C_2}
}{
\exprok {\stdcontext} {\open{e_a}{\rgn}{y}{e_b}} {\tau} {(C_1 \cup C_2)}
}


%%%%%%%%%%% FUNCTION INVOCATION %%%%%%%%%%%
\lgcrule{FUN-APPLY}
{
\exprok {\stdcontext} {e_a} {\inang{\rhobar\,|\,\phi}\taubar \xrightarrow{\rgn} \tau} {C_1} \spc
\exprok {\stdcontext} {\bar{e}} {\bar{\tau'}} {C_2} \spc
\substFn = [\bar{\rho'}/\rhobar]
\\
C_3 = \{\bar{\rho'} \in \A.\rhoenv\} \spc
C_4 = \{\isvalid{\A.\phicx}{\substFn(\phi)}\} \spc
\subtypeok {\A} {\bar{\tau'}} {\substFn(\bar{\tau})} {C_5}
}{
\exprok {\stdcontext} {e_a\inang{\bar{\rho'}}(\bar{e})} {\tau} {\cup_{i=1}^5 C_i}
}

%%%%%%%%%%% METHOD %%%%%%%%%%%
\lgcrule{METHOD}{
CT(B) = \hdOf{B}{\varphi}\{\bar{\tau^f}\,\xbar;\;\bar{d}\} \\
\A = (\rhoenv,\aenv,\phicx) = (\{\rhobar,\rhobarm\},\bar{\tyvar} \extends \bar{\fgjN}, \varphi_m) \spc\spc
C_1 = \{ \tywf{\rhoenv}{\varphi_m} \} \\
\env = \cdot[\thisZ \mapsto B\inang{\bar{\tyvar}}\inang{\rhobar}][\xbar \mapsto \taubar] \spc\spc
\exprok {\stdcontext}{e} {\tau'} {C_2} \spc\spc
\subtypeok {\A} {\tau'} {\tau} {C_3}
}{
\typeok{} {(B, \tau \; m\inang{\rhobarm \,|\, \varphi_m} (\taubar \;  \xbar)\{\C{return} e;\})} {(C_1 \cup C_2 \cup C_3)}
}

%%%%%%%%%%% CLASS %%%%%%%%%%%
\lgcrule{CLASS}{
\A = (\rhoenv, \aenv, \phicx) = (\{\rhoalloc,\rhobar\},\bar{\tyvar} \extends \bar{\fgjN},\varphi) \\
C_1 = \{ \tywf{\rhoenv}{\varphi} \} \spc\spc
\typeok {\A} {\fbN} {C_2} \spc\spc
\typeok {\A} {\bar{\tau^f}} {C_3} \\
C_4 = \{\isvalid{\phicx}{\allocRgn(\bar{\tau^f}) \outlives \rhoalloc \conj \allocRgn(\fbN) = \rhoalloc}\} \\
\typeok {} {\bar{d}} {C_6}
}{
\typeok {} {\hdOf{B}{\varphi}\{\bar{\tau^f}\,\xbar;\;\bar{d}\}} {\bigcup_{i=1}^6 C_i}
}

\caption{Constraint generation}
\label{fig:constraint-gen}
\end{figure*}

\begin{figure*}[t!]

%%%%%%%%%%% Header Box %%%%%%%%%%%
\fbox{  \( \typeok{\A}{\tau}{C} \)}
\\

%%%%%%%%%%% TYPE WELL-FORMEDNESS %%%%%%%%%%%

%%%%%%%%%%% OBJECT TYPE %%%%%%%%%%%
\lgcrule{TWF}
  {
    C = \{ \rgn \in \A.\rhoenv \}
  }
  {
    \typeok {\A} {\ObjZ\inang{\rgn}} {C}
  }

%%%%%%%%%%% CLASS TYPE %%%%%%%%%%%
  \lgcrule{TWF}
  {
    CT(B) = \headerOf{B}\{...\}
    \spc
    \fgjtywf{\aenv}{B\inang{\tbar}}
    \\
    C = \{ \rbar \in \rhoenv, \isvalid{\phicx}{[\rbar/\rhobar, \tbar/\bar{\tyvar}](\phi)} \}
  }
  {
    \typeok {(\rhoenv,\aenv,\phicx)} {B\inang{\rbar}\inang{\tbar}} {C}
  }

%%%%%%%%%%% GENERIC TYPE PARAMETER %%%%%%%%%%%
  \lgcrule{TWF}
  {
    \fgjtywf{\A.\aenv}{T} \spc
    \fgjsubtyp{\A.\aenv}{T}{\ObjZ} \spc
    \\
    C = \{ \rgn \in \A.\rhoenv \}
  }
  {
    \typeok {\A}{T@\rgn} {C}
  }

%%%%%%%%%%% FUNCTION TYPE %%%%%%%%%%%
  \lgcrule{TWF}
  {
    C_1 = \{ \rgn \in \rhoenv \}
    \\
    \rhobar \notin \A.\rhoenv \spc
    \rhoenv' = \rhoenv \cup \{\rhobar\} \spc
    \A' = (\rhoenv', \aenv, \phicx \conj \phi)
    \\
    \tywf{\rhoenv'}{\phi}\spc 
    \typeok{\A'}{\bar{\tau^1}} {C_2} \spc
    \typeok{\A'}{\tau^2} {C_3}
  }
  {
    \typeok{(\rhoenv,\aenv,\phicx)} {\inang{\rhobar \,|\, \phi} \bar{\tau^1} \xrightarrow{\rgn} \tau^2} 
       {C_1 \cup C_2 \cup C_3}
  }

%%%%%%%%%%% REGION TYPE %%%%%%%%%%%
  \lgcrule{TWF}
  { 
    \fgjtywf{\A.\aenv}{T}
  }
  {
    \typeok {\A} {\RgnZ\inang{T}\inang{\toprgn}} {\{\}}
  }

%%%%%%%%%%% FUNCTION SUBTYPING %%%%%%%%%%%
  \lgcrule{FUN-SUBTYPING}
  {
    C_1 = \{ \isvalid{\A.\phicx}{\phi_1 \Rightarrow \phi_2} \}
    \\
    \subtypeok {\A} {\bar{\tau^{11}}} {\bar{\tau^{21}}} {C_2}
    \\
    \subtypeok {\A} {\tau^{22}} {\tau^{12}} {C_3}
  }
  {
    \subtypeok {\A}
      {\inang{\rhobar \,|\, \phi_2}\bar{\tau^{21}} \xrightarrow{\rgn} \tau^{22}}
      {\inang{\rhobar \,|\, \phi_1}\bar{\tau^{11}} \xrightarrow{\rgn} \tau^{12}}
      {C_1 \cup C_2 \cup C_3}
  }

\caption{Type well-formedness constraint generation}
\label{fig:constraint-gen-2}
\end{figure*}
