\section{Related Work}
\label{sec:related-work}

% Tofte and Talpin, A Theory of Stack Allocation in Polymorphically
%     Typed Languages, 
% Tofte and Talpin, Implementation of the Typed Call-by-Value
%     λ-calculus using a Stack of Regions, POPL'94
% Tofte and Talpin, Region-Based Memory Management, IC'97
% Tofte, Birkedal, Elsman, Hallenberg, A Retrospective on Region-based
%     Memory Management, HOSC'2004.
% Yates, A Type-And-Effect System for Encapsulating Memory In Java, 1999
% Salcianu et al, Ownership Types for Safe Region-Based Memory Management in 
%     Real-Time Java, PLDI'03
% Bacchino et al, A Type and Effect System for Deterministic Parallel Java,
%     OOPSLA'09.
% Calcagno et al, Stratified operational semantics for safety and correctness
%     of the region calculus, POPL'01
% Hicks et al, Experience With Safe Manual Memory-Management in
%     Cyclone, ISMM'04
% Grossman et al, Region-based Memory Management in Cyclone, PLDI'02
% Henglein, Makholm, & Niss, A direct approach to control-flow sensitive
%     region-based memory management, PPDP'01.
% Talpin & Jouvelot, Polymorphic Type, Region and Effect Inference, JFP'92.
% Tofte & Birkedal, A Region Inference Algorithm, TOPLAS'98.

Tofte and Talpin in~\cite{tofte93,tofte94,tofte97} introduce the
concept of a region type system to statically enforce the safety of
their region-based memory management scheme in ML. Following their
seminal work, static type systems for safe region-based memory
management have been extensively studied in the context of various
languages and problem settings~\cite{cyclone02, cyclone04, yates99,
MIT03, DPJ09, HMN01, WW01, rust, gpu14}. Our work differs from the
existing proposals in a number of ways. Firstly, our problem setting
includes lexically scoped stack regions and dynamic transferable
regions (both programmer-managed) in context of an object-oriented
programming language equipped with higher-order functions. Second, we
adopt a two-pronged approach to memory safety that relies on a
combination of a simple static type discipline and lightweight runtime
checks. 
% The type system reduces the problem of ensuring safety of all memory
% accesses in a program into a problem of enforcing a state-transition
% discipline on memory accesses over a small subset of objects in the
% program. The later can be done efficiently at runtime. This lets us
% guarantee safety in presence of
In particular, our approach requires neither restrictive static
mechanisms (e.g., linear types and unique pointers) nor expensive
runtime mechanisms (e.g., garbage collection and reference counting)
in order to guarantee safety. Lastly, our region type system comes
equipped with full type inference that eliminates the need to write
region annotations on types to convince the type checker that the
program is safe. 

~\cite{tofte97} elaborates SML programs to introduce stack regions and
do away with GC in a transparent fashion. We too define an
elaboration, but our focus is on introducing region annotations
necessary to prove the safety of an object-oriented program that
already uses regions.
% Similar to their region inference algorithm, our region type inference
% algorithm can deal with region-polymorphic recursion by computing
% fixed points for recursive constraints. 
Our type inference is analogous to their region inference. While their
inference algorithm only ever generates equality constraints, which
can be solved via unification, our type inference algorithm also
generates partial order outlives constraints, which are required to
capture subtle relationships between lifetimes of transferable regions
and stack regions. Consequently, our constraint solving algorithm is
more sophisticated, and is capable of inferring unknown outlives
constraints over region arguments of polymorphic recursive functions.

Cyclone~\cite{cyclone02} equips C with programmer-managed stack
regions, and a typing discipline that statically guarantees the safety
of all pointer dereferences. Later
proposals~\cite{cyclone04,cycloneSCP} extends Cyclone with dynamic
regions. \name differs from Cyclone fundamentally because of its
non-intrusiveness design principle, which requires its safety
mechanisms to not intrude on the the programming practices of C\#.
\name programmers, for example, shouldn’t be forced to abandon
iterators in favor of for-loops, annotate region types, or rewrite
C\#'s standard libraries to use in \name.  Non-intrusiveness is not a
design consideration for Cyclone, which requires C programmers to use
new language constructs and abandon some standard programming idioms
in the interest of preserving safety. For instance, Cyclone
programmers are required to write region types for functions; the type
inference is only intraprocedural. Ensuring safety in presence of
dynamic regions requires using either unique pointers or
reference-counted objects.  Both approaches are intrusive. For
example, unique pointers constrain, or in some cases forbid, the use
of the familiar iterator pattern, which requires creation of aliases
to objects in a collection. Some standard library functions, for
example, those that use caching, may need to be rewritten.  Moreover,
even with unique pointers, safety cannot be guaranteed statically;
checks against \C{NULL} are needed at run-time to enforce safety. For
ref-counted objects, Cyclone requires programmers to use special
functions (\C{alias\_refptr} and \C{drop\_refptr}) to create and
destroy aliases.  Reference count is affected only by these
functions. An alias going out of scope, for instance, does not
decrement the ref-count. The requirement to use additional constructs
to manage aliases makes reference counting more-or-less as intrusive
as unique pointers.

Our work differs from Cyclone also in terms of its technical
contributions. While Cyclone equips C with a range of region
constructs~\cite{cycloneSCP}, the semantics of (a significant subset
of) such constructs, and the safety guarantees of the language are not
formalized. In contrast, the (static and dynamic) semantics of Broom
has been rigorously defined with respect to a well-understood formal
system (FGJ). The safety guarantees have been formalized and proved.
The core of Broom is very simple; the rules that make up static and
dynamic semantics occupy less than a page each. We believe that the
rigor and simplicity of Broom makes it easy to understand the the
underlying ideas, and apply them in various problem settings. Similar
contrast can be made of region type inference in both the languages.
Cyclone’s type inference was only ever described as being similar to
Tofte and Talpin’s, and it’s effectiveness in presence of tracked
pointers is not clear. In contrast, the complete Ocaml (pseudo) code
of Broom's inference algorithm, was given in the supplement and the
ideas underlying type inference have been described elaborately in the
paper

An ownership type system for safe region-based memory management in
real-time Java has been proposed by~\cite{MIT03}.  
% Like us, they too assume a source language with programmer-managed
% memory regions, and focus on proving safety of programs written in
% that language.  
Their source language admits various kinds of regions with
lexically-scoped lifetimes (i.e., stack regions) in order to support
shared-memory concurrency. In contrast, we admit regions with
dynamically determined lifetimes in order to support message-passing
concurrency. We borrow outlives relation from their formal
development, and our type system bears some similarities to theirs.
However, our language also admits parametric polymorphism (generics)
and higher-order functions.
%, whose interaction is non-trivial in context of a region type system. 
% On the other hand, our type system eschews the notions of region kinds
% and ownership, leading to a more succinct formalization. 
Moreover, we establish type safety and transfer safety results that
formalize the guarantees provided by our system. While~\cite{MIT03}'s
language is explicitly typed, our language comes equipped with full
type inference.
% Although there is some support for local
% type inference, region types for methods and classes need to be
% written explicitly. In contrast, our region type inference that
% eliminates any such need.

~\cite{HMN01} proposes a flow-sensitive region-based memory management
for first-order programs that proposes to overcome some of the
drawbacks of~\cite{tofte97} by generalizing~\cite{tofte97}'s approach
to regions with dynamic lifetimes. However, dynamic regions are still
not first-class values of the language, and reference counting is
nonetheless needed to ensure memory safety.~\cite{WW01} extends lambda
calculus with first-class regions with dynamic lifetimes, and imposes
linear typing to control accesses to regions. Our open/close lexical
block for transferable regions traces its orgins to the \C{let!}
expression in~\cite{WW01} and~\cite{wadler90}, which safely relaxes
linear typing restrictions, allowing variables to be temporarily
aliased. We don't have linear typing, thus admit unrestricted
aliasing. 
% Non-linear typing keeps our type system simple, and makes it
% possible to infer types, thus eliminating the annotation burden.
Moreover, ~\cite{WW01}'s linear type system is insufficient to enforce
the invariants needed to ensure safety under region transfers, such as
the absence of references that escape a transferable region.

The idea of using region-based memory management to facilitate the
safe transfer of rich data structures between computational nodes has
been previously explored by~\cite{gpu14} in the context of Scheme
language. However, their setting only includes lexically-scoped
regions for which Tofte and Talpin-style analysis~\cite{tofte97}
suffices. In contrast, our language provides first-class support for
transferable regions with dynamic lifetimes. We require this
generality in order to support streaming query operators, such as the
one shown in Fig.~\ref{fig:motivating-eg}. 

