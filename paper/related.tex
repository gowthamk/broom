\section{Related Work}
\label{sec:related-work}

% Tofte and Talpin, A Theory of Stack Allocation in Polymorphically
%     Typed Languages, 
% Tofte and Talpin, Implementation of the Typed Call-by-Value
%     λ-calculus using a Stack of Regions, POPL'94
% Tofte and Talpin, Region-Based Memory Management, IC'97
% Tofte, Birkedal, Elsman, Hallenberg, A Retrospective on Region-based
%     Memory Management, HOSC'2004.
% Yates, A Type-And-Effect System for Encapsulating Memory In Java, 1999
% Salcianu et al, Ownership Types for Safe Region-Based Memory Management in 
%     Real-Time Java, PLDI'03
% Bacchino et al, A Type and Effect System for Deterministic Parallel Java,
%     OOPSLA'09.
% Calcagno et al, Stratified operational semantics for safety and correctness
%     of the region calculus, POPL'01
% Hicks et al, Experience With Safe Manual Memory-Management in
%     Cyclone, ISMM'04
% Grossman et al, Region-based Memory Management in Cyclone, PLDI'02
% Henglein, Makholm, & Niss, A direct approach to control-flow sensitive
%     region-based memory management, PPDP'01.
% Talpin & Jouvelot, Polymorphic Type, Region and Effect Inference, JFP'92.
% Tofte & Birkedal, A Region Inference Algorithm, TOPLAS'98.

Tofte and Talpin in~\cite{tofte93,tofte94,tofte97} introduce the
concept of a region type system to statically ensure the safety of
region-based memory management in ML. Following their seminal work,
static type systems for safe region-based memory management have been
extensively studied in the context of various languages and problem
settings~\cite{cyclone02, cyclone04, yates99, MIT03, DPJ09, HMN01,
WW01, rust, gpu14}. Our work differs from the existing proposals in a
number of ways. Firstly, our problem setting includes lexically scoped
stack regions and dynamic transferable regions (both
programmer-managed) in context of an object-oriented programming
language equipped with higher-order functions. Second, we adopt a
two-pronged approach to memory safety that relies on a combination of
a simple static type discipline and lightweight runtime checks. 
% The type system reduces the problem of ensuring safety of all memory
% accesses in a program into a problem of enforcing a state-transition
% discipline on memory accesses over a small subset of objects in the
% program. The later can be done efficiently at runtime. This lets us
% guarantee safety in presence of
In particular, our approach requires neither restrictive static
mechanisms (e.g., linear types and unique pointers) nor expensive
runtime mechanisms (e.g., garbage collection and reference counting)
in order to guarantee safety. Lastly, our region type system comes
equipped with full type inference that completely eliminates the need
to write region annotations on types to convince the type checker that
the program is safe. 

~\cite{tofte97} proposes to extend Standard ML, a higher-order
functional language, with lexically-scoped stack regions, and defines
an elaboration from Standard ML to the region-annotated version of
Standard ML. The aim of the elaboration is to introduce stack regions
and do away with GC in a transparent fashion without jeopardizing
memory safety. We too define an elaboration, but our focus is on
introducing region annotations necessary to prove the safety of an
object-oriented program that already uses (stack and dynamic) regions.
Similar to their region inference algorithm, our region type inference
algorithm can deal with region-polymorphic recursion by computing
fixed points for recursive constraints. While their inference
algorithm only ever generates equality constraints, which can be
solved via unification, our type inference algorithm also generates
partial order outlives constraints, which are required to capture
subtle relationships between lifetimes of transferable regions and
stack regions. Consequently, our constraint solving algorithm is more
sophisticated, and is capable of inferring unknown outlives
constraints over region arguments of polymorphic recursive functions.

Cyclone~\cite{cyclone02} equips C with programmer-managed stack
regions, and a typing discipline that statically guarantees the safety
of all pointer dereferences. However, types need to be written
explicitly. While some region annotations can be inferred, type
inference is purely intraprocedural, and cannot infer region types for
functions. Later proposal~\cite{cyclone04} extends Cyclone with
dynamic regions, but has to impose unique pointer restrictions to
guarantee safety. Objects that need to be shared within data
structures are allowed to have multiple references, and safety for
such objects was ensured via runtime reference counting.

An ownership type system for safe region-based memory management in
real-time Java has been proposed by~\cite{MIT03}. Like us, they too
assume a source language with programmer-managed memory regions, and
focus on proving safety of programs written in that language. Their
source language admits various kinds of regions in order to support
shared-memory concurrency. However, all their regions have
lexically-scoped lifetimes. In contrast, we admit regions with
dynamically determined lifetimes in order to support message-passing
concurrency. We borrow outlives relation from their formal
development, and our type system bears some similarities to their's.
However, our language also admits parametric polymorphism (generics)
and higher-order functions, whose interaction is non-trivial in
context of a region type system. On the other hand, our type system
eschews the notions of region kinds and ownership, leading to a more
succinct formalization. Furthermore, we establish a type safety result
that formalizes its guarantees with respect to a well-defined
operational semantics. Like Cyclone, ~\cite{MIT03}'s language is
explicitly typed. Although there is some support for local type
inference, region types for methods and classes need to be written
explicitly. In contrast, we support full type inference that
eliminates any such need.

~\cite{HMN01} proposes a flow-sensitive region-based memory management
for first-order programs that proposes to overcome some of the
drawbacks of~\cite{tofte97} by generalizing~\cite{tofte97}'s approach
to regions with dynamic lifetimes. However, dynamic regions are still
not first-class values of the language, and reference counting is
nonetheless needed to ensure memory safety.~\cite{WW01} extends lambda
calculus with first-class regions with dynamic lifetimes, and imposes
linear typing to control accesses to regions. All type annotations
need to be written explicitly. Moreover, invariants needed to ensure
safety under region transfers, such as the absence of references that
escape a transferable region, cannot be enforced under~\cite{WW01}'s
type system.

The idea of using region-based memory management to facilitate the
safe transfer of rich data structures between computational nodes has
been previously explored by~\cite{gpu14} in the context of Scheme
language. However, their setting only includes lexically-scoped
regions for which Tofte and Talpin~\cite{tofte97} style analysis
suffices. In contrast, our language provides first-class support for
transferable regions with dynamic lifetimes. We require this
generality in order to support streaming query operators, such as the
one shown in Fig.~\ref{fig:motivating-eg}. 

