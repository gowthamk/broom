\section{Related Work}
\label{sec:related-work}

% Tofte and Talpin, A Theory of Stack Allocation in Polymorphically
%     Typed Languages, 
% Tofte and Talpin, Implementation of the Typed Call-by-Value
%     λ-calculus using a Stack of Regions, POPL'94
% Tofte and Talpin, Region-Based Memory Management, IC'97
% Tofte, Birkedal, Elsman, Hallenberg, A Retrospective on Region-based
%     Memory Management, HOSC'2004.
% Yates, A Type-And-Effect System for Encapsulating Memory In Java, 1999
% Salcianu et al, Ownership Types for Safe Region-Based Memory Management in 
%     Real-Time Java, PLDI'03
% Bacchino et al, A Type and Effect System for Deterministic Parallel Java,
%     OOPSLA'09.
% Calcagno et al, Stratified operational semantics for safety and correctness
%     of the region calculus, POPL'01
% Hicks et al, Experience With Safe Manual Memory-Management in
%     Cyclone, ISMM'04
% Grossman et al, Region-based Memory Management in Cyclone, PLDI'02
% Henglein, Makholm, & Niss, A direct approach to control-flow sensitive
%     region-based memory management, PPDP'01.
% Talpin & Jouvelot, Polymorphic Type, Region and Effect Inference, JFP'92.
% Tofte & Birkedal, A Region Inference Algorithm, TOPLAS'98.

Tofte and Talpin in~\cite{tofte93,tofte94,tofte97} introduce the
concept of a region type system to statically ensure the safety of
region-based memory management in ML. Following their seminal work,
static type systems for safe region-based memory management have been
extensively studied in the context of various languages and problem
settings~\cite{cyclone02, cyclone04, yates99, MIT03, DPJ09, HMN01,
WW01, rust}. Our work differs from the existing proposals in a number
of ways. Firstly, our problem setting includes lexically scoped stack
regions and dynamic transferable regions (both programmer-managed) in
context of an object-oriented programming language equipped with
higher-order functions. Second, we adopt a two-pronged approach to
memory safety that relies on a combination of a simple static type
system and lightweight runtime checks. The type system reduces the
problem of ensuring safety of all memory accesses in a program into a
problem of enforcing a state-transition discipline on memory accesses
over a small subset of objects in the program. The later can be done
efficiently at runtime. his lets us guarantee safety in presence of
dynamic memory allocations and deallocations without requiring either
restrictive static mechanisms (e.g., linear types and borrow checker)
or expensive runtime mechanisms (e.g., garbage collection and
reference counting). Lastly, our region type system comes equipped
with full type inference that completely eliminates the need to write
region annotations on types to convince the type checker that the
program is safe. 

% In contrast, ~\cite{tofte97} proposes to
% extend Standard ML, a higher-order functional language, with
% lexically-scoped stack regions. Cyclone~\cite{cyclone02} 


% and defines an elaboration from
% Standard ML to the extended version of Standard ML. The aim of the
% elaboration is to introduce stack regions and do away with GC in a
% transparent fashion without jeopardizing memory safety. We too define
% an elaboration, but our focus is on introducing region annotations
% necessary to prove the safety of an object-oriented program with
% (stack and dynamic) regions. Cyclone
