Fig~\ref{fig:naiad-ex} shows \C{SelectVertex} class, which is a
straightforward implementation of the \C{SELECT} dataflow operator.
\C{SelectVertex} implements \C{onReceive} method, which is called by
the run-time when a message from an upstream actor is ready to be
delivered. A message is usually a sequence of tuples (e.g., key-value
pairs) wrapped inside a transferable region (\C{inRgn}).
\C{SelectVertex} opens its input region \C{inRgn} (line 16), iterates
through the sequence, filtering the input tuples that meet the
selection criterion of the \C{whereFn} (line 19), and applying the
\C{mapFn} to map the selected input tuples to output (line 22).
Observe that the iterator is allocated in the temporary (stack) region
\C{R0} that lives just long enough.  Functions \C{mapFn} and
\C{whereFn} are defined by the users of the dataflow system, hence
they are called \emph{user-defined functions}.  Unlike dataflow system
builders, users are oblivious of regions, hence user-defined functions
are expected to not contain explicit region annotations. While
\C{whereFn} defines the selection criterion that the user is
interested in (e.g: \C{age $\ge$ 18}), \C{mapFn} can either project
attributes of its input tuple amounting to type \C{TOut}, or can
create and return a new \C{TOut} object. Note that, unlike \C{inRgn},
the output transferable region (\C{outRgn}) was opened for allocation
and the \C{mapFn} is applied under this allocation context. Also note
that the input tuple is cloned under this allocation context.
Consequently, the result (\C{outp}) of applying the \C{mapFn} (line
22), which is subsequently added to \C{outMsg}, is guaranteed to be in
the \C{outRgn} regardless of the semantics of \C{mapFn}. On the other
hand, if the input tuple is not cloned (i.e., \C{inp} instead of
\C{inp.clone()} on line 22) and the \C{mapFn} simply projects its
input, then \C{outp} refers to an object inside \C{inp}, which no
longer exists when the \C{outRgn} is transferred to a downstream
actor. Therefore, if \C{mapFn} is applied over \C{inp} instead of
\C{inp.clone()}, then it better be a function that creates and returns
a new \C{TOut} object. Under such scenario, the region type of
\C{SelectVertex.mapFn} should be precise enough to prevent it from
being assigned a function (line 6) that simply returns its input.
Fortunately, \name's region type system (\S~\ref{sec:type-system}) is
capable of capturing such nuances in the type of
\C{SelectVertex.mapFn}. Furthermore, the type can be automatically
inferred by \name's region type inference
(\S~\ref{sec:type-inference}), which can peform the above non-trivial
reasoning on behalf of the programmer.

