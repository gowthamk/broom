% Recent work in high-performance systems written in managed languages
% (such as Java or C\#) has shown that garbage-collection can be a
% significant performance bottleneck.
% A class of these systems, focused on big-data, create numerous large
% data structures with well-defined lifetimes.
%
There is increasing interest in alternative memory management techniques
that strike a balance between the convenience of garbage collection and
the better performance that manual memory management enables, without
compromising safety.
%
In this paper, we present a language and a memory management scheme based
on user-managed memory regions (called \emph{transferable regions})
that allow programmers to exploit knowledge of data structures' lifetimes
to improve performance.

Manual memory management is susceptible to the usual perils of
dangling pointers. A key contribution of this paper is a
refinement-based region type system, for a C\#-like typed
object-oriented language with higher-order functions and generics,
that ensures the memory safety of programs in the presence of
transferable regions.  We complement our type system with a type
inference algorithm, which eliminates the need for programmers to
write region annotations on types.
% The cornerstone of our inference algorithm is a novel constraint solver that
% performs abduction in a partial-order constraint domain to infer weakest solutions
% to recursive constraints.
The inference algorithm has been proven sound, and the constraint
solving algorithm used by the inference has been proven complete.  We
describe a prototype implementation of the inference algorithm and the
solver, and our experience of using it to enforce memory safety in
dataflow programs.


% while facilitating the reuse of existing C\# libraries with no modifications.
% Experiments demonstrate the practical utility of our approach.
