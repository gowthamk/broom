There is an increasing interest in alternative memory management
schemes that seek to the combine the convenience of garbage collection
and the performance of manual memory management in a single language
framework.  Unfortunately, ensuring safety in presence of manual
memory management remains as a challenge. In this paper, we present a
C\#-like object-oriented language called \name that uses a combination
of region type system and lightweight runtime checks to enforce safety
in presence of user-managed memory regions called \emph{transferable
regions}. Unsafe transferable regions have been previously used in
dataflow computations to contain the latency due to unbounded GC
pauses. Our approach shows that it is possible to restore safety
without compromising on the benefits of transferable regions. We prove
the type safety of \name in a formal framework that includes its
C\#-inspired features, such as higher-order functions and generics. We
complement our type system with a type inference algorithm, which
eliminates the need for programmers to write region annotations on
types. The inference algorithm has been proven sound and relatively
complete. We describe a prototype implementation of the inference
algorithm, and our experience of using it to enforce memory safety in
dataflow programs.
% %
% In this paper, we present a language and a memory management scheme based
% on user-managed memory regions (called \emph{transferable regions})
% that allow programmers to exploit knowledge of data structures' lifetimes
% to improve performance.

% Manual memory management is susceptible to the usual perils of
% dangling pointers. A key contribution of this paper is a
% refinement-based region type system, for a C\#-like typed
% object-oriented language with higher-order functions and generics,
% that ensures the memory safety of programs in the presence of
% transferable regions.  
