Recent work in high-performance systems written in managed languages
(such as Java or C\#) has shown that garbage-collection can be a
significant performance bottleneck.  A class of these systems, focused
on big-data, create many and often large data structures with
well-defined lifetimes.  In this paper, we present a language and a
memory management scheme based on user-managed memory regions (called
\emph{transferable regions}) that allow programmers to exploit
knowledge of data structures' lifetimes to achieve significant
performance improvements.

Manual memory management is susceptible to the usual perils of
dangling pointers. A key contribution of this paper is a
refinement-based region type system that ensures the memory safety of
C\# programs in the presence of transferable regions. We complement
our type system with a type inference algorithm that infers principal
region types for first-order programs, and practically useful types
for higher-order programs. This eliminates the need for programmers to
write region annotations on types, while facilitating the reuse of
existing C\# libraries with no modifications. Experiments demonstrate
the practical utility of our approach.

%As a testament to efficacy of our approach, we demonstrate performance
%improvements over a range of realistic dataflow programs that were
%rewritten to use region-based manual memory management, but are
%nevertheless judged to be safe by our type system.
