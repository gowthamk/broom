%Formatting Commands
%-------------------
\newcommand{\C}[1]{\code{#1}}
\renewcommand{\k}[1]{\mathtt{#1}} % Code font in math mode.
\newcommand{\code}[1]{\,{\tt #1}\,}
\newcommand{\conj}{\wedge}
\newcommand{\disj}{\vee}
\newcommand{\seqn}[1]{\overline {#1}}
\newcommand{\tuplee}[1]{\langle #1 \rangle}
\newcommand{\ALT}{~\mid~}
\newcommand{\lamb}[2]{\lambda #1.\,#2}
\newcommand{\spc}[0]{\quad}
\newcommand{\U}[0]{\cup}
\newcommand{\X}[0]{\times}
\newcommand{\xn}[0]{\cap}
\newcommand{\redsto}[3]{#1 \vdash #2 \longrightarrow #3}
\newcommand{\rredsto}[0]{\rightsquigarrow_r }
\newcommand{\evalsto}[0]{\longrightarrow }
\newcommand{\env}[0]{\Gamma}
\newcommand{\finmaparrow}[0]{\overset { Fin }{ \longrightarrow } }
\newcommand{\envwf}[1]{\vdash\,#1}
\newcommand{\tywf}[2]{#1\,\vdash\,#2 \; \texttt{ok}}
\newcommand{\fgjtywf}[2]{#1\,\Vdash\,#2 \; \texttt{ok}}
\newcommand{\isvalid}[2]{#1\,\vdash\,#2}
\newcommand{\isnotvalid}[2]{#1\,\nvdash\,#2}
\newcommand{\hastyp}[3]{#1\,\vdash\,#2:#3}
\newcommand{\hasfgjtyp}[3]{#1\,\Vdash^{}\,#2\,:\,#3}
\newcommand{\hassort}[3]{#1\,\vdash\,#2\,::\,#3}
\newcommand{\subtyp}[3]{#1\,\vdash\,#2 <: #3}
\newcommand{\fgjsubtyp}[3]{#1\,\Vdash\,#2 <: #3}
\newcommand{\eqtyp}[3]{#1\,\vdash\,#2 \equiv #3}
\newcommand{\generalizes}[3]{#1\,\vdash\,#2 \succ #3}
\newcommand{\msfolsig}[0]{{\Lambda \equiv}}
%\newcommand{\msentails}[2]{#1\,\models_L\,#2}
\newcommand{\msentails}[2]{#1\,\models\,#2}
\newcommand{\thesemof}[1]{ \llbracket #1 \rrbracket}
\newcommand{\defeq}[0]{ \triangleq }
\newcommand{\tymap}[0]{\mathcal{F}}
\newcommand{\etaeq}[0]{\eta_{eq}}
\newcommand{\semof}[1]{\thesemof{#1}}
\newcommand{\mssemof}[1]{\thesemof{#1}}
\newcommand{\rwsto}[0]{=}
\newcommand{\rrwsto}[0]{\hookrightarrow}
\newcommand{\rwbindfwd}[0]{\gamma_{\Rightarrow}}
\newcommand{\rwbindbwd}[0]{\gamma_{\Leftarrow}}
\newcommand{\eqhlt}[1]{\underline{#1}}
\newcommand{\tfr}[0]{F_R}
\newcommand{\ground}[1]{\underline{#1}}
\newcommand{\semeq}[0]{\equiv}
\newcommand{\elabsto}[0]{\looparrowright }
\newcommand{\arr}[0]{$\rightarrow$}
\newcommand{\eef}[0]{$\Leftrightarrow$}
\newcommand{\imp}[0]{$\Rightarrow$}
\newcommand{\RNull}[0]{\emptyset}
\newcommand{\cty}[0]{ty}
\newcommand{\hsp}[0]{\hspace*{-0.1in}}
\newcommand{\hspa}[0]{\hspace*{-0.08in}}
% removable:
\newcommand{\subst}[2]{\lbrack #1/#2 \rbrack}
%\newcommand{\absof}[1]{\left\| #1 \right\| }
\newcommand{\absof}[1]{\thesemof{#1}}
\newcommand{\broom}{{\sc Broom}\xspace}
\newcommand{\name}{{\sc Broom}\xspace}
\newcommand{\namec}{{\sc Broomc}\xspace}
\newcommand{\fbname}{{\sc Featherweight Broom}\xspace}
\newcommand{\FB}{{\sc FB}\xspace}
\newcommand{\naiad}{{\sc Naiad}\xspace}
\newcommand{\csolve}{{\sc CSolve}\xspace}
\newcommand{\GK}[1]{{\textcolor{red}{{GK: #1}}}}
\newcommand{\csharp}{C\#~}
\newcommand{\outlives}{\succeq}
\newcommand{\underlives}{\preceq}
\newcommand{\coloneqq}{::=}
\newcommand{\R}[1]{\textrm{#1}}
\newcommand{\N}[1]{{\normalfont #1}}
\newcommand{\M}[1]{\mathtt{#1}}
\newcommand{\ObjZ}{\C{Object}}
\newcommand{\RgnZ}{\C{Region}}
\newcommand{\thisZ}{\C{this}}
\newcommand{\superZ}{\C{super}}
\newcommand{\unitZ}{\C{unit}}
\newcommand{\FuncZ}{\C{Func}}
\newcommand{\unitval}{\C{()}}
\newcommand{\inang}[1]{\langle #1 \rangle}
\newcommand{\rhoalloc}{\rho^a}
\newcommand{\rhobar}{\bar{\rho}}
\newcommand{\rhoallocm}{\rho^a_m}
\newcommand{\rhobarm}{\bar{\rho_m}}
\newcommand{\ralloc}{\pi^a}
\newcommand{\rgn}{r}
\newcommand{\rbar}{\bar{\rgn}}
\newcommand{\tbar}{\bar{T}}
\newcommand{\taubar}{\bar{\tau}}
\newcommand{\xbar}{\bar{x}}
\newcommand{\vbar}{\bar{v}}
\newcommand{\ebar}{\bar{e}}
\newcommand{\extends}{\triangleleft}
\newcommand{\rextends}{\blacktriangleleft}
\renewcommand{\bar}[1]{\overline{#1}}
\newcommand{\angR}{\inang{\rhobar \,|\, \phi}}
\newcommand{\angT}{\inang{\tbar}}
\newcommand{\tyvar}{a}
\newcommand{\tyvarb}{b}
\newcommand{\angAlpha}{\inang{\bar{\tyvar} \extends \bar{\fgjN}}}
\newcommand{\fgjN}{K}
\newcommand{\fbN}{N}
\newcommand{\headerOf}[1]{\C{class}\; #1\angAlpha\angR \extends \fbN}
\newcommand{\substFn}{\mathcal{S}}
% \newcommand{\regionSubstFn}{\mathcal{R}}
% \newcommand{\predSubstFn}{\mathcal{P}}
% \newcommand{\regionSubstFn}{\soln_V}
% \newcommand{\predSubstFn}{\soln_P}
\newcommand{\regionConstants}{R}
\newcommand{\regionVars}{V}
\newcommand{\predVars}{P}
\newcommand{\constraintSet}{C}
\newcommand{\regionDeltaMap}{{\Delta_R}}
\newcommand{\predDeltaMap}{{\Delta_P}}
\newcommand{\mang}{\inang{\bar{\rho_m} \,|\, \phi_m}}
\newcommand{\rhoset}{\Sigma}
\newcommand{\rhomap}{\Sigma}
\newcommand{\rhoenv}{\Delta}
\newcommand{\aenv}{\Theta}
\newcommand{\phicx}{\Phi}
\newcommand{\phictxt}{\phi_{cx}}
\newcommand{\phicstr}{\phi_{cs}}
\newcommand{\phisol}{\phi_{sol}}
\newcommand{\subtypcx}{\rhoenv,\aenv,\phicx}
\newcommand{\A}{{\mathcal{A}}}
\newcommand{\exptycx}[2]{\A,#1,#2}
\newcommand{\void}{\C{void}}
\newcommand{\functy}[3]{\FuncZ\inang{#1}\inang{#2,#3}}
\newcommand{\stmtsemcx}{\A,\env}
\newcommand{\stmtsemcxp}{\A',\env'}
\newcommand{\stmtsemcxpp}{\A'',\env''}
\newcommand{\sredsto}{\Rightarrow}
\newcommand{\stmtsem}[4]{#1 \, \vdash \, #2 \,/\, #3 \sredsto #4}
\newcommand{\okin}[2]{#1 \; \texttt{ok} \; \texttt{in} \; #2}
\newcommand{\lambdaexp}[4]{\lambda@#1\inang{#2}(#3). #4}
\newcommand{\Lambdaexp}[2]{\Lambda\inang{#1}.#2}
\newcommand{\letexp}[3]{\C{let}\;#1\,=\,#2\;\C{in}\;#3}
\newcommand{\letregion}[2]{\C{letregion}\;#1\;\C{in}\;#2}
\newcommand{\open}[4]{\C{open}\;#1\;\C{as}\; #3@#2\;\C{in}\;#4}
\newcommand{\opened}[3]{\C{opened}\;#1(#2)\;\C{in}\;#3}
\newcommand{\letd}[2]{\C{letd}\;#1\;\C{in}\;#2}
\newcommand{\unpackexp}[4]{\C{let}\;(#1,\,#2)\,=\,\C{unpack}\;#3\;\C{in}\;#4}
\newcommand{\invalidexn}{\bot}
\newcommand{\toprgn}{\pi_{\top}}
\newcommand{\csolvestar}{{\sc CSolve*}\xspace}
\newcommand{\elabExpr}{{\sf elabExpr}}
\newcommand{\fields}{{\sf fields}}
\newcommand{\mtype}{{\sf mtype}}
\newcommand{\mbody}{{\sf mbody}}
\newcommand{\bound}{{\sf bound}}
\newcommand{\fresh}{{\sf fresh}}
\newcommand{\typeOk}{{\sf typeOk}}
\newcommand{\subtypeOk}{{\sf subtypeOk}}
\newcommand{\templateTy}{{\sf templateTy}}
\newcommand{\allocRgn}{{\sf allocRgn}}
\newcommand{\shape}{{\sf shape}}
\newcommand{\ctype}{{\sf ctype}}
\newcommand{\override}{{\sf override}}
%\newcommand{\smallskip}{\vspace*{0.1in}}
\newcommand{\elabMeth}{{\sf elabMeth}}
\newcommand{\frv}{{\sf frv}}
\newcommand{\rv}{{\sf RVars}}
\newcommand{\solve}{{\sf solve}}
\newcommand{\valuee}{{\sf value}}
\newcommand{\elabClass}{{\sf elabClass}}
\newcommand{\elabCons}{{\sf elabCons}}
\newcommand{\elabClassTable}{{\sf elabClassTable}}
\newcommand{\CLOSED}{\mathtt{C}}
\newcommand{\OPEN}{\mathtt{O}}
\newcommand{\XFERRED}{{\color{blue}\times}}
\newcommand{\LIVE}{{\color{ForestGreen}\blacksquare}}
\newcommand{\SLIVE}{\LIVE}
\newcommand{\TLIVE}{\LIVE}
% \newcommand{\SLIVE}{{\color{ForestGreen}\widehat{\blacksquare}}}
% \newcommand{\TLIVE}{{\color{ForestGreen}\overrightarrow{\blacksquare}}}
%\newcommand{\FREE}{{\color{blue}\square}}
% \newcommand{\USED}{{\color{blue}\overrightarrow{\blacksquare}}}
\newcommand{\USED}{{\color{blue} \square}}
\newcommand{\transfer}{{\sf transfer}}
\newcommand{\CSP}{(C\lbrack \varphi,\rhobar \rbrack,\rhoenv_{\varphi}, \bar{\rhoenv})}
\newcommand{\csp}{(c\lbrack \varphi,\rhobar \rbrack,\rhoenv_{\varphi},\bar{\rhoenv})}
\newcommand{\Phicx}{\Phi_{cx}}
\newcommand{\Phics}{\Phi_{cs}}
\newcommand{\tynwf}[2]{#1\,\nvdash\,#2 \; \texttt{ok}}
\newcommand{\csid}[1]{\lbrack \mathbf{c_{#1}} \rbrack:\;}
\newcommand{\qqquad}{\quad\quad\quad}
\newcommand{\emptyA}[0]{(\Delta,\cdot,\phicx)}
\newcommand{\emptyASigp}[0]{(\Delta,\cdot,\phicx)}
\newcommand{\emptyASigpp}[0]{(\rhoenv,\cdot,\phicx)}
\newcommand{\emptyADelcup}[1]{(\rhoenv \cup \{#1\},\cdot,\phicx)}
\newcommand{\emptyASigpDelcup}[1]{(\rhoenv \cup \{#1\},
                                    \cdot,\phicx)}
\newcommand{\emptyASigxDelcup}[2]{(dom(\rhomap[#1]),\rhoenv \cup \{#2\},
                                    \cdot,\phicx)}
\newcommand{\emptyADelcupPhicap}[2]{(\rhoenv \cup \{#1\},
                                    \cdot,\phicx \conj \rhoenv \outlives #2)}
\newcommand{\emptyADelcupPhicapp}[2]{(\rhoenv \cup \{#1\},
                                    \cdot,\phicx \conj #2)}
\newcommand{\empTyADelcup}[1]{(\rhomap,\rhoenv \cup \{#1\},\cdot,\phicx)}
\newcommand{\empTyADelcupPhisubst}[2]{(\rhomap,\rhoenv \cup \{#1\},\cdot,[#2]\phicx)}
\newcommand{\vacantA}[1]{(\emptyset,\{#1\},\cdot,true)}
\newcommand{\RgnZT}[1]{\RgnZ\inang{T}\inang{#1}}
\newcommand{\BZ}[0]{B\inang{\tbar}}
\newcommand{\BZT}[1]{B\inang{\tbar}\inang{#1}}
\newcommand{\redstoo}[3]{#2 \longrightarrow #3} %\Delta ignored
\newcommand{\redstocup}[3]{\redstoo{\rhoenv \cup \{#1\}}{#2}{#3}}
\newcommand{\loc}{\mathtt{l}}
\newcommand{\locbar}{\overline{\loc}}
\newcommand{\isLive}{{\sf isLive}}
\newcommand{\mem}{\Sigma}
\newcommand{\status}{s}
\newcommand{\toploc}{\loc_{\top}}
\newcommand{\consistent}{{\sf consistent}}

%Rules
%-----
\newcommand{\rulelabel}[1]{\textrm{\sc {#1}}}
\newcommand{\rulee}[1]{\textrm{\sc {#1}}}
\newcommand{\ilrulelabel}[1]{{\sc #1}}
\newcommand{\RULE}[2]{\frac{\begin{array}{c}#1\end{array}}
                           {\begin{array}{c}#2\end{array}}}

% Math mode
%-----------
\newenvironment{nop}{}{}
\newenvironment{smathpar}{
\begin{nop}\small\begin{mathpar}}{
\end{mathpar}\end{nop}\ignorespacesafterend}

% Theorems Environment
%----------------------
% \newtheorem{theorem}{Theorem}[section]
% \newtheorem{lemma}[theorem]{Lemma}
% \newtheorem{proposition}[theorem]{Proposition}
% \newtheorem{corollary}[theorem]{Corollary}

% \newenvironment{proof}[1][Proof]{\begin{trivlist}
% \item[\hskip \labelsep {\bfseries #1}]}{\end{trivlist}}
% \newenvironment{definition}[1][Definition]{\begin{trivlist}
% \item[\hskip \labelsep {\bfseries #1}]}{\end{trivlist}}
% \newenvironment{example}[1][Example]{\begin{trivlist}
% \item[\hskip \labelsep {\bfseries #1}]}{\end{trivlist}}
% \newenvironment{remark}[1][Remark]{\begin{trivlist}
% \item[\hskip \labelsep {\bfseries #1}]}{\end{trivlist}}

% \newcommand{\qed}{\nobreak \ifvmode \relax \else
      % \ifdim\lastskip<1.5em \hskip-\lastskip
      % \hskip1.5em plus0em minus0.5em \fi \nobreak
      % \vrule height0.75em width0.5em depth0.25em\fi}

\usepackage{listings}
\usepackage{subcaption}
\usepackage{stmaryrd}
\usepackage[skip=5pt]{caption}

\definecolor{Burgundy}{rgb}{0.5, 0.0, 0.13}
\definecolor{Bittersweet}{rgb}{1.0, 0.44, 0.37}
\definecolor{ForestGreen}{rgb}{0.13, 0.55, 0.13}
\definecolor{Bubbles}{rgb}{0.91, 1.0, 1.0}
\definecolor{LighterGray}{rgb}{0.93, 0.93, 0.93}

\newcommand{\lstjava}{\lstset{ %
language=Java, % choose the language of the code
backgroundcolor=\color{LighterGray},   
basicstyle=\footnotesize\ttfamily,       % the size of the fonts that are used for the code
keywordstyle=\color{Burgundy}\bf,
% numbers=left,                   % where to put the line-numbers
numberstyle=\tiny,      % the size of the fonts that are used for the line-numbers
stepnumber=1,                   % the step between two line-numbers. If it is 1 each line will be numbered
numbersep=5pt,                  % how far the line-numbers are from the code
showspaces=false,               % show spaces adding particular underscores
showstringspaces=false,         % underline spaces within strings
showtabs=false,                 % show tabs within strings adding particular underscores
% frame=single,                   % adds a frame around the code
tabsize=2,                      % sets default tabsize to 2 spaces
captionpos=b,                   % sets the caption-position to bottom
breaklines=true,                % sets automatic line breaking
breakatwhitespace=false,        % sets if automatic breaks should only happen at whitespace
commentstyle=\itshape\color{MidnightBlue},
%escapeinside={\%*}{*)},         % if you want to add a comment within your code
morekeywords={var, foreach, unit, letregion, let, in, as, openalloc, open, withroot, not, : , /\\}
}}
\lstnewenvironment{codejava}
    { % \centering
			\lstjava
      \lstset{mathescape=true}%
      \csname lst@setfirstlabel\endcsname}
    { %\centering
      \csname lst@savefirstlabel\endcsname}

\lstnewenvironment{numcodejava}
    { % \centering
			\lstjava
      \lstset{numbers=left,firstnumber=1}%
      \csname lst@setfirstlabel\endcsname}
    { %\centering
      \csname lst@savefirstlabel\endcsname}

\newcommand{\lstml}{\lstset{ %
language=[Objective]Caml, % choose the language of the code
basicstyle=\footnotesize\ttfamily,       % the size of the fonts that are used for the code
keywordstyle=\color{Bittersweet}\bf,
% numbers=left,                   % where to put the line-numbers
numberstyle=\tiny,      % the size of the fonts that are used for the line-numbers
stepnumber=1,                   % the step between two line-numbers. If it is 1 each line will be numbered
numbersep=5pt,                  % how far the line-numbers are from the code
showspaces=false,               % show spaces adding particular underscores
showstringspaces=false,         % underline spaces within strings
showtabs=false,                 % show tabs within strings adding particular underscores
% frame=single,                   % adds a frame around the code
tabsize=2,                      % sets default tabsize to 2 spaces
captionpos=b,                   % sets the caption-position to bottom
breaklines=true,                % sets automatic line breaking
breakatwhitespace=false,        % sets if automatic breaks should only happen at whitespace
commentstyle=\itshape\color{MidnightBlue},
escapeinside={\%*}{*)},         % if you want to add a comment within your code
morekeywords={foreach, do, done, not, : , /\\}
}}
\lstnewenvironment{codeml}
    { % \centering
			\lstml
      \lstset{mathescape=true,numbers=none}%
      \csname lst@setfirstlabel\endcsname}
    { %\centering
      \csname lst@savefirstlabel\endcsname}

\lstnewenvironment{numcodeml}
    { % \centering
			\lstml
      \lstset{mathescape=true,numbers=left}%
      \csname lst@setfirstlabel\endcsname}
    { %\centering
      \csname lst@savefirstlabel\endcsname}

