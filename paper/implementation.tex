\section{Implementation and Evaluation}
\label{sec:implementation}

We have implemented the prototype of \name compiler frontend,
including its region type system and type inference, in 3k+ lines of
OCaml. The input to our system is a program in $\absof{\FB^+}$, an extended
version of $\absof{\FB}$ that includes assignments, conditionals,
loops, more primitive datatypes (\eg strings), and a null value. 
% If needed, a specification file containing region type annotations
% for higher-order functions can also be provided. 
Our implementation of region type inference closely follows the
description given in Sec.~\ref{sec:type-inference}. To solve the
constraints that arise during type inference, we built a solver called
\csolve that implements constraint solving approach based on fixpoint
computation and graph augmentation described in \S~\ref{sec:csolve}. 
% Since graph augmentation is NP-hard, \csolve uses an approximation
% algorithm that repeatedly adds an edge and recomputes paths in graph
% $G_1$ until it is as connected as graph $G_2$. Though this approach
% may not necessarily return the weakest solution to the abduction
% problem (\S~\ref{sec:csolve}), it seems to do so in practice, since
% constraints are not often complex. 
If the input $\absof{\FB^+}$ program does not create any unsafe
references, our compiler annotates it with region types, which act as
a witness to program's memory safety. On contrary, if the input
program does create a reference that is potentially unsafe, then the
type inference fails during the constraint solving phase. We currently
do not implement error localization and feedback mechanisms.

To evaluate the practical utility of our region type system and type
inference, we 
% implemented some of the C\# libraries in $\absof{\FB^+}$, and 
translated some of the Naiad streaming query operator benchmarks
(Naiad vertices) used in~\cite{Broom:HotOS} to $\absof{\FB^+}$, and
used our prototype compiler to assign region types to these programs,
and thus prove their safety. During the process, we found multiple
instances of potential memory safety violations in the $\absof{FB^+}$
translation of benchmarks, which we verified to be present in the
original C\# implementation as well. The cause of all safety
violations is the creation of a reference from the outgoing message (a
transferable region) to the payload of the incoming message. For
example, the implementation of \C{RegionSelectVertex} contains the
following:
\begin{codejava}
  if (this.selectFn(inMsg.payload[i])) {
    outMsg.set(outputOffset, inMsg.payload[i]);
    ...
  } 
\end{codejava}
The \C{outMsg} is later transferred to a downstream actor, where the
reference to \C{inMsg}'s payload becomes unsafe\footnote{This unsafe
reference could have gone unnoticed during experiments
in~\cite{Broom:HotOS} because their experimental setup included only
one actor.}. We eliminated such unsafe references by creating a clone
of \C{inMsg.payload[i]} in \C{outMsg}, and our compiler was
subsequently able to certify the safety of all references.

% We now briefly describe our experience of
% working with these benchmarks. Note that the performance benefits of
% using transferable regions in dataflow systems have been established
% in~\cite{Broom:HotOS}. Our current evaluation only focuses on safety
% aspects.

% \paragraph{Memory safety violation} 
