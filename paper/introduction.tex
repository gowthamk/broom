\newcommand{\TODO}[1]{\textbf{TODO: #1}} \newcommand{\eg}{\emph{e.g.}}
\newcommand{\ie}{\emph{i.e.}}

\section{Introduction} \label{sec:introduction}

Consider the example, from~\cite{Broom:HotOS}, shown in
Fig.~\ref{fig:motivating-eg}.  This code represents the logic for a
streaming query operator.  The operator receives a stream of input
messages, each associated with a time (window) $t$, processed by
method \texttt{onReceive}.  Each input message contains a list of
inputs, each of which is processed by applying a user-defined function
to create a corresponding output.  The operator may receive multiple
messages with the same timestamp (and messages with different
timestamps may be delivered out of order).  A timing-message (an
invocation of method \texttt{OnNotify}) indicates that no more input
messages with a timestamp $t$ will be subsequently delivered.  At this
point, the operator completes the processing for time window $t$ and
sends a corresponding output message to its successor.

\begin{figure}[t!]
\begin{numcodejava}
class MapVertex<TIn, TOut> {
  Func<TIn, TOut> selector;
  Dictionary<Time, List<TOut>> map;
  ...
  void onReceive(Time t, List<TIn> inputs) {
    if (!map.ContainsKey(t))
       map[t] = new List<TOut>();
    foreach (TIn input in inputs) {
      TOut output = selector(input);
      map[t].add(output);
    }
  }
  void onNotify(Time t) {
     List<TOut> outputList = map[t];
     map.Remove(t);
     transfer(successorId, t, outputList); 
  }
}
\end{numcodejava}
\caption{\C{SELECT} dataflow operator}
\label{fig:motivating-eg}
\end{figure}


This example is an instance of a general pattern, where a producer creates a
data structure and passes it to a consumer. In a system where most of the
computation takes this form, and these data structures are very large, as is
the case with many streaming big-data analysis systems, garbage collection
overhead becomes significant~\cite{Broom:HotOS,Nguyen+:facade}.  Furthermore, in a
distributed dataflow system, the GC pause at one node can have a cascading
adverse effect on the performance of other nodes, particularly when real-time
streaming performance is required~\cite{Broom:HotOS, harris15}.  In
particular, a GC pause at an upstream actor can block downstream actors that
are waiting for messages.  However, often much of the GC overhead results
from the collector performing avoidable or unproductive work.  For example,
in the process executing the code from Fig.~\ref{fig:motivating-eg}, GC might
repeatedly traverse the \C{map} data structure, although its objects cannot
be collected until a suitable timing message arrives.

% actors.
% performance degradation due to
% GC overhead becomes aggravated in the distributed 
% For example, a consumer process may trigger a
% GC as soon as it receives a large data structure from a producer
% process, which has just finished its GC. In such case, consumer's GC
% traverses the entire data structure performing little or no work.

% as the collector scans these large data structures For instance, in
% the above example, the list at \C{map[t]} cannot be collected until
% the the timing message arrives, and the list is transferred.
% \C{transfer} operation is complete.  using alternative memory
% management techniques.

An important observation, in the context of processes of the kind
described above, is that the data-structures exchanged between them
can be partitioned into sets of fate-sharing objects with common
lifetimes, which makes them good candidates for a region-based memory
management discipline. A region is a block of memory that is allocated
and freed in one shot, consuming constant time. A region may contain
one or more contiguous range of memory locations, and individual
objects may be dynamically allocated within the region over time,
while they are deallocated en masse when the region is freed.  Thus, a
region is a good fit for a set of fate-sharing objects.
% that is used to realize a data structure with a well-defined
% lifetime. 
In Fig.~\ref{fig:motivating-eg}, the output to be constructed for each
time window $t$ (i.e., \C{map[t]}) can be a separate region that is
allocated when the first message with timestamp $t$ arrives, and
deallocated after \C{map[t]} is transferred in \C{onNotify}.

Region-based memory management, both manual as well as automatic, has
been known for a long time. Manual region-based memory management
suffers from the usual drawbacks, namely the potential for invalid
references and the consequent lack of memory safety. Automatic region-based memory
management systems guarantee memory safety, but impose various
restrictions.  MLKit, which implements the approach pioneered by Tofte
and Talpin~\cite{tofte94,tofte97}, for example, uses lexically scoped
regions.  At runtime, the set of all regions (existing at a point in
time) forms a stack. Thus, the lifetimes of all regions must be
well-nested: it is not possible to have two regions whose lifetimes
overlap, with neither one's lifetime contained within the other.
Unfortunately, the data structures in the above example do not satisfy
this restriction (as the output messages for multiple time windows may
be simultaneously live, without any containment relation between their
lifetimes).  We refer to regions with lexically scoped lifetimes as
\emph{stack regions} and to regions that do not have such a lexically
scoped region as \emph{dynamic} regions.

The goal of this work is a \emph{memory-safe} region-based memory
management technique that supports dynamic regions as first-class
objects. Our focus, in this paper, is on dynamic regions, which can be
safely transferred across address spaces. We refer to such dynamic
regions as \emph{transferable} regions. As with allocation and
deallocation, transferring a memory region is fast, and therefore,
transferring a data structure contained in a region is more efficient
that traversing its objects in heap, and transferring them
independently\footnote{Empirical studies in~\cite{Broom:HotOS} support
this claim}. In the \C{SelectVertex} example from
Fig.~\ref{fig:motivating-eg}, the proposed region to contain the
output for each time window $t$ must be transferable due to the
\C{transfer} operation on Line 16. As it is the case with
\C{SelectVertex}, the transferred data is no longer accessed by the
producer, so the transfer operation in our system deallocates the
region once the transfer is complete.

With respect to memory safety, the key property we wish to ensure is
that there are no invalid references: i.e., references to objects that
were deallocated, or simply never existed. Transfer operation, with no
additional checks, may cause memory safety violations, both at the
producer of the data structure, and its (possibly remote) consumer. At
the producer, any existing references into the data structure become
invalid post transfer. If the data structure contains references to
objects outside its (transferable) region, then such references become
invalid in the context of the consumer. Safety violations of this kind
are particularly unwelcome as the program with GC did not have them to
begin with. Note that, while the references of later kind (i.e.,
references that escape a region) defeat the very purpose of a
transferable region, hence need to be prohibited, references of the
former kind (i.e., references into a region from outside) are not at
odds with the concept of a transferable region, hence need to be
permitted. In fact, allowing such references is crucial to
performance, as any non-trivial program creates temporary objects, and
it is undesirable to allocate them in a transferable region; such
regions are meant for the data being transferred. Since transferable
regions are first class objects in our setting, controlling references
to and from such regions while ensuring their safety without
significantly diluting the performance advantage of regions over GC is
a challenging exercise.

In this paper, we describe an approach that restores memory safety in
presence of transferable regions through a combination of a static
typing discipline and lightweight runtime checks. The cornerstone of
our approach is an \C{open} lexical block for transferable regions,
that ``opens'' a transferable region and guarantees that the region
won't be transferred/freed while it is open. Our observation is that
by nesting a~\cite{tofte94}-style \C{letregion} lexical block, that
delimits the lifetime of a stack region, inside an \C{open} lexical
block for a transferable region, we can guarantee that the
transferable region will remain live as long as the stack region is
live. We say that the former ``outlives'' the later\footnote{We borrow
the \emph{outlives} relation from~\cite{MIT03}. A comparison of our
approach with~\cite{MIT03} can be found in
\S~\ref{sec:related-work}.}, and any references from the stack region
to the transferable region are therefore safe. 
% In general, we can allow references from a region
% $R_0$ to another region $R_1$, provided that we establish that $R_1$ outlives
% $R_0$. 
Next, we note that by controlling the ``outlives'' relationships
between various regions, we only allow safe cross-region references,
while prohibiting unsafe ones. In the above example, an outlives
relationship \emph{from} the stack region \emph{to} the transferable
region means that the references in that direction are allowed, but
not the references in the opposite direction. In contrast, if an
\C{open} block of a transferable region $R_0$ is nested inside an
\C{open} block of another transferable region $R_1$, we do not
establish any outlives relationships, thus declaring our intention to
not allow any cross-region references between $R_0$ and $R_1$.
Finally, we observe that outlives relationships are established based
on the lexical structure of the program, hence a static type system
can enforce them effectively. By assigning region types to objects,
which capture the regions such objects are allocated in, and by
maintaining outlives relationships between various regions, we can
statically decide the safety of all references in the program.

Indeed, the utility of our approach described above is predicated
on the assumption that we can enforce certain invariants on
transferable regions. Firstly, a transferable region cannot be
transferred/freed inside an open block of that region (i.e, while it
is still open).  Secondly, a transferred/freed region cannot be
opened. These are typestate invariants on the transferable region
objects, which are hard to enforce statically due to the presence of
unrestricted aliasing. Techniques like linear types and unique
pointers can be used to restrict aliasing, but the constraints they
impose are often hard to program around. We therefore enforce
typestate invariants at runtime via lightweight checks. In
particular, we define an acceptable state transition discipline for
transferable regions (Fig.~\ref{fig:region-fsm}), and check, at
runtime, whether a given transition of a transferable region (\eg,
from \emph{open} state to \emph{freed} state) is valid or not. The
check is lightweight since it only involves checking a single tag that
captures the current state.  We believe that this is a reasonable
choice since regions are coarse-grained objects manipulated
infrequently, when compared to the fine-grained objects that are
present inside these regions, for which safety is enforced statically.
An added advantage of delegating the enforcement of typestate
invariants to runtime is that our region type system is
simple\footnote{The accompanying material lists all the type rules of our language,
which occupy less than a page.}, which made it possible to formulate a
type inference that completely eliminates the need to write region
type annotations. This, we believe, significantly reduces the
impediment to adopt our approach in practical setting.

% transferableally checking for the second category of
% invariants is easy and does not pose a performance concern (unlike,
% \eg, the first category of constraints); a static typechecker that
% rules out the second category of violations would be very restrictive
% from an expressiveness perspective; finally, the human effort required
% to ensure the second set of invariants is not as burdensome (as
% regions are coarse-grained objects manipulated infrequently), \eg, as
% it would be for the first set of invariants.

% when a producer transfers a
% region containing a data structure to a consumer, in a shared-memory
% context, we enforce an ownership discipline that ensures that the
% producer cannot subsequently modify the data structure. When the same
% transfer happens in a distributed setting, we guarantee that (a). the
% data structure is contained in the region, and can be safely
% transmitted across address spaces, and (b). once the transfer
% concludes, the producer can safely deallocate the memory for the
% data structure without the fear of violating memory safety.

%  Transfer operation,
% with no additional checks and balances, may cause memory safety
% violations, both at the producer of the data structure, and the
% (possibly remote) consumer of the data structure. A key invariant and restriction
% that we utilize
% to achieve this is that no object $o_1$ in one transferable region $D_1$
% can contain a pointer to an object $o_2$ in another transferable region
% $D_2$.  In cases where this is too restrictive, the user has to resort
% to (deep) copying $o_2$ to $D_1$.  This is conceptually quite
% reasonable given the perspective that $D_1$ is a self-contained
% data structure.  Furthermore, this is quite analogous, from a
% performance perspective, to the copying that happens when a garbage
% collector promotes an object from a lower generation to a higher
% generation.  From this perspective, a region may be seen as a
% user-controlled generation.

% This ensures that we can safely free one transferable region (\eg, the
% input-message in our example) without affecting the validity of
% pointers in another transferable region (\eg, the output-message).
% However, this is not sufficient! When processing the input-message
% $D$, we necessarily will have to create pointers that point to objects
% that are internal to $D$. Such pointers may be stack-allocated or even
% reside in the heap (\eg, consider the iterator object used to iterate
% over the list in the input-message).  How do we ensure the safety of
% these pointers?

% We use \emph{stack regions} (regions with lexically-scoped lifetimes),
% as well as the stack, for such temporary pointers (to objects inside a
% transferable region) and use the following protocol to ensure safety
% of such pointers.  To work with a transferable region $D$, we must
% first \emph{open} the region $D$, to indicate that $D$ should not be
% freed during this period. When the processing of the region is
% complete, we \emph{close} the region $D$, to indicate that it is safe
% to free $D$.  In general, a transferable region may be opened and
% closed multiple times before it is eventually transferred or freed.

% Our type-system permits the creation of pointers to internal objects
% of \emph{open transferable} regions, but uses lexical scoping to ensure
% that such pointers are no longer live when the region is closed.
% Thus, such pointers may be either stack-allocated or reside in a
% region whose lifetime is statically guaranteed to be contained within
% the interval when $D$ is open.  As a consequence, the system ensures
% the strong invariant that for a \emph{closed transferable} region $D$ no
% pointer from outside $D$ points to an internal object of $D$.

% The final piece required to ensure memory-safety is ensuring that the
% program correctly follows the above-mentioned protocol for transferable
% regions: \eg, an open region should not be freed and, dually, a freed
% region should not be opened.  (A complete typestate specification of
% the protocol is presented later.)

% Transferable regions are first-class objects in our system. For instance,
% in our running example, we would like to use a dictionary that stores
% transferable regions. However, this means that the program may create
% multiple aliases (pointers) to the same region, which, in turn, means
% that statically checking if the program correctly follows the transferable
% region protocol is hard (undecidable, in fact). Safety can be ensured
% by using, \eg, linear typing to prevent the creation of aliases, but
% this would be quite restrictive. Hence, in our system, we transferableally
% check for adherence to this protocol.

% In summary, our approach reduces memory safety to two sub-problems:
% (a) Ensuring  low-level invariants about pointers to (fine-grained)
% objects inside regions and (b) Ensuring higher-level (typestate)
% invariants about the regions themselves.  We use a type system to
% statically ensure the first set of invariants. We use a transferable check
% to ensure that the second set of invariants are satisfied at runtime.
% %
% This is a novel aspect of our system. We believe that this is a
% reasonable choice: transferableally checking for the second category of
% invariants is easy and does not pose a performance concern (unlike,
% \eg, the first category of constraints); a static typechecker that
% rules out the second category of violations would be very restrictive
% from an expressiveness perspective; finally, the human effort required
% to ensure the second set of invariants is not as burdensome (as
% regions are coarse-grained objects manipulated infrequently), \eg, as
% it would be for the first set of invariants.
% =======
% objects. Though we focus, in this paper, on memory-safety, our
% approach offers other benefits.  When a producer transfers a
% data-structure to a consumer, in a shared-memory context , our
% approach provides an ownership discipline that ensures that the
% producer cannot subsequently modify the data-structure. When the same
% transfer happens in a distributed setting, the system guarantees that
% the data-structure is self-contained and can be safely transmitted
% across address spaces.  We refer to dynamic regions that provide these
% guarantees as \emph{transferable} regions.

% The key memory safety property we wish to ensure is that there are no
% dangling references: i.e., a reference to an object that has been
% freed.  A key invariant and restriction that we utilize to achieve
% this is that no object $o_1$ in one dynamic region $D_1$ can contain a
% pointer to an object $o_2$ in another dynamic region $D_2$.  In cases
% where this is too restrictive, the user has to resort to (deep)
% copying $o_2$ to $D_1$.  This is conceptually quite reasonable given
% the perspective that $D_1$ is a self-contained data-structure.
% Furthermore, this is quite analogous, from a performance perspective,
% to the copying that happens when a garbage collector promotes an
% object from a lower generation to a higher generation.  From this
% perspective, a region may be seen as a user-controlled generation.

% This ensures that we can safely free one dynamic region (\eg, the
% input-message in our example) without affecting the validity of
% pointers in another dynamic region (\eg, the output-message).
% However, this is not sufficient! \dv{for what} When processing the input-message
% $D$, we necessarily will have to create pointers that point to objects
% that are internal to $D$. Such pointers may be stack-allocated or even
% reside in the heap (\eg, consider the iterator object used to iterate
% over the list in the input-message).  How do we ensure the safety of
% these pointers?

% We use \emph{stack regions} (regions with lexically-scoped lifetimes),
% as well as the stack, for such temporary pointers (to objects inside a
% dynamic region) and use the following protocol to ensure safety of
% such pointers.  To work with a dynamic region $D$, we must first
% \emph{open} region $D$, to indicate that $D$ should not be freed
% during this period.  When the processing of the region is complete, we
% \emph{close} the region $D$, to indicate that it is safe to free $D$.
% In general, a dynamic region may be opened and closed multiple times
% before it is finally freed.

% Our type-system permits the creation of pointers to internal objects
% of \emph{open dynamic} regions, but uses lexical scoping to ensure
% that such pointers are no longer live when the region is closed.
% Thus, such pointers may be either stack-allocated or reside in a
% region whose lifetime is statically guaranteed to be contained within
% the interval when $D$ is open.  As a consequence, the system ensures
% the strong invariant that for a \emph{closed dynamic} region $D$ no
% pointer from outside $D$ points to an internal object of $D$.

% The final piece required to ensure memory-safety is ensuring that the
% program correctly follows the above-mentioned protocol for dynamic
% regions: \eg, an open region should not be freed and, dually, a freed
% region should not be opened.  (A complete typestate specification of
% the protocol is presented later.)

% Dynamic regions are first-class objects in our system. For instance,
% in our running example, we would like to use a dictionary that stores
% dynamic regions. However, this means that the program may create
% multiple aliases (pointers) to the same region, which, in turn, means
% that statically checking if the program correctly follows the dynamic
% region protocol is hard (undecidable, in fact). Safety can be ensured
% by using, \eg, linear typing to prevent the creation of aliases, but
% this would be quite restrictive. Hence, in our system, we dynamically
% check for adherence to this protocol.

% In summary, our approach reduces memory safety to two sub-problems:
% (a) Ensuring  low-level invariants about pointers to (fine-grained)
% objects inside regions and (b) Ensuring higher-level (typestate)
% invariants about the regions themselves.  We use a type system to
% statically ensure the first set of invariants. We use a dynamic check
% to ensure that the second set of invariants are satisfied at runtime.
% %
% This is a novel aspect of our system. We believe that this is a
% reasonable choice: dynamically checking for the second category of
% invariants is easy and does not pose a performance concern (unlike,
% \eg, the first category of constraints); a static typechecker that
% rules out the second category of violations would be very restrictive
% from an expressiveness perspective; finally, the human effort required
% to ensure the second set of invariants is not as burdensome (as
% regions are coarse-grained objects manipulated infrequently), \eg, as
% it would be for the first set of invariants.
% >>>>>>> af17b22a358a6165ae10628a5247cd4fa66adc07

\subsection*{Contributions}

The paper makes the following contributions:

\begin{itemize} 
  \item We present \name, a \csharp-like typed
  object-oriented language that eschews garbage collection in favour of
  programmer-managed memory regions . \name extends its core language,
  which includes \emph{lambdas} (higher-order functions) and
  \emph{generics} (parametric polymorphism), with constructs to create,
  manage and destroy static and transferable memory regions. Transferable regions
  are first-class values in \name.
  % Experiments demonstrate that   \naiad~\cite{naiad} dataflow
  % program using programmer-managed memory regions outperform their
  % GC counterparts by a margin of upto 59\%. 

  \item \name is equipped with a region type system that statically
  guarantees safety of all memory accesses in a well-typed program,
  provided that certain typestate invariants on regions hold.  The
  later invariants are enforced via simple runtime checks.
% The correctness of region usage is checked efficiently at run-time.
% The overhead of checking these conditions is less than \GK{x\%} in
% our experiments with \naiad workloads.

  \item We define an operational semantics for \name, and a type
  safety result that clearly defines and proves safety guarantees
  described above.

  \item We describe a region type inference algorithm for \name that
  (a). completely eliminates the need to annotate \name programs with
  region types, and (b). enables seamless interoperability between
  region-aware \name programs and legacy standard library code that is
  region-oblivious. The cornerstone of our inference algorithm is a
  novel constraint solver that performs abduction in a partial-order
  constraint domain to infer weakest solutions to recursive
  constraints.

  \item We describe an implementation of \name frontend in OCaml,
  along with case studies where the region type system was able to
  identify unsafe memory accesses statically.
  
\end{itemize}

  %The rest of the paper is organized as follows. The next section
  %presents an informal overview of \name, and motivates the need for a
  %region type system.  \S~\ref{sec:type-system} formalizes the type
  %system and its safety guarantees. The type inference algorithm is
  %described in \S~\ref{sec:type-inference}. \S~\ref{sec:csolve} focuses
  %on \csolve, the constraint solving algorithm, and its correctness
  %guarantees.  Implementations details and practical extensions to the
  %type system are described in \S~\ref{sec:implementation}.
  %\S~\ref{sec:evaluation} presents experimental evaluation and case
  %studies.  \S~\ref{sec:related} discusses the related work, and
  %\S~\ref{sec:conclusion} concludes.
