\newcommand{\TODO}[1]{\textbf{TODO: #1}} \newcommand{\eg}{\emph{e.g.}}
\newcommand{\ie}{\emph{i.e.}}

\section{Introduction} \label{sec:introduction}

Consider the example, from~\cite{Broom:HotOS}, shown in
Fig.~\ref{fig:motivating-eg}.  This code represents the logic for a
streaming query operator.  The operator receives a stream of input
messages, each associated with a time (window) $t$, processed by
method \texttt{onReceive}.  Each input message contains a list of
inputs, each of which is processed by applying a user-defined function
to create a corresponding output.  The operator may receive multiple
messages with the same timestamp (and messages with different
timestamps may be delivered out of order).  A timing-message (an
invocation of method \texttt{OnNotify}) indicates that no more input
messages with a timestamp $t$ will be subsequently delivered.  At this
point, the operator completes the processing for time window $t$ and
sends a corresponding output message to its successor.

\begin{figure}[t!]
\begin{numcodejava}
class MapVertex<TIn, TOut> {
  Func<TIn, TOut> selector;
  Dictionary<Time, List<TOut>> map;
  ...
  void onReceive(Time t, List<TIn> inputs) {
    if (!map.ContainsKey(t))
       map[t] = new List<TOut>();
    foreach (TIn input in inputs) {
      TOut output = selector(input);
      map[t].add(output);
    }
  }
  void onNotify(Time t) {
     List<TOut> outputList = map[t];
     map.Remove(t);
     transfer(successorId, t, outputList); 
  }
}
\end{numcodejava}
\caption{\C{SELECT} dataflow operator}
\label{fig:motivating-eg}
\end{figure}


This example is an instance of a general pattern, where a producer
creates a data-structure and passes it to a consumer. In a system
where most of the computation takes this form, and these
data-structures are very large, as is the case with many streaming
big-data analysis systems, garbage collection overhead becomes
significant~\cite{Broom:HotOS}.
Furthermore, in a distributed dataflow system, the GC pause at one node can
have a cascading adverse effect on the performance of other nodes, particularly
when real-time streaming performance is required.
\dv{citation missing here, Tim Haris's work and our HotOS submission.}
In particular, a GC pause at an upstream actor can block
downstream actors that are waiting for messages. 
However, often much of the GC overhead results from the collector performing avoidable or unproductive work.
For example, in the process executing the code from
Fig.~\ref{fig:motivating-eg}, GC might repeatedly traverse the \C{map}
data-structure, although its objects cannot be collected until a
suitable timing message arrives.

% actors.
% performance degradation due to
% GC overhead becomes aggravated in the distributed 
% For example, a consumer process may trigger a
% GC as soon as it receives a large data-structure from a producer
% process, which has just finished its GC. In such case, consumer's GC
% traverses the entire data-structure performing little or no work.

% as the collector scans these large data-structures For instance, in
% the above example, the list at \C{map[t]} cannot be collected until
% the the timing message arrives, and the list is transferred.
% \C{transfer} operation is complete.  using alternative memory
% management techniques.

An important observation, in the context of processes of the kind
described above, is that the data-structures exchanged between them
can be partitioned into sets of fate-sharing objects with common
lifetimes, which makes them good candidates for a region-based memory
management discipline. A region is a block of memory that is allocated
and freed in one shot, consuming constant time. A region may contain
one or more contiguous range of memory locations, and individual
objects may be dynamically allocated within the region over time,
while they are deallocated en masse when the region is freed.  Thus, a
region is a good fit for a set of fate-sharing objects.
% that is used to realize a data-structure with a well-defined
% lifetime. 
In this example, the output to be constructed for each time window $t$
can be implemented using a separate region.

Region-based memory management, both manual as well as automatic, has
been known for a long time. Manual region-based memory management
suffers from the usual challenges, namely the potential for dangling
references and lack of memory safety. Automatic region-based memory
management systems guarantee memory safety, but impose various
restrictions.  Tofte and Talpin's system~\cite{tofte94,tofte97} , for
example, uses lexically scoped regions.  At runtime, the set of all
regions (existing at a point in time) forms a stack. Thus, the
lifetimes of all regions must be well-nested: it is not possible to
have two regions whose lifetimes overlap, with neither one's lifetime
contained within the other.  Unfortunately, the data-structures in the
above example do not satisfy this restriction (as the output messages
for multiple time windows may be simultaneously live, without any
containment relation between their lifetimes).  We refer to regions
with lexically scoped lifetimes as \emph{stack regions} and to regions
that do not have such a lexically scoped region as \emph{dynamic}
regions.

The goal of this work is a \emph{memory-safe} region-based memory
management technique that supports dynamic regions as first-class
objects. Though we focus, in this paper, on memory-safety, our
approach offers other benefits.  When a producer transfers a
data-structure to a consumer, in a shared-memory context , our
approach provides an ownership discipline that ensures that the
producer cannot subsequently modify the data-structure. When the same
transfer happens in a distributed setting, the system guarantees that
the data-structure is self-contained and can be safely transmitted
across address spaces.  We refer to dynamic regions that provide these
guarantees as \emph{transferable} regions.

The key memory safety property we wish to ensure is that there are no
dangling references: i.e., a reference to an object that has been
freed.  A key invariant and restriction that we utilize to achieve
this is that no object $o_1$ in one dynamic region $D_1$ can contain a
pointer to an object $o_2$ in another dynamic region $D_2$.  In cases
where this is too restrictive, the user has to resort to (deep)
copying $o_2$ to $D_1$.  This is conceptually quite reasonable given
the perspective that $D_1$ is a self-contained data-structure.
Furthermore, this is quite analogous, from a performance perspective,
to the copying that happens when a garbage collector promotes an
object from a lower generation to a higher generation.  From this
perspective, a region may be seen as a user-controlled generation.

This ensures that we can safely free one dynamic region (\eg, the
input-message in our example) without affecting the validity of
pointers in another dynamic region (\eg, the output-message).
However, this is not sufficient! \dv{for what} When processing the input-message
$D$, we necessarily will have to create pointers that point to objects
that are internal to $D$. Such pointers may be stack-allocated or even
reside in the heap (\eg, consider the iterator object used to iterate
over the list in the input-message).  How do we ensure the safety of
these pointers?

We use \emph{stack regions} (regions with lexically-scoped lifetimes),
as well as the stack, for such temporary pointers (to objects inside a
dynamic region) and use the following protocol to ensure safety of
such pointers.  To work with a dynamic region $D$, we must first
\emph{open} region $D$, to indicate that $D$ should not be freed
during this period.  When the processing of the region is complete, we
\emph{close} the region $D$, to indicate that it is safe to free $D$.
In general, a dynamic region may be opened and closed multiple times
before it is finally freed.

Our type-system permits the creation of pointers to internal objects
of \emph{open dynamic} regions, but uses lexical scoping to ensure
that such pointers are no longer live when the region is closed.
Thus, such pointers may be either stack-allocated or reside in a
region whose lifetime is statically guaranteed to be contained within
the interval when $D$ is open.  As a consequence, the system ensures
the strong invariant that for a \emph{closed dynamic} region $D$ no
pointer from outside $D$ points to an internal object of $D$.

The final piece required to ensure memory-safety is ensuring that the
program correctly follows the above-mentioned protocol for dynamic
regions: \eg, an open region should not be freed and, dually, a freed
region should not be opened.  (A complete typestate specification of
the protocol is presented later.)

Dynamic regions are first-class objects in our system. For instance,
in our running example, we would like to use a dictionary that stores
dynamic regions. However, this means that the program may create
multiple aliases (pointers) to the same region, which, in turn, means
that statically checking if the program correctly follows the dynamic
region protocol is hard (undecidable, in fact). Safety can be ensured
by using, \eg, linear typing to prevent the creation of aliases, but
this would be quite restrictive. Hence, in our system, we dynamically
check for adherence to this protocol.

In summary, our approach reduces memory safety to two sub-problems:
(a) Ensuring  low-level invariants about pointers to (fine-grained)
objects inside regions and (b) Ensuring higher-level (typestate)
invariants about the regions themselves.  We use a type system to
statically ensure the first set of invariants. We use a dynamic check
to ensure that the second set of invariants are satisfied at runtime.
%
This is a novel aspect of our system. We believe that this is a
reasonable choice: dynamically checking for the second category of
invariants is easy and does not pose a performance concern (unlike,
\eg, the first category of constraints); a static typechecker that
rules out the second category of violations would be very restrictive
from an expressiveness perspective; finally, the human effort required
to ensure the second set of invariants is not as burdensome (as
regions are coarse-grained objects manipulated infrequently), \eg, as
it would be for the first set of invariants.

\subsection*{Contributions}

The paper makes the following contributions:

\begin{itemize} \item We present \name, a \csharp-like typed
object-oriented language that eschews garbage collection in favour of
programmer-managed memory regions . \name extends its core language,
which includes \emph{lambdas} (higher-order functions) and
\emph{generics} (parametric polymorphism), with constructs to create,
manage and destroy static and dynamic memory regions. Dynamic regions
are first-class values in \name.
  % Experiments demonstrate that   \naiad~\cite{naiad} dataflow
  % program using programmer-managed memory regions outperform their
  % GC counterparts by a margin of upto 59\%. 

  \item \name is equipped with a region type system that statically
  guarantees safety of all memory accesses in a well-typed program,
  provided that certain typestate invariants on regions hold.  The
  later invariants are enforced via simple runtime checks.
% The correctness of region usage is checked efficiently at run-time.
% The overhead of checking these conditions is less than \GK{x\%} in
% our experiments with \naiad workloads.

  \item We define an operational semantics for \name, and a type
  safety result that clearly defines and proves safety guarantees
  described above.

  \item We describe a region type inference algorithm for \name that
  (a). completely eliminates the need to annotate \name programs with
  region types, and (b). enables seamless interoperability between
  region-aware \name programs and legacy standard library code that is
  region-oblivious. The cornerstone of our inference algorithm is a
  novel constraint solver that performs abduction in a partial-order
  constraint domain to infer weakest solutions to recursive
  constraints.

  \item We describe an implementation of \name frontend in OCaml,
  along with experimental results, and case studies where the region
  type system was able to identify unsafe memory accesses statically.
  
\end{itemize}

%The rest of the paper is organized as follows. The next section
%presents an informal overview of \name, and motivates the need for a
%region type system.  \S~\ref{sec:type-system} formalizes the type
%system and its safety guarantees. The type inference algorithm is
%described in \S~\ref{sec:type-inference}. \S~\ref{sec:csolve} focuses
%on \csolve, the constraint solving algorithm, and its correctness
%guarantees.  Implementations details and practical extensions to the
%type system are described in \S~\ref{sec:implementation}.
%\S~\ref{sec:evaluation} presents experimental evaluation and case
%studies.  \S~\ref{sec:related} discusses the related work, and
%\S~\ref{sec:conclusion} concludes.
