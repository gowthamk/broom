\newcommand{\COMMENT}[1]{}

\section{Overview of \name}
\label{sec:overview}

\name enriches a simple object-oriented language (supporting
parametric polymorphism and lambdas) with a set of region-specific constructs. 
In this section, we present an informal overview of these region-specific constructs.

\subsection{Using Regions in \name}

\paragraph{Stack Regions}
The ``\C{letregion R S}'' construct creates a new stack region,
with a static identifier \C{R}, whose scope is restricted to the
statement \C{S}. The semantics of  \C{letregion} is similar
to Tofte and Talpin~\cite{tofte94}'s \C{letregion} expression:
new objects can be allocated in the newly created region within
the scope of \C{S}, but the region and all objects allocated within it
are freed at the end of \C{S}.

\paragraph{Object Allocation}
The ``\C{new@R T()}'' construct creates a new object of type \C{T} in
the region \C{R}. The specification of the allocation region \C{R} in
this construct is optional.  At runtime, \name maintains a stack of \emph{active}
regions, and we refer to the region at the top of the stack as the \emph{allocation
context}. The statement \C{new T()} allocates the newly created object in
the current allocation context.
%
This is important as it enables \name applications to use existing region-oblivious
C\# libraries. In particular, given a  C\# library function \C{F} (that makes no use of
\name's region constructs), the statement ``\C{letregion R F()}'' invokes \C{F},
but has the effect that all objects allocated by this invocation are allocated in the
new region \C{R}.

\paragraph{Transferable Regions}
\name's \emph{transferable regions} are an encapsulation of a data-structure
that can be transferred between autonomous entities
(\eg, between two concurrently executing threads or actors).
% Transferable regions are similar to stack regions in that they are memory regions
% that support arbitrary object allocations, but they differ from stack regions in the
% following ways.
Hence, unlike stack regions, transferable regions are not constrained to have a
lexically scoped lifetime.
(Hence, we also refer to them as \emph{dynamic} regions.)
% regions whose lifetimes transcend the boundaries of lexical blocks, functions, or
% even programs.

Furthermore, transferable regions, unlike stack regions, are first class values of \name:
they are considered objects of \C{Region} class, and, like objects of other
classes, they are created using the \C{new} keyword, can be passed as
arguments, stored in data structures, and returned from methods.
As mentioned above, a transferable region is intended to encapsulate a single
data-structure, consisting of a collection of objects with a distinguished root
object of some type \C{T}, which we refer to as the region's \emph{root} object.
The class \C{Region} is parametric over the type \C{T} of this root object.

The \C{Region} constructor expects a \C{Func}
object - a higher-order function argument that returns an object of
root type.  The idea is to execute this function with the newly
created transferable region as its allocation context. The object thus
created inside the transferable region will serve as its root object.
This is illustrated by the following code that creates a transferable region
(\C{r}) of type \C{T}:
\begin{codejava}
  Region<T> r = new Region<T>(() => new T())
\end{codejava}


\begin{figure}
\begin{codejava}
  public class Region<T> {
    public Region(Func<void,T> mkRoot);
    public void free();
    public void transfer();
  }
\end{codejava}
\caption{The type signature of the \C{Region} class}
\label{fig:region-class}
\end{figure}

The type signature of the class \C{Region} appears in Fig.~\ref{fig:region-class}.
The method \C{free} deallocates the region (including all objects allocated within it).
The method \C{transfer} transfers the region to a downstream actor as determined by the run-time.
It is an abstraction of two possible forms of  transfer: a transfer between two actors in a shared
memory setting or a transfer between two actors in a distributed, message-passing, setting. 
The precise semantics of \C{transfer} are unimportant in the context of the region type
system and we will not discuss them further.

\paragraph{Open and Closed Regions}
A transferable region must be explicitly \emph{opened} using \name's \C{open} construct
in order to either read or update or allocate objects in the region.
Specifically, the construct ``\C{open rgn as v@R S}'' does the following:
(a). It opens the transferable region handled by \C{rgn} for allocation
(i.e., makes it the current allocation context),
(b). binds the identifier \C{R} to this open region, and
(c). initializes the newly introduced local variable \C{v} to refer
to the root object of the region.
The \C{@R} part of the statement is optional and may be omitted.
The \C{open} construct is intended to simplify the problem of ensuring memory safety,
as will be explained soon.
We refer to a transferable region that has not been opened as a \emph{closed} region.

\paragraph{Motivating Example}
% We now illustrate how the features of \name can be used to 
Fig.~\ref{fig:motivating-eg-in-broom} shows how the motivating example
of Fig.~\ref{fig:motivating-eg} can be written in \name.
The \C{onReceive} method receives its input message in a \emph{transferred}
region (\ie, a \emph{closed} region whose ownership is transferred to the
recipient).
Line 7 creates a new region to store the output for time \C{t},
initializing it to contain an empty list.
Line 9 opens the input region to process it.
Line 10 creates a stack region \C{R0}.
Thus, the temporary objects created by the iteration in line 11,
for example, will be allocated in this stack region that lives just
long enough.
We open the desired output region in line 12, so that the new output
objects created by the invocation of \C{selector} in line 13
are allocated in the output region.
Finally, the input region is freed in line 19.

\begin{figure}[t!]
\begin{numcodejava}
class MapVertex<TIn, TOut> {
  Func<TIn, TOut> userDefinedFunction;
  Dictionary<Time, List<TOut>> map;
  ...
  void onReceive(Time t, Region<List<TIn>> inRgn) {
    if (!map.ContainsKey(t))
       map[t] = new Region<List<TOut>> (
                  () => new List<TOut>());
    open inRgn withroot inMsg {
      letregion R0 {
        foreach (TIn input in inputList) {
          openalloc map[t] withroot outMsg {
            TOut output = userDefinedFunction(input);
            outMsg.add(output);
          }
        }
      }
    }
    inRgn.free();
  }
  void onNotify(Time t) {
     Region<List<TOut>> outputRgn = map[t];
     map.Remove(t);
     successor.onReceive(t, outputRgn.transfer()); 
  }
}
\end{numcodejava}
\caption{\C{SELECT} dataflow operator in \name}
\label{fig:motivating-eg-in-broom}
\end{figure}


\subsection{Memory Safety}
\label{sec:memory-safety}

The key to memory safety in \name is the following restriction:
an object $o_1$ in a region $R_1$ is allowed to store a pointer to
an object $o_2$ in a region $R_2$ only if $R_2$ is guaranteed to outlive $R_1$.
(A similar restriction applies in the case where $o_1$ is a stack-allocated variable.)

Enforcing this restriction is simple in the case of stack regions since the outlives relation
between stack regions can be inferred from their lexical nesting. Unfortunately,
inferring outlives relations between transferable regions is not easy.
\name imposes the following protocol on the use of transferable regions to help simplify
this check.

A transferable region (that has not been freed or transferred) can be in one of two possible
states, \emph{open} or \emph{closed}. A newly created region is in the closed state.
A region must be opened, using the open construct (as explained previously), in order
to read or update or allocate an object within that region.
An open region cannot be freed or transferred. 
In particular, an open region is guaranteed to be live for the entire duration of the open construct.
This allows the type system to infer a valid outlives relation between the opened region
and any stack region that is nested within the open construct.


\begin{figure}
\includegraphics[scale=0.45]{region-fsm.png}
\caption{The lifetime of a dynamic (transferable) region in \name}
\label{fig:region-fsm}
\end{figure}

The protocol for transferable regions is presented as a finite state machine in Fig.~\ref{fig:region-fsm}.
A transferable region starts its lifetime in a \emph{closed} state,
when it is created through \C{new Region} expresion. \name provides

As evident in the above examples, the safety of memory accesses in
\name is now subject to the condition that every transferable region
correctly follows the state transition discipline shown
in Fig.~\ref{fig:region-fsm}. If this is
guaranteed, then \name's region type system statically guarantees the
safety of all memory accesses. In other words, the type system reduces
the problem of ensuring memory safety in \name programs to the problem
of enforcing the state transition discipline for transferable regions.

In \name, this enforcement is done at runtime by explicitly keeping
track of the \emph{current state} for $\RgnZ$ objects, and
checking the validity of every open, transfer, or free operation
and throwing an exception if it is invalid.
The challenge in enforcing this discipline statically is that transferable regions
are first-class objects. Hence, the program can create multiple aliases for
the same region, \eg, open it via one alias and free it via another.
Typestate verification in the presence of aliases is hard.
The checking can be done statically by preventing the creation of aliases using, \eg, linear types
or unique types. However, this would be quite restrictive, in terms of expressiveness.

\name chooses a reasonable tradeoff. Regions are coarse-grained objects, manipulated relatively infrequently,
in comparison to manipulations of the fine-grained objects that reside inside regions. Hence, the programmer burden
as well as the runtime overhead of checking the region's state transition discipline is acceptable.

\paragraph{Cloning}
Note that the example in Fig.~\ref{fig:motivating-eg-in-broom} works correctly only if \C{selector}
creates an output object that does not share any subobjects with its input object (which resides in
the input region and will be freed at the end of the method).
If there is a need for the output object to point to subobjects of the input object, such subobjects
must be cloned (to copy them from the input region to the output region).
Fortunately, \name's region type system (\S~\ref{sec:type-system}) is
capable of capturing such nuances in the type of \C{selector}
and the type checker will ensure correctness.
Furthermore, the type can be automatically inferred by \name's region type
inference (\S~\ref{sec:type-inference}), which can peform the above
non-trivial reasoning on behalf of the programmer.


%\begin{codejava}
%  Region<string> rgn = new Region<List<String>>
%                        (() => new List<String>());
%  open rgn as  strList@R0 {
%    strList.addAtHead("World");
%    strList.addAtHead("Hello");
%  }
%  rgn.transfer();
%\end{codejava}

\COMMENT{
\subsection{Stack Regions and Outlives Relation}

The \C{letregion} blocks can be nested, leading to a stack of regions
that are deallocated in the reverse order in which they are allocated.
Following ~\cite{cyclonepldi02}, we therefore call regions introduced
by \C{letregion} expressions as \emph{stack regions}. The stack
discipline induces an \emph{outlives} relationship among regions created
by nested \C{letregion}s, where the region introduced by the outer
\C{letregion} is guaranteed to outlive the one introduced by the inner
\C{letregion}. It is therefore safe to refer to an object allocated in
outer region from the inner region, but the converse is not true. For
example, consider the following code (assume that class \C{A} has
field \C{x} of type \C{Object}):
\begin{center}
\begin{codejava}
  letregion R0 {
    A a0 = new@R0 A();
    letregion R1 {
      A a1 = new@R1 A();
      a1.x = new@R0 Object(); // safe & legal
      a0.x = new@R1 Object(); // unsafe & illegal
      ...
    }
    ...
  }
\end{codejava}
\end{center}
The code creates two stack regions with identifiers \C{R0} and \C{R1},
where the region \C{R0} outlives the region \C{R1} (denoted as $\C{R0}
\outlives \C{R1}$).  Objects \C{a0} and \C{a1} are allocated in regions
\C{R0} and \C{R1}, respectively. The first assignment statement
assigns to \C{a1.x} an object allocated in outer region (\C{R0}). This
assignment is safe as \C{a1.x} refers to a longer living object, hence
is guaranteed to be a valid reference throughout the lifetime of
\C{a1}.  In contrast, the second assignment is unsafe, as it assigns
to \C{a0.x} an object, whose lifetime is shorter than the lifetime of
\C{a0}, making it unsafe to dereference \C{a0.x} outside the inner
block. 
% Unsafe assignments can also happen indirectly via a function call..
Preempting such unsafe assignments is the \emph{raison d'etre} of the
region type system.

\subsection{Allocation Context and Qualified Region Polymorphism}
\label{sec:alloc-ctxt}

Observe that \C{listIterator} can be called under multiple different
allocation contexts, and each time it returns a \C{ListIterator}
allocated in its allocation context. The iterator object might hold
references to the list, requiring the list to be allocated in a region
that outlives \C{listIterator}\!'s allocation context. However, modulo
this constraint, \C{listIterator} is not concerned about where the
list is allocated. As such, \C{listIterator} is
\emph{region-polymorphic} with respect to (a). its allocation context
argument, and (b). the allocation region of the list, subject to the
constraint that the later outlives the former. We call such region
polymorphism with constraints in \name as \emph{qualified region
polymorphism}. The provision to elide allocation region
specifications, and the ability to infer qualified region-region
polymorphic types are pivotal to interface region-oblivious standard
library code with region-aware application code in \name. 

\subsection{Dynamic (Transferable) Regions}



In a typical dataflow computation, an upstream actor
(e.g., a \C{SELECT} operator) constructs a transferable region, sets
its root to the data structure containing intermediate results, and
then transfers it to a downstream actor (e.g., a \C{COUNT} operator),
which performs further processing. Since a transferable region escapes
the lifetime of the sender, there must be no references from inside of
the transferable region to objects allocated in other memory regions
of the sender. Such references, if exist, may become invalid
references in the context of recipient's address space, jeopardizing
memory safety. \name relies on its region type system to prevent such
unsafe references from being created.

%% USING TEMPORARY STACK REGIONS AND ITS CORRECTNESS

\name lets stack regions to be used as working memory while operating
with the data stored in a transferable region. Consider the following
code, for instance\footnote{For brevity, we drop the region identifier binding part of the
\C{open} expression whenever the identifier is not used.}:
\begin{codejava}
void onReceive(Region<List<String>> rgn)
  open rgn as strList {
    letregion R1 {
      String s = "";
      ListIterator<String> i = strList.listIterator();
      while(i.hasNext()) {
        s += i.getNext();
      }
      print s; //prints "HelloWorld"
    }
  }
  rgn.free();
\end{codejava}
The stack region \C{R1} is being used in the above code to provide
working memory to work with the objects of transferable region
\C{rgn}. Since a transferable region cannot be transferred/freed while
it is still open (Fig~\ref{fig:region-fsm}), \C{rgn} is guaranteed to
outlive the stack region \C{R1} in the above code, making it safe for
the later to contain references to the former. \name therefore allows
such references.

As Fig.~\ref{fig:region-fsm}
indicates, it is not possible to open a region that is already
transferred/freed. The fact that \C{rgn} is open within the block
therefore guarantees that it is not yet transferred/freed, and that
dereferencing \C{strList} within the block is safe. The end of
\C{open} block marks the return of transferable region to the closed
state.  While closed, the region is eligible to be transferred to a
downstream actor, or to be freed.  An actor that receives the
transferable region, receives it in the closed state. It can then
reopen the received region to read its contents, possibly add more
data and transfer it to another actor, or free the region. 

}

