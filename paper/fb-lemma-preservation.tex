\begin{theorem}
\emph{(\textbf{Preservation})}
\label{thm:fb-preservation}
$\forall e, \tau, \rhoenv, \rhomap, \phicx, \rgn$, such that $\rgn \in
\rhoenv$, if $\hastyp{\emptyA,\rgn, \cdot}{e}{\tau}$ and
$\redstoo{(e,\rhomap)}{(e',\rhomap')}$, then 
$\hastyp{\emptyASigp,\rgn, \cdot}{e'}{\tau}$.
\end{theorem}
\begin{proof}
Intros $e$. Induction on $e$. For every subexpressions $e_0$, inductive
hypothesis gives us the following:
\begin{smathpar}
\begin{array}{cr}
  \forall (\tau_0, \rhoenv_0, \rhomap_0, \phicx_0, \rgn_0). \spc 
  (\rgn_0 \in \rhoenv_0)
  \;\conj\; (\hastyp{(dom(\rhomap_0),\rhoenv_0,\cdot,\phicx_0),\rgn_0,
    \cdot}{e_0}{\tau_0}) \;\conj\; (\redstoo{(e_0,\rhomap_0)}
                {(e_0',\rhomap_0')}) & IH1\\
     \Rightarrow \hastyp{(dom(\rhomap_0'),\rhoenv_0,\cdot,\phicx_0),\rgn_0,
    \cdot}{e_0'}{\tau_0} & \\
\end{array}
\end{smathpar}
Cases from induction
\begin{itemize}
\item Case ($e = \unitval$ or $e = x$): Proof is trivial.
\item Case ($e = e_0.f_i$): Intros. Hypothesis:
  \begin{smathpar}
  \begin{array}{cr}
    \rgn \in \rhoenv & H2\\
    \hastyp{\emptyA,\rgn, \cdot}{e}{\tau} & H4\\
    \redstoo{(e,\rhomap)}{(e',\rhomap')} & H6\\
  \end{array}
  \end{smathpar}
  Inverting $H4$:
  \begin{smathpar}
  \begin{array}{cr}
    \hastyp{\emptyA,\rgn, \cdot}{e_0}{\tau'} & H7\\
    \bar{f} :\taubar = \fields(\bound_{\cdot}(\tau')) & H9\\
  \end{array}
  \end{smathpar}
  Inverting $H6$, we get two cases:
  \begin{itemize}
    \item SCase ($\redstoo{(e_0,\rhomap)}{(e_0',\rhomap')}$): In this
    case, $\redstoo{(e_0.f_i,\rhomap)}{(e_0'.f_i,\rhomap')}$. $H7$ and $IH1$ gives:
    \begin{smathpar}
    \begin{array}{cr}
      \hastyp{\emptyASigp,\rgn, \cdot}{e_0'}{\tau'} & H11\\
    \end{array}
    \end{smathpar}
    Proof follows from $H11$ and $H9$.

    \item SCase ($e_0$ is a value $\C{new} \; \fbN(\vbar)$):
    Hypotheses:
    \begin{smathpar}
    \begin{array}{cr}
      \allocRgn(N) \in \rhoenv & H13\\
      \redstoo{(e,\rhomap)}{(v_i,\rhomap)} & H15\\
    \end{array}
    \end{smathpar}
    \end{itemize}
    We need to prove that $\hastyp{(\emptyA,\rgn,\cdot)}{v_i}{\tau_i}$. 
    From $H7$, since $e_0 = \C{new} \; \fbN(\vbar)$:
    \begin{smathpar}
    \begin{array}{cr}
      \hastyp{\emptyA,\rgn, \cdot}{\C{new} \;\fbN(\vbar)}{\tau'} & H16\\
    \end{array}
    \end{smathpar}
    Inverting $H16$ and using $H9$ gives us the proof.

  \item Case ($e = \letregion{\rgn_0}{e_0}$): Intros. Hypothesis:
  \begin{smathpar}
  \begin{array}{cr}
    \rgn \in \rhoenv & H2\\
    \hastyp{\emptyA,\rgn, \cdot}{e}{\tau} & H4\\
    \redstoo{(e,\rhomap)}{(e',\rhomap')} & H6\\
  \end{array}
  \end{smathpar}
  Inverting $H4$:
  \begin{smathpar}
  \begin{array}{cr}
    \rgn_0 \notin \rhoenv & H7\\
    \tywf{\emptyA}{\tau} & H8\\
    \hastyp{\emptyADelcupPhicap{\rgn_0}{\rgn_0}, \rgn_0, \cdot}{e_0}{\tau} & H9\\
  \end{array}
  \end{smathpar}
  Inverting $H6$, we get two cases:
  \begin{itemize}
    \item SCase ($\redstocup{\rgn_0}{(e_0,\rhomap)}{(e_0',\rhomap')}$): In this
    case, $\redstoo{(\letregion{\rgn_0}{e_0},\rhomap)} {(\letregion{\rgn_0}{e_0'},\rhomap')}$. 
    $H9$ and $IH1$ gives:
    \begin{smathpar}
    \begin{array}{cr}
      \hastyp{\emptyADelcupPhicap{\rgn_0}{\rgn_0},\rgn,
          \cdot}{e_0'}{\tau} & H11\\
    \end{array}
    \end{smathpar}
    From $H7$ and $H11$, we can conclude that
    $\hastyp{\emptyASigp,\rgn, \cdot}{\letregion{\rgn_0}{e_0'}}{\tau}$.

    \item SCase ($e_0$ is a value $v_0$): In this case, $\redstoo{(\letregion{\rgn_0}{e_0},\rhomap)}
    {(v_0,\rhomap')}$. From $H2$, $H7-9$, and Lemma~\ref{thm:fb-tywf}, we have:
    \begin{smathpar}
    \begin{array}{cr}
      \hastyp{\emptyA, \rgn, \cdot}{v_0}{\tau} & H13\\
    \end{array}
    \end{smathpar}
    Thus, type is preserved. 
  \end{itemize}

  \item Case ($e = \open{e_a}{\rgn_0}{y}{e_b}$): Intros. Hypotheses:
  \begin{smathpar}
  \begin{array}{cr}
    \rgn \in \rhoenv & H2\\
    \hastyp{\emptyA,\rgn, \cdot}{e}{\tau} & H4\\
    \redstoo{(e,\rhomap)}{(e',\rhomap')} & H6\\
%   \forall (\tau, \rhoenv, \rhomap, \rgn). \rgn \in \rhoenv \conj
%     \hastyp{\emptyA,\rgn, \cdot}{e_0}{\tau} \;
%     \Rightarrow \; \exists(e',\rhomap'). \redstoo{(e_0,\rhomap)}
%                     {(e',\rhomap')} & IH1\\
  \end{array}
  \end{smathpar}
  Inverting $H4$:
  \begin{smathpar}
  \begin{array}{cr}
    \hastyp{\emptyA,\rgn, \cdot}{e_a}{\RgnZ\inang{T}\inang{\toprgn}} & H7\\
    \tywf{\emptyA}{\tau} & H8\\
    \rgn_0 \notin \rhoenv & H9\\
    \hastyp{\emptyASigpDelcup{\rgn_0},\rgn_0,[y \mapsto T@\rgn_0]}{e_b}{\tau} & H11\\
  \end{array}
  \end{smathpar}
  Inverting $H6$, we get many cases:
  \begin{itemize}
    \item SCase ($\redstoo{(e_a,\rhomap)}{(e_a',\rhomap')}$): Since the domain
    of $\rhomap$ monotonically increases during the evaluation, we have:
    \begin{smathpar}
    \begin{array}{cr}
      dom(\rhomap) \subseteq dom(\rhomap') & H13\\
    \end{array}
    \end{smathpar}
    Since strenthening the context trivially preserves typing and
    well-formedness, from $H7-11$ and $H13$, we have:
    \begin{smathpar}
    \begin{array}{cr}
      \hastyp{\emptyASigp,\rgn, \cdot}{e_a}{\RgnZ\inang{T}\inang{\toprgn}} & H15\\
      \tywf{\emptyASigp}{\tau} & H17\\
      \rgn_0 \notin \rhoenv & H19\\
      \hastyp{\emptyASigpDelcup{\rgn_0},\rgn_0,[y \mapsto
      T@\rgn_0]}{e_b}{\tau} & H20\\
    \end{array}
    \end{smathpar}
    From $H15-20$, we have $\hastyp{\emptyASigp,\rgn, \cdot}{e}{\tau}$.

    \item SCase ($e_a$ is a value $v_a$, and $e_b$ steps to $e_b'$): Hypotheses:
    \begin{smathpar}
    \begin{array}{cr}
      v_a = \C{new}\; \RgnZ\inang{T}\inang{\rgn_i}(v_r) & H22\\
      \rgn_i \neq \toprgn & H23\\
      \rhomap(\rgn_i) \neq \XFERRED & H24\\
      \rgn_0 \notin \rhoenv & H26\\
      \redstocup{\rgn_0}{([[\rgn_0/\rgn_i]v_r/y]e_b,
          \rhomap[\rgn_i \mapsto \OPEN])} {(e_b',\rhomap')} & H27\\
      \rhomap'' = \rhomap'[\rgn_i \mapsto \rhomap(\rgn_i)] & H29\\
    \end{array}
    \end{smathpar}
    We need to prove that
    $\hastyp{\emptyASigpp,\rgn,\cdot}{\open{v_a}{\rgn_0}{y}{ e_b'}}{\tau}$.  Note that $dom(\rhomap'') = dom(\rhomap')$. Hence, the proof
    obligation is
    $\hastyp{\emptyASigp,\rgn,\cdot}{\open{v_a}{\rgn_0}{y}{e_b'}}{\tau}$.
    First, since the domain of $\rhomap$ monotonically increases during the
    evaluation, we have:
    \begin{smathpar}
    \begin{array}{cr}
      dom(\rhomap) \subseteq dom(\rhomap') & H31\\
    \end{array}
    \end{smathpar}
    Next, since $e_a = v_a$, from $H7$ and $H22$, we have:
    \begin{smathpar}
    \begin{array}{cr}
      \hastyp{\emptyA,\rgn, \cdot}{\C{new}\;
        \RgnZ\inang{T}\inang{\rgn_i}(v_r)}
        {\RgnZ\inang{T}\inang{\toprgn}} & H33\\
    \end{array}
    \end{smathpar}
    Since $H23$, inversion on $H33$ gives:
    \begin{smathpar}
    \begin{array}{cr}
      \hastyp{\emptyADelcup{\rgn_i},\rgn_i,\cdot}{v_r}{T@\rgn_i} & H34\\
      \rgn_i \notin \rhoenv & H35\\
    \end{array}
    \end{smathpar}
    From $H23$, $H26$, $H34$, $H35$, and Lemma~\ref{thm:fb-renaming}, we have:
    \begin{smathpar}
    \begin{array}{cr}
      \hastyp{\emptyADelcup{\rgn_0},\rgn_0,\cdot}{[\rgn_0/\rgn_i]v_r}{T@\rgn_0} & H36\\
    \end{array}
    \end{smathpar}
    From $H11$, $H36$ and Lemma~\ref{thm:fb-substitution}, we get:
    \begin{smathpar}
    \begin{array}{cr}
      \hastyp{\emptyADelcup{\rgn_0},\rgn_0,
        \cdot}{[[\rgn_0/\rgn_i]v_r/y]e_b}{\tau} & H38\\
    \end{array}
    \end{smathpar}
    $H24$ says that $\rgn_i \in dom(\rhomap)$. Hence, from $H38$:
    \begin{smathpar}
    \begin{array}{cr}
      \hastyp{\emptyASigxDelcup{[\rgn_i \mapsto \OPEN]}{\rgn_0},
          \rgn_0, \cdot}{[[\rgn_0/\rgn_i]v_r/y]e_b}{\tau} & H40\\
    \end{array}
    \end{smathpar}
    From $H40$, $H27$ and $IH1$:
    \begin{smathpar}
    \begin{array}{cr}
      \hastyp{\emptyASigpDelcup{\rgn_0} ,\rgn_0, \cdot}{e_b'}{\tau} & H42\\
    \end{array}
    \end{smathpar}
%   From $H35$, $H26$, and $H42$:
%   \begin{smathpar}
%   \begin{array}{cr}
%     \hastyp{(dom(\rhomap'),\rhoenv \cup \{\rgn_i\},\cdot,true),\rgn_i,
%       \cdot}{[\rgn_i/\rgn_0]e_b'}{[\rgn_i/\rgn_0]\tau} & H44\\
%   \end{array}
%   \end{smathpar}
%   From $H8$ and $H26$, we know that $\rgn_0 \notin \frv(\tau)$. Hence:
%   \begin{smathpar}
%   \begin{array}{cr}
%     \hastyp{(dom(\rhomap'),\rhoenv \cup \{\rgn_i\},\cdot,true),\rgn_i,
%       \cdot}{[\rgn_i/\rgn_0]e_b'}{\tau} & H46\\
%   \end{array}
%   \end{smathpar}
    By strengthening the type context:
    \begin{smathpar}
    \begin{array}{cr}
      \hastyp{\emptyASigpDelcup{\rgn_0} ,\rgn_0,
        \cdot[y \mapsto T@\rgn_0]}{e_b'}{\tau} & H48\\
    \end{array}
    \end{smathpar}
    Since $dom(\rhomap) \subseteq dom(\rhomap')$ (from $H31$), we get the
    following by strengthening the context in $H7-11$:
    \begin{smathpar}
    \begin{array}{cr}
      \hastyp{\emptyASigpp,\rgn, \cdot}{v_a}
        {\RgnZ\inang{T}\inang{\toprgn}} & H50\\
      \tywf{\emptyASigpp}{\tau} & H51\\
    \end{array}
    \end{smathpar}
    From $H48$, $H50$, $H51$, we have the required goal:
    \begin{smathpar}
    \begin{array}{cr}
      \hastyp{\emptyASigp,\rgn,\cdot}{\open{v_a}{\rgn_0}{y}{e_b'}}{\tau}
    \end{array}
    \end{smathpar}
    
    \item SCase ($e_a$ is a value $v_a$, and $e_b$ is a value $v_b$): In this
    case, $\redstoo{(\open{v_a}{\rgn_0}{y}{v_b},\rhomap)} {(v_b,\rhomap)}$. 
    From $H11$:
    \begin{smathpar}
    \begin{array}{cr}
      \hastyp{\emptyADelcup{\rgn_0} ,\rgn_0, [y \mapsto T@\rgn_0]}{v_b}{\tau} & H53\\
    \end{array}
    \end{smathpar}
    Since $\valuee(v_b)$, it has not free variables. Consequently:
    \begin{smathpar}
    \begin{array}{cr}
      \hastyp{\emptyADelcup{\rgn_0} ,\rgn_0,
        \cdot}{v_b}{\tau} & H55\\
    \end{array}
    \end{smathpar}
%   Since $dom(\rhomap) \subseteq dom(\rhomap')$:
%   \begin{smathpar}
%   \begin{array}{cr}
%     \hastyp{(dom(\rhomap'),\rhoenv \cup \{\rgn_0\},\cdot,true),\rgn_0,
%       \cdot}{v_b}{\tau} & H57\\
%   \end{array}
%   \end{smathpar}
    From $H2$, $H8$, $H9$, $H55$ and Lemma~\ref{thm:fb-tywf}, we prove the
    required goal:
    \begin{smathpar}
    \begin{array}{cr}
      \hastyp{\emptyA,\rgn, \cdot}{v_b}{\tau} & \\
    \end{array}
    \end{smathpar}
  % End of cases for open   
  \end{itemize}

  \item Case ($e = \C{new}\; \RgnZT{\toprgn}(e_0)$): 
% Preservation follows trivially from the $IH$ when $e_0$ takes a step. The non-trivial case is when
% $e_0$ is a value $v_0$. 
  Hypotheses:
  \begin{smathpar}
  \begin{array}{cr}
    \rgn \in \rhoenv & H2\\
    \hastyp{\emptyA,\rgn, \cdot}{\C{new}\; \RgnZT{\toprgn}(e_0)}{\tau} & H4\\
    \redstoo{(e,\rhomap)}{(e',\rhomap')} & H6\\
  \end{array}
  \end{smathpar}
  Inverting $H4$, we know that $e_0 = \lambdaexp{\rgn}{\rhoalloc}{}{e_1}$, and:
  \begin{smathpar}
  \begin{array}{cr}
    \hastyp{\vacantA{\rhoalloc},\rhoalloc,\cdot}{e_1} {T@\rhoalloc}& H7\\
%   \hastyp{\emptyA,\rgn, \cdot}{v_0}{\inang{\rhoalloc}\unitZ
%       \xrightarrow{\rgn} T@\rhoalloc} & H7\\
    \fgjtywf{\cdot}{T} & H8\\
%   \rgn_0 \notin \rhoenv & H9\\
%   \hastyp{(dom(\rhomap),\rhoenv \cup \{\rgn_0\},\cdot,true),\rgn_0,[y \mapsto
%   T@\rgn_0]}{e_b}{\tau} & H11\\
  \end{array}
  \end{smathpar}
  From $H7$ and Lemma~\ref{thm:fb-renaming}:
  \begin{smathpar}
  \begin{array}{cr}
    \hastyp{\vacantA{\rgn_i},\rgn_i,\cdot}{[\rgn_i/\rhoalloc]e_1} {T@\rgn_i}& H10\\
  \end{array}
  \end{smathpar}
% Inverting $H7$, we know that $v_0 = \lambdaexp{\rgn}{\rhoalloc}{}{e_1}$, and
% we get the following hypotheses:
% \begin{smathpar}
% \begin{array}{cr}
%   \rhoalloc \notin \rhoenv & H13\\
%   \hastyp{\emptyADelcup{\rhoalloc},\rhoalloc,\cdot}{e}{T@\rhoalloc} & H15\\
% \end{array}
% \end{smathpar}
  Inverting $H6$:
  \begin{smathpar}
  \begin{array}{cr}
    \rgn_i \notin dom(\rhomap) \cup \rhoenv & H17\\
    \rhomap' = \rhomap[\rgn_i \mapsto \CLOSED] & H19\\
    \redstoo{(\C{new}\; \RgnZT{\toprgn}(\lambdaexp{\rgn}{\rhoalloc}{}{e_1}))}
      {(\C{new}\; \RgnZT{\rgn_i}([\rgn_i/\rhoalloc]e_1),\rhomap')} & H21\\
  \end{array}
  \end{smathpar}
  Since strengthening the context preserves typing, strengthening the context for type judgment in
  $H10$ gives us the following:
  \begin{smathpar}
  \begin{array}{cr}
    \hastyp{\emptyASigpDelcup{\rgn_i},\rgn_i,\cdot}{[\rgn_i/\rhoalloc]e}
        {T@\rgn_i} & H24\\
  \end{array}
  \end{smathpar}
% From $H13$, $H17$, $H15$ and Lemma~\ref{thm:fb-renaming}, we get:
% \begin{smathpar}
% \begin{array}{cr}
%   \hastyp{\emptyADelcup{\rgn_i},\rgn_i,\cdot}{[\rgn_i/\rhoalloc]e}
%       {T@\rgn_i} & H23\\
% \end{array}
% \end{smathpar}
% Since $dom(\rhomap) \subseteq dom(\rhomap')$, $H23$ is equivalent to:
% \begin{smathpar}
% \begin{array}{cr}
%   \hastyp{\emptyASigpDelcup{\rgn_i},\rgn_i,\cdot}{[\rgn_i/\rhoalloc]e}
%       {T@\rgn_i} & H24\\
% \end{array}
% \end{smathpar}
  $H19$ implies $\rgn_i \in dom(\rhomap')$. This, and $H17$, $H8$, and $H24$ entail the required goal:
  \begin{smathpar}
  \begin{array}{cr}
    \hastyp{\emptyASigp,\rgn, \cdot}{\C{new}\; \RgnZT{\rgn_i}
        ([\rgn_i/\rhoalloc]e_1)}{\RgnZT{\toprgn}} & \\
  \end{array}
  \end{smathpar}

  \item Case ($e = \C{new} \RgnZT{\rgn_i}(e_0)$, where $\rgn_i \neq \toprgn$):Hypotheses:
  \begin{smathpar}
  \begin{array}{cr}
    \rgn \in \rhoenv & H2\\
    \hastyp{\emptyA,\rgn, \cdot}{\C{new}\; \RgnZT{\rgn_i}(e_0)}{\tau} & H4\\
    \redstoo{(\C{new} \RgnZT{\rgn_i}(e_0),\rhomap)}{(e',\rhomap')} & H6\\
  \end{array}
  \end{smathpar}
  Since $e$ isn't a value,  inverting $H6$ tells us that $\C{new}
  \RgnZT{\rgn_i}(e_0)$ takes a step to $\C{new} \RgnZT{\rgn_i}(e_0')$ when $e_0$
  takes a step to $e_0'$.  The proof for this case is similar to the previous
  case; we invert $H4$ and $H6$, apply inductive hypothesis to derive typing
  judgment for $e_0$ under a context containing $\rgn_i$, and finally apply the
  type rule for $\C{new} \RgnZT{\rgn_i}(e_0')$ (where $\rgn_i
  \neq \toprgn$) to prove the preservation.

  \item Case ($e$ is a lambda expression): $e$ is a value, hence cannot take a
  step. Preservation trivially holds.

  \item Case ($e$ is a method/function call, or a \C{let} expression): Proof
  follows directly from the inductive hypothesis, substitution lemma
  (\ref{thm:fb-substitution}) and renaming lemma (\ref{thm:fb-renaming}).

%End of cases for proof
\end{itemize}

\qed
\end{proof}

