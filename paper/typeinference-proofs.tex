\section{Type Inference: Proofs}

Figs.~\ref{fig:constraint-gen-1} and~\ref{fig:constraint-gen-2} contain the
extra constraint-generation rules not covered in Fig.~\ref{fig:constraint-gen-0}.
We now present a sketch of the proofs of the soundness of type
inference algorithm, and completeness of the constraint solver.

%\begin{theorem}
%\label{thm:constraint-generation-sc}
%Let $C = \consOf{q}$.
%An assignment $\soln$ (for the region and predicate variables in $q$)
%satisfies $C$ iff $q[\soln]$ is well-typed.
%\end{theorem}

\begin{proof}[\textbf{Theorem~\ref{thm:constraint-generation-sc}}]
The correspondence between the static semantics rules and the constraint generation
rules induces a correspondence between derivation trees produced by the
two systems.
% We do not present a formal proof of the theorem.
We can establish the following lemma inductively:
given a derivation tree $\zeta_1$ for
$\exprok {\stdcontext}{e}{\tau}{C}$
and any substitution $\soln$,
we can construct a corresponding \emph{candidate} derivation
tree $\zeta_2$ of $\hastyp {\A[\soln],\env[\soln]}{e[\soln]}{\tau[\soln]}$.
We can show that $\zeta_2$ is a valid derivation tree if $\soln$ satisfies $C$.
(We make use of analogous lemmas for type well-formedness rules as well.)
% \qed
\end{proof}

%\begin{lemma}
%  \label{lemma:gc-is-decomposable}
%  Let $C = \consOf{q}$. Every abduction constraint in $C$ is decomposable.
%\end{lemma}

\begin{proof}[\textbf{Lemma~\ref{lemma:gc-is-decomposable}}]
  By induction over the constraint-generation rules.
  Any context $\A = (\rhoenv,\aenv,\phicx)$ generated by the constraint-generation
  process satisfies the invariant that $\phicx$ is of the form $\varphi \conj \phictxt$
  where $\rhoenv \supseteq \predDeltaMap(\varphi)$.
  The only rule that modifies $\phicx$ is the rule for \C{letregion}
  that adds the set of constraints $\pi_f \outlives \pi$ for every $\pi_f \in \rhoenv$
  as conjuncts to $\phicx$.
% \qed
\end{proof}

We recall the definition of $\predDeltaMap(\varphi)$ first. We work with sets of constraints $C$ that
contain exactly only constraint of the form  $\tywf{\Delta}{\varphi}$ for any predicate variable $\varphi$
and we refer to this $\Delta$ as $\predDeltaMap(\varphi)$. In the sequel, we use the notation
$\deltaPC(\varphi)$ to clarify the dependence on $C$, and will omit the superscript if no confusion
is likely.

%\begin{lemma}
%  \label{lemma:decomposition}
%Consider any decomposable constraint  $\isvalid{\varphi \conj \phi}{\pi_i \outlives \pi_j}$
%where both $\pi_i$ and $\pi_j$ are region constants.
%(a) If $\{ \pi_i, \pi_j \} \subseteq \deltaPC(\varphi)$,
%then $\soln$ satisfies $C \cup \{ \isvalid{\varphi \conj \phi}{\pi_i \outlives \pi_j} \}$
%iff
%$\soln$ satisfies $C \cup \{ \isvalid{\varphi}{\pi_i \outlives \pi_j} \}$
%  (b) If $\{ \pi_i, \pi_j \} \not \subseteq \deltaPC(\varphi)$,
%$\soln$ satisfies $C \cup \{ \isvalid{\varphi \conj \phi}{\pi_i \outlives \pi_j} \}$
%iff
%$\soln$ satisfies $C \cup \{ \isvalid{\phi}{\pi_i \outlives \pi_j} \}$
%\end{lemma}

\begin{proof}[\textbf{Lemma~\ref{lemma:decomposition}}]
  Let $\soln$ be any assignment that satisfies $C$.
  Let $\phi'$ denote $\solnP(\varphi)$.
  The lemma follows once we show that
  $\isvalid{\phi' \conj \phi}{\pi_i \outlives \pi_j}$ iff
  $\isvalid{\phi'}{\pi_i \outlives \pi_j}$ or
  $\isvalid{\phi}{\pi_i \outlives \pi_j}$.
  The reverse implication is trivial, and we establish the forward implication below.

  Assume that $\isvalid{\phi' \conj \phi}{\pi_i \outlives \pi_j}$.
  Recall that we can represent $\phi' \conj \phi$ as a graph, with each vertex
  representing a region identifier, and each outlives-constraint $\pi \outlives \pi'$
  represented as an edge from $\pi$ to $\pi'$. Further,
  $\isvalid{\phi' \conj \phi}{\pi_i \outlives \pi_j}$ holds iff there exists a path
  from $\pi_i$ to $\pi_j$ in this graph.

  Let $\Delta$ denote $\deltaPC(\varphi)$.
  Clearly, $\phi'$ can include an outlives-constraint $\pi \outlives \pi'$ only if
  $\{ \pi, \pi' \} \subseteq \Delta$.
  Thus, $\phi'$ represents a set of edges between vertices in $\Delta$.
  On the other hand, $\phi$ represents a set of edges whose targets are outside $\Delta$
  (from the definition of a decomposable constraint).
  Furthermore, it follows from the definition of a decomposable constraint that
  if $\phi$ includes an outlives-constraint $\pi \outlives \pi'$ for some $\pi \in \Delta$,
  then $\phi$ includes the outlives-constraint $\pi'' \outlives \pi'$ for every $\pi'' \in \Delta$.

  The result follows.
% \qed
\end{proof}

%\begin{theorem}
%  \label{thm:closure}
%Let $C = \consOf{q}$.
%An assignment $\soln$ satisfies $C$ iff $\soln$ satisfies $\satC$.
%\end{theorem}

\begin{proof}[\textbf{Theorem~\ref{thm:closure}}]
  For the forward implication, we use induction to show that every constraint
  added to $\satC$ preserves satisfiability (with respect to an assignment $\soln$).
  The Transitivity and Substitution steps are straightforward. The Abduction Decomposition
  step is justified by Lemmas~\ref{lemma:gc-is-decomposable} and~\ref{lemma:decomposition}.
  The reverse implication is trivial.
% \qed
\end{proof}

\begin{lemma}
  \label{lemma:completely-bound}
Let $C = \consOf{q}$. If $\satC$ is satisfiable, then
every region variable occurring in $\satC$ is bound to some region constant in $\satC$.
\end{lemma}

\begin{proof}
The lemma formalizes the intuition that every region variable will be ``unified'' with
some region constant (by the generated constraints).
%
Region variables are introduced at call-sites and represent the formal region parameters for the call.
Some correspond to region parameters of types of input parameters, and these will be bound by the
actual input parameter expressions. Other region variables correspond to region parameters of the
return type. These are determined by the actual return-expression in the called function/method.
It can be shown that these will be bound to one of ``input region parameters'', provided the called
function/method is typable (\ie, if $\satC$ is satisfiable).

There is one special case where this does not hold: a recursive function
that calls itself in a non-terminating fashion. Such a function never
returns a value, and so the return-value can be typed as anything.
We assume that the return value of such a function is typed to be \C{unit},
which will not introduce any region variables at the call-site of such a function.

% This property follows from several aspects of $\FB$, e.g., it does not admit uninitialized variables.
Note that the region \C{R} where a new object is allocated (by the \C{new} construct) is indicated
by the user (as \C{new@R}). If the user does not specify this, it defaults to the allocation
context region. Thus, this aspect is handled by the frontend, and does not require or use
the constraint solver.

Formally, we can prove via induction that if $\exprok{\stdcontext}{e}{\tau}{C}$ then
every region variable in the result type $\tau$ will be bound to one of the ``live''
region constants (\ie, region constants in $\A$).
% \qed

\end{proof}

\begin{lemma}
  \label{lemma:two-bindings}
Let $C = \consOf{q}$. Assume that $\groundC$ is valid.
Suppose a region variable $\rho$ is bound to two different region constants $\pi_1$
and $\pi_2$ in a context $\varphi \conj \phi$. % in $\satC$.
% Let $\soln$ satisfy $\satC$.
% For any context $\varphi \conj \phi$ that contains $\rho$, we must have
Then, 
(a) $\{ \pi_1, \pi_2 \} \subseteq \predDeltaMap(\varphi)$,
(b) $\isvalid{\thesolnP(\varphi)}{\pi_1 = \pi_2}$.
(c) For any $\soln$ that satisfies $\satC$, we must have $\isvalid{\solnP(\varphi)}{\pi_1 = \pi_2}$.
\end{lemma}

\begin{proof}
  Assume that a region variable $\rho$ is bound to two different
  region constants $\pi_1$ and $\pi_2$ in a context $\varphi \conj \phi$.
  By transitivity, $\satC$ is guaranteed to have the constraints
  $\isvalid{\varphi \conj \phi}{\pi_1 \outlives \pi_2}$
  and $\isvalid{\varphi \conj \phi}{\pi_2 \outlives \pi_1}$.
  By abduction decomposition, one of the following conditions must hold:
  (a) $\{ \pi_1, \pi_2 \} \subseteq \predDeltaMap(\varphi)$ and
  both $\isvalid{\varphi}{\pi_1 \outlives \pi_2}$ and
  $\isvalid{\varphi}{\pi_2 \outlives \pi_1}$ are in $\satC$, or
  (b) $\{ \pi_1, \pi_2 \} \not\subseteq \predDeltaMap(\varphi)$ and
  both $\isvalid{\phi}{\pi_1 \outlives \pi_2}$ and
  $\isvalid{\phi}{\pi_2 \outlives \pi_1}$ are in $\satC$.
  In case (b), $\satC$ will not be satisfiable, so we can ignore this case.
  In case (a), $\varphi$ must be bound to a conjunction that includes both
  $\pi_1 \outlives \pi_2$ and $\pi_2 \outlives \pi_1$ in any satisfying assignment
  for $\satC$.
% \qed
\end{proof}

% \begin{lemma}
% Let $C = \consOf{q}$.
% Let $\satC_1$ denote the subset of $\satC$ obtained by removing all constraints
% that contain a region variable.
% $\satC$ is satisfiable iff $\satC_1$ is satisfiable.
% \end{lemma}
% 
% \begin{proof}
%   The forward implication is trivial.
%   The reverse implication follows from Lemma~\label{lemma:completely-bound}, as shown below.
% \end{proof}

\begin{theorem}
\label{thm:constraint-solver-soundness}
Let $C = \consOf{q}$.
If $\solveCon{C} = \textsc{Some}(\soln)$,
then $\soln$ satisfies $\satC$.
\end{theorem}

\begin{proof}
Assume $\solveCon{C} = \textsc{Some}(\soln)$.

By definition (of $\solveCon{C}$), $\groundC$ (the set of all variable-free constraints in $\satC$)
is satisfiable. Hence, $\soln$ trivially satisfies all constraints in $\groundC$.
%
The definition also ensures that $\soln$ satisfies the well-formedness constraints
for any region variable $\rho$.

Consider any constraint of the form $\isvalid{\varphi}{\pi_i \outlives \pi_j}$ in $\satC$,
where $\pi_i$ and $\pi_j$ are distinct region constants. The definition of $\solnP(\varphi)$
ensures that $\soln$ satisfies this constraint.
Furthermore, $\soln$ must also satisfy the well-formedness constraint
$\tywf{\predDeltaMap(\varphi)}{\varphi}$.
(Otherwise, the application of Abduction Decomposition to this constraint will add
$\isvalid{}{\pi_i \outlives \pi_j}$ to $\satC$. This constraint is invalid, which
contradicts the fact that $\groundC$ is valid.)

Consider any constraint of the form $\isvalid{\varphi \conj \phi}{\pi_i \outlives \pi_j}$ in $\satC$,
It follows from abduction decomposition that either
$\isvalid{\varphi}{\pi_i \outlives \pi_j}$
or
$\isvalid{\phi}{\pi_i \outlives \pi_j}$ must be in $\satC$, which we know is satisfied by $\soln$.
Hence, $\soln$ must satisfy
$\isvalid{\varphi \conj \phi}{\pi_i \outlives \pi_j}$ as well.

This shows that $\soln$ satisfies all outlives-constraints in $\satC$ that have no variables in the consequent.

Consider any constraint of the form $\isvalid{\ell}{F_j(\varphi_j)}$.
Substituting the value $\solnP(\varphi_j)$ reduces this constraint to one of
the form $\isvalid{\ell}{\conj_i \pi_i \outlives \pi_i'}$ where the substitution rule
guarantees $\isvalid{\ell}{\pi_i \outlives \pi_i'}$ is already in $\satC$ and, hence,
satisfied by $\soln$.
Hence, $\soln$ satisfies $\isvalid{\ell}{F_j(\varphi_j)}$ as well.

Consider any constraint of the form $\isvalid{\ell}{r}$ where $r$ contains a single occurrence
of a region variable, say $\rho$.
It follows from Lemma~\ref{lemma:completely-bound} that $\rho$ is bound to some region
constant $\pi$.
Let $r'$ denote $r$ with $\rho$ replaced by $\pi$.
It follows from the transitivity rule that $\satC$ must contain the constraint $\isvalid{\ell}{r'}$.
Since this constraint has no variables on the consequent, we have already shown that $\soln$
must satisfy this constraint ($\isvalid{\ell}{r'}$).
Let $\pi' = \solnR(\rho)$.
We can show (using Lemma~\ref{lemma:two-bindings}) that $\soln$ satisfies $\isvalid{\ell}{\pi = \pi'}$.
Hence, $\soln$ satisfies $\isvalid{\ell}{r}$ as well.

The same idea works for constraints with occurrences of two region variables in the consequent.
% \qed
\end{proof}

\begin{theorem}
\label{thm:constraint-solver-completeness}
Let $C = \consOf{q}$.
If $\solveCon{C} = \textsc{None}$, then $C$ is unsatisfiable.
\end{theorem}

\begin{proof}
We prove the theorem by contradiction.
Assume that $\satC$ is satisfiable.
We will show that the precondition for the first case in the definition of $\textsc{Solve}$
holds and, hence, $\solveCon{C}$ cannot be $\textsc{None}$.

Since $\satC$ is satisfiable, $\groundC$ is trivially valid.
We just need to show that $\thesolnR$ satisfies $\rhoC$.
Recall that $\rhoC$ is the subset of $C$ of well-formedness constraints
on region variables.
So, we just need to show for any constraint $\rho \in \rhoenv$ in $C$ that
$\thesolnR(\rho) \in \rhoenv$.
Since we have assumed that $\satC$ is satisfiable, there exists some $\soln$ that satisfies $\satC$
and, in particular, the constraint $\rho \in \rhoenv$.
Let $\pi$ denote $\soln(\rho)$. Thus, we have $\pi \in \rhoenv$.

Let $\hat{\pi}$ denote $\thesolnR(\rho)$. If $\pi$ and $\hat{\pi}$ are the same region
constant, then $\hat{\pi}\ \in \rhoenv$ and the proof is complete.
%
Otherwise, it follows from the definition of $\thesoln$, that we have some constraint
$\isvalid{\varphi \conj \phi}{\rho = \hat{\pi}}$ in $\satC$.
Since $\soln$ satisfies this constraint, we have
$\isvalid{\solnP(\varphi) \conj \phi}{\pi = \hat{\pi}}$.
We can show that $\{ \pi, \hat{\pi} \} \subseteq \predDeltaMap(\varphi)$
(as in the proof of Lemma~\ref{lemma:two-bindings}).
We can show, by induction over the constraint-generation rules, that if $\Delta$
includes some element of $\predDeltaMap(\varphi)$, then it includes all elements of $\predDeltaMap(\varphi)$.
Hence, it follows that $\hat{\pi} \in \Delta$.
% \qed
% for some predicate variable $\varphi$.
%
% If $\groundC$ is satisfiable, the solver will fail to return an assignment only if the set $S_\rho$ = 
% $\{ \pi \in \regionDeltaMap(\rho) \; | \; \isvalid{\ell}{\rho = \pi} \in \satC \}$ is empty for some region variable $\rho$.
% Consider any such $\rho$.
% Let $\soln$ be a satisfying assignment for $\satC$.
% Let $\pi' = \solnP(\rho)$. Then, we must have $\pi' \in \regionDeltaMap(\rho)$.
% From Lemma~\ref{lemma:completely-bound}, $\rho$ must be bound to some $\pi$ in some context $\ell$.
% Thus, $\isvalid{\ell}{\rho = \pi} \in \satC$.
% Since $\soln$ satisfies this constraint, we have $\isvalid{\ell[\soln]}{\pi' = \pi}$.
% If $\pi$ and $\pi'$ are the same region identifier, we have a contradiction, since
% $\pi \in S_\rho$ and $S_\rho$ cannot be empty.
% If $\pi$ and $\pi'$ are distinct region identifiers, then it follows that the precondition of case (a) must hold
% in Lemma~\ref{lemma:decomposition}, and we must have $\{ \pi, \pi' \} \subseteq \predDeltaMap(\varphi)$.
% We can show, by induction over the constraint-generation rules, that if $\regionDeltaMap(\rho)$ includes
% some element of $\predDeltaMap(\varphi)$, then it includes all elements of $\predDeltaMap(\varphi)$.
% It follows that $\pi \in \regionDeltaMap(\rho)$. Hence, the set $S_\rho$ is non-empty (as it contains $\pi$.
% The result follows.
\end{proof}

%\begin{theorem}
%\label{thm:constraint-solver-sc}
%Let $C = \consOf{q}$.\\
%(Soundness) If $\solveCon{C} = \textsc{Some}(\soln)$, then $\soln$ satisfies $C$.\\
%(Completeness) If $\solveCon{C} = \textsc{None}$, then $C$ is unsatisfiable.
%\end{theorem}

\begin{proof}[\textbf{Theorem~\ref{thm:constraint-solver-sc}}]

Follows from Theorems~\ref{thm:constraint-solver-soundness} and~\ref{thm:constraint-solver-completeness}.
% \qed
\end{proof}

