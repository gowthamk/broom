\documentclass[a4paper,UKenglish]{lipics-v2018}
%This is a template for producing LIPIcs articles. 
%See lipics-manual.pdf for further information.
%for A4 paper format use option "a4paper", for US-letter use option "letterpaper"
%for british hyphenation rules use option "UKenglish", for american hyphenation rules use option "USenglish"
% for section-numbered lemmas etc., use "numberwithinsect"
 
\usepackage{microtype}%if unwanted, comment out or use option "draft"

%\graphicspath{{./graphics/}}%helpful if your graphic files are in another directory

\bibliographystyle{plainurl}% the recommended bibstyle

%%%%%%%%%%%%%%%% OUR PRELIM STUFF BEGINS %%%%%%%%%%%%%%%%%%%%

\usepackage{courier}            % standard fixed width font
\usepackage[scaled]{helvet} % see www.ctan.org/get/macros/latex/required/psnfss/psnfss2e.pdf
% \usepackage{listings}          % format code
\usepackage{enumitem}      % adjust spacing in enums
\usepackage{xspace}
\usepackage{mathtools}
\usepackage{mathpartir}
\usepackage{amssymb}
\usepackage{amsmath}
\usepackage{latexsym}
\usepackage{graphicx}
% \usepackage[usenames,dvipsnames]{color}
\usepackage{listings}
\usepackage{subcaption}
\usepackage{stmaryrd}
\usepackage{comment}
\usepackage{float}
\usepackage{verbatimbox}
\usepackage{breakurl}
% \usepackage[breaklinks,draft=false,backref]{hyperref}
\hypersetup{
     colorlinks   = true,
     citecolor    = blue,
     linkcolor    = blue,
     urlcolor    = blue
}
\usepackage[multiple]{footmisc}

\input{macros}

%%%%%%%%%%%%%%%% OUR PRELIM STUFF ENDS %%%%%%%%%%%%%%%%%%%%

% Author macros::begin %%%%%%%%%%%%%%%%%%%%%%%%%%%%%%%%%%%%%%%%%%%%%%%%
\title{Safe Transferable Regions}

\author{Gowtham Kaki}
       {Purdue University, USA\footnote{Work done during an internship at
Microsoft Research, India.}}
       {gkaki@purdue.edu}
       {}
       {}%mandatory, please use full name; only 1 author per \author macro; first two parameters are mandatory, other parameters can be empty.
\author{G. Ramalingam}
       {Microsoft Research, India}
       {grama@microsoft.com}
       {}
       {}%mandatory, please use full name; only 1 author per \author macro; first two parameters are mandatory, other parameters can be empty.

\authorrunning{G. Kaki and G. Ramalingam}%mandatory. First: Use abbreviated first/middle names. Second (only in severe cases): Use first author plus 'et al.'

\Copyright{Gowtham Kaki and G. Ramalingam}%mandatory, please use full first names. LIPIcs license is "CC-BY";  http://creativecommons.org/licenses/by/3.0/

\subjclass{
	\ccsdesc[500]{Software and its engineering~Allocation / deallocation strategies};\xspace
	\ccsdesc[500]{Software and its engineering~Object oriented languages};\xspace
	\ccsdesc[500]{Software and its engineering~Data flow languages};\xspace
	\ccsdesc[300]{Software and its engineering~Software verification and validation};\xspace
}% mandatory: Please choose ACM 1998 classifications from http://www.acm.org/about/class/ccs98-html . E.g., cite as "F.1.1 Models of Computation". 
\keywords{Memory Safety, Formal Methods, Type System, Type Inference,
Regions, Featherweight Java}% mandatory: Please provide 1-5 keywords

% Author macros::end %%%%%%%%%%%%%%%%%%%%%%%%%%%%%%%%%%%%%%%%%%%%%%%%%

\acknowledgements{We are grateful to Kapil Vaswani, Dimitrios
Vytiniotis, Michael Isard, and Steven Hand for their valuable inputs
during the initial stages of this project. We would also like to thank
the anonymous reviewers for their careful scrutiny and insightful
comments that helped improve the quality of this paper.}%optional

%Editor-only macros:: begin (do not touch as author)%%%%%%%%%%%%%%%%%%%%%%%%%%%%%%%%%%
\EventEditors{Todd Millstein}
\EventNoEds{1}
\EventLongTitle{32nd European Conference on Object-Oriented Programming (ECOOP 2018)}
\EventShortTitle{ECOOP 2018}
\EventAcronym{ECOOP}
\EventYear{2018}
\EventDate{July 16--21, 2018}
\EventLocation{Amsterdam, Netherlands}
\EventLogo{}
\SeriesVolume{109}
\ArticleNo{11}
% Editor-only macros::end %%%%%%%%%%%%%%%%%%%%%%%%%%%%%%%%%%%%%%%%%%%%%%%

\begin{document}

\maketitle

\begin{abstract}
Recent work in high-performance systems written in managed languages
(such as Java or C\#) has shown that garbage-collection can be a
significant performance bottleneck.  A class of these systems, focused
on big-data, create numerous large data structures with well-defined
lifetimes.  In this paper, we present a language and a memory
management scheme based on user-managed memory regions (called
\emph{transferable regions}) that allow programmers to exploit the
knowledge of data structures' lifetimes to achieve significant
performance improvements.

Manual memory management is susceptible to the usual perils of
dangling pointers. A key contribution of this paper is a
refinement-based region type system that ensures the memory safety of
C\# programs in the presence of transferable regions. We complement
our type system with a type inference algorithm that infers principal
region types for first-order programs, and practically useful types
for higher-order programs. This eliminates the need for programmers to
write region annotations on types, while facilitating the reuse of
existing C\# libraries with no modifications. Experiments demonstrate
the practical utility of our approach.

\end{abstract}

\newcommand{\TODO}[1]{\textbf{TODO: #1}} \newcommand{\eg}{\emph{e.g.}}
\newcommand{\ie}{\emph{i.e.}}

\section{Introduction} \label{sec:introduction}

Consider the example, from~\cite{Broom:HotOS}, shown in
Fig.~\ref{fig:motivating-eg}.  This code represents the logic for a
streaming query operator.  The operator receives a stream of input
messages, each associated with a time (window) $t$, processed by
method \texttt{onReceive}.  Each input message contains a list of
inputs, each of which is processed by applying a user-defined function
to create a corresponding output.  The operator may receive multiple
messages with the same timestamp (and messages with different
timestamps may be delivered out of order).  A timing-message (an
invocation of method \texttt{OnNotify}) indicates that no more input
messages with a timestamp $t$ will be subsequently delivered.  At this
point, the operator completes the processing for time window $t$ and
sends a corresponding output message to its successor.

\begin{figure}[t!]
\begin{numcodejava}
class SelectVertex<TIn, TOut> {
  Func<TIn, TOut> selector;
  Dictionary<Time, List<TOut>> map;
  ...
  void onReceive(Time t, List<TIn> inList) {
    if (!map.ContainsKey(t)) map[t] = new List<TOut>();
    foreach (TIn input in inList) {
      TOut output = selector(input);
      map[t].add(output); } }
  void onNotify(Time t) {
     List<TOut> outList = map[t];
     map.Remove(t);
     transfer(successorId, t, outList); }
}
\end{numcodejava}
\caption{\C{SELECT} dataflow operator}
\label{fig:motivating-eg}
\vspace*{-0.2in}
\end{figure}


This example is an instance of a general pattern, where a producer
creates a data-structure and passes it to a consumer. In a system
where most of the computation takes this form, and these
data-structures are very large, as is the case with many streaming
big-data analysis systems, garbage collection overhead becomes
significant~\cite{Broom:HotOS}. This is a cause of concern because,
firstly, in a distributed dataflow system, GC overhead at one node can
have compounding adverse effect on the performance of other nodes.
This can happen, for instance, when a GC is triggered at an upstream
actor that is about to send a message, while the downstream actors are
blocked waiting for messages. Secondly, much of the GC overhead
results from the collector performing avoidable or unproductive work.
For example, in the process executing the code from
Fig.~\ref{fig:motivating-eg}, GC might repeatedly traverse the \C{map}
data-structure, although its objects cannot be collected until a
suitable timing message arrives.

% actors.
% performance degradation due to
% GC overhead becomes aggravated in the distributed 
% For example, a consumer process may trigger a
% GC as soon as it receives a large data-structure from a producer
% process, which has just finished its GC. In such case, consumer's GC
% traverses the entire data-structure performing little or no work.

% as the collector scans these large data-structures For instance, in
% the above example, the list at \C{map[t]} cannot be collected until
% the the timing message arrives, and the list is transferred.
% \C{transfer} operation is complete.  using alternative memory
% management techniques.

An important observation, in the context of processes of kind
described above, is that the data-structures exchanged between them
have can be partitioned into sets of fate-sharing objects with common
lifetimes, which makes them good candidates for a region-based memory
management discipline. A region is a block of memory that is allocated
and freed in one shot, consuming constant time. A region may contain
one or more contiguous range of memory locations, and individual
objects may be dynamically allocated within the region over time,
while they are deallocated en masse when the region is freed.  Thus, a
region is a good fit for a set of fate-sharing objects.
% that is used to realize a data-structure with a well-defined
% lifetime. 
In this example, the output to be constructed for each time window $t$
can be implemented using a separate region.

Region-based memory management, both manual as well as automatic, has
been known for a long time. Manual region-based memory management
suffers from the usual challenges, namely the potential for dangling
references and lack of memory safety. Automatic region-based memory
management systems guarantee memory safety, but impose various
restrictions.  Tofte and Talpin's system~\cite{tofte94,tofte97} , for
example, uses lexically scoped regions.  At runtime, the set of all
regions (existing at a point in time) forms a stack. Thus, the
lifetimes of all regions must be well-nested: it is not possible to
have two regions whose lifetimes overlap, with neither one's lifetime
contained within the other.  Unfortunately, the data-structures in the
above example do not satisfy this restriction (as the output messages
for multiple time windows may be simultaneously live, without any
containment relation between their lifetimes).  We refer to regions
with lexically scoped lifetimes as \emph{stack regions} and to regions
that do not have such a lexically scoped region as \emph{dynamic}
regions.

The goal of this work is a \emph{memory-safe} region-based memory
management technique that supports dynamic regions as first-class
objects. Though we focus, in this paper, on memory-safety, our
approach offers other benefits.  When a producer transfers a
data-structure to a consumer, in a shared-memory context , our
approach provides an ownership discipline that ensures that the
producer cannot subsequently modify the data-structure. When the same
transfer happens in a distributed setting, the system guarantees that
the data-structure is self-contained and can be safely transmitted
across address spaces.  We refer to dynamic regions that provide these
guarantees as \emph{transferable} regions.

The key memory safety property we wish to ensure is that there are no
dangling references: i.e., a reference to an object that has been
freed.  A key invariant and restriction that we utilize to achieve
this is that no object $o_1$ in one dynamic region $D_1$ can contain a
pointer to an object $o_2$ in another dynamic region $D_2$.  In cases
where this is too restrictive, the user has to resort to (deep)
copying $o_2$ to $D_1$.  This is conceptually quite reasonable given
the perspective that $D_1$ is a self-contained data-structure.
Furthermore, this is quite analogous, from a performance perspective,
to the copying that happens when a garbage collector promotes an
object from a lower generation to a higher generation.  From this
perspective, a region may be seen as a user-controlled generation.

This ensures that we can safely free one dynamic region (\eg, the
input-message in our example) without affecting the validity of
pointers in another dynamic region (\eg, the output-message).
However, this is not sufficient! When processing the input-message
$D$, we necessarily will have to create pointers that point to objects
that are internal to $D$. Such pointers may be stack-allocated or even
reside in the heap (\eg, consider the iterator object used to iterate
over the list in the input-message).  How do we ensure the safety of
these pointers?

We use \emph{stack regions} (regions with lexically-scoped lifetimes),
as well as the stack, for such temporary pointers (to objects inside a
dynamic region) and use the following protocol to ensure safety of
such pointers.  To work with a dynamic region $D$, we must first
\emph{open} region $D$, to indicate that $D$ should not be freed
during this period.  When the processing of the region is complete, we
\emph{close} the region $D$, to indicate that it is safe to free $D$.
In general, a dynamic region may be opened and closed multiple times
before it is finally freed.

Our type-system permits the creation of pointers to internal objects
of \emph{open dynamic} regions, but uses lexical scoping to ensure
that such pointers are no longer live when the region is closed.
Thus, such pointers may be either stack-allocated or reside in a
region whose lifetime is statically guaranteed to be contained within
the interval when $D$ is open.  As a consequence, the system ensures
the strong invariant that for a \emph{closed dynamic} region $D$ no
pointer from outside $D$ points to an internal object of $D$.

The final piece required to ensure memory-safety is ensuring that the
program correctly follows the above-mentioned protocol for dynamic
regions: \eg, an open region should not be freed and, dually, a freed
region should not be opened.  (A complete typestate specification of
the protocol is presented later.)

Dynamic regions are first-class objects in our system. For instance,
in our running example, we would like to use a dictionary that stores
dynamic regions. However, this means that the program may create
multiple aliases (pointers) to the same region, which, in turn, means
that statically checking if the program correctly follows the dynamic
region protocol is hard (undecidable, in fact). Safety can be ensured
by using, \eg, linear typing to prevent the creation of aliases, but
this would be quite restrictive. Hence, in our system, we dynamically
check for adherence to this protocol.

In summary, our approach reduces memory safety to two sub-problems:
(a) Ensuring  low-level invariants about pointers to (fine-grained)
objects inside regions and (b) Ensuring higher-level (typestate)
invariants about the regions themselves.  We use a type system to
statically ensure the first set of invariants. We use a dynamic check
to ensure that the second set of invariants are satisfied at runtime.
%
This is a novel aspect of our system. We believe that this is a
reasonable choice: dynamically checking for the second category of
invariants is easy and does not pose a performance concern (unlike,
\eg, the first category of constraints); a static typechecker that
rules out the second category of violations would be very restrictive
from an expressiveness perspective; finally, the human effort required
to ensure the second set of invariants is not as burdensome (as
regions are coarse-grained objects manipulated infrequently), \eg, as
it would be for the first set of invariants.

\subsection*{Contributions}

The paper makes the following contributions:

\begin{itemize} \item We present \name, a \csharp-like typed
object-oriented language that eschews garbage collection in favour of
programmer-managed memory regions . \name extends its core language,
which includes \emph{lambdas} (higher-order functions) and
\emph{generics} (parametric polymorphism), with constructs to create,
manage and destroy static and dynamic memory regions. Dynamic regions
are first-class values in \name.
  % Experiments demonstrate that   \naiad~\cite{naiad} dataflow
  % program using programmer-managed memory regions outperform their
  % GC counterparts by a margin of upto 59\%. 

  \item \name is equipped with a region type system that statically
  guarantees safety of all memory accesses in a well-typed program,
  provided that certain typestate invariants on regions hold.  The
  later invariants are enforced via simple runtime checks.
% The correctness of region usage is checked efficiently at run-time.
% The overhead of checking these conditions is less than \GK{x\%} in
% our experiments with \naiad workloads.

  \item We define an operational semantics for \name, and a type
  safety result that clearly defines and proves safety guarantees
  described above.

  \item We describe a region type inference algorithm for \name that
  (a). completely eliminates the need to annotate \name programs with
  region types, and (b). enables seamless interoperability between
  region-aware \name programs and legacy standard library code that is
  region-oblivious. The cornerstone of our inference algorithm is a
  novel constraint solver that performs abduction in a partial-order
  constraint domain to infer weakest solutions to recursive
  constraints.

  \item We describe an implementation of \name frontend in OCaml,
  along with experimental results, and case studies where the region
  type system was able to identify unsafe memory accesses statically.
  
\end{itemize}

%The rest of the paper is organized as follows. The next section
%presents an informal overview of \name, and motivates the need for a
%region type system.  \S~\ref{sec:type-system} formalizes the type
%system and its safety guarantees. The type inference algorithm is
%described in \S~\ref{sec:type-inference}. \S~\ref{sec:csolve} focuses
%on \csolve, the constraint solving algorithm, and its correctness
%guarantees.  Implementations details and practical extensions to the
%type system are described in \S~\ref{sec:implementation}.
%\S~\ref{sec:evaluation} presents experimental evaluation and case
%studies.  \S~\ref{sec:related} discusses the related work, and
%\S~\ref{sec:conclusion} concludes.


\newcommand{\COMMENT}[1]{}

\section{An Informal Overview of \name} \label{sec:overview}

\name enriches a simple object-oriented language (supporting
parametric polymorphism and lambdas) with a set of region-specific
constructs.  In this section, we present an informal overview of these
region-specific constructs.

\subsection{Using Regions in \name}
\label{sec:alloc-ctxt}

\paragraph{Stack Regions} The ``\C{letregion R \{ S \}}'' construct
creates a new stack region, with a static identifier \C{R}, whose
scope is restricted to the statement \C{S}. The semantics of
\C{letregion} is similar to Tofte and Talpin~\cite{tofte94}'s
\C{letregion} expression: objects can be allocated by \C{S} in the
newly created region while \C{R} is in scope, but the region and all
objects allocated within it are freed at the end of \C{S}.

\paragraph{Object Allocation} The ``\C{new@R T()}'' construct creates
a new object of type \C{T} in the region \C{R}. The specification of
the allocation region \C{R} in this construct is optional.  At
runtime, \name maintains a stack of \emph{active} regions, and we
refer to the region at the top of the stack as the \emph{allocation
context}. The statement \C{new T()} allocates the newly created object
in the current allocation context.
%
This is important as it enables \name applications to use existing
region-oblivious C\# libraries. In particular, given a  C\# library
function \C{f} (that makes no use of \name's region constructs), the
statement ``\C{letregion R \{ f(); \}}'' invokes \C{f}, but has the
effect that all objects allocated by this invocation are allocated in
the new region \C{R}.

\paragraph{Transferable Regions} 
% \name's \emph{transferable regions} are an encapsulation of a
% data-structure that can be transferred between autonomous entities
% (\eg, between two concurrently executing threads or actors).  Hence,
% unlike stack regions, transferable regions are not constrained to have
% a lexically scoped lifetime.  (Hence, we also refer to them as
% \emph{dynamic} regions.)

Transferable regions are first class values of \name: they are objects
of the class \C{Region}, they are created using the \C{new} keyword,
and can be passed as arguments, stored in data structures, and
returned from methods.  A transferable region is intended to
encapsulate a single data-structure, consisting of a collection of
objects with a distinguished root object of some type \C{T}, which we
refer to as the region's \emph{root} object.  The class \C{Region} is
parametric over the type \C{T} of this root object.

The \C{Region} constructor takes as a parameter a function that
constructs the root object: it creates a new region and invokes this
function, with the new region as the allocation context, to create the
root object of the region. The following code illustrates the
creation of a transferable region, whose root is an object of type
\C{A}.
% (\C{r}) of type \C{T}:
\begin{codejava} 
  Region<A> rgn = new Region<A>(() => new A())
\end{codejava} 
In the above code, \C{rgn} is called the \emph{handler} to the newly
created region, and is required to read the contents of the region, or
change its state. The class \C{Region} offers two methods: a \C{free}
method that deallocates the region (and all the objects allocated
within it), and \C{transfer} method that transfers the region to a
(possibly remote) consumer process. 
% It is an abstraction of two possible forms of
% transfer: a transfer between two processes in a shared memory setting
% or a transfer between two processes in a distributed, message-passing,
% setting. 
% The precise semantics of \C{transfer} are unimportant in the
% context of the region type system and we will not discuss them
% further. \dv{this is a bit counter-intuitive.} 

% \begin{figure} \begin{codejava} public class Region<T> { public
% Region(Func<T> mkRoot); public void free(); public void transfer();
% } \end{codejava} \caption{The type signature of the \C{Region}
% class} \label{fig:region-class} \end{figure}

% The type signature of the class \C{Region} appears in
% Fig.~\ref{fig:region-class}.  The method \C{free} deallocates the
% region (including all objects allocated within it).  The method
% \C{transfer} transfers the region to a downstream actor as
% determined by the run-time.  It is an abstraction of two possible
% forms of  transfer: a transfer between two actors in a shared memory
% setting or a transfer between two actors in a distributed,
% message-passing, setting.  The precise semantics of \C{transfer} are
% unimportant in the context of the region type system and we will not
% discuss them further.

\paragraph{Open and Closed Regions} A transferable region must be
explicitly \emph{opened} using \name's \C{open} construct in order to
either read or update or allocate objects in the region.
Specifically, the construct ``\C{open rgn as v@R \{ S \}}'' does the
following: (a). It opens the transferable region handled by \C{rgn}
for allocation (i.e., makes it the current allocation context), (b).
binds the identifier \C{R} to this open region, and (c). initializes
the newly introduced local variable \C{v} to refer to the root object
of the region.  The \C{@R} part of the statement is optional and may
be omitted.  The \C{open} construct is intended to simplify the
problem of ensuring memory safety, as will be explained soon.  We
refer to a transferable region that has not been opened as a
\emph{closed} region. A transferable region can only be transferred or
freed when it is in closed state. The acceptable state transition
discipline over the lifetime of a transferable region is described
in Fig.~\ref{fig:region-fsm}. Enforcement of this discipline is done
at runtime.

\paragraph{Motivating Example}
% We now illustrate how the features of \name can be used to 
Fig.~\ref{fig:motivating-eg-in-broom} shows how the motivating example
of Fig.~\ref{fig:motivating-eg} can be written in \name.  The
\C{onReceive} method receives its input message in a
\emph{transferred} region (\ie, a \emph{closed} region whose ownership
is transferred to the recipient).  Line 7 creates a new region to
store the output for time \C{t}, initializing it to contain an empty
list.  Line 9 opens the input region to process it\footnote{We omit
\C{@R} annotation in \C{open} when we don't need \C{R}.}.  Line 10
creates a stack region \C{R0}.  Thus, the temporary objects created by
the iteration in line 11, for example, will be allocated in this stack
region that lives just long enough.  We open the desired output region
in line 12, so that the new output objects created by the invocation
of \C{selector} in line 13 are allocated in the output region.
Finally, the input region is freed in line 15. The output region at
\C{map[t]} stays as along as input messages with timestamp \C{t} keep
arriving. When the timing message for \C{t} arrives, the \C{onNotify}
method transfers the \C{outRgn} at \C{map[t]} to a downstream actor.
% After the transfer, \C{SelectVertex} is no longer allowed to access
% the \C{outRgn}, facilitating its deallocation.

\begin{figure}[t!]
\begin{numcodejava}
class MapVertex<TIn, TOut> {
  Func<TIn, TOut> userDefinedFunction;
  Dictionary<Time, List<TOut>> map;
  ...
  void onReceive(Time t, Region<List<TIn>> inRgn) {
    if (!map.ContainsKey(t))
       map[t] = new Region<List<TOut>> (
                  () => new List<TOut>());
    open inRgn withroot inMsg {
      letregion R0 {
        foreach (TIn input in inputList) {
          openalloc map[t] withroot outMsg {
            TOut output = userDefinedFunction(input);
            outMsg.add(output);
          }
        }
      }
    }
    inRgn.free();
  }
  void onNotify(Time t) {
     Region<List<TOut>> outputRgn = map[t];
     map.Remove(t);
     successor.onReceive(t, outputRgn.transfer()); 
  }
}
\end{numcodejava}
\caption{\C{SELECT} dataflow operator in \name}
\label{fig:motivating-eg-in-broom}
\end{figure}


% \subsection{Memory Safety} \label{sec:memory-safety}

% Our goal is a type system that can ensure the memory safety of
% programs that use the region constructs described above.  The key to
% memory safety in \name is the following restriction: an object $o_1$
% in a region $R_1$ is allowed to store a pointer to an object $o_2$ in
% a region $R_2$ only if $R_2$ is guaranteed to outlive $R_1$.  (A
% similar restriction applies in the case where $o_1$ is a
% stack-allocated variable.)

% Enforcing this restriction is simple in the case of stack regions
% since the outlives relation between stack regions can be inferred from
% their lexical nesting. Unfortunately, inferring outlives relations
% in presence of transferable regions is not easy.  \name imposes the following
% protocol on the use of transferable regions to help simplify this
% check.

% A transferable region (that has not been freed or transferred) can be
% in one of two possible states, \emph{open} or \emph{closed}. A newly
% created region is in the closed state.  A region must be opened, using
% the open construct (as explained previously), in order to read or
% update or allocate an object within that region.  An open region
% cannot be freed or transferred.  In particular, an open region is
% guaranteed to be live for the entire duration of the open construct.
% This allows the type system to infer a valid outlives relation between
% the opened region and any stack region that is nested within the open
% construct.

\begin{figure} 
\includegraphics[scale=0.45]{region-fsm.png}
\caption{The lifetime of a dynamic (transferable) region in \name}
\label{fig:region-fsm} 
\vspace*{-0.1in} 
\end{figure}

% The protocol for transferable regions is presented as a finite state
% machine in Fig.~\ref{fig:region-fsm}.  
% % A transferable region starts
% % its lifetime in a \emph{closed} state, when it is created.
% %
% The safety of memory accesses in \name is now subject to the condition
% that every transferable region correctly follows the state transition
% discipline of Fig.~\ref{fig:region-fsm}. Under this condition, \name's
% region type system statically guarantees the safety of all memory
% accesses.
% % In other words, the type system reduces the problem of ensuring
% % memory safety in \name programs to the problem of enforcing the
% % state transition discipline for transferable regions.

% In \name, this enforcement is done at runtime by explicitly keeping
% track of the \emph{current state} for $\RgnZ$ objects, and checking
% the validity of every open, transfer, or free operation and throwing
% an exception if it is invalid.  
% % The challenge in enforcing this
% % discipline statically is that transferable regions are first-class
% % objects. Hence, the program can create multiple aliases for the same
% % region, \eg, open it via one alias and free it via another.  Typestate
% % verification in the presence of aliases is hard.  The checking can be
% % done statically by preventing the creation of aliases using, \eg,
% % linear types or unique types. However, this would be quite
% % restrictive, in terms of expressiveness.
% As explained previously, this is a reasonable trade-off in the context
% of \name, as regions are coarse-grained objects, which are manipulated
% infrequently, when compared to fine-grained objects that reside inside
% these regions. Therefore, runtime overhead of checking the region's
% state transition discipline is acceptable.

\begin{figure*}[t!]
%
%\begin{minipage}
\begin{smathpar}
\renewcommand{\arraystretch}{1.2}
\begin{array}{lr} 
\multicolumn{2}{c}{
  {\rgn} \in \mathtt{Static \; region \; ids} \qquad
  {\rho} \in \mathtt{Region \; variables} \qquad
  {\tyvar, \tyvarb} \in \mathtt{Type \; variables} \qquad
  {m} \in \mathtt{Method \; names} \qquad
  {x,y,f} \in \mathtt{Variables \; and \; fields} }\\
\begin{array}{lclcl}
  cn & \in & \M{Class \; names} & \coloneqq & \ObjZ \ALT \RgnZ \ALT A \ALT B\\
  T  & \in & \M{FGJ \; types} & \coloneqq & \tyvar \ALT  \fgjN \ALT \unitZ
       \ALT \bar{T} \rightarrow T \\
  D  & \in & \M{Classes} & \coloneqq & 
       \C{class} \; cn\inang{\bar{\tyvar} \extends \bar{\fgjN}} 
                      \inang{\rhoalloc \rhobar \,|\, \phi}\extends \fbN 
                      \{\bar{\tau} \; \bar{f};\; k\; \bar{d}\}\;\\
  k  & \in & \M{Constructors} & \coloneqq & 
       cn(\bar{\tau} \; \bar{x})\{\C{super}(\bar{x}); \;
                                  \C{this}.\bar{f}\,=\,\bar{x};\}\\
  d  & \in & \M{Methods} & \coloneqq & 
       \tau \; m\inang{\rhoalloc \rhobar \,|\, \phi} (\taubar \; \xbar)
       \{\C{return}\;e;\}\\
\end{array}
\begin{array}{lclcl}
  \fgjN & \in & \M{FGJ \; class \; types} & \coloneqq & cn\inang{\tbar}\\
  \fbN  & \in & \M{FB \; class \; types} & \coloneqq & 
       cn\inang{\tbar}\inang{\ralloc \rbar} \\
  \tau &\in& \M{types} & \coloneqq & T@\rgn  
        \ALT \fbN \ALT \unitZ \\
       &   & & & \ALT \inang{\rhoalloc\rhobar \,|\, \phi}\bar{\tau}
        \xrightarrow{\rgn} \tau \\
  \phi,\Phi &\in& \M{Constraints} & \coloneqq & true 
        \ALT \rho \outlives \rho \\
        & & & & \ALT \rho = \rho \ALT \phi \conj \phi\\
  % v  & \in & \M{Value\;expressions} & \coloneqq & x \ALT \C{new} \; \fbN(\vbar)\\
\end{array}\\
\begin{array}{lclcl}
e  & \in & \M{Expressions} & \coloneqq & \unitval \ALT x \ALT e.f 
     \ALT e.m\inang{\ralloc \rbar}(\ebar) \ALT \C{new}\;\fbN(\ebar) 
     \ALT \lambdaexp{\ralloc}{\rhoalloc\rhobar \,|\, \phi}{\xbar:\taubar} {e}
     \ALT e\inang{\ralloc\rbar}(\bar{e})\\
%    \ALT (\tau)\, e\\
%    \ALT e\,;\,e  \\
   & & & & \ALT \letexp{x}{e}{e}
           \ALT \letregion{\rho}{e} 
           \ALT \open{e}{\rgn}{y}{e} \\
%   & & & & \ALT \unpackexp{\rgn}{x}{e}{e} \ALT e\,;\,e\\
% s  & \in & \M{Statements} & \coloneqq & \tau\;x\;=\;e \ALT x\;=\;e
%      \ALT e.f\;=\;e 
%      \ALT \C{open}\;x\;\C{withroot}\;y\;\{s\} \\
%     & & & & \ALT \C{openalloc}\;x\;\C{withroot}\;y\;\{s\} 
%      \ALT (\rho,\tau\;x)\;=\;\C{unpack}\;e \ALT s\,;\,s\\
\end{array}
\end{array}
\end{smathpar}

\caption{\fbname: Syntax}
\label{fig:fb-syntax}
\end{figure*}


\paragraph{Cloning} Note that in the example from
Fig.~\ref{fig:motivating-eg-in-broom} the object returned by the
\C{selector} (on Line 13) should not contain any references to the
input object, since the input region, where the object resides, will
be freed at the end of the method. If there is a need for the output
object to point to subobjects of the input object, such subobjects
must be cloned (to copy them from the input region to the output
region).  Fortunately, \name's region type system
(\S~\ref{sec:type-system}) is capable of capturing such nuances in the
type of \C{selector} and the type checker will ensure correctness.
Furthermore, the type can be automatically inferred by \name's region
type inference (\S~\ref{sec:type-inference}), which can perform the
above reasoning on behalf of the programmer.


%\begin{codejava} Region<string> rgn = new Region<List<String>> (() =>
%new List<String>()); open rgn as  strList@R0 {
%strList.addAtHead("World"); strList.addAtHead("Hello"); }
%rgn.transfer(); \end{codejava}

\COMMENT{ \subsection{Stack Regions and Outlives Relation}

The \C{letregion} blocks can be nested, leading to a stack of regions
that are deallocated in the reverse order in which they are allocated.
Following ~\cite{cyclonepldi02}, we therefore call regions introduced
by \C{letregion} expressions as \emph{stack regions}. The stack
discipline induces an \emph{outlives} relationship among regions
created by nested \C{letregion}s, where the region introduced by the
outer \C{letregion} is guaranteed to outlive the one introduced by the
inner \C{letregion}. It is therefore safe to refer to an object
allocated in outer region from the inner region, but the converse is
not true. For example, consider the following code (assume that class
\C{A} has field \C{x} of type \C{Object}): \begin{center}
\begin{codejava} letregion R0 { A a0 = new@R0 A(); letregion R1 { A a1
= new@R1 A(); a1.x = new@R0 Object(); // safe & legal a0.x = new@R1
Object(); // unsafe & illegal ...  } ...  } \end{codejava}
\end{center} The code creates two stack regions with identifiers
\C{R0} and \C{R1}, where the region \C{R0} outlives the region \C{R1}
(denoted as $\C{R0} \outlives \C{R1}$).  Objects \C{a0} and \C{a1} are
allocated in regions \C{R0} and \C{R1}, respectively. The first
assignment statement assigns to \C{a1.x} an object allocated in outer
region (\C{R0}). This assignment is safe as \C{a1.x} refers to a
longer living object, hence is guaranteed to be a valid reference
throughout the lifetime of \C{a1}.  In contrast, the second assignment
is unsafe, as it assigns to \C{a0.x} an object, whose lifetime is
shorter than the lifetime of \C{a0}, making it unsafe to dereference
\C{a0.x} outside the inner block. 
% Unsafe assignments can also happen indirectly via a function call..
Preempting such unsafe assignments is the \emph{raison d'etre} of the
region type system.

\subsection{Allocation Context and Qualified Region Polymorphism}
\label{sec:alloc-ctxt}

Observe that \C{listIterator} can be called under multiple different
allocation contexts, and each time it returns a \C{ListIterator}
allocated in its allocation context. The iterator object might hold
references to the list, requiring the list to be allocated in a region
that outlives \C{listIterator}\!'s allocation context. However, modulo
this constraint, \C{listIterator} is not concerned about where the
list is allocated. As such, \C{listIterator} is
\emph{region-polymorphic} with respect to (a). its allocation context
argument, and (b). the allocation region of the list, subject to the
constraint that the later outlives the former. We call such region
polymorphism with constraints in \name as \emph{qualified region
polymorphism}. The provision to elide allocation region
specifications, and the ability to infer qualified region-region
polymorphic types are pivotal to interface region-oblivious standard
library code with region-aware application code in \name. 

\subsection{Dynamic (Transferable) Regions}



In a typical dataflow computation, an upstream actor (e.g., a
\C{SELECT} operator) constructs a transferable region, sets its root
to the data structure containing intermediate results, and then
transfers it to a downstream actor (e.g., a \C{COUNT} operator), which
performs further processing. Since a transferable region escapes the
lifetime of the sender, there must be no references from inside of the
transferable region to objects allocated in other memory regions of
the sender. Such references, if exist, may become invalid references
in the context of recipient's address space, jeopardizing memory
safety. \name relies on its region type system to prevent such unsafe
references from being created.

%% USING TEMPORARY STACK REGIONS AND ITS CORRECTNESS

\name lets stack regions to be used as working memory while operating
with the data stored in a transferable region. Consider the following
code, for instance\footnote{For brevity, we drop the region identifier
binding part of the \C{open} expression whenever the identifier is not
used.}: \begin{codejava} void onReceive(Region<List<String>> rgn) open
rgn as strList { letregion R1 { String s = ""; ListIterator<String> i
= strList.listIterator(); while(i.hasNext()) { s += i.getNext(); }
print s; //prints "HelloWorld" } } rgn.free(); \end{codejava} The
stack region \C{R1} is being used in the above code to provide working
memory to work with the objects of transferable region \C{rgn}. Since
a transferable region cannot be transferred/freed while it is still
open (Fig~\ref{fig:region-fsm}), \C{rgn} is guaranteed to outlive the
stack region \C{R1} in the above code, making it safe for the later to
contain references to the former. \name therefore allows such
references.

As Fig.~\ref{fig:region-fsm} indicates, it is not possible to open a
region that is already transferred/freed. The fact that \C{rgn} is
open within the block therefore guarantees that it is not yet
transferred/freed, and that dereferencing \C{strList} within the block
is safe. The end of \C{open} block marks the return of transferable
region to the closed state.  While closed, the region is eligible to
be transferred to a downstream actor, or to be freed.  An actor that
receives the transferable region, receives it in the closed state. It
can then reopen the received region to read its contents, possibly add
more data and transfer it to another actor, or free the region. 

}



\section{The Core Language}
\label{sec:type-system}

\begin{figure*}[t!]
%
%\begin{minipage}
\begin{smathpar}
\renewcommand{\arraystretch}{1.2}
\begin{array}{lr} 
\multicolumn{2}{c}{
  {\rgn} \in \mathtt{Static \; region \; ids} \qquad
  {\rho} \in \mathtt{Region \; variables} \qquad
  {\tyvar, \tyvarb} \in \mathtt{Type \; variables} \qquad
  {m} \in \mathtt{Method \; names} \qquad
  {x,y,f} \in \mathtt{Variables \; and \; fields} }\\
\begin{array}{lclcl}
  cn & \in & \M{Class \; names} & \coloneqq & \ObjZ \ALT \RgnZ \ALT A \ALT B\\
  T  & \in & \M{FGJ \; types} & \coloneqq & \tyvar \ALT  \fgjN \ALT \unitZ
       \ALT \bar{T} \rightarrow T \\
  D  & \in & \M{Classes} & \coloneqq & 
       \C{class} \; cn\inang{\bar{\tyvar} \extends \bar{\fgjN}} 
                      \inang{\rhoalloc \rhobar \,|\, \phi}\extends \fbN 
                      \{\bar{\tau} \; \bar{f};\; k\; \bar{d}\}\;\\
  k  & \in & \M{Constructors} & \coloneqq & 
       cn(\bar{\tau} \; \bar{x})\{\C{super}(\bar{x}); \;
                                  \C{this}.\bar{f}\,=\,\bar{x};\}\\
  d  & \in & \M{Methods} & \coloneqq & 
       \tau \; m\inang{\rhoalloc \rhobar \,|\, \phi} (\taubar \; \xbar)
       \{\C{return}\;e;\}\\
\end{array}
\begin{array}{lclcl}
  \fgjN & \in & \M{FGJ \; class \; types} & \coloneqq & cn\inang{\tbar}\\
  \fbN  & \in & \M{FB \; class \; types} & \coloneqq & 
       cn\inang{\tbar}\inang{\ralloc \rbar} \\
  \tau &\in& \M{types} & \coloneqq & T@\rgn  
        \ALT \fbN \ALT \unitZ \\
       &   & & & \ALT \inang{\rhoalloc\rhobar \,|\, \phi}\bar{\tau}
        \xrightarrow{\rgn} \tau \\
  \phi,\Phi &\in& \M{Constraints} & \coloneqq & true 
        \ALT \rho \outlives \rho \\
        & & & & \ALT \rho = \rho \ALT \phi \conj \phi\\
  % v  & \in & \M{Value\;expressions} & \coloneqq & x \ALT \C{new} \; \fbN(\vbar)\\
\end{array}\\
\begin{array}{lclcl}
e  & \in & \M{Expressions} & \coloneqq & \unitval \ALT x \ALT e.f 
     \ALT e.m\inang{\ralloc \rbar}(\ebar) \ALT \C{new}\;\fbN(\ebar) 
     \ALT \lambdaexp{\ralloc}{\rhoalloc\rhobar \,|\, \phi}{\xbar:\taubar} {e}
     \ALT e\inang{\ralloc\rbar}(\bar{e})\\
%    \ALT (\tau)\, e\\
%    \ALT e\,;\,e  \\
   & & & & \ALT \letexp{x}{e}{e}
           \ALT \letregion{\rho}{e} 
           \ALT \open{e}{\rgn}{y}{e} \\
%   & & & & \ALT \unpackexp{\rgn}{x}{e}{e} \ALT e\,;\,e\\
% s  & \in & \M{Statements} & \coloneqq & \tau\;x\;=\;e \ALT x\;=\;e
%      \ALT e.f\;=\;e 
%      \ALT \C{open}\;x\;\C{withroot}\;y\;\{s\} \\
%     & & & & \ALT \C{openalloc}\;x\;\C{withroot}\;y\;\{s\} 
%      \ALT (\rho,\tau\;x)\;=\;\C{unpack}\;e \ALT s\,;\,s\\
\end{array}
\end{array}
\end{smathpar}

\caption{\fbname: Syntax}
\label{fig:fb-syntax}
\end{figure*}

\begin{figure*}[t]
%
\begin{minipage}{2.25in}
\begin{smathpar}
\begin{array}{lcl}
  allocRgn(A\inang{\ralloc\rbar}\inang{\tbar}) & = & \ralloc\\
  allocRgn(\inang{\rhoalloc\rhobar \,|\, \phi}\bar{\tau^1}
      \xrightarrow{\ralloc} \tau^2) & = & \ralloc\\
  shape(A\inang{\rhoalloc\rhobar}\inang{\tbar}) & = & A\inang{\tbar}\\
  bound_{\aenv}(\tyvar@\rgn) & = & \aenv(\tyvar)@\rgn\\
  bound_{\aenv}(\fbN) & = & \fbN\\
\end{array}
\end{smathpar}
\end{minipage}
%
\begin{minipage}{1.8in}
\begin{smathpar}
\begin{array}{c}
\renewcommand*{\arraystretch}{1.2}
\RULE
  {
    \\
    B \in \{\ObjZ,\RgnZ\}
  }
  {
    fields(B\inang{\ralloc\rbar}\inang{\tbar}) \;=\; \bullet
  }
\end{array}
\end{smathpar}
\end{minipage}
%
\begin{minipage}{3in}
\begin{smathpar}
\begin{array}{c}
\renewcommand*{\arraystretch}{1.2}
\RULE
  {
    CT(B) = \headerOf{B}\{\bar{\tau^f}\;\bar{f};\,...\}\\
    \substFn = [\rbar/\rhobar, \ralloc/\rhoalloc, \tbar/\bar{\tyvar}] \qquad 
    fields(\substFn(\fbN)) = \bar{g}:\bar{\tau^g}
  }
  {
    fields(B\inang{\ralloc\rbar}\inang{\tbar}) \;=\;
      \bar{g}:\bar{\tau^g},\,\bar{f}:\substFn(\bar{\tau^f})
  }
\end{array}
\end{smathpar}
\end{minipage}
%
\bigskip

\begin{minipage}{3.5in}
\begin{smathpar}
\begin{array}{lcl}
  ctype(\ObjZ\inang{\rgn}) & = & \bullet \\
% ctype(\RgnZ\inang{\rgn}\inang{T}) & = & \inang{\rhoalloc}
%   {\unitZ}\rightarrow{T@\rhoalloc}\\
  ctype(B\inang{\ralloc\rbar}\inang{\tbar}) & = & 
    fields(B\inang{\ralloc\rbar}\inang{\tbar})\\
  mtype(\C{transfer}, \exists\rho.\RgnZ\inang{\rho}\inang{T}) & = & 
    \inang{\rhoalloc} {\unitZ}\rightarrow{\unitZ}\\
  mtype(\C{free}, \exists\rho.\RgnZ\inang{\rho}\inang{T}) & = & 
    \inang{\rhoalloc} {\unitZ}\rightarrow{\unitZ}\\
\end{array}
\end{smathpar}
\end{minipage}
%
\begin{minipage}{3in}
\begin{smathpar}
\begin{array}{c}
\renewcommand*{\arraystretch}{1.2}
\RULE
  {
    CT(B) = \headerOf{B}\{\bar{\tau^f}\;\bar{f};\,k\;\bar{d}\}\\
    m \notin \bar{d} \qquad 
    \substFn = [\rbar/\rhobar, \ralloc/\rhoalloc, \tbar/\bar{\tyvar}]
  }
  {
    mtype (m,B\inang{\ralloc\rbar}\inang{\tbar}) \;=\;
    mtype (m, \substFn(\fbN))
  }
\end{array}
\end{smathpar}
\end{minipage}
%
\bigskip

\begin{minipage}{3.25in}
\begin{smathpar}
\begin{array}{c}
\renewcommand*{\arraystretch}{1.2}
\RULE
  {
    CT(B) = \headerOf{B}\{\bar{\tau^f}\;\bar{f};\,k\;\bar{d}\}\\
    \tau^2 \; m\mang (\bar{\tau^1}\;\bar{x})\{...\} \in \bar{d} \qquad
    \substFn = [\rbar/\rhobar, \ralloc/\rhoalloc, \tbar/\bar{\tyvar}]
  }
  {
    mtype (m,B\inang{\ralloc\rbar}\inang{\tbar}) \;=\;
    \substFn(\mang\bar{\tau^1} \rightarrow \tau^2)
  }
\end{array}
\end{smathpar}
\end{minipage}
%
\begin{minipage}{3.5in}
\begin{smathpar}
\begin{array}{c}
\renewcommand*{\arraystretch}{1.2}
\RULE
  {

    \substFn = \subst{\bar{\rho_2}}{\bar{\rho_1}}
               \subst{\rhoalloc_2}{\rhoalloc_1} \spc
    mtype(m,\fbN) = \inang{\rhoalloc_1\bar{\rho_1},|\, \phi_1}\bar{\tau^{11}} 
                      \rightarrow \tau^{12} \spc \texttt{implies}\\
    \isvalid{\A.\phicx}{\phi_2 \Leftrightarrow \substFn(\phi_1)} 
        \spc \texttt{and} \spc
    \bar{\tau^{21}} = \substFn(\bar{\tau^{11}}) \spc \texttt{and} \spc
    \subtyp{\A}{\tau^{22}} {\substFn(\tau^{12})}
    %\substFn = [\rbar/\rhobar, \ralloc/\rhoalloc, \tbar/\bar{\tyvar}]
  }
  {
    override(\A,\fbN,\inang{\rhoalloc_2\bar{\rho_2},|\, \phi_1}
              \bar{\tau^{21}} \rightarrow \tau^{22})
  }
\end{array}
\end{smathpar}
\end{minipage}
%
\bigskip

\begin{minipage}{5in}
\begin{smathpar}
\begin{array}{c}
  \rhoset,\rhoenv \in 2^{\rho} \qquad
  \aenv \in \tyvar \rightarrow \fgjN \qquad
  \A = (\subtypcx)\\
\end{array}
\end{smathpar}
\end{minipage}
%

\caption{\fbname: Auxiliary Definitions}
\label{fig:fb-auxdef}
\end{figure*}

\begin{figure*}[t!]
%
\textbf{Auxiliary Definitions}\\
\begin{figure*}[t]
%
\begin{minipage}{2.25in}
\begin{smathpar}
\begin{array}{lcl}
  allocRgn(A\inang{\ralloc\rbar}\inang{\tbar}) & = & \ralloc\\
  allocRgn(\inang{\rhoalloc\rhobar \,|\, \phi}\bar{\tau^1}
      \xrightarrow{\ralloc} \tau^2) & = & \ralloc\\
  shape(A\inang{\rhoalloc\rhobar}\inang{\tbar}) & = & A\inang{\tbar}\\
  bound_{\aenv}(\tyvar@\rgn) & = & \aenv(\tyvar)@\rgn\\
  bound_{\aenv}(\fbN) & = & \fbN\\
\end{array}
\end{smathpar}
\end{minipage}
%
\begin{minipage}{1.8in}
\begin{smathpar}
\begin{array}{c}
\renewcommand*{\arraystretch}{1.2}
\RULE
  {
    \\
    B \in \{\ObjZ,\RgnZ\}
  }
  {
    fields(B\inang{\ralloc\rbar}\inang{\tbar}) \;=\; \bullet
  }
\end{array}
\end{smathpar}
\end{minipage}
%
\begin{minipage}{3in}
\begin{smathpar}
\begin{array}{c}
\renewcommand*{\arraystretch}{1.2}
\RULE
  {
    CT(B) = \headerOf{B}\{\bar{\tau^f}\;\bar{f};\,...\}\\
    \substFn = [\rbar/\rhobar, \ralloc/\rhoalloc, \tbar/\bar{\tyvar}] \qquad 
    fields(\substFn(\fbN)) = \bar{g}:\bar{\tau^g}
  }
  {
    fields(B\inang{\ralloc\rbar}\inang{\tbar}) \;=\;
      \bar{g}:\bar{\tau^g},\,\bar{f}:\substFn(\bar{\tau^f})
  }
\end{array}
\end{smathpar}
\end{minipage}
%
\bigskip

\begin{minipage}{3.5in}
\begin{smathpar}
\begin{array}{lcl}
  ctype(\ObjZ\inang{\rgn}) & = & \bullet \\
% ctype(\RgnZ\inang{\rgn}\inang{T}) & = & \inang{\rhoalloc}
%   {\unitZ}\rightarrow{T@\rhoalloc}\\
  ctype(B\inang{\ralloc\rbar}\inang{\tbar}) & = & 
    fields(B\inang{\ralloc\rbar}\inang{\tbar})\\
  mtype(\C{transfer}, \exists\rho.\RgnZ\inang{\rho}\inang{T}) & = & 
    \inang{\rhoalloc} {\unitZ}\rightarrow{\unitZ}\\
  mtype(\C{free}, \exists\rho.\RgnZ\inang{\rho}\inang{T}) & = & 
    \inang{\rhoalloc} {\unitZ}\rightarrow{\unitZ}\\
\end{array}
\end{smathpar}
\end{minipage}
%
\begin{minipage}{3in}
\begin{smathpar}
\begin{array}{c}
\renewcommand*{\arraystretch}{1.2}
\RULE
  {
    CT(B) = \headerOf{B}\{\bar{\tau^f}\;\bar{f};\,k\;\bar{d}\}\\
    m \notin \bar{d} \qquad 
    \substFn = [\rbar/\rhobar, \ralloc/\rhoalloc, \tbar/\bar{\tyvar}]
  }
  {
    mtype (m,B\inang{\ralloc\rbar}\inang{\tbar}) \;=\;
    mtype (m, \substFn(\fbN))
  }
\end{array}
\end{smathpar}
\end{minipage}
%
\bigskip

\begin{minipage}{3.25in}
\begin{smathpar}
\begin{array}{c}
\renewcommand*{\arraystretch}{1.2}
\RULE
  {
    CT(B) = \headerOf{B}\{\bar{\tau^f}\;\bar{f};\,k\;\bar{d}\}\\
    \tau^2 \; m\mang (\bar{\tau^1}\;\bar{x})\{...\} \in \bar{d} \qquad
    \substFn = [\rbar/\rhobar, \ralloc/\rhoalloc, \tbar/\bar{\tyvar}]
  }
  {
    mtype (m,B\inang{\ralloc\rbar}\inang{\tbar}) \;=\;
    \substFn(\mang\bar{\tau^1} \rightarrow \tau^2)
  }
\end{array}
\end{smathpar}
\end{minipage}
%
\begin{minipage}{3.5in}
\begin{smathpar}
\begin{array}{c}
\renewcommand*{\arraystretch}{1.2}
\RULE
  {

    \substFn = \subst{\bar{\rho_2}}{\bar{\rho_1}}
               \subst{\rhoalloc_2}{\rhoalloc_1} \spc
    mtype(m,\fbN) = \inang{\rhoalloc_1\bar{\rho_1},|\, \phi_1}\bar{\tau^{11}} 
                      \rightarrow \tau^{12} \spc \texttt{implies}\\
    \isvalid{\A.\phicx}{\phi_2 \Leftrightarrow \substFn(\phi_1)} 
        \spc \texttt{and} \spc
    \bar{\tau^{21}} = \substFn(\bar{\tau^{11}}) \spc \texttt{and} \spc
    \subtyp{\A}{\tau^{22}} {\substFn(\tau^{12})}
    %\substFn = [\rbar/\rhobar, \ralloc/\rhoalloc, \tbar/\bar{\tyvar}]
  }
  {
    override(\A,\fbN,\inang{\rhoalloc_2\bar{\rho_2},|\, \phi_1}
              \bar{\tau^{21}} \rightarrow \tau^{22})
  }
\end{array}
\end{smathpar}
\end{minipage}
%
\bigskip

\begin{minipage}{5in}
\begin{smathpar}
\begin{array}{c}
  \rhoset,\rhoenv \in 2^{\rho} \qquad
  \aenv \in \tyvar \rightarrow \fgjN \qquad
  \A = (\subtypcx)\\
\end{array}
\end{smathpar}
\end{minipage}
%

\caption{\fbname: Auxiliary Definitions}
\label{fig:fb-auxdef}
\end{figure*}

%

\vspace*{0.1in}
%
\textbf{Type, and Type Constraint Well-formedness}  \; \fbox
  {\(\tywf{\A}{\tau}, \spc 
     \tywf{\rhoenv}{\phi}\)}\\
%
\begin{minipage}{1.25in}
\begin{smathpar}
\begin{array}{c}
\renewcommand*{\arraystretch}{1.2}
\RULE
  {
    \\
    \\
    \rgn \in \A.\rhoenv
  }
  {
    \tywf{\A}{\ObjZ\inang{\rgn}}
  }
\end{array}
\end{smathpar}
\end{minipage}
% %
% \begin{minipage}{1.5in}
% \begin{smathpar}
% \begin{array}{c}
% \renewcommand*{\arraystretch}{1.2}
% \RULE
%   {
%     \rgn \in \A.\rhoenv\\
%     \fgjtywf{\A.\aenv}{\RgnZ\inang{T}}
%   }
%   {
%     \tywf{\A}{\RgnZ\inang{\rgn}\inang{T}}
%   }
% \end{array}
% \end{smathpar}
% \end{minipage}
% %
% \begin{minipage}{1.5in}
% \begin{smathpar}
% \begin{array}{c}
% \renewcommand*{\arraystretch}{1.2}
% \RULE
%   {
%     \fgjtywf{\A.\aenv}{\RgnZ\inang{T}}
%   }
%   {
%     \tywf{\A}{\exists\rho.\RgnZ\inang{\rho}\inang{T}}
%   }
% \end{array}
% \end{smathpar}
% \end{minipage}
%
\begin{minipage}{2.75in}
\begin{smathpar}
\begin{array}{c}
\renewcommand*{\arraystretch}{1.2}
\RULE
  {
    \rgn \in \rhoenv \spc
    \rhobar \notin \A.\rhoenv \\
    \rhoenv' = \rhoenv \cup \{\rhobar\} \spc
    \A' = (\rhoenv', \aenv, \phicx \conj \phi) \\
    \tywf{\rhoenv'}{\phi}\spc 
    \tywf{\A'}{\bar{\tau^1}} \spc
    \tywf{\A'}{\tau^2}
  }
  {
    \tywf{(\rhoenv,\aenv,\phicx)}{\inang{\rhobar \,|\, \phi}
              \bar{\tau^1} \xrightarrow{\rgn} \tau^2}
  }
\end{array}
\end{smathpar}
\end{minipage}
%
\begin{minipage}{1.5in}
\begin{smathpar}
\begin{array}{c}
\renewcommand*{\arraystretch}{1.2}
\RULE
  { 
    \\
    \\
    \fgjtywf{\A.\aenv}{T}
  }
  {
    \tywf{\A}{\RgnZ\inang{T}\inang{\toprgn}}
  }
\end{array}
\end{smathpar}
\end{minipage}
%
\begin{minipage}{1in}
\begin{smathpar}
\begin{array}{c}
\renewcommand*{\arraystretch}{1.2}
\RULE
  {
    \\
    \\
    \rho_0,\rho_1 \in \rhoenv
  }
  {
    \tywf{\rhoenv}{\rho_0 \outlives \rho_1}
  }
\end{array}
\end{smathpar}
\end{minipage}
%

%
\begin{minipage}{3.5in}
\begin{smathpar}
\begin{array}{c}
\renewcommand*{\arraystretch}{1.2}
\RULE
  {
    CT(B) = \headerOf{B}\{...\}\\
    \rbar \in \rhoenv \spc
    \fgjtywf{\aenv}{B\inang{\tbar}}\spc
%   \substFn = [\rbar/\rhobar, \tbar/\bar{\tyvar}] \spc
    \isvalid{\phicx}{[\rbar/\rhobar](\phi)}
  }
  {
    \tywf{(\rhoenv,\aenv,\phicx)}{B\inang{\rbar}\inang{\tbar}}
  }
\end{array}
\end{smathpar}
\end{minipage}
%
\begin{minipage}{1.65in}
\begin{smathpar}
\begin{array}{c}
\renewcommand*{\arraystretch}{1.2}
\RULE
  {
    \fgjtywf{\A.\aenv}{T}\spc
    \rgn \in \A.\rhoenv\\
    \fgjsubtyp{\A.\aenv}{T}{\ObjZ}\spc
  }
  {
    \tywf{\A}{T@\rgn}
  }
\end{array}
\end{smathpar}
\end{minipage}
%
\begin{minipage}{0.75in}
\begin{smathpar}
\begin{array}{c}
\renewcommand*{\arraystretch}{1.2}
\RULE
  {
    \tywf{\rhoenv}{\phi_0} \\ \tywf{\rhoenv}{\phi_1}
  }
  {
    \tywf{\rhoenv}{\phi_0 \wedge \phi_1}
  }
\end{array}
\end{smathpar}
\end{minipage}


% \vspace*{0.1in}
% %
% \textbf{Subtyping}  \; \fbox
%   {\(\subtyp{\A}{\tau_1}{\tau_2}\)}\\
% %
\begin{minipage}{1.2in}
\begin{smathpar}
\begin{array}{c}
\renewcommand*{\arraystretch}{1.2}
  \subtyp{\A}{\tau}{\tau} \qquad
  \subtyp{\A}{\tyvar @\rho}{\aenv(\tyvar) @\rho}\qquad
% \subtyp{\A}{\RgnZ\inang{\rgn}}{\RgnZ\inang{\toprgn}}\qquad
% \subtyp{\A}{\RgnZ\inang{\toprgn}}{\RgnZ\inang{\rgn}}
\end{array}
\end{smathpar}
\end{minipage}
%

\begin{minipage}{1in}
\begin{smathpar}
\begin{array}{c}
\renewcommand*{\arraystretch}{1.2}
\RULE
  {
    \subtyp{\A}{\tau_1}{\tau_2}\\
    \subtyp{\A}{\tau_2}{\tau_3}
  }
  {
    \subtyp{\A}{\tau_1}{\tau_3}
  }
\end{array}
\end{smathpar}
\end{minipage}
% Existential type subtyping is redundant.
%%\begin{minipage}{1.5in}
%%\begin{smathpar}
%%\begin{array}{c}
%%\renewcommand*{\arraystretch}{1.2}
%%\RULE
%%  {
%%    \rho' \;\texttt{not free in}\; \tau
%%  }
%%  {
%%    \subtyp{\A}{\exists\rho.\tau}{\exists\rho'.[\rho'/\rho]\tau}
%%  }
%%\end{array}
%%\end{smathpar}
%%\end{minipage}
%
\begin{minipage}{2.55in}
\begin{smathpar}
\begin{array}{c}
\renewcommand*{\arraystretch}{1.2}
\RULE
  {
    \\
    CT(B) = \headerOf{B}\{...\}
%   \tywf{\A}{B\inang{\tbar}\inang{\pi^a\bar{\pi}}}\spc
%   \substFn = [\rbar/\rhobar, \ralloc/\rhoalloc, \tbar/\bar{\tyvar}]
  }
  {
    \subtyp{\A}{B\inang{\tbar}\inang{\bar{\pi}}}
        {[\rbar/\rhobar, \tbar/\bar{\tyvar}](\fbN)}
  }
\end{array}
\end{smathpar}
\end{minipage}
%
\begin{minipage}{2.75in}
\begin{smathpar}
\begin{array}{c}
\renewcommand*{\arraystretch}{1.2}
\RULE
  {
    \isvalid{\A.\phicx}{\phi_1 \Rightarrow \phi_2} \\
    \subtyp{\A}{\bar{\tau^{11}}}{\bar{\tau^{21}}} \spc
    \subtyp{\A}{\tau^{22}}{\tau^{12}}
  }
  {
    \subtyp{\A}
      {\inang{\rhobar \,|\, \phi_2}\bar{\tau^{21}}
          \xrightarrow{\rgn} \tau^{22}}
      {\inang{\rhobar \,|\, \phi_1}\bar{\tau^{11}}
          \xrightarrow{\rgn} \tau^{12}}
  }
\end{array}
\end{smathpar}
\end{minipage}


% %

\vspace*{0.1in}
\textbf{Expression Typing}  \; \fbox
  {\(\hastyp{\exptycx{\ralloc}{\env}}{e}{\tau}\)}\\
%
\begin{minipage}{1in}
\begin{smathpar}
\begin{array}{l}
\renewcommand*{\arraystretch}{1.2}
\hastyp{\exptycx{\ralloc}{\env}}{\unitval}{\unitZ}\\
\hastyp{\exptycx{\ralloc}{\env}}{x}{\env(\tau)}
% \RULE
%   {
%     \\
%     \env(x) = \tau
%   }
%   {
%     \hastyp{\exptycx{\ralloc}{\env}}{x}{\tau}
%   }
\end{array}
\end{smathpar}
\end{minipage}
%
% \begin{minipage}{1.2in}
% \begin{smathpar}
% \begin{array}{c}
% \renewcommand*{\arraystretch}{1.2}
% \RULE
%   {
%     \\
%     \\
%   }
%   {
%     \hastyp{\exptycx{\ralloc}{\env}}{\unitval}{\unitZ}
%   }
% \end{array}
% \end{smathpar}
% \end{minipage}
%
\begin{minipage}{2in}
\begin{smathpar}
\begin{array}{c}
\renewcommand*{\arraystretch}{1.2}
\RULE
  {
    \hastyp{\exptycx{\ralloc}{\env}}{x}{\tau'}\\
    f:\tau \,\in\, fields(bound_{\A.\aenv}(\tau'))
  }
  {
    \hastyp{\exptycx{\ralloc}{\env}}{e.f}{\tau}
  }
\end{array}
\end{smathpar}
\end{minipage}
%
% --- PACK expressio removed ---
% \begin{minipage}{1.75in}
% \begin{smathpar}
% \begin{array}{c}
% \renewcommand*{\arraystretch}{1.2}
% \RULE
%   {
%     \tywf{\A}{\exists\rho.\tau}\\
%     \hastyp{\exptycx{\ralloc}{\env}}{e}{[\rho_0/\rho]\tau}
%   }
%   {
%     \hastyp{\exptycx{\ralloc}{\env}}
%            {\C{pack} \; (\rho_0,e) \; \C{as} \; \exists\rho.\tau}
%            {\exists\rho.\tau}
%   }
% \end{array}
% \end{smathpar}
% \end{minipage}
%
\bigskip

%
\begin{minipage}{2.1in}
\begin{smathpar}
\begin{array}{c}
\renewcommand*{\arraystretch}{1.2}
\RULE
  {
    \tywf{\exptycx{\ralloc}{\env}}{\fbN} \spc
    ctype(\fbN) = \taubar \\
    \isvalid{\A.\phicx}{\rgn \outlives \ralloc}\spc
    allocRgn(\fbN) = \rgn \\
    \hastyp{\exptycx{\ralloc}{\env}}{\bar{e}}{\bar{\tau'}} \spc
    \subtyp{\A}{\bar{\tau'}}{\taubar}
  }
  {
    \hastyp{\exptycx{\ralloc}{\env}}{\C{new}\; \fbN(\bar{e})}{\fbN}
  }
\end{array}
\end{smathpar}
\end{minipage}
% %
% --- No special treatment of NEW REGION ---
% \begin{minipage}{2.1in}
% \begin{smathpar}
% \begin{array}{c}
% \renewcommand*{\arraystretch}{1.2}
% \RULE
%   {
%     \fgjtywf{\A.\aenv}{\RgnZ\inang{T}}\\
%     \hastyp{\exptycx{\ralloc}{\env}}{e}{\tau} \\
%     \subtyp{\A}{\tau}{\inang{\rhoalloc}
%     {\unitZ}\xrightarrow{\rgn}{T@\rhoalloc}}
%   }
%   {
%     {\exptycx{\ralloc}{\env}}\,\vdash\,
%           {\C{new}\; \RgnZ\inang{\rgn}\inang{T}(e)}\\
%            \hspace*{0.5in}{:\exists\rho.\RgnZ\inang{\rho}\inang{T}}
%   }
% \end{array}
% \end{smathpar}
% \end{minipage}
%
\begin{minipage}{2.75in}
\begin{smathpar}
\begin{array}{c}
\renewcommand*{\arraystretch}{1.2}
\RULE
  {
    \hastyp{\exptycx{\ralloc}{\env}}{e_0}{\tau} \spc
    \rbar \in \A.\rhoenv \\
    mtype(m,bound_{\A.\aenv}(\tau)) = \inang{\rhoalloc\rhobar \,|\, 
        \phi}\bar{\tau^1}\rightarrow{\tau^2} \\
    \substFn = [\rbar/\rhobar, \ralloc/\rhoalloc] \spc
%   \tywf{\A}{\substFn(\bar{\tau^1})} \spc
%   \tywf{\A}{\substFn(\tau^2)} \\
    \hastyp{\exptycx{\ralloc}{\env}}{\bar{e}}{\bar{\tau'}} \spc
    \subtyp{\A}{\bar{\tau'}}{\substFn(\bar{\tau^1})} \spc
    \isvalid{\A.\phicx}{\substFn(\phi)}
  }
  {
    \hastyp{\exptycx{\ralloc}{\env}}{e_0.m\inang{\ralloc\rbar}(\bar{e})} 
           {\substFn(\tau^2)}
  }
\end{array}
\end{smathpar}
\end{minipage}
%
\bigskip

%
\begin{minipage}{3.5in}
\begin{smathpar}
\begin{array}{c}
\renewcommand*{\arraystretch}{1.2}
\RULE
  {
%   \A = (\subtypcx)\spc
    \rhoalloc,\rhobar \notin \A.\rhoenv \spc
%   \rhoenv' = \rhoenv \cup \{\rhoalloc,\rhobar\}\spc
    \A' = (\A.\rhoenv \cup \{\rhoalloc,\rhobar\}, \A.\aenv, 
          \A.\phicx \conj \phi)\\
    \tywf{\A'.\rhoenv}{\phi}\spc
    \tywf{\A'}{\bar{\tau^1}}\spc
    \hastyp{\A',\rhoalloc,\env[\xbar \mapsto \bar{\tau^1}]}{e}{\tau^2}
  }
  {
    \hastyp{\exptycx{\ralloc}{\env}}
           {\lambdaexp{\rhoalloc\rhobar \,|\, \phi}
                      {\xbar:\bar{\tau^1}}{e}}
           {\inang{\rhoalloc\rhobar \,|\, \phi}
            \bar{\tau^1} \xrightarrow{\ralloc} \tau^2}
  }
\end{array}
\end{smathpar}
\end{minipage}
%
\begin{minipage}{3in}
\begin{smathpar}
\begin{array}{c}
\renewcommand*{\arraystretch}{1.2}
\RULE
  {
    \A = (\subtypcx) \spc
    \rgn \notin \rhoenv \spc
    \phi = \rhoenv \outlives \rgn\\
    \A' = (\rhoenv \cup \{\rgn\}, \aenv, \phicx \conj \phi)\spc
    \hastyp{\A',\rgn,\env}{e}{\unitZ}
  }
  {
    \hastyp{\exptycx{\ralloc}{\env}}{\letregion{\rgn}{e}}{\unitZ}
  }
\end{array}
\end{smathpar}
\end{minipage}
%
\bigskip

%
\begin{minipage}{3in}
\begin{smathpar}
\begin{array}{c}
\renewcommand*{\arraystretch}{1.2}
\RULE
  {
   \hastyp{\exptycx{\ralloc}{\env}}{x}
            {\RgnZ\inang{T}\inang{\rgn'}}\\
    \A = (\subtypcx) \spc
    \rgn \notin \rhoenv \spc
    \A' = (\rhoenv \cup \{\rgn\},\aenv,\phicx) \\
    \env' =  \env[y\mapsto T@\rgn]\spc
    \hastyp{\A',\ralloc,\env'}{e}{\unitZ} 
  }
  {
    \hastyp{\exptycx{\ralloc}{\env}}{\open{x}{\rgn}{y}{e}}
            {\unitZ}
  }
\end{array}
\end{smathpar}
\end{minipage}
% % --- No OPENALLOC ---
% \begin{minipage}{3.25in}
% \begin{smathpar}
% \begin{array}{c}
% \renewcommand*{\arraystretch}{1.2}
% \RULE
%   {
%     \A = (\subtypcx) \spc
%     \A' = (\rhoset,\rhoenv \cup \{\rgn\},\aenv,\phicx) \\
%     \hastyp{\exptycx{\ralloc}{\env}}{x}{\RgnZ\inang{\rgn}\inang{T}}\spc
%     \hastyp{\A',\rgn,\env[y\mapsto T@\rgn]}{e}{\unitZ}
%   }
%   {
%     \hastyp{\exptycx{\ralloc}{\env}}{\openalloc{x}{y}{e}}
%             {\unitZ}
%   }
% \end{array}
% \end{smathpar}
% \end{minipage}
%
% % --- BIG LAMBDA not needed ---
% \begin{minipage}{3.75in}
% \begin{smathpar}
% \begin{array}{c}
% \renewcommand*{\arraystretch}{1.2}
% \RULE
%   {
%     \rhoalloc,\rhobar \notin \A.\rhoenv \spc
% %   \rhoenv' = \rhoenv \cup \{\rhoalloc,\rhobar\}\\
%     \A' = (\A.\rhoenv \cup \{\rhoalloc,\rhobar\}, 
%             \A.\aenv, \A.\phicx \conj \phi)\\
%     \tywf{\A'.\rhoenv}{\phi}\spc
%     \hastyp{\A',\ralloc,\env}{e}{\bar{\tau^1} \xrightarrow
%             {\rgn} \tau^2}
%   }
%   {
%     \hastyp{\exptycx{\ralloc}{\env}}
%            {\Lambdaexp{\rhoalloc\rhobar \,|\, \phi}{e}}
%            {\inang{\rhoalloc\rhobar \,|\, \phi}
%             \bar{\tau^1} \xrightarrow{\rgn} \tau^2}
%   }
% \end{array}
% \end{smathpar}
% \end{minipage}
%
\begin{minipage}{3.2in}
\begin{smathpar}
\begin{array}{c}
\renewcommand*{\arraystretch}{1.2}
\RULE
  {
    \rbar \in \A.\rhoenv \spc
    \hastyp{\exptycx{\ralloc}{\env}}{e}
        {\inang{\rhoalloc\rhobar \,|\, \phi}
            \bar{\tau^1} \xrightarrow{\rgn} \tau^2}\\
    \substFn = \subst{\rbar}{\rhobar}
               \subst{\ralloc}{\rhoalloc} \spc
    \isvalid{\A.\phicx}{\substFn(\phi)} \spc
    \hastyp{\exptycx{\ralloc}{\env}}{\bar{e}}
        {\bar{\tau}}\spc
    \subtyp{\A}{\taubar}{\bar{\tau^1}}
  }
  {
    \hastyp{\A,\ralloc,\env}{e\inang{\ralloc\rbar}(\bar{e})}
           {\substFn(\tau^2)}
  }
\end{array}
\end{smathpar}
\end{minipage}
%
\bigskip

% % --- APPLICATION coalesced with instantiation ---
% \begin{minipage}{2in}
% \begin{smathpar}
% \begin{array}{c}
% \renewcommand*{\arraystretch}{1.2}
% \RULE
%   {
%     \hastyp{\exptycx{\ralloc}{\env}}{e_0}
%         {\bar{\tau^1} \xrightarrow{\rgn} \tau^2}\\
%     \hastyp{\exptycx{\ralloc}{\env}}{\bar{e}}
%         {\bar{\tau}}\spc
%     \subtyp{\A}{\taubar}{\bar{\tau^1}}
%   }
%   {
%     \hastyp{\A,\ralloc,\env}{e_0(\bar{e})}
%            {\tau^2}
%   }
% \end{array}
% \end{smathpar}
% \end{minipage}
%
\begin{minipage}{2in}
\begin{smathpar}
\begin{array}{c}
\renewcommand*{\arraystretch}{1.2}
\RULE
  {
    \hastyp{\exptycx{\ralloc}{\env}}{e_1}{\tau_1}\\
    \hastyp{\exptycx{\ralloc}{\env[x\mapsto\tau_1]}}{e_2}{\tau_2}\\
  }
  {
    \hastyp{\exptycx{\ralloc}{\env}}{\letexp{x}{e_1}{e_2}}{\tau_2}
  }
\end{array}
\end{smathpar}
\end{minipage}
%
\begin{minipage}{2in}
\begin{smathpar}
\begin{array}{c}
\renewcommand*{\arraystretch}{1.2}
\RULE
  {
    e_1 \in \{x,\,e.f\}\spc
    \hastyp{\exptycx{\ralloc}{\env}}{e_1}{\tau_1}\\
    \hastyp{\exptycx{\ralloc}{\env}}{e_2}{\tau_2}\spc
    \subtyp{\A}{\tau_2}{\tau_1}
  }
  {
    \hastyp{\exptycx{\ralloc}{\env}}{e_1\,:=\,e_2}{\unitZ}
  }
\end{array}
\end{smathpar}
\end{minipage}
%
% \begin{minipage}{1.1in}
% \begin{smathpar}
% \begin{array}{c}
% % --- UPCAST removed ---
% \renewcommand*{\arraystretch}{1.2}
% \RULE
%   {
%     \hastyp{\exptycx{\ralloc}{\env}}{e}{\tau_1}\\
%     \subtyp{\A}{\tau_1}{\tau}
%   }
%   {
%     \hastyp{\exptycx{\ralloc}{\env}}{(\tau)\,e}{\tau}
%   }
% \end{array}
% \end{smathpar}
% \end{minipage}
% % --- No UNPACK expression ---
% \begin{minipage}{3.25in}
% \begin{smathpar}
% \begin{array}{c}
% \renewcommand*{\arraystretch}{1.2}
% \RULE
%   {
%     \hastyp{\exptycx{\ralloc}{\env}}{e_1}{\exists \rho.\tau}\spc
%     \A = (\subtypcx) \spc
%     \rgn \notin \rhoset \\
%     \A' = (\rhoset \cup \{\rgn\},\rhoenv,\aenv,\phicx) \spc
%     \hastyp{\A',\ralloc,\env[x \mapsto \subst{\rgn}{\rho}\tau]}
%            {e_2}{\tau_2}\spc
%   }
%   {
%     \hastyp{\exptycx{\ralloc}{\env}}{\unpackexp{\rgn}{x}{e_1}{e_2}}
%             {\tau_2}
%   }
% \end{array}
% \end{smathpar}
% \end{minipage}
%
\begin{minipage}{1.5in}
\begin{smathpar}
\begin{array}{c}
\renewcommand*{\arraystretch}{1.2}
\RULE
  {
    \hastyp{\exptycx{\ralloc}{\env}}{e_1}{\unitZ}\\
    \hastyp{\exptycx{\ralloc}{\env}}{e_2}{\tau}\spc
  }
  {
    \hastyp{\exptycx{\ralloc}{\env}}{e_1\,;\,e_2}{\tau}
  }
\end{array}
\end{smathpar}
\end{minipage}
%

%
%%\textbf{Statement Semantics}  \; \fbox
%%  {\(\stmtsem{\ralloc}{s}{\stmtsemcx}{\stmtsemcxp}\)}\\
%%%
\begin{minipage}{2.2in}
\begin{smathpar}
\begin{array}{c}
\renewcommand*{\arraystretch}{1.2}
\RULE
  {
    \\
    \hastyp{\exptycx{\ralloc}{\env}}{e}{\tau'}\\
    \tywf{\A}{\tau}\spc
    \subtyp{\A}{\tau'}{\tau}
  }
  {
    \stmtsem{\ralloc}{\tau\;x\;=\;e}{\stmtsemcx}{\A,\,\env[x\mapsto\tau]}
  }
\end{array}
\end{smathpar}
\end{minipage}
%
\begin{minipage}{2.1in}
\begin{smathpar}
\begin{array}{c}
\renewcommand*{\arraystretch}{1.2}
\RULE
  {
    \\
    e_1 \in \{x,\,e.f\}\spc
    \hastyp{\exptycx{\ralloc}{\env}}{e_1}{\tau_1}\\
    \hastyp{\exptycx{\ralloc}{\env}}{e_2}{\tau_2}\spc
    \subtyp{\A}{\tau_2}{\tau_1}
  }
  {
    \stmtsem{\ralloc}{e_1\;=\;e_2}{\stmtsemcx}{\stmtsemcx}
  }
\end{array}
\end{smathpar}
\end{minipage}
%
\begin{minipage}{2.2in}
\begin{smathpar}
\begin{array}{c}
\renewcommand*{\arraystretch}{1.2}
\RULE
  {
    \A = (\subtypcx) \spc
    \rgn \notin \rhoset \\
    \rhoset' = \rhoset \cup \{\rgn\} \spc
    \rhoenv' = \rhoenv \cup \{\rgn\} \spc
    \phicx' = \phicx \conj \rhoenv \outlives \rgn \\
    \A' = (\rhoset',\rhoenv',\aenv,\phicx') \spc
    \stmtsem{\rgn}{s}{\A',\,\env}{\A'',\,\env'}
  }
  {
    \stmtsem{\ralloc}{\C{letregion}\;\rgn\;\{s\}}{\stmtsemcx}{\stmtsemcx}
  }
\end{array}
\end{smathpar}
\end{minipage}
%
\bigskip

%
\begin{minipage}{3.5in}
\begin{smathpar}
\begin{array}{c}
\renewcommand*{\arraystretch}{1.2}
\RULE
  {
    \hastyp{\exptycx{\ralloc}{\env}}{x}{\RgnZ\inang{\rgn}\inang{T}}\\
    \A = (\subtypcx) \spc
    \rhoenv' = \rhoenv \cup \{\rgn\} \spc
    \A' = (\rhoset,\rhoenv',\aenv,\phicx) \\
    \env' = \env[y\mapsto T@\rgn]\spc
    \stmtsem{\ralloc}{s} {\A',\,\env'}{\A'',\,\env''}
  }
  {
    \stmtsem{\ralloc}{\C{open}\;x\;\C{withroot}\;y\;\{s\}}
            {\stmtsemcx}{\stmtsemcx}
  }
\end{array}
\end{smathpar}
\end{minipage}
%
\begin{minipage}{3.5in}
\begin{smathpar}
\begin{array}{c}
\renewcommand*{\arraystretch}{1.2}
\RULE
  {
    \hastyp{\exptycx{\ralloc}{\env}}{x}{\RgnZ\inang{\rgn}\inang{T}}\\
    \A = (\subtypcx) \spc
    \rhoenv' = \rhoenv \cup \{\rgn\} \spc
    \A' = (\rhoset,\rhoenv',\aenv,\phicx) \\
    \env' = \env[y\mapsto T@\rgn]\spc
    \stmtsem{\rgn}{s} {\A',\,\env'}{\A'',\,\env''}
  }
  {
    \stmtsem{\ralloc}{\C{openalloc}\;x\;\C{withroot}\;y\;\{s\}}
            {\stmtsemcx}{\stmtsemcx}
  }
\end{array}
\end{smathpar}
\end{minipage}
%
\bigskip

%
\begin{minipage}{3.75in}
\begin{smathpar}
\begin{array}{c}
\renewcommand*{\arraystretch}{1.2}
\RULE
  {
    \hastyp{\exptycx{\ralloc}{\env}}{e}{\exists \rho.\tau}\spc
    \A = (\subtypcx) \spc
    \rho' \notin \rhoset \\
    \A' = (\rhoset \cup \{\rho'\},\rhoenv,\aenv,\phicx) \spc
    \tau' = [\rho'/\rho]\tau \spc
    \env' = \env[x\mapsto\tau']
  }
  {
    \stmtsem{\ralloc}{(\rho',\tau'\;x)\;=\;\C{unpack}\;e }
            {\stmtsemcx}{\stmtsemcxp}
  }
\end{array}
\end{smathpar}
\end{minipage}
%
\begin{minipage}{2.5in}
\begin{smathpar}
\begin{array}{c}
\renewcommand*{\arraystretch}{1.2}
\RULE
  {
    \stmtsem{\ralloc}{s_1}
            {\stmtsemcx}{\stmtsemcxp} \spc
    \stmtsem{\ralloc}{s_2}
            {\stmtsemcxp}{\stmtsemcxpp}
  }
  {
    \stmtsem{\ralloc}{s_1\,;\,s_2}
            {\stmtsemcx}{\stmtsemcxpp}
  }
\end{array}
\end{smathpar}
\end{minipage}
%


\caption{\fbname: Static Semantics}
\label{fig:fb-staticsem}
\end{figure*}


%% Trash
%%\env = [\xbar \mapsto \bar{\tau^B},
%%        \thisZ \mapsto B\inang{\rhoalloc\rhobar}\inang{\bar{\tyvar}}]\spc
%%\hastyp{\exptycx{\rhoalloc}{\env}}{\bar{u}}
%%       {\bar{\tau^{u}}}\spc
%%\hastyp{\exptycx{\rhoalloc}{\env}}{\bar{v}}
%%       {\bar{\tau^{v}}}\spc

/Users/gowtham/git/broom/fullversion/broom/paper/fb-morewfrules.tex

\subsection{Syntax}
\label{sec:fb-syntax}

We build on the Featherweight Generic Java (FGJ)~\cite{fgj} formalism
to formalize \name and its region type system. Our development reuses
notations from~\cite{fgj}, and relies on several of its definitions,
such as the definition of type well-formedness for the core
(region-free) language. Due to space constraints, we are unable to
reproduce them here, and instead encourage the reader to refer
to~\cite{fgj}.

Fig~\ref{fig:fb-syntax} describes the syntax of our formal language,
which we call \fbname (\FB). The language is seeded with \ObjZ and
\RgnZ classes. More classes can be defined using the \C{class}
keyword. Class types in \FB are region-annotated variants of class
types in FGJ (also called \emph{core types}). This correspondence is
reflected in the $T@\rgn$ syntax of the region type of an object,
whose core type is $T$, and which is contained in the region $\rgn$.
We let $\rgn$ range over static identifiers of regions in \FB. Note
that all \FB objects are boxed values, hence their \FB types are
always region-annotated. The only unboxed value in \FB is $\unitval$
of type \unitZ.

The expression language of FGJ is impoverished, lacking, among others,
an assignment expression. Note that it is possible to \emph{encode}
assignments to instance variables (object fields) in FGJ by creating a
new object with its fields set to new values. Unfortunately, such
encoding is not semantics preserving in case of \name, as the new
object could be allocated in a region different from the allocation
region of the original object. To avoid such complications, we extend
the expression language with assignments and local variable
declarations (via \C{let} expressions). Instead of distinguishing
between statements (e.g., assignments) and expressions (e.g., method
calls) in \FB, we consider statements as expressions of \unitZ type.
Statements that introduce lexical blocks in \name, such as
\C{letregion} and \C{open} are, are converted to their expression form
using the \C{in} keyword.

The expression language has been equipped with lambda abstraction
($\lambdaexp{...}{\xbar : \taubar}{e}$) and application ($e(\bar{e})$)
expressions to define and apply anonymous functions.  Modulo the angle
braces containing $\rho$'s and $\phi$'s, whose presence will be
justified later, these expressions are uncurried variants of
corresponding expressions from simply typed lambda calculus. Note that
FGJ formalism does not include higher-order functions, hence the
syntactic class (\C{T}) of FGJ types needs to be extended with an
(uncurried) arrow type.

Informally, \FB class definitions are extensions of FGJ class
definitions with regions. A class definition in FGJ can be lifted to
its \FB definition by (a). lifting all core types in the body of the
class (i.e, fields, constructors and methods) to their
region-annotated versions, and (b). parameterizing the class over
region variables that occur free in the region-annotated body of the
class. Note that in \FB, regions, and their static identifiers
($\rgn$) are confined to the expression language. Consequently, the
region annotated types of class fields, constructors and methods do
not refer to identifiers of concrete regions, but rather their place
holders called \emph{region variables} ($\rho$). Informally, region
variables ($\rho$) are to region identifiers ($\rgn$) as type
variables ($a$) are to types ($T$). 

Note that when a generic class definition in FGJ is lifted to a
region-polymorphic definition in \FB, it nonetheless remains generic
with respect to core types. We say that the class is now a
region-polymorphic generic class. Region parameterization of a class
in \FB is independent of its parameterization over core types; they
are not conflated to make the class parametric over region types. For
instance, following is the definition of region-polymorphic generic
\C{Pair} class (The symbol $\extends$ should be read \emph{extends}):
\begin{codejava}[mathescape=true]
class Pair<$\rho^a, \rho_1, \rho_2 \,|\, \rho_1 \outlives \rho^a 
                      \conj \rho_2 \outlives \rho^a$>
          <a $\extends$ Object, b $\extends$ Object> {
  a@$\rho_1$ fst; 
  b@$\rho_2$ snd;
  Pair(a@$\rho_1$ fst, b@$\rho_2$ snd) {
    super(); 
    this.fst = fst; 
    this.snd = snd;
  }
  a@$\rho_1$ getFst() {
    return this.fst;
  }
}
\end{codejava}
Every class definition in \FB is necessarily polymorphic with respect
to the allocation region of its objects. As a rule, allocation region
parameter (denoted $\rho^a$) is the first region parameter of a class.
Besides the allocation region of its objects, \C{Pair} class is also
parametric over the regions its first and second elements are
allocated in. References between objects allocated in different
regions are only allowed if the referred object is guaranteed to
outlive the referring object. In case of \C{Pair} class, this means
that allocation regions ($\rho_1$ and $\rho_2$) of both objects that
make up the pair must outlive the allocation region ($\rho^a$) of the
\C{Pair} object. Such conditions over region parameters of a class
need to be recorded in its header as region constraints ($\phi$) in
order for the class to be judged well-formed by the type system
(Fig.~\ref{fig:fb-morewfrules}). 

To construct objects of the \C{Pair} class, its type and region
parameters need to be instantiated with core types ($T$) and concrete
region identifiers ($\rgn$), respectively. Region instantiation has to
satisfy the stated constraints over region parameters. For example,
the following code snippet appropriately instantiates region
parameters of the \C{Pair} class to construct a \C{Pair} of
{\ObjZ}{\!}s, each allocated in different regions:
\begin{codejava}
letregion $\rgn_0$ {
  Object<$\rgn_0$> fst = new Object<$\rgn_0$>();
  letregion $\rgn_1$ {
    Object<$\rgn_1$> snd = new Object<$\rgn_1$>();
    Pair<Object,Object><$\rgn_1$,$\rgn_0$,$\rgn_1$> p = 
       new Pair<Object,Object><$\rgn_1$,$\rgn_0$,$\rgn_1$>(fst,snd);
  }
}
\end{codejava}
Since $\rgn_0 \outlives \rgn_1 \conj \rgn_1 \outlives \rgn_1$, the
region constraints of the \C{Pair} class are satisfied. Observe that
the region type of pair object \C{p} conveys the fact that (a). it is
allocated in region $\rgn_1$, and (b). it holds references to
object(s) allocated in region $\rgn_0$. On the other hand, if we
choose to allocate the \C{fst} object also in $\rgn_1$, the region
type of \C{p} would be
\C{Pair<\ObjZ,\ObjZ><$\rgn_1$,$\rgn_1$,$\rgn_1$>}, which can be
written simply as \C{Pair<\ObjZ,\ObjZ>@$\rgn_1$}. In general, the
\C{@} notation in a region type of an object \C{x} highlights that
\C{x}, and all the objects reachable from \C{x} via references are
allocated in a single region. We say that \C{x} is \emph{contained} in
the region. The objects of $\ObjZ$ class, by the virtue of
containing no references to other objects, are always contained in
their allocation region. Hence the type $\ObjZ@\rgn$ is equivalent to
the type $\ObjZ\inang{\rgn}$.
%%An object \emph{allocated} in a region $\rgn$ contains its spine in
%%$\rgn$, but can refer to objects allocated in other regions.  On the
%%other hand,

A method definitions ($d$) in \fbname can be region polymorphic with
respect to (a). its allocation context (Sec.~\ref{sec:alloc-ctxt}),
and (b). the regions occuring in the region types of its arguments.
Region parameters on the methods, like those on classes, are
accompanied by constraints ($\phi$) capturing the conditions that the
parameters need to satisfy for the method to be considered well-formed
(Fig.~\ref{fig:fb-morewfrules}). Allocation context ($\rho^a$ or
$\rho^a_m$) is the first and inevitable region parameter of every
method in \FB. If a method is not polymorphic with respect to its
allocation context (for example, if its allocation context needs to be
same as the allocation region of its class), then the monomorphism
needs to be encoded as an explicit equality constraint in $\phi$.  

\fbname supports first-class functions via lambda expressions and
arrow types. A lambda expression defines a region-polymorphic
multi-argument function closure parameterized over function's
allocation context parameter. The arrow type of a function closure is
prenex-quantified over its constrained region arguments
($\inang{\rho^a\rhobar \,|\, \phi}$). Since closures can escape the
context in which they are created, it is important to keep track of
the regions where closures are allocated in order to prevent unsafe
dereferences. The $\rgn^a$ annotation above the arrow in the arrow
type denotes the allocation region of the corresponding closure. Note
that it is important to distinguish between the allocation context
argument of a function and the allocation region of its closure. For
instance, in the following example:
\begin{codejava}
letregion $\rgn$ {
  let f = $\lambda{\inang{\rho^a}}$()$\,\Rightarrow\,$new Object$\inang{\rho^a}$() 
  in f
}
\end{codejava}
The type of \C{f} is $\inang{\rho^a}\unitZ \xrightarrow{\rgn}
\ObjZ\inang{\rho^a}$, coveying that (a). \C{f}'s closure is allocated
in $\rgn$, and (b). when executed under an allocation context
$\rho^a$, the closure returns an object allocated in $\rho^a$.

The expression language admits expressions to instantiate
($e\inang{\ralloc\rbar}$) region parameters, and to apply
($e(\bar{e})$) functions. For a method call, however, region parameter
instantiation and method application are captured in a single
expression ($e.mn\inang{\ralloc \rbar}(\ebar)$). The language also
admits a big-lambda expression, which, when used in conjunction with
the instantiation expression, serves the bureaucratic purpose of
renaming bound region variables in the type of a function closure. 

The 



\renewcommand{\rgn}{\pi}
\section{TO DO Items}

\begin{itemize}
\item Add destructive object-field-update \emph{x.f := y} to the language.
\item Add if-then-else to the language.
\item Omit constructors from classes (and use an implicit constructor).
\item Note that the claim that all region variables will be unified with some concrete
region fails in the case of a method that recursively calls itself (and, hence, does not
terminate). In this case, the region variables in the return type are not constrained.
Similarly for non-terminating mutual recursion. These are uninteresting (in essence, the
return type can be taken to be unit instead).
\item Discuss the safety of virtual method invocation, which provides a limited form
of existentially quantifying regions and instantiating (unpacking) them because a
derived class may have more region parameters than a base class.
\item Discuss the subtyping rule used to check an overriding method (in the static
semantics).
\item Mention that constraints are in the theory of partial orders?
\end{itemize}

\section{Type Inference}
\label{sec:type-inference}

\name's region type system imposes a heavy annotation burden, and
manually annotating C\# standard libraries with region types
can be tedious. We now present our region type inference algorithm
that eliminates the need to write region type annotations.
% except on some higher-order functions.
Formally, the type inference
algorithm is an elaboration function from programs in $\absof{\FB}$
(i.e., \FB without region types, but with \C{letregion} and \C{open}
expressions, similar to the language introduced in
\S~\ref{sec:overview}) to programs in \FB.

\paragraph{Overview.}

Fig.~\ref{fig:type-inference-algo} presents the high-level outline of the type
inference algorithm.
The algorithm consists of the following steps:
\begin{enumerate}
 \item \emph{Region Parametrization}.
   The first step elaborates the input program by introducing \emph{formal region parameters}
   (for each class and method), and \emph{region variables} (representing yet undetermined
   \emph{actual region parameters}). We also introduce for each class and method, a
   \emph{predicate variable} ($\varphi$) to denote an undetermined set of constraints
   over the region parameters of that class/method.

%%    as follows:
%% \begin{enumerate}
%% \item For every class determine a set of \emph{region parameters} ($\rho$)
%%        it should be parametrized over.
%% \item For every field of the class, determine its region parametrized type by
%%        generating as many fresh \emph{actual region parameters} as required by its class (type),
%%        which in turn become formal region parameters of the containing class.
%% \item  For every method and function type, determine its corresponding region
%%    type template by introducing fresh region parameters as required by
%%    the type of each parameter and return value (in addition to the allocation region
%%    parameter).
%%  \item Elaborate every \C{new} expression, method call or function application by
%%    introducing \emph{fresh region variables} (as required by the region type
%%    template of the called method or function).
%% \item Introduce for each class and method, a \emph{predicate variable} ($\varphi$) to denote
%%    an undetermined set of constraints over the region parameters of that
%%    class/method.
%% \end{enumerate}

 \item \emph{Constraint Generation}.
   In the second step, we analyze the program to generate a set of constraints
   (over the region identifiers and the predicate variables)
   that must hold (as per the static semantics in Fig.~\ref{fig:fb-staticsem}).

 \item \emph{Constraint Solving}.
   We solve the generated set of constraints using our fixpoint constraint
   solving algorithm \csolvestar, which reduces the constraint solving problem to
   an abduction problem. If the original program in $\absof{\FB}$ contains unsafe
   references, for example, a reference from a transferable region to a
   stack region, then the constraints generated during the elaboration
   are not satisfiable. In such a case, \csolvestar{} fails to solve
   the constraints.
% in a Herbrand constraint system, and then relies on \csolve,
% our abduction solver for that domain. 

 \item If the solver succeeds, it returns substitution functions $\substFn_\rho$ and
  $\substFn_\varphi$ for free region and predicate variables, respectively, introduced in
  step 1. We apply these substitutions to the elaborated program to produce the final program.


\end{enumerate}



\begin{figure}
\begin{numcodeml}
Infer ($p$) =
  let $q$ = IntroduceRegionParameters($p$) in
  let $(r,C)$ = GenerateConstraints($q$) in
  match (SolveConstraints($C$) with
  | None $\longrightarrow$ None
  | Some ($s$) $\longrightarrow$ Some (ApplySolution $(s,r)$)
\end{numcodeml}

\caption{The type inference algorithm}
\label{fig:type-inference-algo}
\end{figure}

\begin{theorem}
\emph{(\textbf{Soundness})}
For any $p \in \absof{\FB}$, if Infer($p$) returns Some($t$), then
(1) $\absof{t} = p$, and
(2) $t$ is well-typed.
\end{theorem}



%%    This is used to replace every FGJ type in the program
%%    compute a polymorphic region type template
%%     for every class and method from their FGJ types. The templates contain
%%     region variables ($\rho$) to denote unknown region annotations, and
%%     predicate variables ($\varphi$) to denote unknown constraints over
%%     such region annotations. Free region variables are generalized in these
%%     types (which are, hence, polymorphic).
%%  
%%    Second, we make use of the computed region type templates to elaborate
%% expressions by introducing region variables to denote unknown region
%% arguments in \C{new} expressions, method calls and function
%% applications.
%% While elaborating expressions, we also build a system of
%% constraints that capture well-formedness requirements and subtyping
%% relationships between type templates that must hold (as per the static
%% semantics in Fig.~\ref{fb-staticsem}) for the elaboration to be valid.
%% Third, we lift expression elaboration and constraint generation to
%% methods and classes. Finally, we solve the constraints by making use
%% of our fixpoint constraint solving algorithm \csolvestar, which
%% reduces the constraint solving problem to an abduction problem in a
%% Herbrand constraint system, and then relies on \csolve, our abduction
%% solver for that domain. 

% Due to the presence of region-polymorphic
% recursion in \FB, constraints generated by the algorithm can be
% circular. More precisely, constraints generated can assume the form
% $\varphi \Leftrightarrow \phi \wedge F(\varphi)$, where $F$ is a
% non-idempotent substitution function for region variables in
% $\varphi$. In the constraint solving phase, the algorithm then relies
% on a fixpoint constraint solving algorithm called \csolvestar to solve
% the constraints and determine assignments for unknown region and
% predicate variables.

\subsection{Region Parametrization for Classes}
\label{sec:fb-templatization}

Region parametrization is an iterative process involving the following three steps,
the first two of which are mutually dependent on each other.

\emph{Introduction of Formal Region Parameters}.
For every class \C{C}, we identify a sequence of formal region parameters
$\pi_0, \cdots \pi_n$ that \C{C} should be parametric over.

\emph{Introduction of Actual Region Parameters}.
We then replace every instance of class \C{C} in the program by an instance
$\C{C}\langle \rho_0, \cdots, \rho_n \rangle$, where $\rho_0, \cdots, \rho_n$
are fresh identifiers denoting actual region parameters.

\emph{Predicate Variable Introduction}. For every class \C{C}, we introduce
a fresh predicate variable $\varphi$, which represents the yet undetermined
outlives constraints between the formal region parameters of class \C{C}.

We identify the region parameters of classes as follows.

\emph{Non-Recursive Classes}.
The class \C{Object} is defined to have a single region parameter $\pi_0$ (the allocation region).
The region parameters for any other non-recursive class \C{C} is determined
only after the region parameters of any class that \C{C} depends on have been
determined: this includes the base-class \C{B} of \C{C} and the class (type)
of any of its fields.
We replace every dependee type \C{T} in \C{C} by its instantiated type,
using fresh region parameters as needed.
The sequence of region parameters for \C{C} is defined to be
the sequence of region parameters for the base class \C{B} concatenated
with the list of all  fresh region parameters introduced while instantiating the types
of the fields in the class.
(The class inherits its allocation region from its base class. Note that if
a class does not specify an explicit base class, it has an implicit base class
\C{Object}.)

This transformation is illustrated below, using a non-generic \C{Pair} class:

\begin{tabular}{ccc}
\begin{minipage}{0.28\linewidth}
\begin{codejava}
class Pair
  $\extends$ $\ObjZ$
{
  $\ObjZ$ fst;
  $\ObjZ$ snd;
}
\end{codejava}
\end{minipage}
&
$\Rightarrow$
&
\begin{minipage}{0.5\linewidth}
\begin{codejava}
class Pair $\langle \rho_0, \rho_1, \rho_2 \; | \; \varphi \rangle$
  $\extends$ $\ObjZ \langle \rho_0 \rangle$
{
  $\ObjZ \langle \rho_1 \rangle$ fst;
  $\ObjZ \langle \rho_2 \rangle$ snd;
}
\end{codejava}
\end{minipage}
\end{tabular}

\emph{Recursive Classes}.
The region parameters for a recursive class is computed in
a similar fashion, with the following difference: any recursive
field is ignored while instantiating region parameters for the fields of
the class, and the region parameters of the recursive class are computed
as before. We then do parameter instantiation for all recursive fields,
such that their region annotations (the actial region parameters) are
exactly the same as the (formal) region parameters of the class.
The following example illustrates this for a non-generic \C{List} class.
The resulting class represents a linked list with spine in the region
$\rho_0$ and data objects in the region $\rho_1$.

\begin{tabular}{ccc}
\begin{minipage}{0.28\linewidth}
\begin{codejava}
class List
  $\extends$ $\ObjZ$
{
  $\ObjZ$ data;
  List next;
}
\end{codejava}
\end{minipage}
&
$\Rightarrow$
&
\begin{minipage}{0.5\linewidth}
\begin{codejava}
class List $\langle \rho_0, \rho_1 \; | \; \varphi \rangle$
  $\extends$ $\ObjZ \langle \rho_0 \rangle$
{
  $\ObjZ \langle \rho_1 \rangle$ data;
  $\C{List} \langle \rho_0, \rho_1 \rangle$ next;
}
\end{codejava}
\end{minipage}
\end{tabular}

The above technique can be extended to mutually recursive classes in a
straightforward manner, by simultaneously parametrizing them (and
then instantiating them).

\emph{Type-Parametric Classes}.
Type parameters of classes are handled as follows.
Consider a type-parametric class \C{C $\langle$ T $\extends$ B $\rangle$}.
The parametric type \C{T} is instantiated using the number of region parameters
that its bound \C{B} has. If no bound is specified for \C{T}, the bound is taken
to be \C{Object}, and \C{T} is instantiated with one region parameter.

\emph{Function Types}.
Since \FB{} is higher-order, fields of function type are allowed. The parameter instantiation step
instantiates function types as follows: the type of every parameter as well as the return value is
instantiated with fresh region identifiers, as described earlier, and finally these region identifiers
are generalized as formal region parameters of the function type.
For example, the function type $\C{List} \rightarrow \C{Object}$ is instantiated as
$\inang{\rho_0, \rho_1, \rho_2} \C{List} \inang{\rho_0,\rho_1} \rightarrow \C{Object}\inang{\rho_2}$.

%% Likewise, given the region-annotated definition of
%% \C{Pair} class from \S\ref{sec:fb-syntax} a region type template for a
%% method with FGJ type $\C{Pair}\inang{\C{A},\C{B}} \rightarrow \C{A}$
%% is \footnote{In our exposition, we assume that classes $\C{A}$ and
%% $\C{B}$ are trivial subclasses of $\ObjZ$ with no fields/methods. Like
%% $\ObjZ$, they accept one region parameter - the allocation region of
%% their objects.}\footnote{We abuse arrow notation to also represent
%% types of methods, but unlike function types, there is no allocation
%% region annotation atop the arrow in a method type.}
%% $\inang{\rhoalloc_0,\rho_1,\rho_2,\rho_3 \,|\, \varphi_0}
%% \C{Pair}\inang{\C{A},\C{B}} \inang{\rho_1,\rho_2,\rho_3} \rightarrow
%% \unitZ$, where $\rhoalloc_0$ and $\rho_{1-3}$ are fresh region
%% variables, and $\varphi_0$ is a fresh predicate variable denoting
%% unknown constraints over $\rhoalloc_0$ and $\rho_{1-3}$.

\subsection{Region Parametrization for Methods}

As the next step, we introduce region parameters for every method.
We do this by instantiating the types of all parameters and the
return value (of the method) using fresh region identifiers (as explained previously),
and then generalizing these region identifiers as formal region parameters
of the method. As for the classes, we also introduce a fresh predicate variable
$\varphi$ for every method.

We then consider every method invocation in the program, and introduce
fresh region variables representing the (yet unknown) actual region
parameters for this particular invocation.
%
We similarly perform instantiation for every constructor invocation
of the form \C{new T($\ldots$)}, by instantiating the type \C{T} as
before, turning it into \C{new T$\langle \rho_0, \cdots, \rho_n \rangle$($\ldots$)},
where $\rho_0, \cdots, \rho_n$ are fresh region variables.

\subsection{Constraints}
\label{sec:fb-constraintsem}

%% ************ MACROS *************
\newcommand{\hdOf}[2]{\C{class}\; #1\angAlpha\inang{\rhobar \,|\, #2} \extends \fbN}

% mathpartir's inferrule: seems to have problems with linebreaks
%% \newcommand{\genconstraint}[2]{\inferrule*{#1}{#2}}
%% \newcommand{\gcarrow}{\leadsto}
%% \newcommand{\gctransforms}{\models}

%% \newcommand{\gcrule}[2]{%
%% \begin{smathpar}\begin{array}{c}%
%% \renewcommand*{\arraystretch}{1.2}%
%% \RULE {#1} {#2}%
%% \end{array}\end{smathpar}%
%% }

% \newcommand{\SRULE}[2]{\frac{\begin{array}{c}\renewcommand*{\arraystretch}{1.5} #1\end{array}}
%                            {\begin{array}{c}\renewcommand*{\arraystretch}{1.5} #2\end{array}}}
\newcommand{\SRULE}[2]{\frac{\begin{array}{c} #1 \end{array}}{\begin{array}{c} #2 \end{array}}}

\newcommand{\beginrules}{%
\renewcommand*{\arraystretch}{1.2}%
\begin{smathpar}\begin{array}{lc}%
}
\newcommand{\myendrules}{\end{array}\end{smathpar}}

\newcommand{\lgcrule}[3]{%
[\rulelabel{#1}] & \SRULE {#2} {#3}%
\\[0.2cm]%
}

\newcommand{\lgcfact}[2]{%
[\rulelabel{#1}] & {#2} %
\\[0.2cm]%
}

\newcommand{\nl}{\\}

\newcommand{\subtypesym}{<:}
\newcommand{\typeok}[3]{{#1\,\vdash\,#2 \; \texttt{ok} \, \lhd #3}}
\newcommand{\exprok}[4]{{#1} \, \vdash \, {#2} : {#3} \, \lhd {#4}}
\newcommand{\subtypeok}[4]{{#1} \, \vdash \, {#2}  \subtypesym {#3} \, \lhd {#4}} 
\newcommand{\stdcontext}{\A,\env}

%% ************ END MACROS *************

\begin{figure*}[t!]

%%%%%%%%%%% Header Box %%%%%%%%%%%
\fbox{  \( \exprok{\stdcontext}{e}{\tau}{C} \)}
\\

\beginrules

%%%%%%%%%%% () and x %%%%%%%%%%%
\lgcfact{UNIT}{\exprok{\stdcontext}{\unitval}{\unitZ}{\{\}}}

\lgcfact{VAR}{\exprok{\stdcontext}{x}{\env(\tau)}{\{\}}}

% \begin{minipage}{1.2in}
% \begin{smathpar}
% \begin{array}{l}
% \renewcommand*{\arraystretch}{1.2}
% \exprok{\stdcontext}{\unitval}{\unitZ}{\{\}} \\
% \exprok{\stdcontext}{x}{\env(\tau)}{\{\}}
% \end{array}
% \end{smathpar}
% \end{minipage}

%%%%%%%%%%% FIELD-ACCESS: e.f %%%%%%%%%%%
\lgcrule{FIELD-ACCESS}
  {
    \exprok{\stdcontext}{e}{\tau'}{C} \spc
    \bar{f}:\taubar \,=\, \fields(\bound_{\A.\aenv}(\tau'))
  }
  {
    \exprok{\stdcontext}{e.f_i}{\tau_i}{C}
  }

%%%%%%%%%%% NEW %%%%%%%%%%%
  \lgcrule{NEW}
  {
    \typeok {\A} {\fbN} {C_1} \spc
    \exprok {\stdcontext} {\bar{e}} {\bar{\tau'}} {C_2}
    %% \spc C_3 = \{ \isvalid{\A.\phicx}{\allocRgn(\fbN)=\ralloc} \}
    \nl
    \fields(\fbN) = \bar{f} : \taubar \spc
    \subtypeok {\A} {\bar{\tau'}} {\bar{\tau}} {C_3}
  }
  {
    \exprok {\stdcontext}   {\C{new} \fbN(\bar{e})} {\fbN} {\cup_{i=1}^3 C_i}
  }

%%%%%%%%%%% LET %%%%%%%%%%%
  \lgcrule{LET}
  {
    \exprok{\stdcontext}{e_1}{\tau_1}{C_1} \spc
    \exprok{\A,{\env[x\mapsto\tau_1]}}{e_2}{\tau_2}{C_2} \\
  }
  {
    \exprok{\stdcontext}{\letexp{x}{e_1}{e_2}}{\tau_2}{C_1 \cup C_2}
  }

%%%%%%%%%%% NEW-REGION %%%%%%%%%%%
  \lgcrule{NEW-REGION}
  {
    \typeok {\A.\aenv} {T} {C_1} \spc
    \rgn \in \A.\rhoenv
    \nl
    \exprok{(\{\rho\},\A.\aenv,true),{\cdot}}{e}{T@\rho}{C_2}
  }
  {
    \exprok {\stdcontext} {\C{new}\; \RgnZ\inang{T} \inang{\toprgn} (\lambdaexp{\rgn}{\rho}{}{e})}
        {\RgnZ\inang{T} \inang{\toprgn}} {C_1 \cup C_2}
  }

%%%%%%%%%%% METHOD-INV %%%%%%%%%%%
  \lgcrule{METHOD-INV}
  {
    \exprok {\stdcontext} {e_0} {\tau} {C_1} \spc C_4 = \{ \rbar \in \A.\rhoenv \}
    \nl
    \mtype(m,\bound_{\A.\aenv}(\tau)) = \inang{\rhobar \,|\, \phi}\bar{\tau^1}\rightarrow{\tau^2}
    \nl
%   \substFn = [\rbar/\rhobar] \\
    \typeok {\A} {\inang{\rhobar \,|\,\phi}\bar{\tau^1}\rightarrow{\tau^2}} {C_2}
       \spc
       \exprok {\stdcontext} {\bar{e}} {[\rbar/\rhobar](\bar{\tau^1})} {C_3}
    \nl
%   \subtyp{\A}{\bar{\tau'}}{\substFn(\bar{\tau^1})} \spc
    C_5 = \{ \isvalid{\A.\phicx}{[\rbar/\rhobar](\phi)} \}
  }
  {
    \exprok {\stdcontext} {e_0.m\inang{\rbar}(\bar{e})} 
       {[\rbar/\rhobar](\tau^2)} {C_1 \cup C_2 \cup C_3 \cup C_4 \cup C_5}
  }

%%%%%%%%%%% LET-REGION %%%%%%%%%%%

\lgcrule{LET-REGION}{
   \A = (\rhoenv,\aenv,\phicx)  \spc \rgn \notin \rhoenv \spc
      \A' = (\rhoenv \cup \{\rgn\}, \aenv, \phicx \conj (\rhoenv \outlives \rgn))
   \nl
   \exprok{\A',\env} {e_a} {\tau} {C_1} \spc
      \typeok {\A} {\tau} {C_2}
}{
   \exprok{\stdcontext} {\letregion{\rgn}{e_a}} {\tau} {(C_1 \cup C_2)}
}

\myendrules

\caption{Constraint generation}
\label{fig:constraint-gen-0}
\end{figure*}



The constraint generation algorithm mimics the static type checker, but accumulates
constraints that must hold for the type checking to succeed.

\paragraph{Syntax of Constraints.}
The constraints are expressed in terms of region identifiers ($\pi$ and $\rho$)
and predicate variables ($\varphi$).

Recall that \emph{region identifiers} are used in four different roles.
A region identifier may serve as
(a) A \emph{formal region parameter} of a class or method, or
(b) a \emph{static region identifier} introduced by a \C{letregion} construct, or
(c) an  \emph{open transferable region identifier} introduced by an \C{open} construct, or
(d) a \emph{region variable}, introduced to represent an unknown actual region parameter
of a method invocation or object alloation,
which will be bound to a region identifier of one of the three
preceding kinds by the end of the type inference.

Every predicate variable $\varphi$ denotes an unknown \emph{region-constraint},
%% over a set of fixed formal region parameters,
where the set of region-constraints $\phi$ is defined by:
\begin{smathpar}
\begin{array}{lcl}
\phi & \coloneqq & true \ALT \rho \outlives \rho \ALT \rho = \rho \ALT \phi \conj \phi \\
\end{array}
\end{smathpar}
We will use the term \emph{validity constraint} to denote an entailment constraint
of the form $\isvalid{\varphi}{\rho_1 \outlives \rho_2}$.

Our constraints also make uses of \emph{pending substitutions}\footnote{we borrowed this terminology from~\cite{ltpldi08}} $F$.
A pending substitution serves to bind formal region parameters in $\varphi$ to the actual region parameters
used in a particular context:
\begin{smathpar}
\begin{array}{lcl}
F & \coloneqq & \cdot \ALT [\rho/\rho]F \\
\end{array}
\end{smathpar}
E.g., in the validity constraint $\isvalid{\rgn_1 \outlives
\rgn_2}{[\rgn_1/\rho_1][\rgn_2/\rho_2]\varphi}$, the pending substitution
is $[\rgn_1/\rho_1][\rgn_2/\rho_2]$. Any concrete formula (call it
$\phisol$) over variables $\rho_1$ and $\rho_2$ is a solution to
$\varphi$ if and only the formula obtained by substituting $\rgn_1$
and $\rgn_2$ for $\rho_1$ and $\rho_2$ (resp.) in $\phisol$ is
deducible from $\rgn_1 \outlives \rgn_2$.

The constraints generated are of the following kinds:
\begin{itemize}

\item Well-formedness constraints of form $\rho \in \rhoenv$,
restricting the domain of unification for a region variable ($\rho$)
to a constant set $\rhoenv = \{ \pi_1, \cdots, \pi_n \}$ of regions in scope,

\item Well-formedness constraints of form $\tywf{\rhoenv}{\varphi}$, restricting the domain of a predicate
variable ($\varphi$) to the set of all possible constraint formulas over a fixed set of 
regions ($\rhoenv = \{ \pi_1, \cdots, \pi_n \}$) in scope, and

\item Validity constraints of the form $\isvalid{\varphi_{\C{C}}}{\varphi_{\C{B}}}$
  (e.g., to indicate that the region-constraint of a derived class \C{C} is stricter than
  the region-constraint of its base class \C{B}).

\item Validity constraints of the form $\isvalid{\varphi_{\C{C}}}{{F(\varphi_{\C{T}})}}$
  (e.g., to indicate the region-constraint of a class \C{C} must imply the
  region-constraint of each of its fields).

\item Validity constraints of form $\isvalid{\varphi_i \conj \phictxt} {\phicstr}$
where $\varphi_i$ is a predicate variable (representing the precondition of a
method to be determined), $\phicstr$ is a region constraint that is \emph{required}
to hold at a particular program point (within the method), and $\phictxt$ is
a region constraint that is \emph{known} to hold at that program point.

%
%% Formulas $\phictxt$ and $\phicstr$ are concrete, i.e., free
%% of predicate variables and pending substitutions. While $\phictxt$
%% captures relationships that are \emph{known} to hold between concrete
%% region identifiers (i.e., $\rgn$'s) when the constraint was generated,
%% $\phicstr$ captures relationships that are \emph{required} to hold
%% among region varibles (i.e., $\rho$'s), or relationships between
%% region variables and identifiers.
%

\item Validity constraints of the form $\isvalid{\varphi_i \conj \phictxt} {F_j(\varphi_j)}$
generated by an invocation of a method with precondition $\varphi_j$ (where $\phictxt$ and
$\varphi_i$ are as above).

\end{itemize}

\paragraph{Constraint Generation.}

\paragraph{Constraint Solution}
Solving the set ($C$) of constraints entails finding an
assignment for each predicate variable $(\varphi$) and each region
variable ($\rho$) that occurs free in $C$, that satisfies all
the validity constraints as well as the well-formedness constraints on
$\varphi$ and $\rhobar$.
% To simplify presentation, we think of $C$ as being parameterized on
% $\varphi$ and $\rhobar$, and write it as $C\lbrack \varphi,\rhobar
% \rbrack$. We now formalize the constraint satisfaction problem, and
% its solution.

\begin{definition}
\emph{(\textbf{Constraint Satisfaction Problem (CSP)})}
A constraint satisfaction problem is a tuple
$(\bar{\pi}, \bar{\rho}, \bar{\varphi}, \regionDeltaMap, \predDeltaMap, C)$,
where $\bar{\pi}$ is a set of region constants,
$\bar{\rho}$ is a set of region variables,
$\bar{\varphi}$ is a set of predicate variables,
$\regionDeltaMap : \bar{\rho} \rightarrow 2^{\bar{\pi}}$ is a function that
specifies a unification domain for each region variable,
$\predDeltaMap : \bar{\varphi} \rightarrow 2^{\bar{\pi}}$ is a function that
specifies the domain for each predicate variable,
and $C$ is a set of validity constraints, where the $i$'th validity constraint
assumes one of the following forms:
\begin{center}
\(
    \isvalid{\phictxt^{i} \conj \varphi}
            {\phicstr^{i}}\qquad
    \isvalid{\phictxt^{i} \conj \varphi}
            {F_i(\varphi)}
\)
\end{center}
The solution to the constraint satisfaction problem is a pair $(\regionSubstFn,\predSubstFn)$,
where $\regionSubstFn$ is a map from $\rhobar$ to $\bar{\pi}$
and $\predSubstFn$ is a map from $\bar{\varphi}$ to a region-constraint formula such that
\begin{itemize}
  \item $\regionSubstFn(\rho) \in \regionDeltaMap(\rho)$, for every $\rho \in \bar{\rho}$,

  \item $\predSubstFn$ is well-formed under $\predDeltaMap$
    (i.e., $\tywf{\predDeltaMap(\varphi)}{\predSubstFn(\varphi)}$, for every $\varphi \in \bar{\varphi}$).

  \item Every sequent in $C$ is valid after substitutions $\predSubstFn$ and $
    \regionSubstFn$.
  %% $C\lbrack (\regionSubstFn,\predSubstFn) \rbrack$ is valid.
\end{itemize}
\end{definition}

\textbf{TODO: The following not yet integrated into above.}

\paragraph{Constraints Example 1} Consider the $\C{Pair}$ class
template from \S\ref{sec:fb-templatization}. Following constraints are
generated during its elaboration (Constrains are identified with
$\mathbf{c_i}$'s. Some trivial constraints, such as $\rho_4 \in
\rhoenv_0$ and $\rho_5 \in \rhoenv_1$, where $\rhoenv_0 =
\{\rhoalloc_0,\rho_{0-4}\}$ and $\rhoenv_1 = \rhoenv_0 \cup
\{\rho_5\}$, have been elided): 
\begin{smathpar}
\begin{array}{l}
  \csid{1} \tywf{\rhoenv_0}{\varphi_0} \qquad
  \csid{2} \isvalid{\varphi_0}{\rho_0 \outlives \rhoalloc_0 \conj \rho_1
     \succeq \rhoalloc_0 \conj \rho_4 = \rhoalloc_0} \\
  \csid{3} \isvalid{\varphi_0}{\rho_2 = \rho_0} \spc
  \csid{4} \isvalid{\varphi_0}{\rho_3 = \rho_1} \\
  \csid{5} \isvalid{\varphi_0 \conj \varphi_1} {\rho_5 = \rho_0}\spc
  \csid{6} \tywf{\rhoenv_1}{\varphi_1} \qquad
\end{array}
\end{smathpar}

\paragraph{Constraints Example 2} Let us add to the \C{Pair} class a
contrived method $\C{alt}$ that accepts a \C{Region} object \C{r}, a
\C{Pair<A,A>} object \C{q}, and an \C{A} object \C{y}. It assigns
\C{y} to \C{fst} and \C{snd} fields of \C{q}, and calls itself
recursively with the same region, a new \C{Pair} object allocated in a
local region, and an \C{A} object referred by the \C{snd} field of the
pair inside the region. \C{alt} never terminates.  Elaboration phase
elaborates the method to the following region-annotated
definition\footnote{In reality, elaboration uses new region variables
as parameters to the constructor and method calls, and then generates
constraints that unify them with actuals. In our examples, to avoid
clutter due to trivial constraints,we coalesced both steps and show
the actuals instead.}(The original definition of \C{alt} can be
obtained by erasing all the region annotations from the elaborated
version):
% \begin{codejava}
% unit alt<$\rhoalloc_2$,$\rho_{6-10}$ | $\varphi_2$>(Pair<A,A><$\rho_6$,$\rho_7$,$\rho_8$> p,
%                           A<$\rho_9$> x, A<$\rho_{10}$> y) {
%   p.fst = x; p.snd = y; 
%   this.alt<$\rhoalloc_2$,$\rho_7$,$\rho_8$,$\rho_{10}$,$\rho_9$>(A,y,x);
% }
% \end{codejava}
\begin{codejava}
unit alt<$\rhoalloc_2$,$\rho_{6-9}\,$|$\,\varphi_2$>(Region<Pair<A,A>><$\toprgn$> r, 
                Pair<A,A><$\rho_{6-8}$> q, A<$\rho_{9}$> y) {
  q.fst := y; q.snd := y; 
  open r as p@$\rgn_0$ in
    letregion $\rgn_1$ in
      let x = new Pair<A,A><$\rgn_1$,$\rgn_0$,$\rgn_0$>
                      (p.fst,p.fst) in
        alt<$\rgn_{1}$,$\rgn_{1}$,$\rgn_{0}$,$\rgn_{0}$,$\rgn_{0}$>(r,x,p.snd)
}
\end{codejava}
% Note that, to avoid clutter, we have already resolved appropriate
% region arguments to the recursive call, instead of introducing new
% region variables and generating equality constraints on them. 
Constraints generated during the elaboration are shown below
(let $\rhoenv_2 = \{\rhoalloc_0,\rho_{0-4},\rhoalloc_2,\rho_{6-9}\}$ ):
\begin{smathpar}
\begin{array}{l}
\csid{7} \tywf{\rhoenv_2}{\varphi_2}\qquad
\csid{8} \isvalid{\varphi_1 \conj \varphi_2}
    {\rho_7 \outlives \rho_6 \conj \rho_8 \outlives \rho_6} \\
\csid{9} \isvalid{\varphi_1 \conj \varphi_2}{\rho_7 = \rho_9} \qquad
\csid{10} \isvalid{\varphi_1 \conj \varphi_2}{\rho_{8} = \rho_9} \\
\csid{11} \isvalid{\varphi_1 \conj \varphi_2 \conj \rgn_0 \outlives
\rgn_1} 
    {[\rgn_1/\rhoalloc_2][\rgn_1/\rho_6][\rgn_0/\rho_{7-9}]\varphi_2}
\end{array}
\end{smathpar}

% \paragraph{Constraints Example 1} The original source of all
% outlives constraints is the validity constraint $C_3$ in $\elabClass$
% function (Fig.~\ref{fig:fb-elabmeth}). In context of the $\C{Pair}$
% class template from \S\ref{sec:fb-templatization}, the constraint is
% as following:
% \begin{center}
%   \(\isvalid{\varphi_0}{\rho_0 \succeq \rhoalloc_0 \conj \rho_0
%   \succeq \rhoalloc_0 \conj \rho_4 = \rhoalloc_0}\)
% \end{center}

% \paragraph{Constraints Example 2} Consider a contrived method
% $\C{alt}$ that accepts an argument $\C{p}$ of type
% $\C{Pair}\inang{\C{A},\C{A}}$, and two objects ($\C{x}$ and $\C{y}$)
% of type $A$. It then assigns one object to $\C{p.fst}$ and other to
% $\C{p.snd}$, and calls itself recursively without ever terminating.
% Which one is assigned to $\C{fst}$, and which to $\C{snd}$, is
% alternated between recursive calls. The region-annotated definition of
% $\C{alt}$ is shown below:
% \begin{codejava}
% unit alt<$\rhoalloc$,$\rho_{0-4}$ | $\varphi_0$>(Pair<A,A><$\rho_0$,$\rho_1$,$\rho_2$> p,
%                           A<$\rho_3$> x, A<$\rho_4$> y) {
%   return p.fst = x; p.snd = y; 
%          this.alt<$\rhoalloc$,$\rho_1$,$\rho_2$,$\rho_4$,$\rho_3$>(A,y,x)
% }
% \end{codejava}
% Note that, to avoid clutter, we have already resolved appropriate
% region arguments to the recursive call, instead of introducing new
% region variables and generating equality constraints on them. Rest of
% the validity constraints generated while elaborating $\C{alt}$ to the
% above region-annotated definition are shown below:
% \begin{center}
% \(\isvalid{\varphi_0}{\rho_1 \outlives \rho_0 \conj \rho_2 \outlives \rho_0}\)
% $\quad$
% \(\isvalid{\varphi_0}{\rho_1 = \rho_3}\)
% $\quad$
% \(\isvalid{\varphi_0}{\rho_2 = \rho_4}\)\\
% \(\isvalid{\varphi_0}{[\rho_3/\rho_4][\rho_4/\rho_3]\varphi_0}\)
% \end{center}
% The first constraint is generated by $\typeOk$ on the type of $\C{p}$.
% Second and third are generated by the the assignment expressions, and
% the last constraint is generated from the recursive call.

\subsection{Solving the CSP}

\paragraph{Properties of generated constraints}
The set of generated constraints have two interesting properties that allow us to
solve the constraints efficiently.
%
The first property is that every region variable is forced to be unified with some
region constant by the constraints, except in one special case.
This follows because $\FB$ does not admit uninitialized variables.
(The only case where this does not hold is in the case of a recursive function
that calls itself in a non-terminating fashion: in this case, the function never
returns a value, and so the return-value can be typed as anything. In the sequel,
we assume that the return value of such a function is typed to be \C{unit}.)
% , and the type system uniquely determines the region parameters of every value
%
The second property is that the \emph{context region constraint} $\phictxt$
occurring on the antecedent of any validity constraint is a conjunction of
outlives-constraints of the form $\pi \outlives \pi_s$ where $\pi_s$
is a static region identifier and $\pi$ is a region constant (either a formal
region parameter, or static region identifier, or open transferable region identifier).
Furthermore, if $\phictxt$ includes any conjunct $\pi \outlives \pi_s$, then
it includes every conjunct $\pi_f \outlives \pi_s$ for every $\pi_f \in \predDeltaMap(\varphi)$.

\paragraph{Constraint Solver.}
Solving the constraints is an iterative process, consisting of the following
steps, and this process fails at any point if it determines that no solution
exists for the set of constraints:

\begin{enumerate}

\item
    A constraint of the form $\isvalid{\varphi}{\rho_i = \rho_j}$ unifies
$\rho_i$ and $\rho_j$ if at least one of them is a region variable.
If $\rho_i$ and $\rho_j$ are distinct region constants, then the constraint
solver fails.

\item
When a region variable $\rho$ is unified with a region constant $\pi$,
if $\pi \in \regionDeltaMap(\rho)$, we update $\regionSubstFn(\rho)$ to be $\pi$,
and replace every occurrence of $\rho$ by $\pi$ in the set of constraints.
If $\pi \not\in \regionDeltaMap(\rho)$, the constraint solver fails.

\item
\label{item:context}
Consider any constraint of the form $\isvalid{\varphi \conj \phictxt}{\pi_i \outlives \pi_j}$,
where $\pi_i$ and $\pi_j$ are both region constants.
If $\{ \pi_i, \pi_j \} \subseteq \predDeltaMap(\varphi)$, then we simply add
$\pi_i \outlives \pi_j$ as an additional conjunct to $\predSubstFn(\varphi)$.
Otherwise, we check if $\isvalid{\phictxt}{\pi_i \outlives \pi_j}$.
If this entailment does not hold, the constraint solver fails, since no valid solution is possible.

\item
  When any outlives constraint $\phi$ is added to $\predSubstFn(\varphi)$,
  then for every existing validity constraint $\isvalid{\varphi' \conj \phictxt}{F(\varphi)}$,
  we produce a new validity constraint $\isvalid{\varphi' \conj \phictxt}{F(\phi)}$. 
\end{enumerate}

Point~\ref{item:context} makes use of the special structure of the generated
constraints: namely that $\phictxt$  is a conjunction of outlives-constraints of the
form $\pi \outlives \pi_s$ where $\pi_s$ is a static region identifier and $\pi$ is a
region constant. As a result, we can show that $\isvalid{\varphi \conj \phictxt}{\pi_i \outlives \pi_j}$
iff $\isvalid{\varphi}{\pi_i \outlives \pi_j}$ or $\isvalid{\phictxt}{\pi_i \outlives \pi_j}$.

This iterative process is a standard fixed point computation process.
It is possible to encode this computation using a set of Datalog rules, and compute the
solution using any Datalog engine.
Alternatively, note that the above computation, in the absence of context constraints
$\phictxt$ on the left-hand-side of the entailment, is a special instance of context-free
reachability in graphs. Algorithms for context-free reachability can be adapted to
incorporate the above treatment of context constraints $\phictxt$.

%% \emph{Class Constraints}. 
%% Computing the class constraint $\varphi_{\C{C}}$ corresponding to a class \C{C} is
%% fairly straightforward. It is the conjunction of
%% \begin{itemize}
%% \item $\conj_{1 \leq i \leq n} (\rho_i \outlives \rho_0)$
%% \item $\varphi_{\C{B}}$
%% \item $[\rho_a/\rho_f]\varphi_{\C{F}}$
%% \end{itemize}

\subsubsection{Non-Recursive Constraints}
\label{sec:csolve}

We first describe how we solve a non-recursive CSP $\csp$ where
$c\lbrack \varphi,\rhobar \rbrack$ is of the form $\isvalid{\Phicx
\conj \varphi}{\Phics}$. 

Our approach is based on the observation that $\FB$ does not admit
null values or uninitialized variables, thus forcing every region
variable to be unified with some concrete region.
%
The constraint formula $\Phics$ captures all such unification constraints on $\rhobar$.
Since the set of all concrete region identifiers is the
unification domain ($\rhoenv$) of CSP, this means for every $\rho_i$
with a well-formedness constraint as $\rho_i \in \rhoenv_i$, there
exists a $\rgn \in \rhoenv$ such that and $\isvalid{\Phicx \conj
\Phics}{\rho = \rgn}$. However, $\rgn$ may not belong to $\rhoenv_i$,
in which case the well-formedness constraint on $\rho_i$ is not
satisfied, and constraint solving must fail.  Therefore, there exists
a unique assignment $\substFn$ such that $c\lbrack
\phi_{sol},\substFn(\rhobar)$ is satisfied, regardless of
$\phi_{sol}$.

To obtain $\phisol$, we make use of another observation. Let $c\lbrack
\varphi, \substFn(\rhobar) \rbrack$ be the following constraint:
\begin{center}
\(
  \isvalid{ \Phicx \conj \varphi}
        {\Phics'}
\)
\end{center}
Consider a maximally weak formula $\phi$ such that
$\isvalid{\Phicx}{\phi \Leftrightarrow \Phics'}$. 
% Since $\phi$ is
% maximally weak, there does not exist a $\phi'$ satisfying all the
% below three conditions:
% \begin{itemize}
% \item $\isvalid{\Phicx}{\phi' \Leftrightarrow \Phics'}$.
% \item $\isvalid{\cdot} {\phi \Rightarrow \phi'}$.
% \item $\isnotvalid{\cdot}{\phi' \Rightarrow \phi}$.
% \end{itemize}
Clearly, $\tywf{\rhoenv}{\phi}$. However, since $\rhoenv_{\varphi}
\subseteq \rhoenv$, we have two cases:
\begin{itemize}

\item Case $\tywf{\rhoenv_{\varphi}}{\phi}$: This means that $\phi$ is
a solution to $\varphi$. 
% Furthermore, it is the weakest solution, because it is equivalent to
% $\Phics$ under $\Phicx \conj \bigwedge_j(\rho_j =
% \substFn(\rho_j))$, and any other solution has to stronger than
% $\Phics$ under the same context.  

\item Case $\tynwf{\rhoenv_{\varphi}}{\phi}$: In this case, $\phi$
contains at least one equality or outlives constraint on two region
identifiers, $\rgn_i$ and $\rgn_j$, where (a). $\rgn_i,\rgn_j \in
\rhoenv - \rhoenv_{\varphi}$, or (b). $\rgn_i \in \rhoenv -
\rhoenv_{\varphi}$ and $\rgn_j \in \rhoenv_{\varphi}$, such that the
constraint is not implied by $\Phicx$ (if it is implied, then $\phi$
is not maximally weak). Let us denote such constraint on $\rgn_i$ and
$\rgn_j$ as $\phi_{ij}$. Now, let us consider a solution $\phisol$ to
$\varphi$, which means that $\isvalid{\Phicx \conj \phisol}{\Phics'}$.
Since $\isvalid{\Phicx}{\phi \Leftrightarrow \Phics'}$, we have
$\isvalid {\Phicx \conj \phisol}{\phi}$. Since
$\isvalid{\phi}{\phi_{ij}}$, we have $\isvalid {\Phicx \conj
\phisol}{\phi_{ij}}$, although $\isnotvalid{\Phicx}{\phi_{ij}}$. But
this is impossible. To see why, recall that all the constraints on
identifiers in $\rhoenv_{\varphi}$ occuring in $\Phicx$ are subsumed
by $(\rhoenv - \rhoenv_{\varphi}) \outlives \rhoenv_{\varphi}$.
Therefore, it is impossible to derive $\phi_{ij}$ from $\Phicx$ by
only adding constraints on $\rgn \in \rhoenv_{\varphi}$. Hence, such a
solution $\phisol$ cannot exist.
\end{itemize}

\textbf{TODO: Is the above inequality correct?}

The above discussion hints at an algorithm to compute a solution to
$\varphi$: find a maximally weak $\phisol$ such that $\isvalid{\Phicx}
{\phisol \Leftrightarrow \Phics'}$. If $\tywf{\rhoenv_{\varphi}}
{\phisol}$, then $\phisol$ is the solution. Otherwise, there is no
solution to $\varphi$. We now describe a graph-based algorithm to
compute maximally weak $\phisol$.

\paragraph{Definitions} Given a constraint formula $\phi$, we define
its graph encoding $G(\phi)=(V(\phi),E(\phi))$ as a digraph whose
vertices ($V(\phi)$) are free region variables and identifiers in
$\phi$, and whose edges ($E(\phi)$) denote outlives constraints
in\footnote{We alternate between viewing a constraint formula ($\phi$)
as a set of constraints and a conjunction of constraints in this
description} $\phi$.  That is, if $\phi$ contains a constraint $\rho_1
\outlives \rho_2$, then $\rho_1,\rho_2 \in V(\phi)$ and
$(\rho_1,\rho_2) \in E(\phi)$. Equality constraints are treated as a
conjunction of symmetric outlives constraints for the purpose of graph
encoding. Conversely, given a digraph $G$, we define its constraint
encoding $\Phi(G)$ in a straightforward manner. We say that a graph
$G_1=(V_1,E_1)$ is as connected as graph $G_2=(V_2,E_2)$ if $V_2
\subseteq V_1$, and for every $\rho_1, \rho_2 \in V_2$, if there
exists a path between $\rho_1$ and $\rho_2$ in $G_2$, then there must
exist a path between same pair of vertices in $G_1$. 

A maximally weak $\phisol$ that satisfies $\isvalid{\phictxt}{\phisol
\Leftrightarrow \phicstr}$ is a constraint encoding of the smallest
graph $G$ (i.e., $\Phi(G)$) such that $G(\phictxt) \cup G$ is as
connected as $G(\phicstr)$. The problem of computing such a $G$ is
equivalent to the problem finding of finding minimum number of edges
to add to $G(\phictxt)$ such that it is as connected as $G(\phicstr)$.
Algorithms to solve the later problem are known to
exist~\cite{siam92}.

Solving the constraints $c_{12}$ and $c_{13}$ by reducing them to
graph augmentation problems, as described above, results in the
following solutions for $\varphi_0$ and $\varphi_1$ (symmetric
outlives constraints are replaced with equalities):
\begin{center}
\(
  \varphi_1 \Leftrightarrow \rho_0 \outlives \rhoalloc_0 \conj \rho_1
     \succeq \rhoalloc_0 \conj \rho_4 = \rhoalloc_0 \conj 
     \rho_2 = \rho_0 \conj \rho_3 = \rho_1
\)\\
\(
  \varphi_2 \Leftrightarrow \rho_5 = \rho_0
\)
\end{center}

\subsubsection{Solving Recursive Constraints}

\textbf{TODO: FP iteration should use $\Phicx$}

Substituting the above solutions for $\varphi_0$ and $\varphi_1$ in
$c_{14}$ gives us a recursive constraint of the following form:
\begin{center}
\(
  \isvalid{\phictxt \conj \varphi}{\phicstr \conj F(\varphi)}
\)
\end{center}
Where $\phictxt$ and $\phicstr$ are concrete constraint formulas, and
$F$ is a substitution function, not necessarily idempotent. We now
extend constraint solving to recursive CSP $\csp$, where $c\lbrack
\varphi,\rhobar \rbrack$ assumes the form $\isvalid{\Phicx \conj
\varphi}{\Phics \conj \bigwedge_i F_i(\varphi)}$. For the sake of
brevity, we define $G(\varphi) = \Phics \conj \bigwedge_i
F_i(\varphi)$, and use $F$ in $c\lbrack \varphi,\rhobar \rbrack$: 
\begin{center}
\(
  \isvalid{\Phicx \conj \varphi}{G(\varphi)}
\)
\end{center}
To solve the above recursive constraint, we start with the observation
that the set of all constaints ($\phi$) over $\rhoenv \cup
\{\rhobar\}$ is a lattice, where $\phi_1 \le \phi_2 \triangleq
\isvalid{\cdot}{\phi_1 \Rightarrow \phi_2}$. Note that $G$ is a
monotone over the lattice:
\begin{center}
  $\forall \phi_1,\phi_2.\; \isvalid{\cdot}{(\phi_1 \Rightarrow
  \phi_2) \Rightarrow (G(\phi_1) \Rightarrow G(\phi_2))}$
\end{center}
From Knaster-Tarski's theorem, we know that $G$ has a greatest fixed
point ($\phi_f$), with following properties:
\begin{itemize}
\item \textbf{Property 1} $\phi_f = G(\phi_f)$, which means
$\isvalid{\cdot}{\phi_f \Leftrightarrow G(\phi_f)}$
\item \textbf{Property 2} $\forall \phi_f'$ such that $\phi_f' =
G(\phi_f')$, we have $\phi_f' \le \phi_f$, which means $\isvalid
{\cdot} {\phi_f' \Rightarrow \phi_f}$.
\end{itemize}
We therefore compute the greatest fixed point ($\phi_f$) of $G$, and
convert the recursive constraint to the following non-recursive
constraint:
\begin{center}
\(
  \isvalid{\Phicx \conj \varphi}{\phi_f}
\)
\end{center}
The technique described in \S~\ref{sec:csolve} now suffices to solve
the above constraint.

Let $H(\varphi_2)$ denote the consequent of the recursive constraint
$c_{14}$. Its greatest fixed point ($\phi_f$) is shown below:
\begin{center}
\(
  \rho_7 \outlives \rho_6 \conj \rho_8 \outlives \rho_6 \conj \rho_7 =
  \rho_9 \conj \rho_8 = \rho_9 \conj \rgn_0 \outlives \rgn_1 \conj
  \rgn_0 = \rgn_0
\)
\end{center}
Computing a maximally weak $\phisol$ such that $\isvalid{\varphi_1
\conj \rgn_0 \outlives \rgn_1}{\phisol \Leftrightarrow \phi_f}$
results in the following solution for $\varphi_2$ that meets its
well-formedness requirement ($\tywf{\{\rhoalloc_0,\rho_{}0-4,
\rhoalloc_2, \rho_{6-9}\}}{\varphi_2} $):
\begin{center}
\(
  \rho_7 \outlives \rho_6 \conj \rho_8 \outlives \rho_6 \conj \rho_7 =
  \rho_9 \conj \rho_8 = \rho_9
\)
\end{center}



% Note that $F$ is a non-idempotent substitution function with following
% properties:
% \begin{itemize}
% \item $F$ is a monotone: 
% \begin{center}
%   $\forall \phi_1,\phi_2.\; \isvalid{\cdot}{(\phi_1 \Rightarrow
%   \phi_2) \Rightarrow (F(\phi_1) \Rightarrow F(\phi_2))}$
% \end{center}

% \item $F$ distributes over conjunction (mind the syntactic equality):
% \begin{center}
%   $\forall \phi_1,\phi_2.\; F(\phi_1 \conj \phi_2) = F(\phi_1) \conj
%   F(\phi_2)$
% \end{center}
% \end{itemize}

% Define $G(\varphi) = \Phics \conj F(\varphi)$. Note that $G$ is also a
% monotone, hence $G$ has greatest fixed point (GFP) in the lattice. Let
% $\phi_f$ be the GFP of $G$. Following properties ensue:
% \begin{itemize}
% \item \textbf{Property 1} $\phi_f = G(\phi_f)$, which means
% $\isvalid{\cdot}{\phi_f \Leftrightarrow G(\phi_f)}$
% \item \textbf{Property 2} $\forall \phi_f'$ such that $\phi_f' =
% G(\phi_f')$, we have $\phi_f' \le \phi_f$, which means $\isvalid
% {\cdot} {\phi_f' \Rightarrow \phi_f}$.
% \end{itemize}

% \begin{theorem}
% $\phi_f$ is a solution.
% \end{theorem}
% \begin{proof}
% $\phi_f$ is GFP of $G$. Therefore, $\isvalid{\cdot}{\phi_f
% \Leftrightarrow G(\phi_f)}$. By strengthening the context, we get
% $\isvalid{\Phicx}{\phi_f \Leftrightarrow G(\phi_f)}$. It follows that
% $\isvalid{\Phicx \conj \phi_f}{G(\phi_f)}$. Hence, $\phi_f$ is a
% solution for $\varphi$.
% \end{proof}

% \begin{theorem}
% If $\phisol$ is a solution of $\varphi$, then $\isvalid{\Phicx \conj
% \phisol}{\phi_f}$
% \end{theorem}
% \begin{proof}
%   Since $\phisol$ is a solution, $\isvalid{\Phicx}{\phisol \Rightarrow
%   G(\phisol)}$. Therefore, $\isvalid{}{\Phicx \conj \phisol \Rightarrow
%   \Phicx \conj G(\phisol)}$. Therefore, $\isvalid{}{G(\Phicx) \conj
%   G(\phisol) \Rightarrow G(\Phicx) \conj G^2(\phisol)}$

% \end{proof}

\subsection{Modular Constraint Solving}
\label{sec:fb-constraintsolving}

Our constraint generation algorithm traverses the entire program,
performing elaboration and collecting constraints, which are
subsequently solved en masse. The motivation behind the whole-program
approach to constraint generation is twofold: it simplifies
elaboration functions and makes presentation easier, and second, it
naturally generalizes to mutual recursion. Nonetheless, we do not
intend our type inference to be a whole-program analysis for (a). it
preempts opportunities for separate compilation and dynamic linking,
and (b). it is expensive and an overkill in most practical cases. We
therefore reclaim the compositionality of type inference by solving
the constraints in a compositional fashion. In more practical terms
this means that our constraint solving algorithm visits and solves
every constraint (or, every set of mutually dependent constraints)
only once. It composes computed solutions to solve other constraints
that depend on the solved constraints. Importantly, the failure to
solve a dependent constraint does not result in backtracking. We now
describe our compositional constraint solving algorithm in detail. To
simplify the presentation of our algorithm, we assume that are no
mutually recursive definitions in the source program. Recursive
definitions are nonetheless allowed.

\paragraph{Terminology} In a validity constraint, a predicate variable
occuring on the left side of the turnstile is said to occur
negatively, or with \emph{negative polarity}. In contrast, a predicate
variable occuring on the right side is said to occur positively, or
with \emph{positive polarity}. A validity constraint
\emph{constrains} the set of predicate variables that occur
negatively in the constraint, while it \emph{uses} the set of
predicate variables that occur positively. A constraint is said to be
\emph{recursive} if it constrains and uses a predicate variable.
% A pair of validity
% constraints are said to be \emph{mutually dependent} if there exists a
% pair ($\varphi_1$, $\varphi_2$) of predicate variables that occur
% with opposite polarities in both the constraints. In such case, the
% pair ($\varphi_1$,$\varphi_2$) of predicate variables are also said to
% be mutually dependent. We call the transitive closure of mutual
% dependency relation as \emph{transitive dependency} relation. Like
% mutual dependency, transitive dependency is also extended to predicate
% variables.

Given a set of validity constraints, we first build a dependency graph
($G_c$) with constraints as nodes, and dependencies between them
captured as edges. There exists an edge from a constraint $c_2$ to a
constraint $c_1$ in the graph (i.e., $c_2$ \emph{depends on} $c_1$) if
$c_1$ constrains a predicate variable that $c_2$ uses. 

The above condition intuitively corresponds to a case, where
expression or type, whose region elaboration is constrained by $c_2$
refers to a method or a class, whose unknown precondition is
constrained by $c_1$. The common predicate variable ($\varphi$)
represents the unknown precondition in this case. The dependency from
$c_2$ to $c_1$ means that $c_1$ must be solved to compute $\varphi$
before $c_2$ is solved, thus enforcing the rule that the precondition
of a method must not depend on its calling context. 
% Note that the first condition results in self-loops over recursive
% constraints in the dependency graph.  Since mutually recursive
% definitions refer each other, the first condition also results in
% bidirectional dependencies between mutually recursive constraints
% (and self-loops over recursive constraints). 

Next, we convert the dependency graph over constraints into a
dependency DAG ($G_C$) over sets of constraints, where each set
represents a strongly connected component in the dependency graph.

%% \paragraph{Example} The dependency DAG ($G_C$) over validity
%% constraints from the \C{Pair} example (\S~\ref{sec:constraints}) is
%% shown below: %in Fig.~\ref{fig:pair-deps}. 
%% \vspace*{-0.08in}
%% \begin{figure}[H]
%% \includegraphics[scale=0.6]{DepDAG.png}
%% \end{figure}
%% \vspace*{-0.08in}
%% Each node (labeled $C_i$) is a set of constraints that belong to a
%% strongly connected component in the dependency graph ($G_c$), hence
%% are mutually dependent. All dependencies, except the self-dependency
%% on $\mathbf{c_{10}}$, are type-2 dependencies.

A dependency DAG makes the dependencies between constraints explicit.
Constraints in each set are mutually dependent, and need to be solved
simultaneously, whereas constraints in different sets can be solved as
per any valid topological ordering of the graph's transpose.
Accordingly, we obtain a topological ordering of nodes in the graph
$G_{{C}}^{T}$ ($G_C$'s transpose), and solve the sets of constraints
in that order. The solutions obtained after solving a constraint set
are applied to the constraints in subsequent sets before attempting to
solve them.  Consequently, when the turn of a constraint set ($C$)
arrives during the constraint solving process, it satisfies certain
properties:
\begin{itemize}
\item There exists only one predicate variable ($\varphi$) that is either
constrained or used by the constraints in the set ($C$). The variable is
called set's \emph{subject}. This property follows from (a). the fact
that all the dependency constraints have already been solved (and
solutions applied), and (b). the assumption that there are no mutually
recursive definitions. 
\item All the constraints that constrain the set's subject are present
in the set. This follows from our definition of the dependency relation.
\end{itemize}

%% For the DAG in figure above, we consider the topological order $[C_1,
%% C_2, C_3]$ of its transpose, and solve the sets of constraints in that
%% order.

\subsection{Expression Elaboration}

\begin{figure}

\begin{codeml}
$\elabExpr(CT, \A, \ralloc, \env, e)$ = 
  match $e$ with
  | $\C{new} \fgjN(\bar{e})$ $\longrightarrow$ 
    let $\fbN$ = $\templateTy(\fgjN)$ in
    let $C_1$ = $\typeOk({\A},{\fbN})$ in
    let $C_2$ = match $\fgjN$ with $\RgnZ\inang{T}$ $\longrightarrow$ $\top$
          | _ $\longrightarrow$ $\{\isvalid{\A.\phicx}{\allocRgn(\fbN)=\ralloc}\}$ in
    let ($\_:\taubar$) = $\fields(\fbN)$
    let $(\bar{e'}:\bar{\tau'}, C_3)$ = $\elabExpr(\A,\ralloc,\env,\bar{e})$ in
    let $C_4$ = $\subtypeOk({\A},{\bar{\tau'}},{\taubar})$ in
      ($\C{new} \fbN(\bar{e'}) : \fbN$,$\bigcup_{i=1}^4 C_i$)
  | $e_a(\bar{e})$ $\longrightarrow$ 
    let ($e_a':\inang{\rhoalloc\rhobar\,|\,\phi}\taubar \xrightarrow{\rgn} \tau$,$C_1$) = 
                $\elabExpr(\A,\ralloc,\env,e_a)$ in
    let $\bar{\rho'}$ = $\bar{\fresh_\rho()}$ in
    let $C_2$ = $\{\bar{\rho'} \in \A.\rhoenv\}$ in
    let $\substFn$ = $[\bar{\rho'}/\rhobar][\ralloc/\rhoalloc]$ in
    let $C_3$ = $\{\isvalid{\A.\phicx}{\substFn(\phi)}\}$ in
%*   %let $(C_3,C_4)$ = $(\typeOk(\substFn(\taubar)), \typeOk(\substFn(\tau)))$ in 
*)   let $(\bar{e'}:\bar{\tau'}, C_4)$ = $\elabExpr(\A,\ralloc,\env,\bar{e})$ in
    let $C_5$ = $\subtypeOk({\A},{\bar{\tau'}},{\bar{\substFn(\tau)}})$ in
      ($e_a'\inang{\ralloc\bar{\rho'}}(\bar{e'}):\substFn(\tau)$,$\bigcup_{i=1}^5 C_i$)
  | $\letregion{\rgn}{e_a}$ $\longrightarrow$
    let $\rgn'$ = $\fresh_{\rgn}()$ in 
    let $(\rhoenv,\aenv,\phicx)$ = $\A$ in
    let $\A'$ = ($\rhoenv \cup \{\rgn'\}, \aenv, \phicx \conj (\rhoenv \outlives \rgn')$) in
    let ($e_a':\tau$,$C_1$) = $\elabExpr(\A',\rgn',\env,[\rgn'/\rgn]e_a)$ in
      ($\letregion{\rgn'}{e_a'}:\tau$,$C_1$)
  | $\open{e_a}{\rgn}{y}{e_b}$ $\longrightarrow$ 
    let ($e_a':\RgnZ\inang{T}\inang{\rho}$,$C_1$) = 
                $\elabExpr(\A,\ralloc,\env,e_a)$ in
    let $\rgn'$ = $\fresh_{\rgn}()$ in 
    let $(\rhoenv,\aenv,\phicx)$ = $\A$ in
    let ($\A'$,$\env'$) = ($(\rhoenv \cup \{\rgn'\},\aenv,\phicx)$,$\env[y\mapsto T@\rgn']$) in
    let ($e_b':\tau$,$C_2$) = $\elabExpr(\A',\rgn',\env',[\rgn'/\rgn]e_a)$ in
      ($\open{e_a'}{\rgn'}{y}{e_b'} : \tau$, $C_1 \cup C_2$)
  | _ $\longrightarrow$ ...
\end{codeml}

\caption{Constraint generation for expressions in $\absof{\FB}$}
\label{fig:fb-elabexpr}
\end{figure}

\begin{figure}

\begin{codeml}
$\elabMeth(B,d)$ = 
  let $\headerOf{B}\{\bar{\tau^f}\,\xbar;\;k\;\bar{d}\}$ = $CT(B)$ in
  let $T \; m(\tbar \; \xbar)\{\C{return} e;\}$ = $d$ in
  let $(\tau,\taubar)$ = $(\templateTy(T),\templateTy(\tbar))$ in
  let $(\rhoalloc_m,\rhobarm,\varphi)$ = $(\fresh_\rho(),\frv(\tau\taubar),\fresh_\varphi())$ in
  let ($\rhoenv$,$\aenv$,$\phicx$) = $(\{\rhoalloc,\rhobar,\rhoalloc_m,\rhobarm\},\bar{\tyvar} \extends \bar{\fgjN},\phi \conj \varphi)$ in
  let $\env$ = $\cdot[\thisZ \mapsto B\inang{\bar{\tyvar}}\inang{\rhoalloc\rhobar}][\xbar \mapsto \taubar]$ in
  let $d_m$ = $\tau \; m\inang{\rhoalloc_m\rhobarm \,|\, \varphi} (\taubar \; \xbar)\{\C{return} \unitval;\}$ in
  let $C_B$ = $\headerOf{B}\{\bar{\tau^f}\,\xbar;\;k\;\bar{d}d_m\}$ in
  let _ = $CT := CT[B \mapsto C_B]$ in
  let ($e':\tau'$,$C_1$) = $\elabExpr (\A,\rhoalloc_m,\env,e)$ in
  let $C_2$ = $\subtypeOk({\A},{\tau'},{\tau})$ in
  let ($\substFn_\rho$,$\phi_{sol}$) = $\solve(C_1 \cup C_2,\{\rhoalloc,\rhobar,\rhoallocm,\rhobarm\})$ in
  let $d_m'$ = $\tau \; m\inang{\rhoalloc_m\rhobarm \,|\, \phi_{sol}} (\taubar \; \xbar)\{\C{return} \substFn_\rho(e');\}$ in
  let $C_B'$ = $\headerOf{B}\{\bar{\tau^f}\,\xbar;\;k\;\bar{d}d_m'\}$ in
  let _ = $CT := CT[B \mapsto C_B']$ in
    $d_m'$
\end{codeml}

\caption{Method elaboration in $\absof{\FB}$}
\label{fig:fb-elabmeth}
\end{figure}


%Elaborating $\absof{\FB}$ expressions to $\FB$ expressions involves
%(a). replacing core types in variable declarations and \C{new}
%expressions with fresh region type templates, and (b). explicitly
%instantiating region parameters of methods with fresh region variables
%in method calls and function applications. This elaboration is
%performed with respect to the polymorphic type templates of classes
%and methods computed as per \S\ref{sec:fb-templatization}. 

Function $\elabExpr$, shown in Fig.~\ref{fig:fb-elabexpr}, performs
this elaboration for a subset of expressions in $\absof{\FB}$, whose
corresponding $\FB$ expressions have been ascribed static semantics in
Fig.~\ref{fig:fb-staticsem}. $\elabExpr$ is defined under the same
context as the expression typing judgment in
Fig.~\ref{fig:fb-staticsem} with symbols $\A$,$\ralloc$, and $\env$
retaining their meaning. The function traverses expressions in a
syntax-directed manner of a type checker, introducing fresh region
type templates for unknown region types, while generating constraints
over region and predicate variables. The precise nature of generated
constraints is explained in \S\ref{sec:fb-constraintsem}, but in
summary, they capture the relationships between the type templates of
various subexpressions and their well-formedness. Note that
$\elabExpr$ returns the region type template of the subexpression,
which is used to generate constraints for the expression. Functions
$\typeOk$ and $\subtypeOk$ (definitions not shown) used by $\elabExpr$
implement type well-formedness and subtype judgments from
Fig.~\ref{fig:fb-staticsem}, respectively.

\subsection{Method and Class Elaboration}

Functions $\elabMeth$ and $\elabClass$ shown in
Fig.~\ref{fig:fb-elabmeth} lift expression elaboration to method and
class definitions, respectively. Both functions first build a context
($\A$) containing a set ($ \rhoenv$) of region variables denoting
regions that are currently live, a map ($\aenv$) mapping type
variables to their bounds, and a constraint formula ($\phicx$)
capturing constraints over live region variables. We use predicate
variables ($\varphi$ and $\varphi_m$) to capture constraints over
variables in $\rhoenv$ denoting the fact that such constraints are yet
to be inferred.

Function $\elabMeth$ elaborates a method definition of class $B$. It
calls $\elabExpr$ with the context $\A$, its allocation context
parameter ($\rhoallocm$), and a type environment ($\env$) that
contains region type bindings for all the arguments of the method,
including the implicit $\C{this}$ argument. The region type template
returned by $\elabExpr$ for the method body is checked against its
expected type (derived from the type template of the method)
generating more constraints. The function then returns the elaborated
method definition and the set of constraints.

$\elabClass$ elaborates the definition of a class $B$. It relies on
$\elabCons$\footnote{The definition of $\elabCons$ is straightforward,
hence not shown.} and $\elabMeth$ functions to elaborate $B$'s
constructor ($k$) and method definitions ($\bar{d}$), respectively. To
the set of constraints returned by these functions, $\elabClass$ adds
constraints generated by checking the well-formedness of the type
templates of its superclass and fields, and also a new constraint
capturing a couple of safety conditions: first, the allocation regions
of objects referred by the instance variables should outlive the
allocation region of the instance itself, and second, the allocation
regions of a class type and its superclass type must be the same.

Function $\elabClassTable$ (Fig.~\ref{fig:fb-elabmeth}) elaborates
every definition in the class table $CT$, while accumulating
constraints.

\section{Other Aspects}

\paragraph{Modularity Aspects of Type Inference.}
The type inference algorithm, as presented, traverses the entire program to
generate the set of constraints, which are solved en masse, using an iterative
fixed point computation. However, the type inference can be realized in a
modular and compositional fashion, subject only to the restrictions imposed
by recursion.

In the elaboration phase, we can process a class \C{C} only after any class
\C{B} that \C{C} depends on has been processed: class \C{C} depends on
class \C{B} if \C{B} is either \C{C}'s base class or the type of any field
of \C{C} depends on \C{B}. In effect, this means that any collection of
mutually recursive classes must be processed together. Non-recursive
dependences can be handled in a compositional fashion: if class \C{C}
depends on \C{B} non-recursively, then the elaboration can be done for
\C{B} first, and then \C{C} can be processed.

The same idea applies to the constraint-solving phase as well.
Given a set of constraints, we say that a predicate variable $\varphi_1$
\emph{directly-depends} on another predicate variable $\varphi_2$ if the set of
constraints includes a constraint $\isvalid{\varphi_1 \conj \phictxt}{F(\varphi_2)}$.
We say that $\varphi_1$ \emph{depends} on $\varphi_2$ if $\varphi_1$ transitively
depends on $\varphi_2$.
The constraint solver needs to process any collection of mutually dependent
predicate variables together.
In effect, this requires the type inference to process any collection of
mutually recursive methods together.
However, methods that are not mutually recursive can be processed separately.

\renewcommand{\rgn}{r}

\section{Implementation and Evaluation}
\label{sec:implementation}

We have implemented the prototype of \name compiler frontend,
including its region type system and type inference, in 3k+ lines of
OCaml. Our implementation is called \namec. The input to \namec is a
program in $\absof{\FB^+}$, an extended version of $\absof{\FB}$ that
includes assignments, conditionals, loops, more primitive datatypes
(\eg, integers), and a null value. 
% If needed, a specification file containing region type annotations
% for higher-order functions can also be provided. 
Our implementation of region type inference and constraint solving
closely follows the description given in
Sec.~\ref{sec:type-inference}. The one difference is that our
constraint solver uses an approximation algorithm to compute the
minimal graph augmentation (\S~\ref{sec:csolve}).
% To solve the
% constraints that arise during type inference, we built a solver called
% \csolve that implements constraint solving approach based on fixpoint
% computation and graph augmentation described in \S~\ref{sec:csolve}. 
% Since graph augmentation is NP-hard, \csolve uses an approximation
% algorithm that repeatedly adds an edge and recomputes paths in graph
% $G_1$ until it is as connected as graph $G_2$. Though this approach
% may not necessarily return the weakest solution to the abduction
% problem (\S~\ref{sec:csolve}), it seems to do so in practice, since
% constraints are not often complex. 
% If the input $\absof{\FB^+}$ program does not create any unsafe
% references, \namec annotates it with region types, which act as a
% witness to program's memory safety. If potential safety violations are
% encountered, then the type inference fails during the constraint
% solving phase. 

\begin{figure}
\begin{codejava}
class LinkedList<T><R5,R4 | R4$\outlives$R5> {
  ListNode<T><R5,R4> head;
  ..
  List<T><R17,R4> rev<R17,R4 | R4$\outlives$R17>(unit u) {
    List<T><R17,R4> xs = 
        new List<T><R17,R4>(this.head.val);
    ListNode<T><R5,R4> cur = this.head.next;
    while (!cur == Null) {
      xs.add<R17>(cur.val)
      cur = cur.next;
    }
    return xs;
  }
\end{codejava}

\caption{Region-annotated definition of \C{rev} computed by \namec}
\label{fig:rev}
\vspace*{-0.15in}
\end{figure}

To evaluate the practical utility of our region type system and type
inference, we 
% implemented some of the C\# libraries in $\absof{\FB^+}$, and 
performed two kinds of experiments. First, we implemented some of the
microbenchmarks ($\le$100 LOC), which are the standard libraries such
as pairs, lists, list iterators, etc., in $\absof{\FB^+}$, and used
our inference engine to infer their principal region types. Since a
library class is region-oblivious, if it is well-typed as per the core
type system, then \namec must be able to automatically construct its
region-type-annotated definition without fail. As expected, \namec was
able to infer principal region types for all the library classes,
under 10ms. Fig~\ref{fig:rev} shows the region-type-annotated
definition computed for the list reverse method. Observe that \namec
was able to infer that the list and its data (of type \C{T}) can be
allocated in different regions, as long as the later outlives the
former. This allows, for instance, a \C{preOrder} method to traverse a
tree in a transferable region, and return a list of its nodes, where
the list itself is allocated in the stack region. 

Next, we translated some of the Naiad streaming query operator
benchmarks (Naiad vertices) used in~\cite{Broom:HotOS} to
$\absof{\FB^+}$, and used \namec to verify their safety. During the
process, we found multiple instances of potential memory safety
violations in the $\absof{FB^+}$ translation of benchmarks, which we
verified to be present in the original C\# implementation as well. The
cause of all safety violations is the creation of a reference from the
outgoing message (a transferable region) to the payload of the
incoming message. For example, the implementation of \C{SelectVertex}
contains the following:
\begin{codejava}
  if (this.selector(inMsg.payload[i])) {
    outMsg.set(outputOffset, inMsg.payload[i]);
    ...
  } 
\end{codejava}
The \C{outMsg} is later transferred to a downstream actor, where the
reference to \C{inMsg}'s payload becomes unsafe\footnote{This unsafe
reference could have gone unnoticed during experiments
in~\cite{Broom:HotOS} because their experimental setup included only
one actor.}. We eliminated such unsafe references by creating a clone
of \C{inMsg.payload[i]} in \C{outMsg}, and our compiler was
subsequently able to certify the safety of all references. 

\dv{Hmmmm what is \C{inMsg.payload[i]} in this case? Is it an object? The example
  demonstrates that we should really have passed in a region that stored other
  transferrable regions inside, i.e. we should proably make the \C{inMsg.payload[i]} a
  region itself. This way we don't have to pay the cost of clone. I think that might be
  an interesting additional discussion in this paragraph. Otherwise I am left as a reader
  with this bittersweat feeling that ``ok you fixed it but you introduced a copy''. But if
  the payload was also on it own transferrable region we would be fine right? We probably
  need a \C{GiveUp} primitive of a region too as we will transfer only the root transferrable
  region (\C{inMsg}) but we want to declare that we've given up ownership on all the nested
  ones too.}

Our experience with Naiad benchmarks suggests that, although our
approach cannot statically enforce memory safety, \dv{Valid point,
  also worth re-visiting in the light of the above?}
it is nonetheless useful, particularly because the static verification comes at no
additional cost to the developer. If the developer is satisfied with
the confidence gained by static verification, she may even choose to
turn the runtime safety checks off. Since we did not implement code
generation in \namec, we could not measure the runtime overhead of
checks needed to enforce safety. However, the number of LOC peforming
operations on \C{Region} objects relative to the total LOC never
exceeds 8\% in Naiad benchmarks. 
% \C{SelectVertex}, for example, contains one each of
% \C{open} and \C{transfer} operations on one region in its 45LOC,
% whereas, \C{JoinVertex} contains 5 \C{open}s, 2 \C{transfer}s, and 1
% \C{free} operation.

% We now briefly describe our experience of
% working with these benchmarks. Note that the performance benefits of
% using transferable regions in dataflow systems have been established
% in~\cite{Broom:HotOS}. Our current evaluation only focuses on safety
% aspects.

% \paragraph{Memory safety violation} 


\dv{I belive we are missing a discussion and future work section. This section should
  give sketches of what would it take to integrate other modern language features, e.g. static fields,
  value types? Also another point we could have is discussion on what would it take to integrate this
  work with the garbage collector?}


\section{Related Work}
\label{sec:related-work}

% Tofte and Talpin, A Theory of Stack Allocation in Polymorphically
%     Typed Languages, 
% Tofte and Talpin, Implementation of the Typed Call-by-Value
%     λ-calculus using a Stack of Regions, POPL'94
% Tofte and Talpin, Region-Based Memory Management, IC'97
% Tofte, Birkedal, Elsman, Hallenberg, A Retrospective on Region-based
%     Memory Management, HOSC'2004.
% Yates, A Type-And-Effect System for Encapsulating Memory In Java, 1999
% Salcianu et al, Ownership Types for Safe Region-Based Memory Management in 
%     Real-Time Java, PLDI'03
% Bacchino et al, A Type and Effect System for Deterministic Parallel Java,
%     OOPSLA'09.
% Calcagno et al, Stratified operational semantics for safety and correctness
%     of the region calculus, POPL'01
% Hicks et al, Experience With Safe Manual Memory-Management in
%     Cyclone, ISMM'04
% Grossman et al, Region-based Memory Management in Cyclone, PLDI'02
% Henglein, Makholm, & Niss, A direct approach to control-flow sensitive
%     region-based memory management, PPDP'01.
% Talpin & Jouvelot, Polymorphic Type, Region and Effect Inference, JFP'92.
% Tofte & Birkedal, A Region Inference Algorithm, TOPLAS'98.

% Tofte and Talpin in~\cite{tofte93,tofte94,tofte97} introduce the
% concept of a region type system to statically enforce the safety of
% their region-based memory management scheme in ML.
Following Tofte and Talpin's seminal work in~\cite{tofte93,tofte94,tofte97},
static type systems for safe region-based memory
management have been extensively studied in the context of various
languages and problem settings~\cite{cyclone02,cyclone04,yates99,MIT03,DPJ09,HMN01,WW01,rust,gpu14}.
Our work differs from the
existing proposals in one or more of the following respects.

\begin{enumerate}
\item 
   Our design choice focuses on ensuring memory-safety while giving programmers
   control over  region management and allocation of objects in regions.
   In contrast, some systems automate all aspects of memory management.
   This is a convenience-performance trade-off.

\item We support both lexically scoped (stack) regions and dynamic transferable
regions (both programmer-managed).

\item We exploit a combination of a simple static type discipline and lightweight
  runtime checks to ensure memory safety.
  % adopt a two-pronged approach to memory safety that relies on a
In particular, our approach circumvents the need for restrictive
static mechanisms (e.g., linear types and unique pointers) or
expensive runtime mechanisms (e.g., garbage collection and reference
counting) in order to guarantee safety.

\item We present a full (interprocedural) type inference algorithm
that eliminates the need to write region annotations on types.

\item Our underlying language is an object-oriented programming language,
equipped with higher-order functions and parameterized (generic) types.
These language features necessitate some non-trivial choices in the design
of the region-parametricity aspect of the language, which also have an
impact on aspects such type inference.

\end{enumerate}

Tofte and Talpin's approach~\cite{tofte97} uses compiler-managed lexically
scoped (stack) regions (as a replacement for GC).
Our type inference is analogous to theirs in some respects,
while differing in others.
Their inference algorithm only generates equality constraints, solvable via unification.
Our type inference algorithm generates partial order outlives constraints.
% which are required to capture relationships between lifetimes of transferable regions and stack regions.
Consequently, our constraint solving algorithm is
more sophisticated, and is capable of inferring unknown outlives
constraints over region arguments of polymorphic recursive functions.

Walker and Watkins~\cite{WW01} extend lambda calculus with first-class regions
with dynamic lifetimes, and impose linear typing to control accesses to regions.
Our open/close lexical block for transferable regions traces its origins to the \C{let!}
expression in~\cite{WW01} and~\cite{wadler90}, which safely relaxes
linear typing restrictions, allowing variables to be temporarily
aliased.
We don't use linear typing (for references to regions), thus admit unrestricted aliasing,
but use lightweight runtime checks for safety.
% Non-linear typing keeps our type system simple, and makes it
% possible to infer types, thus eliminating the annotation burden.
Moreover, ~\cite{WW01}'s linear type system is insufficient to enforce
the invariants needed to ensure safety under region transfers, such as
the absence of references that escape a transferable region.

Cyclone~\cite{cyclone02} equips C with programmer-managed stack
regions, and a typing discipline that statically guarantees the safety
of all pointer dereferences. Later
proposals~\cite{cyclone04,cycloneSCP} extends Cyclone with dynamic
regions. \name differs from Cyclone in its non-intrusiveness design principle,
which requires its safety
mechanisms to not intrude on the programming practices of C\#.
\name programmers, for example, shouldn't be forced to abandon
iterators in favor of for-loops, annotate region types, or rewrite
C\#'s standard libraries to use in \name. Cyclone requires C
programmers to use new language constructs and abandon some standard
programming idioms in the interest of preserving safety. For instance,
Cyclone programmers are required to write region types for functions;
the type inference is only intraprocedural. Ensuring safety in
presence of dynamic regions requires using either unique pointers or
reference-counted objects.  Both approaches are intrusive. For
example, unique pointers constrain, or in some cases forbid, the use
of the familiar iterator pattern, which requires creation of aliases
to objects in a collection. Some standard library functions, for
example, those that use caching, may need to be rewritten.  Moreover,
even with unique pointers, safety cannot be guaranteed statically;
checks against \C{NULL} are needed at run-time to enforce safety. For
ref-counted objects, Cyclone requires programmers to use special
functions (\C{alias\_refptr} and \C{drop\_refptr}) to create and
destroy aliases.  Reference count is affected only by these functions.
An alias going out of scope, for instance, does not decrement the
ref-count. The requirement to use additional constructs
to manage aliases makes reference counting more-or-less as intrusive
as unique pointers.

Our work differs from Cyclone also in terms of its technical
contributions. While Cyclone equips C with a range of region
constructs~\cite{cycloneSCP}, the semantics of (a significant subset
of) such constructs, and the safety guarantees of the language are not
formalized. In contrast, the (static and dynamic) semantics of Broom
has been rigorously defined with respect to a well-understood formal
system (FGJ). The safety guarantees have been formalized and proved.
%The core of Broom is very simple; the rules that make up static and
%dynamic semantics occupy less than a page each. We believe that the
%rigor and simplicity of Broom makes it easy to understand the the
%underlying ideas, and apply them in various problem settings.
Similar contrast can be made of region type inference in both the languages.
Cyclone's type inference was only ever described as being similar to
Tofte and Talpin's, and its effectiveness in presence of tracked
pointers is not clear.
In contrast, a detailed type inference algorithm is one of our core contributions.
%In contrast, the complete Ocaml (pseudo) code
%of Broom's inference algorithm, was given in the supplement and the
%ideas underlying type inference have been described elaborately in the
%paper

Our region type system can also be thought of as a specialized
ownership type system~\cite{OwnershipSurvey}, where each region is the
owner of all objects allocated in the region.
%
An ownership type system for safe region-based memory management in
real-time Java has been proposed by~\cite{MIT03}.  
% Like us, they too assume a source language with programmer-managed
% memory regions, and focus on proving safety of programs written in
% that language.  
Their language permits only lexically-scoped (stack) regions.
% (in the context of shared-memory concurrency).
In contrast, we permit regions with dynamically determined lifetimes.
% (in the context of message-passing concurrency).
% We borrow outlives relation from their formal
% development, and our type system bears some similarities to theirs.
Our language also admits generics and higher-order functions.
We also establish type safety and transfer safety results that
formalize the guarantees provided by our system. While~\cite{MIT03}'s
language is explicitly typed, our language comes equipped with full
type inference.
However, several inference algorithms have been proposed in the context
of other ownership type systems.
Our type inference algorithm is also novel compared to these existing
ownership inference algorithms, which are based on, e.g.,
pointer analysis~\cite{HuangEtAl:ECOOP12} or boolean satisfiability~\cite{DietlEtAl:ECOOP11}.
(See Section 5.2 of~\cite{OwnershipSurvey} for a more comprehensive discussion
of ownership inference algorithms.)
Some distinguishing characteristics of our algorithm is that it is customized
to our problem, does not use any pointer analysis algorithm (which
can be a source of imprecision) or SAT solvers (which can be a source of inefficiency),
and comes with relative completeness guarantees.


~\cite{HMN01} proposes a flow-sensitive approach for first-order
programs to generalize~\cite{tofte97}'s approach to dynamic
regions.~\cite{CR04} describes a flow-insensitive and
context-sensitive analysis that transforms Java programs to use
(dynamic) regions.  However, neither~\cite{HMN01} nor~\cite{CR04}
supports dynamic regions as first-class objects; they cannot be stored
in data structures or passed to methods. Furthermore,
while~\cite{HMN01} requires reference counting to ensure memory
safety,~\cite{CR04} comes with no formal safety guarantees.
%
\cite{gpu14} uses regions to safely transfer data between the CPU and GPU
in the context of Scheme. However, their setting only includes lexically-scoped
regions for which Tofte and Talpin-style analysis~\cite{tofte97}
suffices. In contrast, we provide first-class support for
transferable regions with dynamic lifetimes.
% We require this generality in order to support streaming query operators, such as the
% one shown in Fig.~\ref{fig:motivating-eg}. 


\bibliography{broom}

\clearpage

%%
%% Appendix
%%

\appendix

%\section{Operational Semantics}
%
%Fig.~\ref{fig:fb-opsem} contains a definition of the operational semantics for \fbname.
%
\section{\fbname}

% \subsection{Full Syntax}
% \input{fb-syntax-full}

% The full syntax of \FB, including expressions that manifest only at
% runtime, is shown in Fig.~\ref{fig:fb-syntax-full}. Such expressions
% include:
% \begin{itemize}

% \item Memory locations ($\loc$) corresponding to the memory regions,

% \item A special expression $\RgnZT{\toploc\loc}(e)$ that evaluates to
% a region handler object (the syntax of this expression is captured by
% the $\C{new}\;\fbN(\overline{e})$ construct). $\toploc$ is the
% location of the special $\toprgn$ region where region handlers are
% stored. $\loc$ is the location of the corresponding transferable
% region.

% \item A $\letd{\loc}{e}$ expression that results when
% $\letregion{\pi}{e}$ expression takes a step (see
% Fig.~\ref{fig:fb-opsem-2}).  $\loc$ is the location of the newly
% allocated region.  

% \item An $\opened{\loc}{s}{e}$ expression that
% results when $\C{open}\;{\RgnZT{\toploc\loc}(v)} ...$ expression takes
% a step. $\loc$ is the location of the newly open transferable region.
% The symbol $s$ denotes the typestate of the
% transferable region before it is opened\footnote{Operational semantics
% lets a transferable region to be opened while it is already open. This
% allows methods to safely open a transferable region argument
% regardless of the calling context.}. The typestate can assume the
% values of $\USED$ (allocated and closed), $\LIVE$ (allocated and
% live), and $\XFERRED$ (transferred or freed).

% \end{itemize}

\subsection{Full Static Semantics}

\begin{figure*}[t]
%
\begin{minipage}{2.25in}
\begin{smathpar}
\begin{array}{lcl}
  allocRgn(A\inang{\ralloc\rbar}\inang{\tbar}) & = & \ralloc\\
  allocRgn(\inang{\rhoalloc\rhobar \,|\, \phi}\bar{\tau^1}
      \xrightarrow{\ralloc} \tau^2) & = & \ralloc\\
  shape(A\inang{\rhoalloc\rhobar}\inang{\tbar}) & = & A\inang{\tbar}\\
  bound_{\aenv}(\tyvar@\rgn) & = & \aenv(\tyvar)@\rgn\\
  bound_{\aenv}(\fbN) & = & \fbN\\
\end{array}
\end{smathpar}
\end{minipage}
%
\begin{minipage}{1.8in}
\begin{smathpar}
\begin{array}{c}
\renewcommand*{\arraystretch}{1.2}
\RULE
  {
    \\
    B \in \{\ObjZ,\RgnZ\}
  }
  {
    fields(B\inang{\ralloc\rbar}\inang{\tbar}) \;=\; \bullet
  }
\end{array}
\end{smathpar}
\end{minipage}
%
\begin{minipage}{3in}
\begin{smathpar}
\begin{array}{c}
\renewcommand*{\arraystretch}{1.2}
\RULE
  {
    CT(B) = \headerOf{B}\{\bar{\tau^f}\;\bar{f};\,...\}\\
    \substFn = [\rbar/\rhobar, \ralloc/\rhoalloc, \tbar/\bar{\tyvar}] \qquad 
    fields(\substFn(\fbN)) = \bar{g}:\bar{\tau^g}
  }
  {
    fields(B\inang{\ralloc\rbar}\inang{\tbar}) \;=\;
      \bar{g}:\bar{\tau^g},\,\bar{f}:\substFn(\bar{\tau^f})
  }
\end{array}
\end{smathpar}
\end{minipage}
%
\bigskip

\begin{minipage}{3.5in}
\begin{smathpar}
\begin{array}{lcl}
  ctype(\ObjZ\inang{\rgn}) & = & \bullet \\
% ctype(\RgnZ\inang{\rgn}\inang{T}) & = & \inang{\rhoalloc}
%   {\unitZ}\rightarrow{T@\rhoalloc}\\
  ctype(B\inang{\ralloc\rbar}\inang{\tbar}) & = & 
    fields(B\inang{\ralloc\rbar}\inang{\tbar})\\
  mtype(\C{transfer}, \exists\rho.\RgnZ\inang{\rho}\inang{T}) & = & 
    \inang{\rhoalloc} {\unitZ}\rightarrow{\unitZ}\\
  mtype(\C{free}, \exists\rho.\RgnZ\inang{\rho}\inang{T}) & = & 
    \inang{\rhoalloc} {\unitZ}\rightarrow{\unitZ}\\
\end{array}
\end{smathpar}
\end{minipage}
%
\begin{minipage}{3in}
\begin{smathpar}
\begin{array}{c}
\renewcommand*{\arraystretch}{1.2}
\RULE
  {
    CT(B) = \headerOf{B}\{\bar{\tau^f}\;\bar{f};\,k\;\bar{d}\}\\
    m \notin \bar{d} \qquad 
    \substFn = [\rbar/\rhobar, \ralloc/\rhoalloc, \tbar/\bar{\tyvar}]
  }
  {
    mtype (m,B\inang{\ralloc\rbar}\inang{\tbar}) \;=\;
    mtype (m, \substFn(\fbN))
  }
\end{array}
\end{smathpar}
\end{minipage}
%
\bigskip

\begin{minipage}{3.25in}
\begin{smathpar}
\begin{array}{c}
\renewcommand*{\arraystretch}{1.2}
\RULE
  {
    CT(B) = \headerOf{B}\{\bar{\tau^f}\;\bar{f};\,k\;\bar{d}\}\\
    \tau^2 \; m\mang (\bar{\tau^1}\;\bar{x})\{...\} \in \bar{d} \qquad
    \substFn = [\rbar/\rhobar, \ralloc/\rhoalloc, \tbar/\bar{\tyvar}]
  }
  {
    mtype (m,B\inang{\ralloc\rbar}\inang{\tbar}) \;=\;
    \substFn(\mang\bar{\tau^1} \rightarrow \tau^2)
  }
\end{array}
\end{smathpar}
\end{minipage}
%
\begin{minipage}{3.5in}
\begin{smathpar}
\begin{array}{c}
\renewcommand*{\arraystretch}{1.2}
\RULE
  {

    \substFn = \subst{\bar{\rho_2}}{\bar{\rho_1}}
               \subst{\rhoalloc_2}{\rhoalloc_1} \spc
    mtype(m,\fbN) = \inang{\rhoalloc_1\bar{\rho_1},|\, \phi_1}\bar{\tau^{11}} 
                      \rightarrow \tau^{12} \spc \texttt{implies}\\
    \isvalid{\A.\phicx}{\phi_2 \Leftrightarrow \substFn(\phi_1)} 
        \spc \texttt{and} \spc
    \bar{\tau^{21}} = \substFn(\bar{\tau^{11}}) \spc \texttt{and} \spc
    \subtyp{\A}{\tau^{22}} {\substFn(\tau^{12})}
    %\substFn = [\rbar/\rhobar, \ralloc/\rhoalloc, \tbar/\bar{\tyvar}]
  }
  {
    override(\A,\fbN,\inang{\rhoalloc_2\bar{\rho_2},|\, \phi_1}
              \bar{\tau^{21}} \rightarrow \tau^{22})
  }
\end{array}
\end{smathpar}
\end{minipage}
%
\bigskip

\begin{minipage}{5in}
\begin{smathpar}
\begin{array}{c}
  \rhoset,\rhoenv \in 2^{\rho} \qquad
  \aenv \in \tyvar \rightarrow \fgjN \qquad
  \A = (\subtypcx)\\
\end{array}
\end{smathpar}
\end{minipage}
%

\caption{\fbname: Auxiliary Definitions}
\label{fig:fb-auxdef}
\end{figure*}

\renewcommand{\rgn}{r}
\renewcommand{\rbar}{\overline{r}}

\begin{figure*}[!h]
%
\textbf{Subtyping}  \; \fbox
  {\(\subtyp{\A}{\tau_1}{\tau_2}\)}\\
%
\begin{minipage}{1.8in}
\begin{smathpar}
\begin{array}{c}
\renewcommand*{\arraystretch}{1.2}
  \subtyp{\A}{\tau}{\tau} \\
  \subtyp{(\Delta,\aenv,\phicx)}{\tyvar @\rho}{\aenv(\tyvar) @\rho}\qquad
% \subtyp{\A}{\RgnZ\inang{\rgn}}{\RgnZ\inang{\toprgn}}\qquad
% \subtyp{\A}{\RgnZ\inang{\toprgn}}{\RgnZ\inang{\rgn}}
\end{array}
\end{smathpar}
\end{minipage}
%
\begin{minipage}{2.7in}
\begin{smathpar}
\begin{array}{c}
\renewcommand*{\arraystretch}{1.2}
\RULE
  {
    \\
    CT(B) = \headerOf{B}\{...\}
  }
  {
    \subtyp{\A}{B\inang{\tbar}\inang{\rbar}}
        {[\rbar/\rhobar, \tbar/\bar{\tyvar}](\fbN)}
  }
\end{array}
\end{smathpar}
\end{minipage}
%

\begin{minipage}{1.5in}
\begin{smathpar}
\begin{array}{c}
\renewcommand*{\arraystretch}{1.2}
\RULE
  {
    \subtyp{\A}{\tau_1}{\tau_2}\\
    \subtyp{\A}{\tau_2}{\tau_3}
  }
  {
    \subtyp{\A}{\tau_1}{\tau_3}
  }
\end{array}
\end{smathpar}
\end{minipage}
%
\begin{minipage}{2.75in}
\begin{smathpar}
\begin{array}{c}
\renewcommand*{\arraystretch}{1.2}
\RULE
  {
    \isvalid{\A.\phicx}{\phi_1 \Rightarrow \phi_2} \\
    \subtyp{\A}{\bar{\tau^{11}}}{\bar{\tau^{21}}} \spc
    \subtyp{\A}{\tau^{22}}{\tau^{12}}
  }
  {
    \subtyp{\A}
      {\inang{\rhobar \,|\, \phi_2}\bar{\tau^{21}}
          \xrightarrow{\rgn} \tau^{22}}
      {\inang{\rhobar \,|\, \phi_1}\bar{\tau^{11}}
          \xrightarrow{\rgn} \tau^{12}}
  }
\end{array}
\end{smathpar}
\end{minipage}


%
\bigskip

\textbf{Type, and Type Constraint Well-formedness}  \; \fbox
  {\(\tywf{\A}{\tau}, \spc 
     \tywf{\mem}{\phi}\)}\\
/Users/gowtham/git/broom/fullversion/broom/paper/fb-tywfrules-full.tex
%
\bigskip

\textbf{Expression Typing}  \; \fbox
  {\(\hastyp{\exptycx{\env}}{e}{\tau}\)}\\
/Users/gowtham/git/broom/fullversion/broom/paper/fb-exptyprules-full.tex
%
\bigskip

\caption{\fbname: Static Semantics}
\label{fig:fb-staticsem}
\end{figure*}

\renewcommand{\rgn}{\pi}
\renewcommand{\rbar}{\overline{\pi}}


Figs.~\ref{fig:fb-staticsem-1} and~\ref{fig:fb-staticsem-2} show full
static semantics of \FB. Fig.~\ref{fig:fb-staticsem-1} contains subtyping
rules, and rules to check well-formedness of \FB types, type
constraints, methods, and class definitions. Method well-formedness
rule makes use of the expression typing judgment defined in
Fig.~\ref{fig:fb-staticsem-2}. Auxiliary definitions used in
Figs.~\ref{fig:fb-staticsem-1} and~\ref{fig:fb-staticsem-2} are
defined in Fig.~\ref{fig:fb-auxdef}. As described in
\S~\ref{sec:type-system}, all judgments are parameterized over the
class table ($CT$). The premise $CT(B) = \headerOf{B}\{...\}$
used in some of the rules means that the definition of class $B$ is
present in $CT$ and that the definition is well-formed.

\subsection{Operational Semantics}

% \newcommand{\opsemrule}[3]{%
% \begin{minipage}{#1}\begin{smathpar}\begin{array}{c}%
% \renewcommand*{\arraystretch}{1.2}%
% \RULE {#2} {#3}%
% \end{array}\end{smathpar}\end{minipage}%
% }
% \newcommand{\lopsemrule}[4]{%
% \begin{minipage}{#1}\begin{smathpar}\begin{array}{c}%
% \renewcommand*{\arraystretch}{1.2}%
% [\rulelabel{#4}] \spc \RULE {#2} {#3}%
% \end{array}\end{smathpar}\end{minipage}%
% }
% %
% \newcommand{\anobjty}[0]{B\inang{\tbar}\inang{\ralloc\locbar}}
% \newcommand{\anobj}[0]{\C{new} \; \anobjty(\bar{v})}
% %

\begin{figure*}[t!]

%
\fbox {\(\redstoo{\Delta}{(e,\mem)}{(e',\mem')}\)}\\

\lopsemrule{1.5in}{
    \redstoo{\Delta}{(e,\mem)}{(e',\mem')}
}{
    \redstoo{\Delta}{(E\lbrack e \rbrack, \mem)}
            {(E\lbrack e' \rbrack, \mem')}
	  }{EvalOrder}

\lopsemrule{1.2in}{
    \redstoo{\Delta}{(e,\mem)}{\invalidexn}
  }{
    \redstoo{\Delta}{(E\lbrack e \rbrack, \mem)}
            {\invalidexn}
	  }{Exception}
%

\lopsemrule{2.7in}{
%   \allocRgn(A\inang{\rgn\locbar}\inang{\tbar}) & = & \rgn
%   \allocRgn(\fbN) \in \rhoenv \spc
    \mem(\loc) = \LIVE \spc
    \fields(A\inang{\tbar}\inang{\loc\locbar}) = \bar{f}:\taubar
}{
    \redstoo{\Delta}{((\C{new} \; A\inang{\tbar}\inang{\loc\locbar}(\vbar)).f_i,\mem)}{(v_i,\mem)}
}{FieldAccess}

\lopsemrule{3.85in}{
%   \allocRgn(\fbN) \in \rhoenv \spc
    \mem(\loc) = \LIVE \spc
    \mbody(m,A\inang{\tbar}\inang{\loc\locbar}) = \rho\rhobar.\bar{x}.e 
%   \redstoo{\Delta}{(, \mem)}{(e',\mem')}
}{
    \redstoo{\Delta}{((\C{new}\;A\inang{\tbar}\inang{\loc\locbar}(\bar{v})).m\inang{\loc'\overline{\loc'}}
                      (\bar{v'}),\mem)}
            {([\loc'\overline{\loc'}/\rho\rhobar][\bar{v'}/\xbar]
                [\C{new} \; A\inang{\tbar}\inang{\loc\locbar}(\bar{v})/\thisZ]\,e,\mem)}
	  }{MethodInv}

\lopsemrule{2.85in}{
    v_a = \lambdaexp{\loc'}{\rho\rhobar}
                        {\taubar \; \xbar}{e} \spc
    \mem(\loc') = \LIVE  \spc 
}{
    \redstoo{\Delta}{(v_a\inang{\loc\locbar}(\bar{v}) ,\mem)}
            {([\bar{v}/\xbar][\loc\locbar/\rho\rhobar]\,e,\mem)}
	  }{FnApply}

%
% \lopsemrule{2in}{
% %   \rgn \notin \rhoenv \spc
% %   \fresh(\rgn') \spc
%     \redstoo{\Delta}{(e,\mem)}{\invalidexn}
% }{
%     \redstoo{\Delta}{(\letregion{\rgn}{e},\mem)}{\invalidexn}
%   }{EXCEPTION}

\lopsemrule{2in}{
%   \fgjN = \RgnZ\inang{T} \spc
%   \rgn \in dom(\mem) \spc
    \mem(\loc) = \USED \spc
    \redstocup{\loc}{(e,\mem[\loc \mapsto \LIVE])}{\invalidexn}
}{
    \redstoo{\Delta}{(\C{new} \; \RgnZT{\toploc\loc} (e),\mem)}
            {\invalidexn}
	  }{Exception}

\lopsemrule{2in}{
%   \fgjN = \RgnZ\inang{T} \spc
%   \rgn \in dom(\mem) \spc
}{
    \redstoo{\Delta}{(\letexp{x}{v}{e},\mem)}
            {([v/x]e,\mem)}
	  }{LetExp}

% \lopsemrule{3.2in}{
%     \not\exists \loc.~\mem(\loc) = \FREE
% }{
%     \redstoo{\Delta}{(\letregion{\rgn}{e},\mem)}{\invalidexn}
%   }{EXCEPTION}


% \lopsemrule{3.3in}{
%     \not\exists \loc.~\mem(\loc) = \FREE
% }{
%     \redstoo{\Delta}{(\C{new} \; \RgnZ\inang{T}\inang{\toprgn}
%                 (\lambdaexp{\loc'}{\rho}{}{e}),\mem)}
%             {\invalidexn}
%         }{EXCEPTION}
% \lopsemrule{2.5in}{
%     \mem(\rgn_r) \neq \XFERRED \spc
% %   \fresh(\rgn_1) \spc
% %  \rgn_0 \notin \rhoenv \\
%     \redsto{\Delta \cup \{\rgn\}}{([[\rgn/\rgn_r]v_r/x]e_b,
%       \mem[\rgn_r \mapsto \OPEN])}{\invalidexn}
% }{
%     \redstoo{\Delta}{(\open{(\C{new} \; \RgnZ\inang{T}\inang{\rgn_r}(v_r))}
%                    {\rgn}{x}{e_b},\mem)} 
%             {\invalidexn}
%     }{EXCEPTION}

%
\bigskip

\textbf{Evaluation Context} \fbox {\(E\)}\\
\begin{smathpar}
\begin{array}{lcl}
E & \coloneqq & \bullet \ALT (\bullet).f \ALT \bullet.m\inang{\locbar}(\ebar) \ALT
      v.m\inang{\locbar}(...,\bullet,...) \ALT \C{new}\; \fbN(...,\bullet,...) \ALT
      \C{new} \; \RgnZ\inang{T}\inang{\toprgn}(\bullet) \ALT 
      \bullet\inang{\locbar}(\ebar) \\
  &  & \ALT v\inang{\locbar}(...,\bullet,...) \ALT
       \letexp{x}{\bullet}{e} \ALT \open{\bullet}{\rgn}{y}{e} 
%      \ALT \opened{\loc}{\status}{\bullet} \ALT \letd{\loc}{\bullet}
%     The following should be forbidden. See NEW-REGION rule 2.
%      \ALT \C{new} \; \RgnZ\inang{T}\inang{\toploc\loc}(\bullet)
\end{array}
\end{smathpar}

\caption{\fbname: Operational semantics (part 1)}
\label{fig:fb-opsem-1}
\end{figure*}

\begin{figure*}[t!]

\lopsemrule{3.2in}{
%   \fresh(\rgn')\spc
%   \mem(\loc) = \FREE \spc
    \loc \not\in dom(\mem) \spc
    \mem' = \mem[\loc \mapsto \SLIVE]
%   \redstoo{\Delta}{(e,\mem)}{(e',\mem')}
}{
    \redstoo{\Delta}{(\letregion{\rgn}{e},\mem)}{(\letd{\loc}{[\loc/\rgn]e},\mem')}
  }{LetRegionBegin}

\lopsemrule{3in}{
    \redstocup{\loc}{(e,\mem)}{(e',\mem')}
}{
    \redstoo{\Delta}{(\letd{\loc}{e},\mem)}{(\letd{\loc}{e'},\mem')}
  }{LetRegion}

\lopsemrule{3in}{
%   \rgn \notin \rhoenv
%   \mem' = \mem[\loc \mapsto \FREE]
    \mem' = \mem[\loc \mapsto \XFERRED]
}{
    \redstoo{\Delta}{(\letd{\loc}{v},\mem)}{(v,\mem)}
  }{LetRegionEnd}

\lopsemrule{3.3in}{
%   \fgjN = \RgnZ\inang{T} \spc
%   \rgn \in \rhoenv \spc
%   We need Delta here to ensure Delta U {pi_r} in next rule is sound. 
%   \rgn_r \notin \Delta \cup dom(\Sigma) \spc
%   \rgn \notin dom(\mem) \cup \rhoenv \spc
%   \mem' = \mem[\rgn_r \mapsto \CLOSED]
%   \mem(\loc) = \LIVE \spc
%   \mem(\loc') = \FREE \spc
    \loc \not\in dom(\mem) \spc
    \mem' = \mem[\loc \mapsto \USED]
}{
    \redstoo{\Delta}{(\C{new} \; \RgnZ\inang{T}\inang{\toprgn}
                (v),\mem)}
            {(\C{new} \; \RgnZ\inang{T}\inang{\toploc\loc}
                (v\inang{\loc}()),\mem')}
	      }{NewRegion}

\lopsemrule{3.3in}{
%   \mem(\loc) = \USED \spc % Type system cannot guarantee this.
    \redstocup{\loc}{(e,\mem[\loc \mapsto \LIVE]) }{(e',\mem')}
}{
    \redstoo{\Delta}{(\C{new} \; \RgnZ\inang{T}\inang{\toploc\loc}
                (e),\mem)}
            {(\C{new} \; \RgnZ\inang{T}\inang{\toploc\loc}
                (e'),\mem')} %[\loc\mapsto\USED])}
	      }{NewRegion}

% \lopsemrule{2.8in}{
%     \redsto{\Delta \cup \{\rgn_r\}}{(e,\mem)}{(e',\mem')} \spc
% %   \fgjN = \RgnZ\inang{T} \spc
%     \rgn_r \in dom(\mem)
% }{
%     \redstoo{\Delta}{(\C{new} \; \RgnZ\inang{T}\inang{\rgn_r}
%                 (e),\mem)}
%             {(\C{new} \; \RgnZ\inang{T}\inang{\rgn_r}
%                 (e'),\mem')}
%         }{NEW-REGION}

\lopsemrule{4in}{
    v_a = \C{new} \; \RgnZ\inang{T}\inang{\toploc\loc}(v) \spc
    \mem(\loc) = \USED ~\texttt{or}~ \mem(\loc) = \TLIVE \spc
%   \fresh(\rgn_1)\\
%   \rgn_0 \notin \rhoenv \\
    \mem' = \mem[\loc \mapsto \TLIVE]
%   \mem'' = \mem'[\rgn_r \mapsto \mem(\rgn_r)]
}{
    \redstoo{\Delta}{(\open{v_a}{\rgn}{x}{e_b},\mem)} 
            {(\opened{\loc}{\mem(\loc)}{[v/x][\loc/\rgn]e_b},\mem')}
	  }{Open}

\lopsemrule{2.7in}{
    \redstocup{\loc}{(e,\mem)}{(e',\mem')}
}{
    \redstoo{\Delta}{(\opened{\loc}{\status}{e},\mem)}
    {(\opened{\loc}{\status}{e'},\mem')}
  }{Opened}

\lopsemrule{2.7in}{
%   v_a = \C{new} \; \RgnZ\inang{T}\inang{\}(v_r) \spc
%   \mem(\rgn_r) \neq \XFERRED \spc
%   \rgn_0 \notin \rhoenv \spc
    \mem' = \mem[\loc \mapsto \status]
}{
    \redstoo{\Delta}{(\opened{\loc}{\status}{v},\mem)} {(v,\mem')}
  }{OpenEnd}


\lopsemrule{3in}{
    v_a = \C{new} \; \RgnZ\inang{T}\inang{\toploc\loc}(v) \spc
    \mem(\loc) \neq \USED ~\texttt{and}~ \mem(\loc) \neq \LIVE
%   \rgn_0 \notin \rhoenv \\
%   \redstocup{\rgn_0}{([[\rgn_0/\rgn]v_r/x]e_b,
%     \mem[\rgn \mapsto \OPEN])}{(e_b',\mem')}
}{
    \redstoo{\Delta}{(\open{v_a}{\rgn}{x}{e_b},\mem)} 
            {\invalidexn}
	  }{OpenTransferred}


\lopsemrule{3.25in}{
%   \ralloc \in \rhoenv \spc
%   \fbN = \RgnZ\inang{T}\inang{\rgn_r}\spc
%   \redstoo{\Delta}{(, \mem)}{(e',\mem')}
    \mem(\loc) = \USED \spc
    \mem' = \mem[\loc \mapsto \XFERRED]
}{
    \redstoo{\Delta}{((\C{new}\;\RgnZ\inang{T}\inang{\toploc\loc}(v)).\transfer(),\mem)}
            {(\unitval,\mem')}
	  }{Transfer}

\lopsemrule{2.5in}{
%   \fbN = \RgnZ\inang{T}\inang{\rgn}\spc
%   \ralloc \in \rhoenv \spc
%   \redstoo{\Delta}{(, \mem)}{(e',\mem')}
    \mem(\loc) = \TLIVE \spc
}{
    \redstoo{\Delta}{((\C{new}\;\RgnZ\inang{T}\inang{\toploc\loc}(v)).\transfer(),\mem)}
            {\invalidexn}
	  }{TransferOpened}

% \lopsemrule{2.5in}{
%     \not\exists \loc.~\mem(\loc) = \SLIVE \spc
%     \mem(\loc') = \XFERRED
% }{
%     \redstoo{\Delta}{(e,\mem)}{(e,\mem[\loc'\mapsto\FREE])}
%     }{GARBAGE-COLLECT}


\caption{\fbname: Operational semantics (part 2)}
\label{fig:fb-opsem-2}
\end{figure*}

Figs.~\ref{fig:fb-opsem-1} and~\ref{fig:fb-opsem-2} show the
operational semantics of \fbname (Fig.~\ref{fig:fb-opsem-2} is also
contained in the main paper as Fig.~\ref{fig:fb-opsem}. It is included
here for the sake of completeness). 

\section{Type System: Proofs}

We first define the $\consistent$ relation between $\Delta$ and
$\mem$:

\begin{definition}[\consistent($\Delta$,$\mem$)]
A set $\Delta \in 2^{\rgn}$ of region annotations is said to be
consistent with a map $\mem \in \loc \rightarrow s$ from region
locations to typestates if and only if forall $\loc\in\Delta$,
$\mem(\loc) = \LIVE$.
\end{definition}

Consistency between $\Delta$ and $\mem$ is preserved by the reduction
relation:

\begin{lemma}[\textbf{consistency preservation}]
\label{lem:consistency}
$\forall(e,\Delta,\mem,\phicx,\tau)$. if $\consistent(\Delta,\mem)$
and $\tywf{\Delta}{\phicx}$ and
$\hastyp{(\Delta,\cdot,\phicx),\cdot}{e}{\tau}$ and
$\redstoo{\Delta}{(e,\mem)}{(e',\mem')}$, then
$\consistent(\Delta,\mem')$.
\end{lemma}
\begin{proof}
Proof is by induction on $\redstoo{\Delta}{(e,\mem)}{(e',\mem')}$.
For the rules where $\mem'=\mem$, proof is trivial. For the rules
where $\mem'$ is a result of executing a subexpression under
$\Delta$, proof follows from the inductive hypothesis. Remaining rules
are discussed below:
% \begin{smathpar}
% \begin{array}{cl}
%   \forall(\Delta,\mem,\phicx,\tau).~\consistent(\Delta,\mem)
%   ~\conj ~\tywf{\Delta}{\phicx}
%   ~\conj ~\hastyp{(\Delta,\cdot,\phicx),\cdot}{e_0}{\tau} & \\
%   \hspace*{1in}
%   ~\redstoo{\Delta}{(e_0,\mem)}{(e_0',\mem')}
%   ~\Rightarrow~ \consistent(\Delta,\mem') & IH\\
% \end{array}
% \end{smathpar}
\begin{itemize}
  \item Rule $\rulelabel{LetRegionBegin}$: All live regions in $\mem$
  are also live in $\mem'$. Proof follows
  \item Rule $\rulelabel{LetRegion}$: By inductive hypothesis, $\mem'$
  is consistent with $\Delta \cup \{\loc\}$. Hence, it is consistent
  with $\Delta$.
  \item Rule $\rulelabel{LetRegionEnd}$: Here, $\loc$ is set to
  $\XFERRED$ in $\mem'$. However, since $e$ is the \C{letd}
  expression, and the type rule for the \C{letd} expression gives us
  $\loc \not\in \Delta$. Hence $\Delta$ is consisten with $\mem'$.
  \item Rule $\rulelabel{NewRegion}$: $\mem'$ extends $\mem$ with a
  new binding. Consistency is trivially preserved.
  \item Rule \#2 of $\rulelabel{NewRegion}$ : Since $\mem(\loc)=\USED$
  and $\consistent(\Delta,\mem)$, $\loc \not\in \Delta$. The proof
  follows from inductive hypothesis.
  \item Rule $\rulelabel{Open}$: As $\mem'=\mem[\loc\mapsto\LIVE]$,
  hence $\Delta$ remains consistent with $\mem'$. Here, we also take
  note of the fact that if the result expression (\C{opened}) is
  tagged with a typestate of $\USED$, then $\loc\notin \Delta$.
  \item Rule $\rulelabel{Opened}$: Inductive hypothesis guarantees
  that $\Delta\cup\{\loc\}$ is consistent with $\mem'$. Hence,
  $\Delta$ is consistent with $\mem'$. We also take note of the fact
  that the typestate tagged with the $\C{opened}$ expression remains
  invariant during reduction.
  \item Rule $\rulelabel{OpenEnd}$: $\mem'=\mem[\loc\mapsto s]$, where
  $s$ is the typestate tagged with the \C{opened} expression. As clear
  from the $\rulelabel{Open}$ rule, $s$ can be either $\USED$ or
  $\LIVE$, and if $s$ is $\USED$ then $\loc\notin\Delta$. Hence, in
  either case $\Delta$ remains consistent with $\mem$.
  \item Rules $\rulelabel{Transfer}$: changes binding for a non-live
  location $\loc$. Hence, consistency is preserved.
\end{itemize} 
\qed
\end{proof}

\begin{lemma}[value substitution preserves typing]
\label{lem:substitution}
$\forall(e,z,\tau_1,\tau_2,\Delta,\phicx)$, if $\hastyp{(\Delta,\cdot,
\phicx),\cdot[x\mapsto\tau_1]}{e}{\tau_2}$ and $\hastyp{(\Delta,\cdot,
\phicx),\cdot}{v}{\tau_1}$, then $\hastyp{(\Delta,\cdot,
\phicx),\cdot} {[v/x]e}{\tau_2}$.
\end{lemma}
\begin{proof}
The proof is by induction on typing derivation and follows on the
lines of similar proof for FGJ.
\qed
\end{proof}


% \begin{lemma}[weakening]
% $\forall(v,\tau,\Delta,\Delta_0,\phicx,\phicx_0)$, 
% if 
% $\hastyp{(\Delta \cup \Delta_0,\cdot, \phicx \conj
% \phicx_0),\cdot}{v}{\tau}$
% and
% $\tywf{(\Delta,\cdot,\phicx)}{\tau}$
% then 
% $\hastyp{(\Delta,\cdot, \phicx),\cdot} {v}{\tau}$.
% \end{lemma}
% \begin{proof}
% The proof is by induction on $\hastyp{(\Delta \cup \Delta_0,\cdot, \phicx \conj
% \phicx_0),\cdot}{v}{\tau}$. Since $v$ is a value, we have few cases:

% \begin{itemize}
%   \item Case ($v = \C{new}\; \fbN_0(\vbar)$ and $\tau = \fbN_0$): By
%   inverting 
% \end{itemize}

% \end{proof}

\begin{lemma}[progress]
\label{lem:progress}
$\forall e, \tau, \mem, \rhomap, \phicx$, if $\consistent(\Delta,\mem)$ and 
$\tywf{\Delta}{\phicx}$ and
$\hastyp{\emptyA,\cdot}{e}{\tau}$, then one of the following holds:\\
  \begin{smathpar}
  \begin{array}{rl}
    (i) & \exists (e',\rhomap').\;\redstoo{\Delta}{(e,\rhomap)}{(e',\rhomap')}\\
    (ii) & \valuee(e)\\
    (iii) & \redstoo{\Delta}{(e,\rhomap)}{\invalidexn}\\
  \end{array}
  \end{smathpar}
\end{lemma}
\begin{proof}
Proof is by induction on the typing derivation of $e:\tau$. Most cases
follow from the inductive hypothesis (IH), which claims that if a
subexpression has a typing derivation, then it can make progress. We
will consider cases where all subexpressions are values, but the
expression itself is not a value.
\begin{itemize}
  \item $B\inang{\tbar}{\locbar}(\vbar).f_i$ case: This expression has
  a type only if $\tywf{\emptyA}{B\inang{\tbar}\inang{\locbar}}$, which is
  possible only if $\locbar \in \Delta$. Hence $e$ can make progress.

  \item $\C{let} \; x = v \; \C{in} \; e$ case: Always takes step via
  \rulelabel{LetExp}.

  \item $\C{new}\; \RgnZT{\toprgn}(v)$ case: takes a step by
  \rulelabel{NewRegion} rule to $\C{new}\; \RgnZT{\toploc\loc}
  (v\inang{\loc}())$.

  \item Method call case: Since $\mtype$ is defined, $\mbody$ is also
  defined, and the execution takes a step by $\rulelabel{MethodInv}$.

  \item Lambda application case: takes step by \rulelabel{FnApply}

  \item \C{letregion} case: takes a step by \rulelabel{LetRegionBegin}
  rule.

  \item \C{open} case: takes a step by \rulee{Open} rule, or throws an
  exception by \rulee{OpenTransferred} rule, depending on the typestate
  of the region location.

  \item \C{letd} case: The subexpression is typed under $\Delta \cup
  \{\loc\}$, hence the subexpression takes a step under $\Delta \cup
  \{\loc\}$. This allows \C{letd} expression take a step via
  \rulee{LetRegion} rule. If subexpression is a value, then \C{letd}
  takes a step via \rulee{LetRegionEnd} rule.

  \item \C{opened} case: similar to \C{letd} case.
\end{itemize}
\qed
\end{proof}

\begin{lemma}[preservation]
\label{lem:preservation}
$\forall e, \tau, \Delta, \mem$, such that $\consistent(\Delta,\mem)$
and $\tywf{\rhoenv}{\phicx}$, if $\hastyp{\emptyA,
\cdot}{e}{\tau}$, and $\redstoo{\Delta}{(e,\mem)}{(e',\mem')}$, then 
$\hastyp{\emptyASigp,\cdot}{e'}{\tau}$.
\end{lemma}
\begin{proof}
  Proof is by induction on the reduction step. Most cases follow from
  the inductive hypothesis, which asserts that if a subexpression
  takes a step, it preserves its type under the same $\Delta$ and
  $\phicx$. We consider interesting cases below:
  \begin{itemize}
    \item \rulee{FieldAccess} case: $\C{new}\; A\inang{\tbar}
    \inang{\loc\locbar}(\vbar).f_i$ takes a step to $v_i$. The field
    access expression has a type of $i^{th}$ field returned by
    $\fields$ definition, and so does $v_i$, if ${new}\;
    A\inang{\tbar} \inang{\loc\locbar}(\vbar)$ has to be typable.
    Hence the type is preserved.

    \item \rulee{MethodInv} case: From the method invocation type rule
    and method well-formedness condition, method body ($e'$) has a type
    $\tau^2$ under $(\{\loc,\locbar,\rhobar\},\cdot,\phi),\cdot[\xbar
    \mapsto \overline{\tau^1}]$. Also,
    $\tywf{\loc,\locbar,\rhobar}{\phi}$. Since $\overline{\loc'} \neq \rhobar$
    (locations are never equal to region variables), we have:
    \begin{center}
    $\hastyp{(\{\loc,\locbar,\overline{\loc'}\},\cdot,[\overline{\loc'}/\rhobar]\phi),\cdot[\xbar
    \mapsto \overline{\tau^1}]}{e'}{\tau^2}$ and
    $\tywf{\{\loc,\locbar,\overline{\loc'}\}}{[\overline{\loc'}/\rhobar]\phi}$
    \end{center}
    Since $\{\{\loc,\locbar,\overline{\loc'}\} \subseteq \Delta$, and
    $\phicx \vdash [\overline{\loc'}/\rhobar]\phi$, we can strengthen
    the context and derive:
    \begin{center}
      $\hastyp{(\Delta,\cdot,\phicx),\cdot[\xbar \mapsto \overline{\tau^1}]}{e'}{\tau^2}$
    \end{center}
    Applying the substitution lemma (Lemma~\ref{lem:substitution}) and
    inductive hypothesis gives the proof.

    \item \rulee{FnApply} case: proceeds on the similar lines as
    \rulee{MethodInv} case, except that no strengthening is needed;
    function body is typed under a context that includes $\Delta$.

    \item \rulee{LetRegionBegin} case: Since $\loc \notin dom(\mem)$,
    it follows that $\loc \notin \Delta$. Rest of the premises
    required to apply the \C{letd} type rule on result expression are
    obtained from the \C{letregion} type rule of the initial
    expression ($e$), by substituing $\loc$ for $\pi$.

    \item \rulee{LetRegionEnd} case: $e$ is a \C{letd} expression
    which reduces to a value $v$. Invering the typing derivation of
    $e$ yeilds the premise that $v$ is well-typed under the context
    $(\Delta \cup \{\loc\},\cdot,\phicx \conj \Delta \outlives \loc)$.
    However, we have to prove that $v$ is well-typed under the smaller
    context $(\Delta,\cdot,\phicx)$. We carry out this proof by
    induction on the structure of value $v$:
    \begin{itemize}
      \item If $v$ is an object value (e.g.,
      $B\inang{\tbar}\inang{\locbar}(\vbar)$), IH yeilds the
      well-typedness of $\vbar$ under $(\Delta,\cdot,\phicx)$, and the
      premise 
      $\tywf{(\Delta,\cdot,\phicx)} {B\inang{\tbar}\inang{\locbar}}$,
      obtained by inverting the type judgment for \C{letd}, yeilds the
      well-typedness of whole value.

      \item A region handler value ($\RgnZT{\toploc\loc'}(v)$) is
      well-typed under any context.

      \item If $v$ is a function closure, then it is well-tuped under
      $(\Delta,\cdot,\phicx)$ only if it doesn't trap any references
      to $\loc$, the \C{letd} location. Since closure's type needs to
      be well-formed under $(\Delta,\cdot,\phicx)$, its allocation
      region belongs to $\Delta$, hence outlives $\loc$. Since the
      closure typing rule requires all free region variables of the
      closure body outlive the allocation region of closure, it
      follows that all free region variables strictly outlive $\loc$,
      hence they belong to $\Delta$. Hence the function closure is
      well-typed outside \C{letd}.
    \end{itemize}

  \item \rulee{OpenEnd} case: The proof is similar to the \C{letd}
  case.
  \qed
  \end{itemize}
\end{proof}

\begin{proof}[\textbf{Theorem~\ref{thm:fb-type-safety}}]
Follows from Lemmas~\ref{lem:consistency},~\ref{lem:progress},
and~\ref{lem:preservation}.
\qed
\end{proof}

\begin{figure*}[t!]

\beginrules

%%%%%%%%%%% LAMBDA %%%%%%%%%%%

\lgcrule{LAMBDA}
  {
    \rgn \in \A.\rhoenv \spc
    \rhobar \notin \A.\rhoenv
    \spc
%   \rhoenv' = \rhoenv \cup \{\rhoalloc,\rhobar\}\spc
    \A' = (\A.\rhoenv \cup \{\rhobar\}, \A.\aenv, 
          \A.\phicx \conj \phi)\spc
    \tywf{\A'.\rhoenv}{\phi}
    \\
    \typeok {\A'} {\bar{\tau^1}} {C} \spc
    \typeok {\A'} {\tau^2} {C} \spc
    \exprok {\A',\env[\xbar \mapsto \bar{\tau^1}]} {e} {\tau^2} {C}
  }
  {
    \exprok {\stdcontext}
           {\lambdaexp{\rgn}{\rhobar \,|\, \phi} {\xbar:\bar{\tau^1}}{e}}
           {\inang{\rhobar \,|\, \phi} \bar{\tau^1} \xrightarrow{\rgn} \tau^2}
	   {C}
  }

%%%%%%%%%%% OPEN-REGION %%%%%%%%%%%
\lgcrule{OPEN}{
   \exprok {\stdcontext} {e_a} {\RgnZ\inang{T}\inang{\rho}} {C_1} \spc
   \A = (\rhoenv,\aenv,\phicx) \spc
   \rgn \notin \rhoenv
   \\
   % (\A',\env') = ((\rhoenv \cup \{\rgn\},\aenv,\phicx),\env[y\mapsto T@\rgn] \spc
   \exprok {(\rhoenv \cup \{\rgn\},\aenv,\phicx),\env[y\mapsto T@\rgn]} {e_b} {\tau} {C_2}
}{
   \exprok {\stdcontext} {\open{e_a}{\rgn}{y}{e_b}} {\tau} {(C_1 \cup C_2)}
}


%%%%%%%%%%% FUNCTION INVOCATION %%%%%%%%%%%
\lgcrule{FUN-APPLY}
{
\exprok {\stdcontext} {e_a} {\inang{\rhobar\,|\,\phi}\taubar \xrightarrow{\rgn} \tau} {C_1} \spc
\exprok {\stdcontext} {\bar{e}} {\bar{\tau'}} {C_2} \spc
\substFn = [\bar{\rho'}/\rhobar]
\\
C_3 = \{\bar{\rho'} \in \A.\rhoenv\} \spc
C_4 = \{\isvalid{\A.\phicx}{\substFn(\phi)}\} \spc
\subtypeok {\A} {\bar{\tau'}} {\substFn(\bar{\tau})} {C_5}
}{
\exprok {\stdcontext} {e_a\inang{\bar{\rho'}}(\bar{e})} {\tau} {\cup_{i=1}^5 C_i}
}

%%%%%%%%%%% METHOD %%%%%%%%%%%
\lgcrule{METHOD}{
CT(B) = \hdOf{B}{\varphi}\{\bar{\tau^f}\,\xbar;\;\bar{d}\} \\
\A = (\rhoenv,\aenv,\phicx) = (\{\rhobar,\rhobarm\},\bar{\tyvar} \extends \bar{\fgjN}, \varphi_m) \spc\spc
C_1 = \{ \tywf{\rhoenv}{\varphi_m} \} \\
\env = \cdot[\thisZ \mapsto B\inang{\bar{\tyvar}}\inang{\rhobar}][\xbar \mapsto \taubar] \spc\spc
\exprok {\stdcontext}{e} {\tau'} {C_2} \spc\spc
\subtypeok {\A} {\tau'} {\tau} {C_3}
}{
\typeok{} {(B, \tau \; m\inang{\rhobarm \,|\, \varphi_m} (\taubar \;  \xbar)\{\C{return} e;\})} {(C_1 \cup C_2 \cup C_3)}
}

%%%%%%%%%%% CLASS %%%%%%%%%%%
\lgcrule{CLASS}{
\A = (\rhoenv, \aenv, \phicx) = (\{\rhoalloc,\rhobar\},\bar{\tyvar} \extends \bar{\fgjN},\varphi) \\
C_1 = \{ \tywf{\rhoenv}{\varphi} \} \spc\spc
\typeok {\A} {\fbN} {C_2} \spc\spc
\typeok {\A} {\bar{\tau^f}} {C_3} \\
C_4 = \{\isvalid{\phicx}{\allocRgn(\bar{\tau^f}) \outlives \rhoalloc \conj \allocRgn(\fbN) = \rhoalloc}\} \\
\typeok {} {\bar{d}} {C_5}
}{
\typeok {} {\hdOf{B}{\varphi}\{\bar{\tau^f}\,\xbar;\;\bar{d}\}} {\bigcup_{i=1}^5 C_i}
}

\myendrules

\caption{Constraint generation}
\label{fig:constraint-gen-1}
\end{figure*}

\begin{figure*}[t!]

\beginrules

%%%%%%%%%%% Header Box %%%%%%%%%%%
\fbox{  \( \typeok{\A}{\tau}{C} \)}
\\

%%%%%%%%%%% TYPE WELL-FORMEDNESS %%%%%%%%%%%

%%%%%%%%%%% OBJECT TYPE %%%%%%%%%%%
\lgcrule{TWF}
  {
    C = \{ \rgn \in \A.\rhoenv \}
  }
  {
    \typeok {\A} {\ObjZ\inang{\rgn}} {C}
  }

%%%%%%%%%%% CLASS TYPE %%%%%%%%%%%
  \lgcrule{TWF}
  {
    CT(B) = \headerOf{B}\{...\}
    \spc
    \fgjtywf{\aenv}{B\inang{\tbar}}
    \\
    C = \{ \rbar \in \rhoenv, \isvalid{\phicx}{[\rbar/\rhobar, \tbar/\bar{\tyvar}](\phi)} \}
  }
  {
    \typeok {(\rhoenv,\aenv,\phicx)} {B\inang{\rbar}\inang{\tbar}} {C}
  }

%%%%%%%%%%% GENERIC TYPE PARAMETER %%%%%%%%%%%
  \lgcrule{TWF}
  {
    \fgjtywf{\A.\aenv}{T} \spc
    \fgjsubtyp{\A.\aenv}{T}{\ObjZ} \spc
    \\
    C = \{ \rgn \in \A.\rhoenv \}
  }
  {
    \typeok {\A}{T@\rgn} {C}
  }

%%%%%%%%%%% FUNCTION TYPE %%%%%%%%%%%
  \lgcrule{TWF}
  {
    C_1 = \{ \rgn \in \rhoenv \}
    \\
    \rhobar \notin \A.\rhoenv \spc
    \rhoenv' = \rhoenv \cup \{\rhobar\} \spc
    \A' = (\rhoenv', \aenv, \phicx \conj \phi)
    \\
    \tywf{\rhoenv'}{\phi}\spc 
    \typeok{\A'}{\bar{\tau^1}} {C_2} \spc
    \typeok{\A'}{\tau^2} {C_3}
  }
  {
    \typeok{(\rhoenv,\aenv,\phicx)} {\inang{\rhobar \,|\, \phi} \bar{\tau^1} \xrightarrow{\rgn} \tau^2} 
       {C_1 \cup C_2 \cup C_3}
  }

%%%%%%%%%%% REGION TYPE %%%%%%%%%%%
  \lgcrule{TWF}
  { 
    \fgjtywf{\A.\aenv}{T}
  }
  {
    \typeok {\A} {\RgnZ\inang{T}\inang{\toprgn}} {\{\}}
  }

%%%%%%%%%%% FUNCTION SUBTYPING %%%%%%%%%%%
  \lgcrule{FUN-SUBTYPING}
  {
    C_1 = \{ \isvalid{\A.\phicx}{\phi_1 \Rightarrow \phi_2} \}
    \\
    \subtypeok {\A} {\bar{\tau^{11}}} {\bar{\tau^{21}}} {C_2}
    \\
    \subtypeok {\A} {\tau^{22}} {\tau^{12}} {C_3}
  }
  {
    \subtypeok {\A}
      {\inang{\rhobar \,|\, \phi_2}\bar{\tau^{21}} \xrightarrow{\rgn} \tau^{22}}
      {\inang{\rhobar \,|\, \phi_1}\bar{\tau^{11}} \xrightarrow{\rgn} \tau^{12}}
      {C_1 \cup C_2 \cup C_3}
  }

\myendrules

\caption{Type well-formedness constraint generation}
\label{fig:constraint-gen-2}
\end{figure*}

\renewcommand{\rgn}{\myownthrowexception}
\begin{lemma}
The antecedent of any generated validity constraint is of the form
$\varphi \conj \phictxt$ where $\varphi$ is a predicate variable and
$\phictxt$ is a conjunction of zero or more outlives-constraints.
Let $\pi_1 \outlives \pi_2$ be a conjunct in $\phictxt$.
Then, $\pi_2 \notin \predDeltaMap(\varphi)$.
Furthermore, if $\pi_1 \in \predDeltaMap(\varphi)$, then for
every $\pi_f \in \predDeltaMap(\varphi)$, $\pi_f \outlives \pi_2$ is
a conjunct in $\phictxt$.
\end{lemma}

\begin{proof}
  By induction over the constraint-generation rules.
  Any context $\A = (\rhoenv,\aenv,\phicx)$ generated by the constraint-generation
  process satisfies the invariant that $\phicx$ is of the form $\varphi \conj \phictxt$
  where $\rhoenv \supseteq \predDeltaMap(\varphi)$.
  The only rule that modifies $\phicx$ is the rule for \C{letregion}
  that adds the set of constraints $\pi_f \outlives \pi$ for every $\pi_f \in \rhoenv$
  as conjuncts to $\phicx$.
\end{proof}


\begin{theorem}
Let $p$, $q$, $r$ and $C$ denote the values of the corresponding
variables in an execution of the type inference algorithm.
\begin{enumerate}
\item $\absof{q}$ = $p$
\item $\absof{q[\sigma]}$ = $p$ for any substitution $\sigma$.
\item If $\sigma$ is any solution to $C$, then $q[\sigma]$ is well-typed.
\item If there is any substitution $\sigma$ such that $q[\sigma]$ is well-typed, then
$C$ has a solution.
\item If SolveConstraints($C$) returns Some($s$), then $s$ is a solution to $C$.
\item If $C$ has a  solution, then SolveConstraints($C$) will return some solution.
\end{enumerate}
\end{theorem}

\section{Other Aspects}

\paragraph{Modularity Aspects of Type Inference.}
The type inference algorithm, as presented, traverses the entire program to
generate the set of constraints, which are solved en masse, using an iterative
fixed point computation. However, the type inference can be realized in a
modular and compositional fashion, subject only to the restrictions imposed
by recursion.

In the elaboration phase, we can process a class \C{C} only after any class
\C{B} that \C{C} depends on has been processed: class \C{C} depends on
class \C{B} if \C{B} is either \C{C}'s base class or the type of any field
of \C{C} depends on \C{B}. In effect, this means that any collection of
mutually recursive classes must be processed together. Non-recursive
dependences can be handled in a compositional fashion: if class \C{C}
depends on \C{B} non-recursively, then the elaboration can be done for
\C{B} first, and then \C{C} can be processed.

The same idea applies to the constraint-solving phase as well.
Given a set of constraints, we say that a predicate variable $\varphi_1$
\emph{directly-depends} on another predicate variable $\varphi_2$ if the set of
constraints includes a constraint $\isvalid{\varphi_1 \conj \phictxt}{F(\varphi_2)}$.
We say that $\varphi_1$ \emph{depends} on $\varphi_2$ if $\varphi_1$ transitively
depends on $\varphi_2$.
The constraint solver needs to process any collection of mutually dependent
predicate variables together.
In effect, this requires the type inference to process any collection of
mutually recursive methods together.
However, methods that are not mutually recursive can be processed separately.


%%
%% Bibliography
%%

\end{document}
