\section{Type Inference}
\label{sec:type-inference}

We formalize region type inference as an elaboration function from
programs in $\absof{\FB}$ (i.e., region-erased \FB) to programs in
\FB.  The type inference algorithm alternates between a constraint
generation phase and a constraint solving phase. During the generation
phase, inference algorithm introduces region variables ($\rho$) for
unknown region annotations on types, and constraint variables (denoted
$\varphi$) for unknown region constraints ($\phi$) on methods and
functions. Subsequently, the algorithm makes use of static semantics
in Fig.~\ref{fig:fb-staticsem} to generate constraints on region and
constraint variables. Due to the presence of region-polymorphic
recursion in \FB, constraints generated by the algorithm can be
circular. More precisely, constraints generated can assume the form
$\varphi \Leftrightarrow \phi \wedge F(\varphi)$, where $F$ is a
non-idempotent substitution function for region variables in
$\varphi$. In the constraint solving phase, the algorithm then relies
on a fixpoint constraint solving algorithm called \csolvestar to solve
the constraints and determine assignments for unknown region and
constraint variables.

\begin{figure}

\begin{codeml}
$\elabExpr(CT, \A, \ralloc, \env, e)$ = 
  match $e$ with
  | $\C{new} \fgjN(\bar{e})$ $\longrightarrow$ 
    let $\fbN$ = $\templateTy(\fgjN)$ in
    let $C_1$ = $\typeOk({\A},{\fbN})$ in
    let $C_2$ = match $\fgjN$ with $\RgnZ\inang{T}$ $\longrightarrow$ $\top$
          | _ $\longrightarrow$ $\{\isvalid{\A.\phicx}{\allocRgn(\fbN)=\ralloc}\}$ in
    let ($\_:\taubar$) = $\fields(\fbN)$
    let $(\bar{e'}:\bar{\tau'}, C_3)$ = $\elabExpr(\A,\ralloc,\env,\bar{e})$ in
    let $C_4$ = $\subtypeOk({\A},{\bar{\tau'}},{\taubar})$ in
      ($\C{new} \fbN(\bar{e'}) : \fbN$,$\bigcup_{i=1}^4 C_i$)
  | $e_a(\bar{e})$ $\longrightarrow$ 
    let ($e_a':\inang{\rhoalloc\rhobar\,|\,\phi}\taubar \xrightarrow{\rgn} \tau$,$C_1$) = 
                $\elabExpr(\A,\ralloc,\env,e_a)$ in
    let $\bar{\rho'}$ = $\bar{\fresh_\rho()}$ in
    let $C_2$ = $\{\bar{\rho'} \in \A.\rhoenv\}$ in
    let $\substFn$ = $[\bar{\rho'}/\rhobar][\ralloc/\rhoalloc]$ in
    let $C_3$ = $\{\isvalid{\A.\phicx}{\substFn(\phi)}\}$ in
%*   %let $(C_3,C_4)$ = $(\typeOk(\substFn(\taubar)), \typeOk(\substFn(\tau)))$ in 
*)   let $(\bar{e'}:\bar{\tau'}, C_4)$ = $\elabExpr(\A,\ralloc,\env,\bar{e})$ in
    let $C_5$ = $\subtypeOk({\A},{\bar{\tau'}},{\bar{\substFn(\tau)}})$ in
      ($e_a'\inang{\ralloc\bar{\rho'}}(\bar{e'}):\substFn(\tau)$,$\bigcup_{i=1}^5 C_i$)
  | $\letregion{\rgn}{e_a}$ $\longrightarrow$
    let $\rgn'$ = $\fresh_{\rgn}()$ in 
    let $(\rhoenv,\aenv,\phicx)$ = $\A$ in
    let $\A'$ = ($\rhoenv \cup \{\rgn'\}, \aenv, \phicx \conj (\rhoenv \outlives \rgn')$) in
    let ($e_a':\tau$,$C_1$) = $\elabExpr(\A',\rgn',\env,[\rgn'/\rgn]e_a)$ in
      ($\letregion{\rgn'}{e_a'}:\tau$,$C_1$)
  | $\open{e_a}{\rgn}{y}{e_b}$ $\longrightarrow$ 
    let ($e_a':\RgnZ\inang{T}\inang{\rho}$,$C_1$) = 
                $\elabExpr(\A,\ralloc,\env,e_a)$ in
    let $\rgn'$ = $\fresh_{\rgn}()$ in 
    let $(\rhoenv,\aenv,\phicx)$ = $\A$ in
    let ($\A'$,$\env'$) = ($(\rhoenv \cup \{\rgn'\},\aenv,\phicx)$,$\env[y\mapsto T@\rgn']$) in
    let ($e_b':\tau$,$C_2$) = $\elabExpr(\A',\rgn',\env',[\rgn'/\rgn]e_a)$ in
      ($\open{e_a'}{\rgn'}{y}{e_b'} : \tau$, $C_1 \cup C_2$)
  | _ $\longrightarrow$ ...
\end{codeml}

\caption{Constraint generation for expressions in $\absof{\FB}$}
\label{fig:fb-elabexpr}
\end{figure}

\begin{figure}

\begin{codeml}
$\elabMeth(B,d)$ = 
  let $\headerOf{B}\{\bar{\tau^f}\,\xbar;\;k\;\bar{d}\}$ = $CT(B)$ in
  let $T \; m(\tbar \; \xbar)\{\C{return} e;\}$ = $d$ in
  let $(\tau,\taubar)$ = $(\templateTy(T),\templateTy(\tbar))$ in
  let $(\rhoalloc_m,\rhobarm,\varphi)$ = $(\fresh_\rho(),\frv(\tau\taubar),\fresh_\varphi())$ in
  let ($\rhoenv$,$\aenv$,$\phicx$) = $(\{\rhoalloc,\rhobar,\rhoalloc_m,\rhobarm\},\bar{\tyvar} \extends \bar{\fgjN},\phi \conj \varphi)$ in
  let $\env$ = $\cdot[\thisZ \mapsto B\inang{\bar{\tyvar}}\inang{\rhoalloc\rhobar}][\xbar \mapsto \taubar]$ in
  let $d_m$ = $\tau \; m\inang{\rhoalloc_m\rhobarm \,|\, \varphi} (\taubar \; \xbar)\{\C{return} \unitval;\}$ in
  let $C_B$ = $\headerOf{B}\{\bar{\tau^f}\,\xbar;\;k\;\bar{d}d_m\}$ in
  let _ = $CT := CT[B \mapsto C_B]$ in
  let ($e':\tau'$,$C_1$) = $\elabExpr (\A,\rhoalloc_m,\env,e)$ in
  let $C_2$ = $\subtypeOk({\A},{\tau'},{\tau})$ in
  let ($\substFn_\rho$,$\phi_{sol}$) = $\solve(C_1 \cup C_2,\{\rhoalloc,\rhobar,\rhoallocm,\rhobarm\})$ in
  let $d_m'$ = $\tau \; m\inang{\rhoalloc_m\rhobarm \,|\, \phi_{sol}} (\taubar \; \xbar)\{\C{return} \substFn_\rho(e');\}$ in
  let $C_B'$ = $\headerOf{B}\{\bar{\tau^f}\,\xbar;\;k\;\bar{d}d_m'\}$ in
  let _ = $CT := CT[B \mapsto C_B']$ in
    $d_m'$
\end{codeml}

\caption{Method elaboration in $\absof{\FB}$}
\label{fig:fb-elabmeth}
\end{figure}

