\section{Type Inference}
\label{sec:type-inference}

We formalize region type inference as an elaboration function from
programs in $\absof{\FB}$ (i.e., region-erased \FB) to programs in
\FB.  The type inference algorithm alternates between a constraint
generation phase and a constraint solving phase. During the generation
phase, inference algorithm introduces region variables ($\rho$) for
unknown region annotations on types, and constraint variables (denoted
$\varphi$) for unknown region constraints ($\phi$) on methods and
functions. Subsequently, the algorithm makes use of static semantics
in Fig.~\ref{fig:fb-staticsem} to generate constraints on region and
constraint variables. Due to the presence of region-polymorphic
recursion in \FB, constraints generated by the algorithm can be
circular. More precisely, constraints generated can assume the form
$\varphi \Leftrightarrow \phi \wedge F(\varphi)$, where $F$ is a
non-idempotent substitution function for region variables in
$\varphi$. In the constraint solving phase, the algorithm then relies
on a fixpoint constraint solving algorithm called \csolvestar to solve
the constraints and determine assignments for unknown region and
constraint variables.

\begin{figure}

\begin{codeml}
$\elabExpr(\A,\ralloc,\env,e)$ = 
  match $e$ with
  | $\unitval$ $\longrightarrow$ ($\unitval:\unitZ$,$\top$)
  | $x$ $\longrightarrow$ ($x:\env(x)$,$\top$)
  | $e_a.f$ $\longrightarrow$ 
    let ($e_a':\tau_a$,$C_1$) = $\elabExpr(\A,\ralloc,\env,e_a)$ in
    let $F$ = $\fields(\tau_a)$ in
      ($e_a'.f:F(f)$,$C_1$)
  | $e_a.m(\bar{e})$ $\longrightarrow$ 
    let ($e_a':\tau_a$,$C_1$) = $\elabExpr(\A,\ralloc,\env,e_a)$ in
    let $\inang{\rhoalloc\rhobar\,|\,\phi}\taubar \rightarrow \tau$ = $\mtype(m,\bound(\tau_a))$ in
    let $\bar{\rho'}$ = $\fresh({\sf length}(\rhobar))$ in
    let $C_2$ = $\bar{\rho'} \in \A.\rhoenv$ in
    let $\substFn$ = $[\bar{\rho'}/\rhobar][\ralloc/\rhoalloc]$ in
    let $C_3$ = $\substFn(\phi)$ in
%*   %let $(C_3,C_4)$ = $(\typeOk(\substFn(\taubar)), \typeOk(\substFn(\tau)))$ in 
*)   let $(\bar{e'}:\bar{\tau'}, C_4)$ = $\elabExpr(\A,\ralloc,\env,\bar{e})$ in
    let $C_5$ = $\subtyp{\A}{\bar{\tau'}}{\bar{\substFn(\tau)}}$ in
      ($e_a'.m\inang{\ralloc\bar{\rho'}}(\bar{e'}):\substFn(\tau)$,$\bigwedge_{i=1}^5 C_i$)
  | $\C{new} \fgjN(\bar{e})$ $\longrightarrow$ 
    let $\fbN$ = $\templateTy(\fbN)$ in
    let ($C_1$,$C_2$) = ($\tywf{\A}{\fbN}$,$\allocRgn(\fbN)=\ralloc$) in
    let ($_:\taubar$) = $\fields(\fbN)$
    let $(\bar{e'}:\bar{\tau'}, C_3)$ = $\elabExpr(\A,\ralloc,\env,\bar{e})$ in
    let $C_4$ = $\subtyp{\A}{\bar{\tau'}}{\taubar}$ in
      ($\C{new} \fbN(\bar{e'}) : \fbN$,$\bigwedge_{i=1}^4 C_i$)
  | $e_a(\bar{e})$ $\longrightarrow$ 
    let ($e_a':\inang{\rhoalloc\rhobar\,|\,\phi}\taubar \xrightarrow{\rgn} \tau$,$C_1$) = 
                $\elabExpr(\A,\ralloc,\env,e_a)$ in
    let $\bar{\rho'}$ = $\fresh({\sf length}(\rhobar))$ in
    let $C_2$ = $\bar{\rho'} \in \A.\rhoenv$ in
    let $\substFn$ = $[\bar{\rho'}/\rhobar][\ralloc/\rhoalloc]$ in
    let $C_3$ = $\substFn(\phi)$ in
%*   %let $(C_3,C_4)$ = $(\typeOk(\substFn(\taubar)), \typeOk(\substFn(\tau)))$ in 
*)   let $(\bar{e'}:\bar{\tau'}, C_4)$ = $\elabExpr(\A,\ralloc,\env,\bar{e})$ in
    let $C_5$ = $\subtyp{\A}{\bar{\tau'}}{\bar{\substFn(\tau)}}$ in
      ($e_a'\inang{\ralloc\bar{\rho'}}(\bar{e'}):\substFn(\tau)$,$\bigwedge_{i=1}^5 C_i$)
  | $\letregion{\rgn}{e_a}$ $\longrightarrow$
    let _ = ${\sf assert}(\rgn \notin \A.\rhoenv)$ in
    let $(\rhoenv,\aenv,\phicx)$ = $\A$ in
    let $\A'$ = ($\rhoenv \cup \{\rgn\}, \aenv, \phicx \conj (\rhoenv \outlives \rgn)$) in
      $\elabExpr(\A',\rgn,\env,e_a)$ 
  | $\open{e_a}{\rgn}{y}{e_b}$ $\longrightarrow$ 
    let ($e_a':\RgnZ\inang{T}\inang{\rgn'}$,$C_1$) = 
                $\elabExpr(\A,\ralloc,\env,e_a)$ in
    let _ = ${\sf assert}(\rgn \notin \A.\rhoenv)$ in
    let $(\rhoenv,\aenv,\phicx)$ = $\A$ in
      $\elabExpr((\rhoenv \cup \{\rgn\},\aenv,\phicx),\rgn,\env[y\mapsto T@\rgn],e_a)$ 
\end{codeml}

\caption{Constraint generation for expressions in $\absof{\FB}$}
\label{fig:fb-constraintgen}
\end{figure}

