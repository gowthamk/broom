\section{Type Inference}
\label{sec:type-inference}

\name's region type system imposes a heavy annotation burden, and
manually annotating C\# standard libraries with region types
can be tedious. We now present our region type inference algorithm
that eliminates the need to write region type annotations.
% except on some higher-order functions.
Formally, the type inference
algorithm is an elaboration function from programs in $\absof{\FB}$
(i.e., \FB without region types, but with \C{letregion} and \C{open}
expressions, similar to the language introduced in
\S~\ref{sec:overview}) to programs in \FB.

\paragraph{Overview.}

Fig.~\ref{fig:type-inference-algo} presents the high-level outline of the type
inference algorithm.
The algorithm consists of the following steps:
\begin{enumerate}
 \item \emph{Region Parametrization}.
   The first step elaborates the input program by introducing \emph{formal region parameters}
   (for each class and method), and \emph{region variables} (representing yet undetermined
   \emph{actual region parameters}). We also introduce for each class and method, a
   \emph{predicate variable} ($\varphi$) to denote an undetermined set of constraints
   over the region parameters of that class/method.

%%    as follows:
%% \begin{enumerate}
%% \item For every class determine a set of \emph{region parameters} ($\rho$)
%%        it should be parametrized over.
%% \item For every field of the class, determine its region parametrized type by
%%        generating as many fresh \emph{actual region parameters} as required by its class (type),
%%        which in turn become formal region parameters of the containing class.
%% \item  For every method and function type, determine its corresponding region
%%    type template by introducing fresh region parameters as required by
%%    the type of each parameter and return value (in addition to the allocation region
%%    parameter).
%%  \item Elaborate every \C{new} expression, method call or function application by
%%    introducing \emph{fresh region variables} (as required by the region type
%%    template of the called method or function).
%% \item Introduce for each class and method, a \emph{predicate variable} ($\varphi$) to denote
%%    an undetermined set of constraints over the region parameters of that
%%    class/method.
%% \end{enumerate}

 \item \emph{Constraint Generation}.
   In the second step, we analyze the program to generate a set of constraints
   (over the region identifiers and the predicate variables)
   that must hold (as per the static semantics in Fig.~\ref{fig:fb-staticsem}).

 \item \emph{Constraint Solving}.
   We solve the generated set of constraints using our fixpoint constraint
   solving algorithm \csolvestar, which reduces the constraint solving problem to
   an abduction problem. If the original program in $\absof{\FB}$ contains unsafe
   references, for example, a reference from a transferable region to a
   stack region, then the constraints generated during the elaboration
   are not satisfiable. In such a case, \csolvestar{} fails to solve
   the constraints.
% in a Herbrand constraint system, and then relies on \csolve,
% our abduction solver for that domain. 

 \item If the solver succeeds, it returns substitution functions $\substFn_\rho$ and
  $\substFn_\varphi$ for free region and predicate variables, respectively, introduced in
  step 1. We apply these substitutions to the elaborated program to produce the final program.
\end{enumerate}

\begin{figure}
\begin{numcodeml}
Infer ($p$) =
  let $q$ = IntroduceRegionParameters($p$) in
  let $(r,C)$ = GenerateConstraints($q$) in
  match (SolveConstraints($C$) with
  | None $\longrightarrow$ None
  | Some ($s$) $\longrightarrow$ Some (ApplySolution $(s,r)$)
\end{numcodeml}

\caption{The type inference algorithm}
\label{fig:type-inference-algo}
\end{figure}

We later on establish the soundness of the type inference algorithm, and also establish
partial completeness results.

\subsection{Region Parametrization for Classes}
\label{sec:fb-templatization}

Region parametrization is an iterative process involving the following three steps,
the first two of which are mutually dependent on each other.

\emph{Introduction of Formal Region Parameters}.
For every class \C{C}, we identify a sequence of formal region parameters
$\pi_0, \cdots \pi_n$ that \C{C} should be parametric over.

\emph{Introduction of Actual Region Parameters}.
We then replace every instance of class \C{C} in the program by an instance
$\C{C}\langle \rho_0, \cdots, \rho_n \rangle$, where $\rho_0, \cdots, \rho_n$
are fresh identifiers denoting actual region parameters.

\emph{Predicate Variable Introduction}. For every class \C{C}, we introduce
a fresh predicate variable $\varphi$, which represents the yet undetermined
outlives constraints between the formal region parameters of class \C{C}.

We identify the region parameters of classes as follows.

\emph{Non-Recursive Classes}.
The class \C{Object} is defined to have a single region parameter $\pi_0$ (the allocation region).
The region parameters for any other non-recursive class \C{C} is determined
only after the region parameters of any class that \C{C} depends on have been
determined: this includes the base-class \C{B} of \C{C} and the class (type)
of any of its fields.
We replace every dependee type \C{T} in \C{C} by its instantiated type,
using fresh region parameters as needed.
The sequence of region parameters for \C{C} is defined to be
the sequence of region parameters for the base class \C{B} concatenated
with the list of all  fresh region parameters introduced while instantiating the types
of the fields in the class.
(The class inherits its allocation region from its base class. Note that if
a class does not specify an explicit base class, it has an implicit base class
\C{Object}.)

This transformation is illustrated below, using a non-generic \C{Pair} class:

\begin{tabular}{ccc}
\begin{minipage}{0.28\linewidth}
\begin{codejava}
class Pair
  $\extends$ $\ObjZ$
{
  $\ObjZ$ fst;
  $\ObjZ$ snd;
}
\end{codejava}
\end{minipage}
&
$\Rightarrow$
&
\begin{minipage}{0.5\linewidth}
\begin{codejava}
class Pair $\langle \rho_0, \rho_1, \rho_2 \; | \; \varphi \rangle$
  $\extends$ $\ObjZ \langle \rho_0 \rangle$
{
  $\ObjZ \langle \rho_1 \rangle$ fst;
  $\ObjZ \langle \rho_2 \rangle$ snd;
}
\end{codejava}
\end{minipage}
\end{tabular}

\emph{Recursive Classes}.
The region parameters for a recursive class is computed in
a similar fashion, with the following difference: any recursive
field is ignored while instantiating region parameters for the fields of
the class, and the region parameters of the recursive class are computed
as before. We then do parameter instantiation for all recursive fields,
such that their region annotations (the actual region parameters) are
exactly the same as the (formal) region parameters of the class.
The following example illustrates this for a non-generic \C{List} class.
The resulting class represents a linked list with spine in the region
$\rho_0$ and data objects in the region $\rho_1$.

\begin{tabular}{ccc}
\begin{minipage}{0.28\linewidth}
\begin{codejava}
class List
  $\extends$ $\ObjZ$
{
  $\ObjZ$ data;
  List next;
}
\end{codejava}
\end{minipage}
&
$\Rightarrow$
&
\begin{minipage}{0.5\linewidth}
\begin{codejava}
class List $\langle \rho_0, \rho_1 \; | \; \varphi \rangle$
  $\extends$ $\ObjZ \langle \rho_0 \rangle$
{
  $\ObjZ \langle \rho_1 \rangle$ data;
  $\C{List} \langle \rho_0, \rho_1 \rangle$ next;
}
\end{codejava}
\end{minipage}
\end{tabular}

The above technique can be extended to mutually recursive classes in a
straightforward manner, by simultaneously parametrizing them (and
then instantiating them).

\emph{Type-Parametric Classes}.
Type parameters of classes are handled as follows.
Consider a type-parametric class \C{C $\langle$ T $\extends$ B $\rangle$}.
The parametric type \C{T} is instantiated using the number of region parameters
that its bound \C{B} has. If no bound is specified for \C{T}, the bound is taken
to be \C{Object}, and \C{T} is instantiated with one region parameter.

\emph{Function Types}.
Since \FB{} is higher-order, fields of function type are allowed. The parameter instantiation step
instantiates function types as follows: the type of every parameter as well as the return value is
instantiated with fresh region identifiers, as described earlier, and finally these region identifiers
are generalized as formal region parameters of the function type.
For example, the function type $\C{List} \rightarrow \C{Object}$ is instantiated as
$\inang{\rho_0, \rho_1, \rho_2} \C{List} \inang{\rho_1,\rho_2} \rightarrow \C{Object}\inang{\rho_0}$.

%% Likewise, given the region-annotated definition of
%% \C{Pair} class from \S\ref{sec:fb-syntax} a region type template for a
%% method with FGJ type $\C{Pair}\inang{\C{A},\C{B}} \rightarrow \C{A}$
%% is \footnote{In our exposition, we assume that classes $\C{A}$ and
%% $\C{B}$ are trivial subclasses of $\ObjZ$ with no fields/methods. Like
%% $\ObjZ$, they accept one region parameter - the allocation region of
%% their objects.}\footnote{We abuse arrow notation to also represent
%% types of methods, but unlike function types, there is no allocation
%% region annotation atop the arrow in a method type.}
%% $\inang{\rhoalloc_0,\rho_1,\rho_2,\rho_3 \,|\, \varphi_0}
%% \C{Pair}\inang{\C{A},\C{B}} \inang{\rho_1,\rho_2,\rho_3} \rightarrow
%% \unitZ$, where $\rhoalloc_0$ and $\rho_{1-3}$ are fresh region
%% variables, and $\varphi_0$ is a fresh predicate variable denoting
%% unknown constraints over $\rhoalloc_0$ and $\rho_{1-3}$.

\subsection{Region Parametrization for Methods}

As the next step, we introduce region parameters for every method.
We do this by instantiating the types of all parameters and the
return value (of the method) using fresh region identifiers (as explained previously),
and then generalizing these region identifiers as formal region parameters
of the method. As for the classes, we also introduce a fresh predicate variable
$\varphi$ for every method.

We then consider every method invocation in the program, and introduce
fresh region variables representing the (yet unknown) actual region
parameters for this particular invocation.
%
We similarly perform instantiation for every constructor invocation
of the form \C{new T($\ldots$)}, by instantiating the type \C{T} as
before, turning it into \C{new T$\langle \rho_0, \cdots, \rho_n \rangle$($\ldots$)},
where $\rho_0, \cdots, \rho_n$ are fresh region variables.

\subsection{Constraints}
\label{sec:fb-constraintsem}

The constraint generation algorithm mimics the static type checker, but accumulates
constraints that must hold for the type checking to succeed.

\paragraph{Syntax of Constraints.}
The constraints are expressed in terms of region identifiers ($\pi$ and $\rho$)
and predicate variables ($\varphi$).

\emph{Region identifiers} are of two kinds: \emph{region constants} and \emph{region variables}.
A \emph{region variable} is introduced to represent an unknown actual region parameter
of a method invocation or object alloation, which will be bound to a region constant by the end of the
type inference.
A region constant may be 
(a) a \emph{formal region parameter} of a class or method, or
(b) a \emph{static region identifier} introduced by a \C{letregion} construct, or
(c) an  \emph{open transferable region identifier} introduced by an \C{open} construct.

Every predicate variable $\varphi$ denotes an unknown \emph{region-constraint},
%% over a set of fixed formal region parameters,
where the set of region-constraints $\phi$ is defined by:
\begin{smathpar}
\begin{array}{lcl}
\phi & \coloneqq & true \ALT \rho \outlives \rho \ALT \rho = \rho \ALT \phi \conj \phi \\
\end{array}
\end{smathpar}
We will use the term \emph{validity constraint} to denote an entailment constraint
of the form $\isvalid{\varphi}{\rho_1 \outlives \rho_2}$.

Our constraints also make uses of \emph{pending substitutions} $F$:
% \footnote{we borrowed this terminology from~\cite{ltpldi08}}
A pending substitution serves to bind formal region parameters in $\varphi$ to the actual region parameters
used in a particular context:
\begin{smathpar}
\begin{array}{lcl}
F & \coloneqq & \cdot \ALT [\rho/\rho]F \\
\end{array}
\end{smathpar}
E.g., in the validity constraint $\isvalid{\rgn_1 \outlives
\rgn_2}{[\rgn_1/\rho_1][\rgn_2/\rho_2]\varphi}$, the pending substitution
is $[\rgn_1/\rho_1][\rgn_2/\rho_2]$. Any concrete formula (call it
$\phisol$) over variables $\rho_1$ and $\rho_2$ is a solution to
$\varphi$ if and only if the formula obtained by substituting $\rgn_1$
and $\rgn_2$ for $\rho_1$ and $\rho_2$ (resp.) in $\phisol$ is
deducible from $\rgn_1 \outlives \rgn_2$.

The constraints generated are of the following kinds:
\begin{itemize}

\item Well-formedness constraints of form $\rho \in \rhoenv$,
restricting the domain of unification for a region variable ($\rho$)
to a constant set $\rhoenv = \{ \pi_1, \cdots, \pi_n \}$ of regions in scope,

\item Well-formedness constraints of form $\tywf{\rhoenv}{\varphi}$, restricting the domain of a predicate
variable ($\varphi$) to the set of all possible constraint formulas over a fixed set of 
regions ($\rhoenv = \{ \pi_1, \cdots, \pi_n \}$) in scope, and

\item Validity constraints of the form $\isvalid{\varphi_{\C{C}}}{\varphi_{\C{B}}}$
  (e.g., to indicate that the region-constraint of a derived class \C{C} is stricter than
  the region-constraint of its base class \C{B}).

\item Validity constraints of the form $\isvalid{\varphi_{\C{C}}}{{F(\varphi_{\C{T}})}}$
  (e.g., to indicate the region-constraint of a class \C{C} must imply the
  region-constraint of each of its fields).

\item Validity constraints of form $\isvalid{\varphi_i \conj \phictxt} {\phicstr}$
where $\varphi_i$ is a predicate variable (representing the precondition of a
method to be determined), $\phicstr$ is a region constraint that is \emph{required}
to hold at a particular program point (within the method), and $\phictxt$ is
a region constraint that is \emph{known} to hold at that program point.

%
%% Formulas $\phictxt$ and $\phicstr$ are concrete, i.e., free
%% of predicate variables and pending substitutions. While $\phictxt$
%% captures relationships that are \emph{known} to hold between concrete
%% region identifiers (i.e., $\rgn$'s) when the constraint was generated,
%% $\phicstr$ captures relationships that are \emph{required} to hold
%% among region varibles (i.e., $\rho$'s), or relationships between
%% region variables and identifiers.
%

\item Validity constraints of the form $\isvalid{\varphi_i \conj \phictxt} {F_j(\varphi_j)}$
generated by an invocation of a method with precondition $\varphi_j$ (where $\phictxt$ and
$\varphi_i$ are as above).

\end{itemize}

% mathpartir's inferrule: seems to have problems with linebreaks
\newcommand{\genconstraint}[2]{\inferrule*{#1}{#2}}
\newcommand{\gcarrow}{\leadsto}
\newcommand{\gctransforms}{\models}

\newcommand{\gcrule}[2]{%
\begin{smathpar}\begin{array}{c}%
\renewcommand*{\arraystretch}{1.2}%
\RULE {#1} {#2}%
\end{array}\end{smathpar}%
}

\newcommand{\minigcrule}[3]{%
\begin{minipage}{#1}\begin{smathpar}\begin{array}{c}%
\renewcommand*{\arraystretch}{1.2}%
\RULE {#2} {#3}%
\end{array}\end{smathpar}\end{minipage}%
}

\newcommand{\subtypesym}{<:}

\newcommand{\typeok}[3]{{#1\,\vdash\,#2 \; \texttt{ok} \, \lhd #3}}
\newcommand{\exprok}[4]{{#1} \, \vdash \, {#2} : {#3} \, \lhd {#4}}
\newcommand{\subtypeok}[4]{{#1} \, \vdash \, {#2}  \subtypesym {#3} \, \lhd {#4}} 

\newcommand{\stdcontext}{\exptycx{\ralloc}{\env}}

\begin{figure*}[t]
% \begin{mathpar}

% NEW
  \gcrule
  {
    \typeok {\A} {\fbN} {C_1} \spc
    \exprok {\stdcontext} {\bar{e}} {\bar{\tau'}} {C_2} \spc
    C_3 = \{ \isvalid{\A.\phicx}{\allocRgn(\fbN)=\ralloc} \} \spc
    \fields(\fbN) = \bar{f} : \taubar \spc
    \subtypeok {\A} {\bar{\tau'}} {\bar{\tau}} {C_4}
  }
  {
    \exprok {\stdcontext}   {\C{new} \fbN(\bar{e})} {\fbN} {\cup_{i=1}^4 C_i}
  }

% FUNCTION INVOCATION
\gcrule
{
\exprok {\stdcontext} {e_a} {\inang{\rho_0 \rhobar\,|\,\phi}\taubar \xrightarrow{\rgn} \tau} {C_1} \spc
\exprok {\stdcontext} {\bar{e}} {\bar{\tau'}} {C_2} \spc
\substFn = [\bar{\rho'}/\rhobar][\rho_0'/\rho_0]
\\
C_3 = \{\rho_0' \bar{\rho'} \in \A.\rhoenv\} \spc
C_4 = \{\isvalid{\A.\phicx}{\substFn(\phi)}\} \spc
\subtypeok {\A} {\bar{\tau'}} {\substFn(\bar{tau})} {C_5}
}{
\exprok {\stdcontext} {e_a\inang{\rho_0'\bar{\rho'}}(\bar{e})} {\tau} {\cup_{i=1}^5 C_i}
}

% Original from elaboration algorithm:
% $e_a(\bar{e})$ $\longrightarrow$ 
%    let ($e_a':\inang{\rhoalloc\rhobar\,|\,\phi}\taubar \xrightarrow{\rgn} \tau$,$C_1$) = 
%                $\elabExpr(\A,\ralloc,\env,e_a)$ in
%    let $\bar{\rho'}$ = $\bar{\fresh_\rho()}$ in
%    let $C_2$ = $\{\bar{\rho'} \in \A.\rhoenv\}$ in
%    let $\substFn$ = $[\bar{\rho'}/\rhobar][\ralloc/\rhoalloc]$ in
%    let $C_3$ = $\{\isvalid{\A.\phicx}{\substFn(\phi)}\}$ in
%%*   %let $(C_3,C_4)$ = $(\typeOk(\substFn(\taubar)), \typeOk(\substFn(\tau)))$ in 
%*)   let $(\bar{e'}:\bar{\tau'}, C_4)$ = $\elabExpr(\A,\ralloc,\env,\bar{e})$ in
%    let $C_5$ = $\subtypeOk({\A},{\bar{\tau'}},{\bar{\substFn(\tau)}})$ in
%      ($e_a'\inang{\ralloc\bar{\rho'}}(\bar{e'}):\substFn(\tau)$,$\bigcup_{i=1}^5 C_i$)

% LET-REGION
\gcrule{
\A= (\rhoenv,\aenv,\phicx)  \spc
\rgn \notin \rhoenv \spc
\A' = (\rhoenv \cup \{\rgn\}, \aenv, \phicx \conj (\rhoenv \outlives \rgn))  \spc
\exprok{\A',\rgn,\env} {e_a} {\tau} {C_1} \spc
\typeok {\A} {\tau} {C_2}
}{
\exprok{\stdcontext} {\letregion{\rgn}{e_a}} {\tau} {(C_1 \cup C_2)}
}

% Original from elaboration algorithm:
%  | $\letregion{\rgn}{e_a}$ $\longrightarrow$
%    let $\rgn'$ = $\fresh_{\rgn}()$ in 
%    let $(\rhoenv,\aenv,\phicx)$ = $\A$ in
%    let $\A'$ = ($\rhoenv \cup \{\rgn'\}, \aenv, \phicx \conj (\rhoenv \outlives \rgn')$) in
%    let ($e_a':\tau$,$C_1$) = $\elabExpr(\A',\rgn',\env,[\rgn'/\rgn]e_a)$ in
%      ($\letregion{\rgn'}{e_a'}:\tau$,$C_1$)

\gcrule{
\exprok {\stdcontext} {e_a} {\RgnZ\inang{T}\inang{\rho}} {C_1} \spc
\A = (\rhoenv,\aenv,\phicx) \spc
\rgn \notin \rhoenv \spc
(\A',\env') = ((\rhoenv \cup \{\rgn\},\aenv,\phicx),\env[y\mapsto T@\rgn] \spc
\exprok {\A',\rgn,\env'} {e_b} {\tau} {C_2}
}{
\exprok {\stdcontext} {\open{e_a}{\rgn}{y}{e_b}} {\tau} {C_1 \cup C_2}
}

% Original from elaboration algorithm:
%  | $\open{e_a}{\rgn}{y}{e_b}$ $\longrightarrow$ 
%    let ($e_a':\RgnZ\inang{T}\inang{\rho}$,$C_1$) = 
%                $\elabExpr(\A,\ralloc,\env,e_a)$ in
%    let $\rgn'$ = $\fresh_{\rgn}()$ in 
%    let $(\rhoenv,\aenv,\phicx)$ = $\A$ in
%    let ($\A'$,$\env'$) = ($(\rhoenv \cup \{\rgn'\},\aenv,\phicx)$,$\env[y\mapsto T@\rgn']$) in
%    let ($e_b':\tau$,$C_2$) = $\elabExpr(\A',\rgn',\env',[\rgn'/\rgn]e_a)$ in
%      ($\open{e_a'}{\rgn'}{y}{e_b'} : \tau$, $C_1 \cup C_2$)



% \end{mathpar}
\caption{Constraint generation}
\label{fig:constraint-gen}
\end{figure*}


\paragraph{Constraint Generation.}
The constraint generation algorithm is a direct adaption of the type checker:
each type checking rule is modified to produce a set of constraints that must hold
for the type checker to succeed.
%
Fig.~\ref{fig:constraint-gen} illustrates this for the object allocator \C{new} construct.


\paragraph{Constraint Solution}
Solving the set ($C$) of constraints entails finding an
assignment for each predicate variable $(\varphi$) and each region
variable ($\rho$) that occurs free in $C$, that satisfies all
the validity constraints as well as the well-formedness constraints on
$\varphi$ and $\rhobar$.
% To simplify presentation, we think of $C$ as being parameterized on
% $\varphi$ and $\rhobar$, and write it as $C\lbrack \varphi,\rhobar
% \rbrack$. We now formalize the constraint satisfaction problem, and
% its solution.

\begin{definition}
\emph{(\textbf{Constraint Satisfaction Problem (CSP)})}
A constraint satisfaction problem is a tuple
$(\regionConstants, \regionVars, \predVars, \regionDeltaMap, \predDeltaMap, \constraintSet)$,
where $\regionConstants$ is a set of region constants,
$\regionVars$ is a set of region variables,
$\predVars$ is a set of predicate variables,
$\regionDeltaMap : \regionVars \rightarrow 2^{\regionConstants}$ is a function that
specifies a unification domain for each region variable,
$\predDeltaMap : \predVars \rightarrow 2^{\regionConstants}$ is a function that
specifies the domain for each predicate variable,
and $\constraintSet$ is a set of validity constraints in one of the following forms:
\begin{center}
\(
    \isvalid{\phictxt \conj \varphi}{\phicstr}\qquad
    \isvalid{\phictxt \conj \varphi}{F(\varphi)}
\)
\end{center}
The solution to the constraint satisfaction problem is a pair $(\regionSubstFn,\predSubstFn)$,
where $\regionSubstFn$ is a map from $\regionVars$ to $\regionConstants$
and $\predSubstFn$ is a map from $\predVars$ to a region-constraint formula such that
\begin{itemize}
  \item $\regionSubstFn(\rho) \in \regionDeltaMap(\rho)$, for every $\rho \in \regionVars$,

  \item $\predSubstFn$ is well-formed under $\predDeltaMap$
    (i.e., $\tywf{\predDeltaMap(\varphi)}{\predSubstFn(\varphi)}$, for every $\varphi \in \predVars$).

  \item Every sequent in $C$ is valid after substitutions $\predSubstFn$ and $
    \regionSubstFn$.
  %% $C\lbrack (\regionSubstFn,\predSubstFn) \rbrack$ is valid.
\end{itemize}
\end{definition}

\subsection{Solving the CSP}

\paragraph{Properties of generated constraints}
The set of generated constraints have two interesting properties that allow us to
solve the constraints efficiently.
%
The first property is that every region variable is guaranteed to be unified with some
region constant by the constraints\footnote{
There is one special case where this does not hold: a recursive function
that calls itself in a non-terminating fashion. Such a function never
returns a value, and so the return-value can be typed as anything.
We assume that the return value of such a function is typed to be \C{unit},
which will not introduce any region variables at the call-site of such a function.}
This follows because $\FB$ does not admit uninitialized variables.
% , and the type system uniquely determines the region parameters of every value
%
The second property is that the \emph{context region constraint} $\phictxt$
occurring on the antecedent of any validity constraint is a conjunction of
outlives-constraints of the form $\pi \outlives \pi_s$ where $\pi_s$
is a static region identifier and $\pi$ is a region constant (either a formal
region parameter, or static region identifier, or open transferable region identifier).
Furthermore, if $\phictxt$ includes any conjunct $\pi \outlives \pi_s$, then
it includes every conjunct $\pi_f \outlives \pi_s$ for every $\pi_f \in \predDeltaMap(\varphi)$.

\paragraph{Constraint Solver.}
Solving the constraints is an iterative process, consisting of the following
steps, and this process fails at any point if it determines that no solution
exists for the set of constraints:

\begin{enumerate}

\item
    A constraint of the form $\isvalid{\varphi}{\rho_i = \rho_j}$ unifies
$\rho_i$ and $\rho_j$ if at least one of them is a region variable.
If $\rho_i$ and $\rho_j$ are distinct region constants, then the constraint
solver fails.

\item
When a region variable $\rho$ is unified with a region constant $\pi$,
if $\pi \in \regionDeltaMap(\rho)$, we update $\regionSubstFn(\rho)$ to be $\pi$,
and replace every occurrence of $\rho$ by $\pi$ in the set of constraints.
If $\pi \not\in \regionDeltaMap(\rho)$, the constraint solver fails.

\item
\label{item:context}
Consider any constraint of the form $\isvalid{\varphi \conj \phictxt}{\pi_i \outlives \pi_j}$,
where $\pi_i$ and $\pi_j$ are both region constants.
If $\{ \pi_i, \pi_j \} \subseteq \predDeltaMap(\varphi)$, then we simply add
$\pi_i \outlives \pi_j$ as an additional conjunct to $\predSubstFn(\varphi)$.
Otherwise, we check if $\isvalid{\phictxt}{\pi_i \outlives \pi_j}$.
If this entailment does not hold, the constraint solver fails, since no valid solution is possible.

\item
  When any outlives constraint $\phi$ is added to $\predSubstFn(\varphi)$,
  then for every existing validity constraint $\isvalid{\varphi' \conj \phictxt}{F(\varphi)}$,
  we produce a new validity constraint $\isvalid{\varphi' \conj \phictxt}{F(\phi)}$. 
\end{enumerate}

Point~\ref{item:context} makes use of the special structure of the generated
constraints: namely that $\phictxt$  is a conjunction of outlives-constraints of the
form $\pi \outlives \pi_s$ where $\pi_s$ is a static region identifier and $\pi$ is a
region constant. As a result, we can show that $\isvalid{\varphi \conj \phictxt}{\pi_i \outlives \pi_j}$
iff $\isvalid{\varphi}{\pi_i \outlives \pi_j}$ or $\isvalid{\phictxt}{\pi_i \outlives \pi_j}$.

\paragraph{Algorithmic Aspects.}
The iterative process described above is a standard fixed point computation.
We can encode this computation using a set of Datalog rules,
and compute the solution using any Datalog engine.
Alternatively, algorithms for context-free reachability can be adapted
to solve the constraints, as explained below.

Given a set $S$ of outlives constraints, we define the directed
graph $G(S)=(V(S),E(S))$ as follows.
Every distinct region identifier $\rho$ in $S$ is represented by a vertex,
which we will also refer to as $\rho$.
Every outlives constraint $\rho_1 \outlives \rho_2$ is represented by
an edge from $\rho_1$ to $\rho_2$.
Recall that an equality constraint $\rho_1 = \rho_2$ is shorthand
for the pair of constraints $\rho_1 \outlives \rho_2$ and
$\rho_2 \outlives \rho_1$: thus, an equality constraint is represented
by a pair of edges.
%
It is easy to see that $S \models \rho_1 \outlives \rho_2$ iff
there exists a path from $\rho_1$ to $\rho_2$ in $G(S)$.
%
Thus, simple graph reachability (transitive closure) algorithms can be
used to identify all the logical consequences of a set of outlives
constraints.

Given a set of context-free validity constraints, we can generalize the
above approach and reduce the problem to one of context-free reachability
in graph~\cite{Reps:Reachability} (as usual for context-sensitive interprocedural analysis).
%
Algorithms for context-free reachability can also be adapted to incorporate
the above treatment of context constraints $\phictxt$.
%
Further details omitted.

\paragraph{Modularity Aspects.}
The type inference algorithm, as presented, traverses the entire program to
generate the set of constraints, which are solved en masse, using an iterative
fixed point computation. However, the type inference can be realized in a
modular and compositional fashion, subject only to the restrictions imposed
by recursion.

In the elaboration phase, we can process a class \C{C} only after any class
\C{B} that \C{C} depends on has been processed: class \C{C} depends on
class \C{B} if \C{B} is either \C{C}'s base class or the type of any field
of \C{C} depends on \C{B}. In effect, this means that any collection of
mutually recursive classes must be processed together. Non-recursive
dependences can be handled in a compositional fashion: if class \C{C}
depends on \C{B} non-recursively, then the elaboration can be done for
\C{B} first, and then \C{C} can be processed.

The same idea applies to the constraint-solving phase as well.
Given a set of constraints, we say that a predicate variable $\varphi_1$
\emph{directly-depends} on another predicate variable $\varphi_2$ if the set of
constraints includes a constraint $\isvalid{\varphi_1 \conj \phictxt}{F(\varphi_2)}$.
We say that $\varphi_1$ \emph{depends} on $\varphi_2$ if $\varphi_1$ transitively
depends on $\varphi_2$.
The constraint solver needs to process any collection of mutually dependent
predicate variables together.
In effect, this requires the type inference to process any collection of
mutually recursive methods together.
However, methods that are not mutually recursive can be processed separately.

%% \emph{Class Constraints}. 
%% Computing the class constraint $\varphi_{\C{C}}$ corresponding to a class \C{C} is
%% fairly straightforward. It is the conjunction of
%% \begin{itemize}
%% \item $\conj_{1 \leq i \leq n} (\rho_i \outlives \rho_0)$
%% \item $\varphi_{\C{B}}$
%% \item $[\rho_a/\rho_f]\varphi_{\C{F}}$
%% \end{itemize}

\paragraph{Soundness and Completeness.}

The type inference algorithm is sound:

\begin{theorem}
\emph{(\textbf{Soundness})}
For any $p \in \absof{\FB}$, if Infer($p$) returns Some($t$), then
(1) $\absof{t} = p$, and
(2) $t$ is well-typed.
\end{theorem}

Furthermore, the second phase of the type inference algorithm is complete as well.
If there is any possible instantiation (of the region variables and predicate variables) of
the elaborated program produced by the first phase that type checks, then the second phase
will identify it.

Incompleteness exists only in a few choices made during the first phase (elaboration phase), as
explained below.

\begin{enumerate}
\item Our technique for determining the set of region parameters for a recursive class
incorporates a specific heuristic, namely that the recursive occurrence of the class has
the same parameters, in the same order, as the class itself. This technique will fail, for
example. if the program uses a recursive list type whose elements alternatively come
from two different regions. Such a program would require the following elaboration,
which is beyond the scope of our approach:
% , $R_1$ and $R_2$.
% both of which are region parameters of the class.
\begin{tabular}{ccc}
\begin{minipage}{0.3\linewidth}
\begin{codejava}
class List
  $\extends$ $\ObjZ$
{
  $\ObjZ$ data;
  List next;
}
\end{codejava}
\end{minipage}
&
$\Rightarrow$
&
\begin{minipage}{0.5\linewidth}
\begin{codejava}
class List $\langle \rho_0, \rho_1, \rho_2 \; | \; \varphi \rangle$
  $\extends$ $\ObjZ \langle \rho_0 \rangle$
{
  $\ObjZ \langle \rho_1 \rangle$ data;
  $\C{List} \langle \rho_0, \rho_2, \rho_1 \rangle$ next;
}
\end{codejava}
\end{minipage}
\end{tabular}

\item Our technique for region parametrization also uses a heuristic in the case of
higher order programs. Consider a higher-order method that has a function $f$ as a
parameter. We have a choice in terms of where the region-parameters introduced for
$f$ are bound (i.e., quantified). We always bind these region-parameters within the type
of $f$ (essentially, requiring $f$ to be region-polymorphic). This may also be a potential
source of incompleteness in the type-inference.

The treatment of class fields with a function type also uses the same (incomplete)
heuristic.

\end{enumerate}

Region annotations provided by the user can help the type-inference overcome these
sources of incompleteness.

\subsection{Elaboration and Constraint Generation: Details}

\begin{figure}

\begin{codeml}
$\elabExpr(CT, \A, \ralloc, \env, e)$ = 
  match $e$ with
  | $\C{new} \fgjN(\bar{e})$ $\longrightarrow$ 
    let $\fbN$ = $\templateTy(\fgjN)$ in
    let $C_1$ = $\typeOk({\A},{\fbN})$ in
    let $C_2$ = match $\fgjN$ with $\RgnZ\inang{T}$ $\longrightarrow$ $\top$
          | _ $\longrightarrow$ $\{\isvalid{\A.\phicx}{\allocRgn(\fbN)=\ralloc}\}$ in
    let ($\_:\taubar$) = $\fields(\fbN)$
    let $(\bar{e'}:\bar{\tau'}, C_3)$ = $\elabExpr(\A,\ralloc,\env,\bar{e})$ in
    let $C_4$ = $\subtypeOk({\A},{\bar{\tau'}},{\taubar})$ in
      ($\C{new} \fbN(\bar{e'}) : \fbN$,$\bigcup_{i=1}^4 C_i$)
  | $e_a(\bar{e})$ $\longrightarrow$ 
    let ($e_a':\inang{\rhoalloc\rhobar\,|\,\phi}\taubar \xrightarrow{\rgn} \tau$,$C_1$) = 
                $\elabExpr(\A,\ralloc,\env,e_a)$ in
    let $\bar{\rho'}$ = $\bar{\fresh_\rho()}$ in
    let $C_2$ = $\{\bar{\rho'} \in \A.\rhoenv\}$ in
    let $\substFn$ = $[\bar{\rho'}/\rhobar][\ralloc/\rhoalloc]$ in
    let $C_3$ = $\{\isvalid{\A.\phicx}{\substFn(\phi)}\}$ in
%*   %let $(C_3,C_4)$ = $(\typeOk(\substFn(\taubar)), \typeOk(\substFn(\tau)))$ in 
*)   let $(\bar{e'}:\bar{\tau'}, C_4)$ = $\elabExpr(\A,\ralloc,\env,\bar{e})$ in
    let $C_5$ = $\subtypeOk({\A},{\bar{\tau'}},{\bar{\substFn(\tau)}})$ in
      ($e_a'\inang{\ralloc\bar{\rho'}}(\bar{e'}):\substFn(\tau)$,$\bigcup_{i=1}^5 C_i$)
  | $\letregion{\rgn}{e_a}$ $\longrightarrow$
    let $\rgn'$ = $\fresh_{\rgn}()$ in 
    let $(\rhoenv,\aenv,\phicx)$ = $\A$ in
    let $\A'$ = ($\rhoenv \cup \{\rgn'\}, \aenv, \phicx \conj (\rhoenv \outlives \rgn')$) in
    let ($e_a':\tau$,$C_1$) = $\elabExpr(\A',\rgn',\env,[\rgn'/\rgn]e_a)$ in
      ($\letregion{\rgn'}{e_a'}:\tau$,$C_1$)
  | $\open{e_a}{\rgn}{y}{e_b}$ $\longrightarrow$ 
    let ($e_a':\RgnZ\inang{T}\inang{\rho}$,$C_1$) = 
                $\elabExpr(\A,\ralloc,\env,e_a)$ in
    let $\rgn'$ = $\fresh_{\rgn}()$ in 
    let $(\rhoenv,\aenv,\phicx)$ = $\A$ in
    let ($\A'$,$\env'$) = ($(\rhoenv \cup \{\rgn'\},\aenv,\phicx)$,$\env[y\mapsto T@\rgn']$) in
    let ($e_b':\tau$,$C_2$) = $\elabExpr(\A',\rgn',\env',[\rgn'/\rgn]e_a)$ in
      ($\open{e_a'}{\rgn'}{y}{e_b'} : \tau$, $C_1 \cup C_2$)
  | _ $\longrightarrow$ ...
\end{codeml}

\caption{Constraint generation for expressions in $\absof{\FB}$}
\label{fig:fb-elabexpr}
\end{figure}

\newcommand{\hdOf}[2]{\C{class}\; #1\angAlpha\inang{\rhoalloc\rhobar \,|\, #2} \extends \fbN}
\begin{figure}

\begin{codeml}
$\elabMeth(CT, B, \tau \; m\inang{\rhoalloc_m\rhobarm \,|\, \varphi_m} (\taubar \; \xbar)\{\C{return} e;\})$ = 
  let $\hdOf{B}{\varphi}\{\bar{\tau^f}\,\xbar;\;k\;\bar{d}\}$ = $CT(B)$ in
  let ($\rhoenv$,$\aenv$,$\phicx$) as $\A$ = 
            $(\{\rhoalloc,\rhobar,\rhoalloc_m,\rhobarm\},\bar{\tyvar} \extends \bar{\fgjN},\varphi \conj \varphi_m)$ in
  let $C_1$ = $\{\tywf{\rhoenv}{\varphi_m}\}$ in
  let $\env$ = $\cdot[\thisZ \mapsto B\inang{\bar{\tyvar}}\inang{\rhoalloc\rhobar}][\xbar \mapsto \taubar]$ in
  let ($e':\tau'$,$C_2$) = $\elabExpr (\A,\rhoalloc_m,\env,e)$ in
  let $C_3$ = $\subtypeOk({\A},{\tau'},{\tau})$ in
    ($\tau \; m\inang{\rhoalloc_m\rhobarm \,|\, \varphi_m} (\taubar \;
    \xbar)\{\C{return} e';\}$, $C_1 \cup C_2 \cup C_3$)
\end{codeml}

\begin{codeml}
$\elabClass(CT,B)$ = 
  let $\hdOf{B}{\varphi}\{\bar{\tau^f}\,\xbar;\;k\;\bar{d}\}$ = $CT(B)$ in
  let ($\rhoenv$,$\aenv$,$\phicx$) as $\A$ = $(\{\rhoalloc,\rhobar\},\bar{\tyvar} \extends \bar{\fgjN},\varphi)$ in
  let $C_1$ = $\tywf{\rhoenv}{\varphi}$ in
  let ($C_2$,$C_3$) = ($\typeOk({\A},{\fbN})$,$\typeOk({\A},{\bar{\tau^f}})$) in
  let $C_4$ = $\{\isvalid{\phicx}{\allocRgn(\bar{\tau^f}) \outlives \rhoalloc \conj \allocRgn(\fbN) = \rhoalloc}\}$ in
  let ($k'$,$C_5$) = $\elabCons(B,k)$ in
  let ($\bar{d'}$,$C_6$) = $\elabMeth(B,\bar{d})$ in
    ($\hdOf{B}{\varphi}\{\bar{\tau^f}\,\xbar;\;k'\;\bar{d'}\}$, $\bigcup_{i=1}^6 C_i$)
\end{codeml}
%
% \begin{codeml}
% $\elabClassTable(CT)$ = 
%   let ($CT'$,$C$) = (ref $\cdot$, ref $\emptyset$) in
%   let _ = foreach $B \in dom(CT)$ do
%             let ($D_B$,$C_B$) = $\elabClass(CT,B)$ in
%               $CT'$ := $CT'[B \mapsto D_B]$;
%               $C$ := $C \cup C_B$
%            done in
%   let ($\substFn_\rho$,$\substFn_\varphi$) = $\solve(C)$ in
%     $\substFn_\varphi(\substFn_\rho(CT')) $
%       
% \end{codeml}

\caption{Method, class and class table elaboration}
\label{fig:fb-elabmeth}
\end{figure}


%Elaborating $\absof{\FB}$ expressions to $\FB$ expressions involves
%(a). replacing core types in variable declarations and \C{new}
%expressions with fresh region type templates, and (b). explicitly
%instantiating region parameters of methods with fresh region variables
%in method calls and function applications. This elaboration is
%performed with respect to the polymorphic type templates of classes
%and methods computed as per \S\ref{sec:fb-templatization}. 

We now present an integrated algorithm that processes all the methods
in a given program, elaborating them (by introducing region variables
to stand for actual region parameters) and simultaneously generating
the set of constraints that must be solved.

Function $\elabExpr$, shown in Fig.~\ref{fig:fb-elabexpr}, performs
this elaboration for a subset of expressions in $\absof{\FB}$, whose
corresponding $\FB$ expressions have been ascribed static semantics in
Fig.~\ref{fig:fb-staticsem}. $\elabExpr$ is defined under the same
context as the expression typing judgment in
Fig.~\ref{fig:fb-staticsem} with symbols $\A$,$\ralloc$, and $\env$
retaining their meaning. The function traverses expressions in a
syntax-directed manner of a type checker, introducing fresh region
type templates for unknown region types, while generating constraints
over region and predicate variables.
Note that $\elabExpr$ returns the type of the subexpression,
which is used to generate constraints for the expression. Functions
$\typeOk$ and $\subtypeOk$ (definitions not shown) used by $\elabExpr$
implement type well-formedness and subtype judgments from
Fig.~\ref{fig:fb-staticsem}, respectively.

Functions $\elabMeth$ and $\elabClass$ shown in
Fig.~\ref{fig:fb-elabmeth} lift expression elaboration to method and
class definitions, respectively. Both functions first build a context
($\A$) containing a set ($ \rhoenv$) of region variables denoting
regions that are currently live, a map ($\aenv$) mapping type
variables to their bounds, and a constraint formula ($\phicx$)
capturing constraints over live region variables. We use predicate
variables ($\varphi$ and $\varphi_m$) to capture constraints over
variables in $\rhoenv$ denoting the fact that such constraints are yet
to be inferred.

Function $\elabMeth$ elaborates a method definition of class $B$. It
calls $\elabExpr$ with the context $\A$, its allocation context
parameter ($\rhoallocm$), and a type environment ($\env$) that
contains region type bindings for all the arguments of the method,
including the implicit $\C{this}$ argument. The type 
returned by $\elabExpr$ for the method body is checked against its
expected type (derived from the type of the method)
generating more constraints. The function then returns the elaborated
method definition and the set of constraints.

$\elabClass$ elaborates the definition of a class $B$. It relies on
$\elabCons$\footnote{The definition of $\elabCons$ is straightforward,
hence not shown.} and $\elabMeth$ functions to elaborate $B$'s
constructor ($k$) and method definitions ($\bar{d}$), respectively. To
the set of constraints returned by these functions, $\elabClass$ adds
constraints generated by checking the well-formedness of the type
templates of its superclass and fields, and also a new constraint
capturing a couple of safety conditions: first, the allocation regions
of objects referred by the instance variables should outlive the
allocation region of the instance itself, and second, the allocation
regions of a class type and its superclass type must be the same.

Function $\elabClassTable$ (Fig.~\ref{fig:fb-elabmeth}) elaborates
every definition in the class table $CT$, while accumulating
constraints.

\section{TO DO Items}

\begin{itemize}
\item Add destructive object-field-update \emph{x.f := y} to the language.
\item Add if-then-else to the language.
\item Omit constructors from classes (and use an implicit constructor).
\item Discuss the safety of virtual method invocation, which provides a limited form
of existentially quantifying regions and instantiating (unpacking) them because a
derived class may have more region parameters than a base class.
\item Discuss the subtyping rule used to check an overriding method (in the static
semantics).
\item Mention that constraints are in the theory of partial orders?
\item Update discussion of examples
\end{itemize}

\textbf{TODO: Miscellaneous points to be integrated.}

To see why, recall that all the constraints on
identifiers in $\rhoenv_{\varphi}$ occuring in $\Phicx$ are subsumed
by $(\rhoenv - \rhoenv_{\varphi}) \outlives \rhoenv_{\varphi}$.
Therefore, it is impossible to derive $\phi_{ij}$ from $\Phicx$ by
only adding constraints on $\rgn \in \rhoenv_{\varphi}$. Hence, such a
solution $\phisol$ cannot exist.

\textbf{TODO: Is the above inequality correct?}



\subsection{Example}

\paragraph{Constraints Example 1} Consider the $\C{Pair}$ class
template from \S\ref{sec:fb-templatization}. Following constraints are
generated during its elaboration (Constrains are identified with
$\mathbf{c_i}$'s. Some trivial constraints, such as $\rho_4 \in
\rhoenv_0$ and $\rho_5 \in \rhoenv_1$, where $\rhoenv_0 =
\{\rhoalloc_0,\rho_{0-4}\}$ and $\rhoenv_1 = \rhoenv_0 \cup
\{\rho_5\}$, have been elided): 
\begin{smathpar}
\begin{array}{l}
  \csid{1} \tywf{\rhoenv_0}{\varphi_0} \qquad
  \csid{2} \isvalid{\varphi_0}{\rho_0 \outlives \rhoalloc_0 \conj \rho_1
     \succeq \rhoalloc_0 \conj \rho_4 = \rhoalloc_0} \\
  \csid{3} \isvalid{\varphi_0}{\rho_2 = \rho_0} \spc
  \csid{4} \isvalid{\varphi_0}{\rho_3 = \rho_1} \\
  \csid{5} \isvalid{\varphi_0 \conj \varphi_1} {\rho_5 = \rho_0}\spc
  \csid{6} \tywf{\rhoenv_1}{\varphi_1} \qquad
\end{array}
\end{smathpar}

\paragraph{Constraints Example 2} Let us add to the \C{Pair} class a
contrived method $\C{alt}$ that accepts a \C{Region} object \C{r}, a
\C{Pair<A,A>} object \C{q}, and an \C{A} object \C{y}. It assigns
\C{y} to \C{fst} and \C{snd} fields of \C{q}, and calls itself
recursively with the same region, a new \C{Pair} object allocated in a
local region, and an \C{A} object referred by the \C{snd} field of the
pair inside the region. \C{alt} never terminates.  Elaboration phase
elaborates the method to the following region-annotated
definition\footnote{In reality, elaboration uses new region variables
as parameters to the constructor and method calls, and then generates
constraints that unify them with actuals. In our examples, to avoid
clutter due to trivial constraints,we coalesced both steps and show
the actuals instead.}(The original definition of \C{alt} can be
obtained by erasing all the region annotations from the elaborated
version):
% \begin{codejava}
% unit alt<$\rhoalloc_2$,$\rho_{6-10}$ | $\varphi_2$>(Pair<A,A><$\rho_6$,$\rho_7$,$\rho_8$> p,
%                           A<$\rho_9$> x, A<$\rho_{10}$> y) {
%   p.fst = x; p.snd = y; 
%   this.alt<$\rhoalloc_2$,$\rho_7$,$\rho_8$,$\rho_{10}$,$\rho_9$>(A,y,x);
% }
% \end{codejava}
\begin{codejava}
unit alt<$\rhoalloc_2$,$\rho_{6-9}\,$|$\,\varphi_2$>(Region<Pair<A,A>><$\toprgn$> r, 
                Pair<A,A><$\rho_{6-8}$> q, A<$\rho_{9}$> y) {
  q.fst := y; q.snd := y; 
  open r as p@$\rgn_0$ in
    letregion $\rgn_1$ in
      let x = new Pair<A,A><$\rgn_1$,$\rgn_0$,$\rgn_0$>
                      (p.fst,p.fst) in
        alt<$\rgn_{1}$,$\rgn_{1}$,$\rgn_{0}$,$\rgn_{0}$,$\rgn_{0}$>(r,x,p.snd)
}
\end{codejava}
% Note that, to avoid clutter, we have already resolved appropriate
% region arguments to the recursive call, instead of introducing new
% region variables and generating equality constraints on them. 
Constraints generated during the elaboration are shown below
(let $\rhoenv_2 = \{\rhoalloc_0,\rho_{0-4},\rhoalloc_2,\rho_{6-9}\}$ ):
\begin{smathpar}
\begin{array}{l}
\csid{7} \tywf{\rhoenv_2}{\varphi_2}\qquad
\csid{8} \isvalid{\varphi_1 \conj \varphi_2}
    {\rho_7 \outlives \rho_6 \conj \rho_8 \outlives \rho_6} \\
\csid{9} \isvalid{\varphi_1 \conj \varphi_2}{\rho_7 = \rho_9} \qquad
\csid{10} \isvalid{\varphi_1 \conj \varphi_2}{\rho_{8} = \rho_9} \\
\csid{11} \isvalid{\varphi_1 \conj \varphi_2 \conj \rgn_0 \outlives
\rgn_1} 
    {[\rgn_1/\rhoalloc_2][\rgn_1/\rho_6][\rgn_0/\rho_{7-9}]\varphi_2}
\end{array}
\end{smathpar}

% \paragraph{Constraints Example 1} The original source of all
% outlives constraints is the validity constraint $C_3$ in $\elabClass$
% function (Fig.~\ref{fig:fb-elabmeth}). In context of the $\C{Pair}$
% class template from \S\ref{sec:fb-templatization}, the constraint is
% as following:
% \begin{center}
%   \(\isvalid{\varphi_0}{\rho_0 \succeq \rhoalloc_0 \conj \rho_0
%   \succeq \rhoalloc_0 \conj \rho_4 = \rhoalloc_0}\)
% \end{center}

% \paragraph{Constraints Example 2} Consider a contrived method
% $\C{alt}$ that accepts an argument $\C{p}$ of type
% $\C{Pair}\inang{\C{A},\C{A}}$, and two objects ($\C{x}$ and $\C{y}$)
% of type $A$. It then assigns one object to $\C{p.fst}$ and other to
% $\C{p.snd}$, and calls itself recursively without ever terminating.
% Which one is assigned to $\C{fst}$, and which to $\C{snd}$, is
% alternated between recursive calls. The region-annotated definition of
% $\C{alt}$ is shown below:
% \begin{codejava}
% unit alt<$\rhoalloc$,$\rho_{0-4}$ | $\varphi_0$>(Pair<A,A><$\rho_0$,$\rho_1$,$\rho_2$> p,
%                           A<$\rho_3$> x, A<$\rho_4$> y) {
%   return p.fst = x; p.snd = y; 
%          this.alt<$\rhoalloc$,$\rho_1$,$\rho_2$,$\rho_4$,$\rho_3$>(A,y,x)
% }
% \end{codejava}
% Note that, to avoid clutter, we have already resolved appropriate
% region arguments to the recursive call, instead of introducing new
% region variables and generating equality constraints on them. Rest of
% the validity constraints generated while elaborating $\C{alt}$ to the
% above region-annotated definition are shown below:
% \begin{center}
% \(\isvalid{\varphi_0}{\rho_1 \outlives \rho_0 \conj \rho_2 \outlives \rho_0}\)
% $\quad$
% \(\isvalid{\varphi_0}{\rho_1 = \rho_3}\)
% $\quad$
% \(\isvalid{\varphi_0}{\rho_2 = \rho_4}\)\\
% \(\isvalid{\varphi_0}{[\rho_3/\rho_4][\rho_4/\rho_3]\varphi_0}\)
% \end{center}
% The first constraint is generated by $\typeOk$ on the type of $\C{p}$.
% Second and third are generated by the the assignment expressions, and
% the last constraint is generated from the recursive call.

