\section{Appendix}

\subsection{Full Static Semantics}

\begin{figure*}[h!]
%
\begin{minipage}{2.6in}
\begin{smathpar}
\begin{array}{lcl}
  \allocRgn(A\inang{\rgn\rbar}\inang{\tbar}) & = & \rgn\\
  \allocRgn(\inang{\rhobar \,|\, \phi}\bar{\tau^1}
      \xrightarrow{\rgn} \tau^2) & = & \rgn\\
  \shape(A\inang{\rbar}\inang{\tbar}) & = & A\inang{\tbar}\\
\end{array}
\end{smathpar}
\end{minipage}
%
\begin{minipage}{2.6in}
\begin{smathpar}
\begin{array}{lcl}
  \bound_{\aenv}(\tyvar@\rgn) & = & \aenv(\tyvar)@\rgn\\
  \bound_{\aenv}(\fbN) & = & \fbN\\
  \fields(\ObjZ\inang{\rgn}) & = & \bullet \\
\end{array}
\end{smathpar}
\end{minipage}
%
% \begin{minipage}{\textwidth}
% \begin{smathpar}
% \begin{array}{lcl}
%   \ctype(\ObjZ\inang{\rgn}) & = & \bullet \\
%   \ctype(B\inang{\tbar}\inang{\rbar}) & = & 
%     \fields(B\inang{\tbar}\inang{\rbar})\\
% \end{array}
% \end{smathpar}
% \end{minipage}
%

%
\begin{minipage}{\textwidth}
\begin{smathpar}
\begin{array}{c}
\renewcommand*{\arraystretch}{1.2}
\RULE
  {
    CT(B) = \headerOf{B}\{\bar{\tau^f}\;\bar{f};\,...\}\spc
    \substFn = [\rbar/\rhobar, \tbar/\bar{\tyvar}] \qquad 
    \fields(\substFn(\fbN)) = \bar{g}:\bar{\tau^g}
  }
  {
    \fields(B\inang{\tbar}\inang{\rbar}) \;=\;
      \bar{g}:\bar{\tau^g},\,\bar{f}:\substFn(\bar{\tau^f})
  }
\end{array}
\end{smathpar}
\end{minipage}
%

\begin{minipage}{\textwidth}
\begin{smathpar}
\begin{array}{c}
\renewcommand*{\arraystretch}{1.2}
\RULE
  {
    CT(B) = \headerOf{B}\{\bar{\tau^f}\;\bar{f};\,\bar{d}\}\spc
    m \notin \bar{d} \qquad 
    \substFn = [\rbar/\rhobar, \tbar/\bar{\tyvar}]
  }
  {
    \mtype (m,B\inang{\tbar}\inang{\rbar}) \;=\;
    \mtype (m, \substFn(\fbN))
  }
\end{array}
\end{smathpar}
\end{minipage}
%

\begin{minipage}{\textwidth}
\begin{smathpar}
\begin{array}{c}
\renewcommand*{\arraystretch}{1.2}
\RULE
  {
    CT(B) = \headerOf{B}\{\bar{\tau^f}\;\bar{f};\,\bar{d}\}\spc
    \tau^2 \; m\mang (\bar{\tau^1}\;\bar{x})\{...\} \in \bar{d} \spc
    \substFn = [\rbar/\rhobar, \tbar/\bar{\tyvar}]
  }
  {
    \mtype (m,B\inang{\tbar}\inang{\rbar}) \;=\;
    \substFn(\mang\bar{\tau^1} \rightarrow \tau^2)
  }
\end{array}
\end{smathpar}
\end{minipage}
%
\begin{minipage}{\textwidth}
\begin{smathpar}
\begin{array}{c}
\renewcommand*{\arraystretch}{1.2}
\RULE
  {

    \mtype(m,\fbN) = \inang{\bar{\rho_1}\,|\, \phi_1}\bar{\tau^{11}} 
                      \rightarrow \tau^{12} \spc \texttt{implies} \\
    \isvalid{\A.\phicx}
                  {\phi_2 \Leftrightarrow \subst{\bar{\rho_2}}{\bar{\rho_1}}(\phi_1)}
    \spc \texttt{and} \spc
    \bar{\tau^{21}} = \subst{\bar{\rho_2}}{\bar{\rho_1}}(\bar{\tau^{11}}) \spc 
    \texttt{and} \spc \subtyp{\A}{\tau^{22}} {\subst{\bar{\rho_2}}{\bar{\rho_1}}(\tau^{12})}
%   \substFn = \subst{\bar{\rho_2}}{\bar{\rho_1}}
%              \subst{\rhoalloc_2}{\rhoalloc_1}
    %\substFn = [\rbar/\rhobar, \ralloc/\rhoalloc, \tbar/\bar{\tyvar}]
  }
  {
    \A \vdash \override(m,\fbN,\inang{\bar{\rho_2}\,|\, \phi_1}
              \bar{\tau^{21}} \rightarrow \tau^{22})
  }
\end{array}
\end{smathpar}
\end{minipage}
%

\caption{\fbname: auxiliary definitions}
\label{fig:fb-auxdef}
\end{figure*}


\begin{figure*}[!h]
%
\textbf{Subtyping}  \; \fbox
  {\(\subtyp{\A}{\tau_1}{\tau_2}\)}\\
%
\begin{minipage}{1.8in}
\begin{smathpar}
\begin{array}{c}
\renewcommand*{\arraystretch}{1.2}
  \subtyp{\A}{\tau}{\tau} \\
  \subtyp{(\Delta,\aenv,\phicx)}{\tyvar @\rho}{\aenv(\tyvar) @\rho}\qquad
% \subtyp{\A}{\RgnZ\inang{\rgn}}{\RgnZ\inang{\toprgn}}\qquad
% \subtyp{\A}{\RgnZ\inang{\toprgn}}{\RgnZ\inang{\rgn}}
\end{array}
\end{smathpar}
\end{minipage}
%
\begin{minipage}{2.7in}
\begin{smathpar}
\begin{array}{c}
\renewcommand*{\arraystretch}{1.2}
\RULE
  {
    \\
    CT(B) = \headerOf{B}\{...\}
  }
  {
    \subtyp{\A}{B\inang{\tbar}\inang{\rbar}}
        {[\rbar/\rhobar, \tbar/\bar{\tyvar}](\fbN)}
  }
\end{array}
\end{smathpar}
\end{minipage}
%

\begin{minipage}{1.5in}
\begin{smathpar}
\begin{array}{c}
\renewcommand*{\arraystretch}{1.2}
\RULE
  {
    \subtyp{\A}{\tau_1}{\tau_2}\\
    \subtyp{\A}{\tau_2}{\tau_3}
  }
  {
    \subtyp{\A}{\tau_1}{\tau_3}
  }
\end{array}
\end{smathpar}
\end{minipage}
%
\begin{minipage}{2.75in}
\begin{smathpar}
\begin{array}{c}
\renewcommand*{\arraystretch}{1.2}
\RULE
  {
    \isvalid{\A.\phicx}{\phi_1 \Rightarrow \phi_2} \\
    \subtyp{\A}{\bar{\tau^{11}}}{\bar{\tau^{21}}} \spc
    \subtyp{\A}{\tau^{22}}{\tau^{12}}
  }
  {
    \subtyp{\A}
      {\inang{\rhobar \,|\, \phi_2}\bar{\tau^{21}}
          \xrightarrow{\rgn} \tau^{22}}
      {\inang{\rhobar \,|\, \phi_1}\bar{\tau^{11}}
          \xrightarrow{\rgn} \tau^{12}}
  }
\end{array}
\end{smathpar}
\end{minipage}

%

\textbf{Type, and Type Constraint Well-formedness}  \; \fbox
  {\(\tywf{\A}{\tau}, \spc 
     \tywf{\rhoenv}{\phi}\)}\\
%
\begin{minipage}{1.95in}
\begin{smathpar}
\begin{array}{c}
\renewcommand*{\arraystretch}{1.2}
\RULE
  {
    \rgn \in \rhoenv
  }
  {
    \tywf{(\rhoenv,\aenv,\phicx)}{\ObjZ\inang{\rgn}}
  }
\end{array}
\end{smathpar}
\end{minipage}
%
%
\begin{minipage}{1.5in}
\begin{smathpar}
\begin{array}{c}
\renewcommand*{\arraystretch}{1.2}
\RULE
  {
    \rgn_0,\rgn_1 \in \rhoenv
%   \mem(\rgn_0) = \LIVE \spc
%   \mem(\rgn_1) = \LIVE
  }
  {
    \tywf{\rhoenv}{\rgn_0 \outlives \rgn_1}
  }
\end{array}
\end{smathpar}
\end{minipage}
%
%
\begin{minipage}{1.8in}
\begin{smathpar}
\begin{array}{c}
\renewcommand*{\arraystretch}{1.2}
\RULE
  {
    \tywf{\rhoenv}{\phi_0} \spc
    \tywf{\rhoenv}{\phi_1}
  }
  {
    \tywf{\rhoenv}{\phi_0 \wedge \phi_1}
  }
\end{array}
\end{smathpar}
\end{minipage}
%

% 
\begin{minipage}{\textwidth}
\begin{smathpar}
\begin{array}{c}
\renewcommand*{\arraystretch}{1.2}
\RULE
  {
    \rgn \in \rhoenv \spc
    \rhobar \not\in \rhoenv\spc
%   \rhobar \notin dom(\mem) \\
%   \mem' = \mem[\rhobar \mapsto \overline{\LIVE}] \spc
    \rhoenv' = \rhoenv \cup \{\rhobar\} \spc
    \A' = (\rhoenv', \aenv, \phicx \conj \phi) \spc
    \tywf{\rhoenv'}{\phi}\spc 
    \tywf{\A'}{\bar{\tau^1}} \spc
    \tywf{\A'}{\tau^2}
  }
  {
    \tywf{(\rhoenv,\aenv,\phicx)}{\inang{\rhobar \,|\, \phi}
              \bar{\tau^1} \xrightarrow{\rgn} \tau^2}
  }
\end{array}
\end{smathpar}
\end{minipage}
%


%
\textbf{Class Well-formedness}  \; \fbox
  {\(B \; \mathtt{ok}\)}\\
%
\begin{minipage}{\textwidth}
\begin{smathpar}
\begin{array}{c}
\renewcommand*{\arraystretch}{1.2}
\RULE
  {
    \rhoenv = \{\rho,\rhobar\} \spc
    \aenv = [\bar{\tyvar} \mapsto \bar{\fgjN}] \spc
    \phicx = \phi \spc
    \A = (\subtypcx)\spc
%   \A = (\{\rho,\rhobar\},[\bar{\tyvar} \mapsto \bar{\fgjN}],\phi)\spc
    \tywf{\rhoenv}{\phi} \spc
    \fgjtywf{\aenv}{\bar{\fgjN}} \spc
    \okin{\bar{d}}{B}
    \\
    \tywf{\A}{\fbN,\bar{\tau^f}} \spc
    \shape(\fbN) \neq \RgnZ\inang{T}\spc
%   \tywf{\A}{\bar{\tau^f}} \spc
    \isvalid{\phicx}{\allocRgn(\bar{\tau^f}) \outlives \rho} \spc
%   \ctype(\fbN) = \bar{\tau^{\fbN}} \spc
%    k = B(\bar{\tau^{\fbN}}\;\xbar,\bar{\tau^f}\;\bar{y})
%        \{\superZ(\xbar);\,\thisZ.\bar{f} = \bar{y};\} \spc
    \allocRgn(\fbN) = \rho \spc
  }
  {
    \C{class}\; B\angAlpha\inang{\rho\rhobar \,|\, \phi} \extends \fbN
    \{\bar{\tau^f}\;\bar{f};\, \bar{d}\} \;{\texttt{ok}}
  }
\end{array}
\end{smathpar}
\end{minipage}
%
%

\caption{\fbname: subtyping and well-formedness rules (contiunuation
of Fig.~\ref{fig:fb-staticsem})}
\label{fig:fb-staticsem-1}
\end{figure*}

\begin{figure*}[!t]
\textbf{Expression Typing}  \; \fbox {\(\hastyp{\exptycx{\env}{\rgn}}{e}{\tau}\)}\\

%
\begin{minipage}{0.4\textwidth}
\begin{smathpar}
\begin{array}{c}
\renewcommand*{\arraystretch}{1.2}
\RULE
  {
    \hastyp{\exptycx{\env}{\rgn}}{e_1}{\tau_1}\\
    \hastyp{\exptycx{\env[x\mapsto\tau_1]}{\rgn}}{e_2}{\tau_2}\\
  }
  {
    \hastyp{\exptycx{\env}{\rgn}}{\letexp{x}{e_1}{e_2}}{\tau_2}
  }
\end{array}
\end{smathpar}
\end{minipage}
%
%
\begin{minipage}{0.58\textwidth}
\begin{smathpar}
\begin{array}{c}
\renewcommand*{\arraystretch}{1.2}
\RULE
  {
%   \A = (\subtypcx)\\
%   \loc \in \rhoenv \spc
%   \hastyp{\exptycx{\env}}{e}{\tau}
    \A = (\subtypcx) \spc
    \loc \notin \rhoenv \\
    \A' = (\rhoenv\cup\{\loc\}, \aenv, \phicx \conj \Delta \outlives
    \loc)\\
    \hastyp{\A',\env,\loc}{e}{\tau} \spc
    \tywf{\A}{\tau}
  }
  {
    \hastyp{\exptycx{\env}{\rgn}}{\letd{\loc}{e}}{\tau}
  }
\end{array}
\end{smathpar}
\end{minipage}
%

%
\begin{minipage}{\textwidth}
\begin{smathpar}
\begin{array}{c}
\renewcommand*{\arraystretch}{1.2}
\RULE
  {
    \A = (\subtypcx)\spc
    \A' = (\rhoenv\cup\{\loc\}, \aenv,\phicx) \spc
%   \mem(\loc) = \USED \spc
    \tywf{\A'}{T@\loc} \spc
%   \fgjN = \RgnZ\inang{T}\spc
    \hastyp{\A',\env, \loc}{e}{T@\loc}
  }
  {
    \hastyp{\exptycx{\env}{\rgn}}{\C{new}\;
    \RgnZT{\toploc\loc}(e)} {\RgnZT{\toprgn}}
  }
\end{array}
\end{smathpar}
\end{minipage}
%

%
\begin{minipage}{3.4in}
\begin{smathpar}
\begin{array}{c}
\renewcommand*{\arraystretch}{1.2}
\RULE
  {
%   \A = (\subtypcx)\\
%   \loc \in \rhoenv \spc
%   \hastyp{\exptycx{\env}}{e}{\tau}
    \A = (\subtypcx) \spc
%   \loc \notin \rhoenv \\ % An open region can be opened again.
%   \loc \notin dom(\mem) \spc
    \A' = (\rhoenv\cup\{\loc\},\aenv,\phicx) \\
    \loc \in \Delta ~\texttt{implies}~ s = \LIVE \spc
    \hastyp{\A',\env,\loc}{e}{\tau} \spc
    \tywf{\A}{\tau}
  }
  {
    \hastyp{\exptycx{\env}{\rgn}}{\opened{\loc}{s}{e}}{\tau}
  }
\end{array}
\end{smathpar}
\end{minipage}
%
% %
% \begin{minipage}{2in}
% \begin{smathpar}
% \begin{array}{c}
% \renewcommand*{\arraystretch}{1.2}
% \RULE
%   {
%     \\
%     \\
%     \hastyp{\exptycx{\env}{\rgn}}{e}{\RgnZ\inang{T}\inang{\toprgn}}
%   }
%   {
%     \hastyp{\exptycx{\env}{\rgn}}{e.\transfer(\ldots)}{\unitZ}
%   }
% \end{array}
% \end{smathpar}
% \end{minipage}
%
\begin{minipage}{1.2in}
\begin{smathpar}
\begin{array}{c}
\renewcommand*{\arraystretch}{1.2}
\RULE
  {
    \hastyp{\exptycx{\env}{\rgn}}{e}{\tau} \\
    \subtyp{\A}{\tau}{\tau'}
  }
  {
    \hastyp{\exptycx{\env}{\rgn}}{e}{\tau'}
  }
\end{array}
\end{smathpar}
\end{minipage}
%

\caption{\fbname: expression typing (contiunuation of Fig.~\ref{fig:fb-staticsem})}
\label{fig:fb-staticsem-2}
\end{figure*}

% Figs.~\ref{fig:fb-staticsem-1} and~\ref{fig:fb-staticsem-2} show full
% static semantics of \FB. Fig.~\ref{fig:fb-staticsem-1} contains subtyping
% rules, and rules to check well-formedness of \FB types, type
% constraints, methods, and class definitions. Method well-formedness
% rule makes use of the expression typing judgment defined in
% Fig.~\ref{fig:fb-staticsem-2}. Auxiliary definitions used in
% Figs.~\ref{fig:fb-staticsem-1} and~\ref{fig:fb-staticsem-2} are
% defined in Fig.~\ref{fig:fb-auxdef}. As described in
% \S~\ref{sec:type-system}, all judgments are parameterized over the
% class table ($CT$). The premise $CT(B) = \headerOf{B}\{...\}$
% used in some of the rules means that the definition of class $B$ is
% present in $CT$ and that the definition is well-formed.

\subsection{Full Operational Semantics}

% Figs.~\ref{fig:fb-opsem-1} and~\ref{fig:fb-opsem-2} show the
% operational semantics of \fbname (Fig.~\ref{fig:fb-opsem-2} is also
% contained in the main paper as Fig.~\ref{fig:fb-opsem}. It is included
% here for the sake of completeness). 
\begin{figure*}[h!]

%
\fbox {\(\redstoo{\Delta}{(e,\mem)}{(e',\mem')}\)}\\

\lopsemrule{\textwidth}{
    \redstoo{\Delta}{(e,\mem)}{(e',\mem')}
}{
    \redstoo{\Delta}{(E\lbrack e \rbrack, \mem)}
            {(E\lbrack e' \rbrack, \mem')}
	  }{EvalOrder}
%
\vspace*{-0.1in}
%

\lopsemrule{\textwidth}{
    \redstoo{\Delta}{(e,\mem)}{\invalidexn}
  }{
    \redstoo{\Delta}{(E\lbrack e \rbrack, \mem)}
            {\invalidexn}
	  }{Exception}
%

\lopsemrule{\textwidth}{
%   \allocRgn(A\inang{\rgn\locbar}\inang{\tbar}) & = & \rgn
%   \allocRgn(\fbN) \in \rhoenv \spc
    \mem(\loc) = \LIVE \spc
    \fields(A\inang{\tbar}\inang{\loc\locbar}) = \bar{f}:\taubar
}{
    \redstoo{\Delta}{((\C{new} \; A\inang{\tbar}\inang{\loc\locbar}(\vbar)).f_i,\mem)}{(v_i,\mem)}
}{FieldAccess}

\lopsemrule{\textwidth}{
%   \allocRgn(\fbN) \in \rhoenv \spc
    \mem(\loc) = \LIVE \spc
    \mbody(m,A\inang{\tbar}\inang{\loc\locbar}) = \rho\rhobar.\bar{x}.e 
%   \redstoo{\Delta}{(, \mem)}{(e',\mem')}
}{
    \redstoo{\Delta}{((\C{new}\;A\inang{\tbar}\inang{\loc\locbar}(\bar{v})).m\inang{\loc'\overline{\loc'}}
                      (\bar{v'}),\mem)}
            {\\([\loc'\overline{\loc'}/\rho\rhobar][\bar{v'}/\xbar]
                [\C{new} \; A\inang{\tbar}\inang{\loc\locbar}(\bar{v})/\thisZ]\,e,\mem)}
	  }{MethodInv}

\lopsemrule{\textwidth}{
    v_a = \lambdaexp{\loc'}{\rho\rhobar}
                        {\taubar \; \xbar}{e} \spc
    \mem(\loc') = \LIVE  \spc 
}{
    \redstoo{\Delta}{(v_a\inang{\loc\locbar}(\bar{v}) ,\mem)}
            {([\bar{v}/\xbar][\loc\locbar/\rho\rhobar]\,e,\mem)}
	  }{FnApply}

%
% \lopsemrule{\textwidth}{
% %   \rgn \notin \rhoenv \spc
% %   \fresh(\rgn') \spc
%     \redstoo{\Delta}{(e,\mem)}{\invalidexn}
% }{
%     \redstoo{\Delta}{(\letregion{\rgn}{e},\mem)}{\invalidexn}
%   }{EXCEPTION}

\lopsemrule{\textwidth}{
%   \fgjN = \RgnZ\inang{T} \spc
%   \rgn \in dom(\mem) \spc
    \mem(\loc) = \USED \spc
    \redstocup{\loc}{(e,\mem[\loc \mapsto \LIVE])}{\invalidexn}
}{
    \redstoo{\Delta}{(\C{new} \; \RgnZT{\toploc\loc} (e),\mem)}
            {\invalidexn}
	  }{Exception}

\lopsemrule{\textwidth}{
%   \fgjN = \RgnZ\inang{T} \spc
%   \rgn \in dom(\mem) \spc
}{
    \redstoo{\Delta}{(\letexp{x}{v}{e},\mem)}
            {([v/x]e,\mem)}
	  }{LetExp}

\textbf{Evaluation Context} \fbox {\(E\)}\\
\begin{smathpar}
\begin{array}{lcl}
E & \coloneqq & \bullet \ALT (\bullet).f \ALT \bullet.m\inang{\locbar}(\ebar) \ALT
      v.m\inang{\locbar}(...,\bullet,...) \ALT \C{new}\;
      \fbN(...,\bullet,...) \\
  &  & \ALT \C{new} \; \RgnZ\inang{T}\inang{\toprgn}(\bullet)
       \ALT \bullet\inang{\locbar}(\ebar)\ALT
       v\inang{\locbar}(...,\bullet,...)\\
  &  & \ALT \letexp{x}{\bullet}{e} \ALT \open{\bullet}{\rgn}{y}{e} 
%      \ALT \opened{\loc}{\status}{\bullet} \ALT \letd{\loc}{\bullet}
%     The following should be forbidden. See NEW-REGION rule 2.
%      \ALT \C{new} \; \RgnZ\inang{T}\inang{\toploc\loc}(\bullet)
\end{array}
\end{smathpar}

\caption{\fbname: operational semantics (continuation of
Fig.~\ref{fig:fb-opsem})}
\label{fig:fb-opsem-1}
\end{figure*}
\clearpage





\subsection{Full Constraint Generation Semantics}

\newcommand{\localokin}[2]{#1 \; \texttt{ok} \; \texttt{in} \; #2}
\newcommand{\exptycxFix}[1]{\A,#1,r}

\begin{figure*}[!h]

\beginrules

%%%%%%%%%%% Header Box %%%%%%%%%%%
\multicolumn{2}{l}{
  \textbf{Expression Typing Constraint Generation} \; \fbox{  \( \exprok{\stdcontext}{e}{\tau}{C} \)}
}
\\[0.3cm]

%%%%%%%%%%% () and x %%%%%%%%%%%
\lgcfact{Unit}{\exprok{\stdcontext}{\unitval}{\unitZ}{\{\}}}

\lgcfact{Var}{\exprok{\stdcontext}{x}{\env(\tau)}{\{\}}}

% \begin{minipage}{1.2in}
% \begin{smathpar}
% \begin{array}{l}
% \renewcommand*{\arraystretch}{1.2}
% \exprok{\stdcontext}{\unitval}{\unitZ}{\{\}} \\
% \exprok{\stdcontext}{x}{\env(\tau)}{\{\}}
% \end{array}
% \end{smathpar}
% \end{minipage}

%%%%%%%%%%% FIELD-ACCESS: e.f %%%%%%%%%%%
\lgcrule{FieldAccess}
  {
    \exprok{\stdcontext}{e}{\tau'}{C} \spc
    \bar{f}:\taubar \,=\, \fields(\bound_{\A.\aenv}(\tau'))
  }
  {
    \exprok{\stdcontext}{e.f_i}{\tau_i}{C}
  }

%%%%%%%%%%% LET %%%%%%%%%%%
  \lgcrule{Let}
  {
    \exprok{\stdcontext}{e_1}{\tau_1}{C_1} \spc
    \exprok{\A,{\env[x\mapsto\tau_1]},r}{e_2}{\tau_2}{C_2} \\
  }
  {
    \exprok{\stdcontext}{\letexp{x}{e_1}{e_2}}{\tau_2}{C_1 \cup C_2}
  }

%%%%%%%%%%% METHOD-INV %%%%%%%%%%%
  \lgcrule{MethodInv}
  {
    \exprok {\stdcontext} {e_0} {\tau} {C_1} \spc C_2 = \{ r \rbar \in \A.\rhoenv \}
    \nl
    \mtype(m,\bound_{\A.\aenv}(\tau)) = \inang{\rho \rhobar \,|\, \phi}\bar{\tau^1}\rightarrow{\tau^2}
    \nl
%   \substFn = [\rbar/\rhobar] \\
    \typeok {\A} {\inang{\rho\rhobar \,|\,\phi}\bar{\tau^1}\rightarrow{\tau^2}} {C_3}
       \spc
       \exprok {\stdcontext} {\bar{e}} {[r\rbar/\rho\rhobar](\bar{\tau^1})} {C_4}
    \nl
%   \subtyp{\A}{\bar{\tau'}}{\substFn(\bar{\tau^1})} \spc
    C_5 = \{ \isvalid{\A.\phicx}{[r\rbar/\rho\rhobar](\phi)} \}
  }
  {
    \exprok {\stdcontext} {e_0.m\inang{r \rbar}(\bar{e})} 
       {[r \rbar / \rho \rhobar](\tau^2)} {C_1 \cup C_2 \cup C_3 \cup C_4 \cup C_5}
  }

%%%%%%%%%%% LAMBDA %%%%%%%%%%%

\lgcrule{Lambda}
  {
    \rgn \in \A.\rhoenv \spc
    \rho\rhobar \notin \A.\rhoenv
    \spc
%   \rhoenv' = \rhoenv \cup \{\rhoalloc,\rhobar\}\spc
    \A' = (\A.\rhoenv \cup \{\rho\rhobar\}, \A.\aenv, 
          \A.\phicx \conj \phi)
    \\
    \tywf{\A'.\rhoenv}{\phi}\spc
    \typeok {\A'} {\bar{\tau^1}} {C_1} \spc
    \typeok {\A'} {\tau^2} {C_2}
    \\
    \exprok {\A',\env[\xbar \mapsto \bar{\tau^1}],\rho} {e} {\tau^2} {C_3}
    % \spc
    % C_4 = \bigcup\limits_{\pi'\in \frv(e)\setminus\{\rhobar\}}\{\pi'\outlives\pi\}
  }
  {
    \exprok {\stdcontext}
           {\lambdaexp{\rgn}{\rho\rhobar \,|\, \phi} {\xbar:\bar{\tau^1}}{e}}
           {\inang{\rho\rhobar \,|\, \phi} \bar{\tau^1} \xrightarrow{\rgn} \tau^2}
	   {\cup_{i=1}^{4} C_i}
  }


%%%%%%%%%%% SUBTYPING %%%%%%%%%%%
\lgcrule{SubTyping}{
    \exprok {\exptycxFix{\env}} {e} {\tau} {C_1} \spc  \subtypeok {\A} {\tau} {\tau'} {C_2}
}{
    \exprok {\exptycxFix{\env}} {e} {\tau'} {C_1 \cup C_2}
}

%%%%%%%%%%% METHOD %%%%%%%%%%%
\multicolumn{2}{l}{
   \textbf{Method Well-formedness Constraint Generation}  \; \fbox{$\okinok{d}{B}{C}$}
}
\\[0.3cm]

\lgcrule{Method}{
CT(B) = \C{class}\; B \angAlpha \inang{\rhobar \,|\, \varphi} \extends \fbN
   \{ \cdots \}
   % \{ \bar{\tau^f}\,\xbar;\;\bar{d} \}
\nl
\A = (\rhoenv,\aenv,\phicx) = (\{\rhobar,\rho_m,\rhobarm\}, [\bar{\tyvar} \mapsto \bar{\fgjN}], \varphi_m) \spc\spc
C_1 = \{ \tywf{\rhoenv}{\varphi_m} \}
\nl
\env = \cdot[\thisZ \mapsto B\inang{\bar{\tyvar}}\inang{\rhobar}][\xbar \mapsto \bar{\tau^1}]
\spc
\mtype(m,\fbN) = \inang{\rhobarm \,|\, \phi_m}\bar{\tau^1} \rightarrow \tau^2
\nl
\exprok {\A,\env,\rho_m}{e} {\tau^2} {C_2}
\spc \typeok {\A} {\bar{\tau^1}} {C_3}
\spc \typeok {\A} {\tau^2} {C_4}
% \subtypeok {\A} {\tau'} {\tau} {C_3}
}{
\okinok {\tau^2 \; m\inang{\rho_m \rhobarm \,|\, \varphi_m} (\bar{\tau^1} \;  \xbar)\{\C{return} e;\}}
   {B}
   {(C_1 \cup C_2 \cup C_3 \cup C_4)}
}

%%%%%%%%%%% CLASS %%%%%%%%%%%
\multicolumn{2}{l}{
   \textbf{Class Well-formedness Constraint Generation}  \; \fbox{$\classok{B}{C}$}
}
\\[0.3cm]

\lgcrule{Class}{
  \A = (\rhoenv, \aenv, \phicx) = (\{\rho,\rhobar\}, [\bar{\tyvar} \mapsto \bar{\fgjN}],\varphi) \\
C_1 = \{ \tywf{\rhoenv}{\varphi} \} \spc\spc
\fgjtywf{\aenv}{\bar{\fgjN}} \spc\spc
\typeok {\A} {\fbN} {C_2} \spc\spc
\typeok {\A} {\bar{\tau^f}} {C_3} \\
C_4 = \{\isvalid{\phicx}{\allocRgn(\bar{\tau^f}) \outlives \rho \conj \allocRgn(\fbN) = \rho}\} \\
\okinok {\bar{d}} {B} {C_5}
}{
% \typeok {} {\hdOf{B}{\varphi}\{\bar{\tau^f}\,\xbar;\;\bar{d}\}} {\bigcup_{i=1}^5 C_i}
\classok
  {\C{class}\; B\angAlpha \inang{\rho \rhobar \,|\, \varphi} \extends \fbN
      \{\bar{\tau^f}\,\xbar;\;\bar{d}\}}
  {\bigcup_{i=1}^5 C_i}
}


\myendrules

\caption{\fbname: Constraint generation rules part 2 (Continuation of
Fig.~\ref{fig:constraint-gen-0})}
\label{fig:constraint-gen-1}
\end{figure*}

\newpage

\begin{figure*}[t!]

\beginrules

%%%%%%%%%%% Header Box %%%%%%%%%%%
\multicolumn{2}{l}{
  \textbf{Subtyping Constraint Generation} \;
\fbox{  \( \subtypeok{\A}{\tau_1}{\tau_2}{C} \)}
}
\\[0.3cm]

%%%%%%%%%%% SUBTYPING: REFLEXIVE %%%%%%%%%%%

  \lgcfact{Reflexivity}{
    \subtypeok{\A}{\tau}{\tau}{\{\}}
  }

%%%%%%%%%%% SUBTYPING: UNIFICATION %%%%%%%%%%%
  \lgcfact{Unify}{
    \subtypeok{\A}{\tau}{[\pi/\rho](\tau)}{ \{ \pi \outlives \rho, \rho \outlives \pi \} }
  }

%%%%%%%%%%% SUBTYPING: TRANSITIVITY %%%%%%%%%%%

  \lgcrule{Transitivity}{
    \subtypeok{\A}{\tau_1}{\tau_2}{C_1} \spc
    \subtypeok{\A}{\tau_2}{\tau_3}{C_2}
  }{
    \subtypeok{\A}{\tau_1}{\tau_3}{C_1 \cup C_2}
  }

%%%%%%%%%%% FUNCTION SUBTYPING %%%%%%%%%%%
  \lgcrule{FnSubtyping}
  {
    C_1 = \{ \isvalid{\A.\phicx}{\phi_1 \Rightarrow \phi_2} \}
    \\
    \subtypeok {\A} {\bar{\tau^{11}}} {\bar{\tau^{21}}} {C_2}
    \spc
    \subtypeok {\A} {\tau^{22}} {\tau^{12}} {C_3}
  }
  {
    \subtypeok {\A}
      {\inang{\rhobar \,|\, \phi_2}\bar{\tau^{21}} \xrightarrow{\rgn} \tau^{22}}
      {\inang{\rhobar \,|\, \phi_1}\bar{\tau^{11}} \xrightarrow{\rgn} \tau^{12}}
      {C_1 \cup C_2 \cup C_3}
  }

%%%%%%%%%%% Header Box %%%%%%%%%%%
\multicolumn{2}{l}{
  \textbf{Type Well-formedness Constraint Generation} \;
  \fbox{  \( \typeok{\A}{\tau}{C} \)}
}
\\[0.3cm]

%%%%%%%%%%% TYPE WELL-FORMEDNESS %%%%%%%%%%%

%%%%%%%%%%% OBJECT TYPE %%%%%%%%%%%
\lgcrule{ObjectType}
  {
    C = \{ \rgn \in \rhoenv \}
  }
  {
    \typeok {(\rhoenv,\aenv,\phicx)} {\ObjZ\inang{\rgn}} {C}
  }

%%%%%%%%%%% CLASS TYPE %%%%%%%%%%%
  \lgcrule{ClassType}
  {
    CT(B) = \headerOf{B}\{...\}
    \spc
    \fgjtywf{\aenv}{B\inang{\tbar}}
    \\
    C = \{ \rbar \in \rhoenv, \isvalid{\phicx}{[\rbar/\rhobar](\phi)} \}
  }
  {
    \typeok {(\rhoenv,\aenv,\phicx)} {B\inang{\tbar}\inang{\rbar}} {C}
  }

%%%%%%%%%%% GENERIC TYPE PARAMETER %%%%%%%%%%%
  \lgcrule{TypeParam}
  {
    \fgjtywf{\aenv}{T} \spc
    \fgjsubtyp{\aenv}{T}{\ObjZ} \spc
    \spc
    C = \{ \rgn \in \rhoenv \}
  }
  {
    \typeok {(\rhoenv,\aenv,\phicx)}{T@\rgn} {C}
  }

%%%%%%%%%%% FUNCTION TYPE %%%%%%%%%%%
  \lgcrule{FnType}
  {
    C_1 = \{ \rgn \in \rhoenv \}
    \\
    \rhobar \notin \rhoenv \spc
    \rhoenv' = \rhoenv \cup \{\rhobar\} \spc
    \A' = (\rhoenv', \aenv, \phicx \conj \phi)
    \\
    \typeconsok{\rhoenv'}{\phi} {C_2} \spc 
    \typeok{\A'}{\bar{\tau^1}} {C_3} \spc
    \typeok{\A'}{\tau^2} {C_4}
  }
  {
    \typeok{(\rhoenv,\aenv,\phicx)} {\inang{\rhobar \,|\, \phi} \bar{\tau^1} \xrightarrow{\rgn} \tau^2} 
       {C_1 \cup C_2 \cup C_3 \cup C_4}
  }

%%%%%%%%%%% REGION TYPE %%%%%%%%%%%
  \lgcrule{RegionType}
  { 
    \fgjtywf{\aenv}{T}
  }
  {
    \typeok {(\rhoenv,\aenv,\phicx)} {\RgnZ\inang{T}\inang{\toprgn}} {\{\}}
  }

\myendrules

\caption{\fbname: Constraint generation rules part 3 (Continuation of
Fig.~\ref{fig:constraint-gen-1})}
\label{fig:constraint-gen-2}
\end{figure*}
\clearpage

