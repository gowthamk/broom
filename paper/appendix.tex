%\section{Operational Semantics}
%
%Fig.~\ref{fig:fb-opsem} contains a definition of the operational semantics for \fbname.
%
\section{\fbname}

\subsection{Full Syntax}
\renewcommand{\rgn}{r}
\renewcommand{\rbar}{\overline{r}}
\begin{figure*}[t!]
%
\begin{smathpar}
\renewcommand{\arraystretch}{1.2}
\begin{array}{lclcl} 
\multicolumn{5}{c}{
  {\pi} \in \mathtt{Static \; region \; ids} \qquad
  {\rho} \in \mathtt{Region \; variables} \qquad
  {\loc} \in \mathtt{Memory\; locations} }\\
\multicolumn{5}{c}{
  {\tyvar, \tyvarb} \in \mathtt{Type \; variables} \qquad
  {m} \in \mathtt{Method \; names} \qquad
  {x,y,f} \in \mathtt{Variables \; and \; fields} }\\
cn & \in & \M{Class \; names} & \coloneqq & \ObjZ \ALT \RgnZ \ALT A \ALT B\\
     \fgjN & \in & \M{FGJ \; class \; types} & \coloneqq & cn\inang{\tbar}\\
T  & \in & \M{FGJ \; types} & \coloneqq & \tyvar \ALT  \fgjN \ALT \unitZ
     \ALT \bar{T} \rightarrow T \\
r  & \in & \M{Region\; annotations} & \coloneqq & \rho \ALT \pi \ALT \loc\\
\status & \in & \mathtt{Region\; Typestate} & \coloneqq & \FREE \ALT
                  \USED \ALT \LIVE \ALT \XFERRED\\
\fbN  & \in & \M{Region-annotated \; class \; types} & \coloneqq & 
     cn\inang{\tbar}\inang{\rbar} \\
\tau &\in& \M{types} & \coloneqq & T@\rgn  
      \ALT \fbN \ALT \unitZ 
      \ALT \inang{\rhobar \,|\, \phi}\bar{\tau}
      \xrightarrow{\rgn} \tau \\
C  & \in & \M{Class \; definitions} & \coloneqq & 
     \C{class} \; cn\inang{\bar{\tyvar} \extends \bar{\fgjN}} 
                    \inang{\rhobar \,|\, \phi}\extends \fbN 
                    \{\bar{\tau} \; \bar{f};\; \bar{d}\}\;\\
%k  & \in & \M{Constructors} & \coloneqq & 
%     cn(\bar{\tau} \; \bar{x})\{\C{super}(\bar{x}); \;
%                                \C{this}.\bar{f}\,=\,\bar{x};\}\\
d  & \in & \M{Methods} & \coloneqq & 
     \tau \; m\inang{\rhobar \,|\, \phi} (\taubar \; \xbar)
     \{\C{return}\;e;\}\\
\phi,\phicx &\in& \M{Region\;constraints} & \coloneqq & true 
      \ALT \rgn \outlives \rgn \ALT \rgn = \rgn \ALT \phi \conj \phi\\
e  & \in & \M{Expressions} & \coloneqq & \unitval \ALT x \ALT e.f 
     \ALT e.m\inang{\rbar}(\ebar) \ALT \C{new}\;\fbN(\ebar)
     \ALT \lambdaexp{\rgn}{\rhobar \,|\, \phi}
                    {\xbar:\taubar} {e}
           \ALT e\inang{\rbar}(\bar{e})\\
   & & & & \ALT \letexp{x}{e}{e} \ALT \letregion{\rgn}{e} 
           \ALT \open{e}{\rgn}{y}{e}\\
   & & & & \ALT \letd{\loc}{e} \ALT \opened{\loc}{\status}{e}\\
\end{array}
\end{smathpar}

\caption{\fbname: Syntax}
\label{fig:fb-syntax}
\end{figure*}

\renewcommand{\rgn}{\pi}
\renewcommand{\rbar}{\overline{\pi}}


The full syntax of \FB, including expressions that manifest only at
runtime, is shown in Fig.~\ref{fig:fb-syntax-full}. Such expressions
include:
\begin{itemize}

\item Memory locations ($\loc$) corresponding to the memory regions,

\item A special expression $\RgnZT{\toploc\loc}(e)$ that evaluates to
a region handler object (the syntax of this expression is captured by
the $\C{new}\;\fbN(\overline{e})$ construct). $\toploc$ is the
location of the special $\toprgn$ region where region handlers are
stored. $\loc$ is the location of the corresponding transferable
region.

\item A $\letd{\loc}{e}$ expression that results when
$\letregion{\pi}{e}$ expression takes a step (see
Fig.~\ref{fig:fb-opsem-2}).  $\loc$ is the location of the newly
allocated region.  

\item An $\opened{\loc}{s}{e}$ expression that
results when $\C{open}\;{\RgnZT{\toploc\loc}(v)} ...$ expression takes
a step. $\loc$ is the location of the newly open transferable region.
The symbol $s$ denotes the typestate of the
transferable region before it is opened\footnote{Operational semantics
lets a transferable region to be opened while it is already open. This
allows methods to safely open a transferable region argument
regardless of the calling context.}. The typestate can assume the
values of $\USED$ (allocated and closed), $\LIVE$ (allocated and
live), and $\XFERRED$ (transferred or freed).

\end{itemize}

\subsection{Full Static Semantics}

\begin{figure*}[t]
%
\begin{minipage}{2.25in}
\begin{smathpar}
\begin{array}{lcl}
  allocRgn(A\inang{\ralloc\rbar}\inang{\tbar}) & = & \ralloc\\
  allocRgn(\inang{\rhoalloc\rhobar \,|\, \phi}\bar{\tau^1}
      \xrightarrow{\ralloc} \tau^2) & = & \ralloc\\
  shape(A\inang{\rhoalloc\rhobar}\inang{\tbar}) & = & A\inang{\tbar}\\
  bound_{\aenv}(\tyvar@\ralloc) & = & \aenv(\tyvar)@\ralloc\\
  bound_{\aenv}(\fbN) & = & \fbN\\
\end{array}
\end{smathpar}
\end{minipage}
%
\begin{minipage}{1.8in}
\begin{smathpar}
\begin{array}{c}
\renewcommand*{\arraystretch}{1.2}
\RULE
  {
    \\
    B \in \{\ObjZ,\RgnZ\}
  }
  {
    fields(B\inang{\ralloc\rbar}\inang{\tbar}) \;=\; \bullet
  }
\end{array}
\end{smathpar}
\end{minipage}
%
\begin{minipage}{3in}
\begin{smathpar}
\begin{array}{c}
\renewcommand*{\arraystretch}{1.2}
\RULE
  {
    CT(B) = \headerOf{B}\{\bar{\tau^f}\;\bar{f};\,...\}\\
    \substFn = [\rbar/\rhobar, \ralloc/\rhoalloc, \tbar/\bar{\tyvar}] \qquad 
    fields(\substFn(\fbN)) = \bar{g}:\bar{\tau^g}
  }
  {
    fields(B\inang{\ralloc\rbar}\inang{\tbar}) \;=\;
      \bar{g}:\bar{\tau^g},\,\bar{f}:\substFn(\bar{\tau^f})
  }
\end{array}
\end{smathpar}
\end{minipage}
%
\bigskip

\begin{minipage}{3.5in}
\begin{smathpar}
\begin{array}{lcl}
  ctype(\ObjZ\inang{\rgn}) & = & \bullet \\
% ctype(\RgnZ\inang{\rgn}\inang{T}) & = & \inang{\rhoalloc}
%   {\unitZ}\rightarrow{T@\rhoalloc}\\
  ctype(B\inang{\ralloc\rbar}\inang{\tbar}) & = & 
    fields(B\inang{\ralloc\rbar}\inang{\tbar})\\
  mtype(\C{transfer}, \RgnZ\inang{\rgn}\inang{T}) & = & 
    \inang{\rhoalloc} {\unitZ}\rightarrow{\unitZ}\\
  mtype(\C{free}, \RgnZ\inang{\rgn}\inang{T}) & = & 
    \inang{\rhoalloc} {\unitZ}\rightarrow{\unitZ}\\
\end{array}
\end{smathpar}
\end{minipage}
%
\begin{minipage}{3in}
\begin{smathpar}
\begin{array}{c}
\renewcommand*{\arraystretch}{1.2}
\RULE
  {
    CT(B) = \headerOf{B}\{\bar{\tau^f}\;\bar{f};\,k\;\bar{d}\}\\
    m \notin \bar{d} \qquad 
    \substFn = [\rbar/\rhobar, \ralloc/\rhoalloc, \tbar/\bar{\tyvar}]
  }
  {
    mtype (m,B\inang{\ralloc\rbar}\inang{\tbar}) \;=\;
    mtype (m, \substFn(\fbN))
  }
\end{array}
\end{smathpar}
\end{minipage}
%
\bigskip

\begin{minipage}{3.25in}
\begin{smathpar}
\begin{array}{c}
\renewcommand*{\arraystretch}{1.2}
\RULE
  {
    CT(B) = \headerOf{B}\{\bar{\tau^f}\;\bar{f};\,k\;\bar{d}\}\\
    \tau^2 \; m\mang (\bar{\tau^1}\;\bar{x})\{...\} \in \bar{d} \qquad
    \substFn = [\rbar/\rhobar, \ralloc/\rhoalloc, \tbar/\bar{\tyvar}]
  }
  {
    mtype (m,B\inang{\ralloc\rbar}\inang{\tbar}) \;=\;
    \substFn(\mang\bar{\tau^1} \rightarrow \tau^2)
  }
\end{array}
\end{smathpar}
\end{minipage}
%
\begin{minipage}{3.5in}
\begin{smathpar}
\begin{array}{c}
\renewcommand*{\arraystretch}{1.2}
\RULE
  {

    \substFn = \subst{\bar{\rho_2}}{\bar{\rho_1}}
               \subst{\rhoalloc_2}{\rhoalloc_1} \spc
    mtype(m,\fbN) = \inang{\rhoalloc_1\bar{\rho_1},|\, \phi_1}\bar{\tau^{11}} 
                      \rightarrow \tau^{12} \spc \texttt{implies}\\
    \isvalid{\A.\phicx}{\phi_2 \Leftrightarrow \substFn(\phi_1)} 
        \spc \texttt{and} \spc
    \bar{\tau^{21}} = \substFn(\bar{\tau^{11}}) \spc \texttt{and} \spc
    \subtyp{\A}{\tau^{22}} {\substFn(\tau^{12})}
    %\substFn = [\rbar/\rhobar, \ralloc/\rhoalloc, \tbar/\bar{\tyvar}]
  }
  {
    override(\A,\fbN,\inang{\rhoalloc_2\bar{\rho_2},|\, \phi_1}
              \bar{\tau^{21}} \rightarrow \tau^{22})
  }
\end{array}
\end{smathpar}
\end{minipage}
%
\bigskip

\begin{minipage}{5in}
\begin{smathpar}
\begin{array}{c}
  \rhoset,\rhoenv \in 2^{\rho} \qquad
  \aenv \in \tyvar \rightarrow \fgjN \qquad
  \A = (\subtypcx)\\
\end{array}
\end{smathpar}
\end{minipage}
%

\caption{\fbname: Auxiliary Definitions}
\label{fig:fb-auxdef}
\end{figure*}

\begin{figure*}[!ht]
%
\textbf{Subtyping}  \; \fbox
  {\(\subtyp{\A}{\tau_1}{\tau_2}\)}\\
%
\begin{minipage}{1.8in}
\begin{smathpar}
\begin{array}{c}
\renewcommand*{\arraystretch}{1.2}
  \subtyp{\A}{\tau}{\tau} \\
  \subtyp{(\Delta,\aenv,\phicx)}{\tyvar @\rho}{\aenv(\tyvar) @\rho}\qquad
% \subtyp{\A}{\RgnZ\inang{\rgn}}{\RgnZ\inang{\toprgn}}\qquad
% \subtyp{\A}{\RgnZ\inang{\toprgn}}{\RgnZ\inang{\rgn}}
\end{array}
\end{smathpar}
\end{minipage}
%
\begin{minipage}{2.55in}
\begin{smathpar}
\begin{array}{c}
\renewcommand*{\arraystretch}{1.2}
\RULE
  {
    \\
    CT(B) = \headerOf{B}\{...\}
  }
  {
    \subtyp{\A}{B\inang{\tbar}\inang{\rbar}}
        {[\rbar/\rhobar, \tbar/\bar{\tyvar}](\fbN)}
  }
\end{array}
\end{smathpar}
\end{minipage}
%

\begin{minipage}{1.5in}
\begin{smathpar}
\begin{array}{c}
\renewcommand*{\arraystretch}{1.2}
\RULE
  {
    \subtyp{\A}{\tau_1}{\tau_2}\\
    \subtyp{\A}{\tau_2}{\tau_3}
  }
  {
    \subtyp{\A}{\tau_1}{\tau_3}
  }
\end{array}
\end{smathpar}
\end{minipage}
%
\begin{minipage}{2.75in}
\begin{smathpar}
\begin{array}{c}
\renewcommand*{\arraystretch}{1.2}
\RULE
  {
    \isvalid{\A.\phicx}{\phi_1 \Rightarrow \phi_2} \\
    \subtyp{\A}{\bar{\tau^{11}}}{\bar{\tau^{21}}} \spc
    \subtyp{\A}{\tau^{22}}{\tau^{12}}
  }
  {
    \subtyp{\A}
      {\inang{\rhobar \,|\, \phi_2}\bar{\tau^{21}}
          \xrightarrow{\rgn} \tau^{22}}
      {\inang{\rhobar \,|\, \phi_1}\bar{\tau^{11}}
          \xrightarrow{\rgn} \tau^{12}}
  }
\end{array}
\end{smathpar}
\end{minipage}


%
\bigskip

\textbf{Type, and Type Constraint Well-formedness}  \; \fbox
  {\(\tywf{\A}{\tau}, \spc 
     \tywf{\rhoenv}{\phi}\)}\\
%
\begin{minipage}{1.25in}
\begin{smathpar}
\begin{array}{c}
\renewcommand*{\arraystretch}{1.2}
\RULE
  {
    \\
    \\
    \rgn \in \rhoenv
  }
  {
    \tywf{(\rhoenv,\aenv,\phicx)}{\ObjZ\inang{\rgn}}
  }
\end{array}
\end{smathpar}
\end{minipage}
% 
\begin{minipage}{2.75in}
\begin{smathpar}
\begin{array}{c}
\renewcommand*{\arraystretch}{1.2}
\RULE
  {
    \rgn \in \rhoenv \spc
    \rhobar \not\in \rhoenv\\
%   \rhobar \notin dom(\mem) \\
%   \mem' = \mem[\rhobar \mapsto \overline{\LIVE}] \spc
    \rhoenv' = \rhoenv \cup \{\rhobar\} \spc
    \A' = (\rhoenv', \aenv, \phicx \conj \phi) \\
    \tywf{\rhoenv'}{\phi}\spc 
    \tywf{\A'}{\bar{\tau^1}} \spc
    \tywf{\A'}{\tau^2}
  }
  {
    \tywf{(\rhoenv,\aenv,\phicx)}{\inang{\rhobar \,|\, \phi}
              \bar{\tau^1} \xrightarrow{\rgn} \tau^2}
  }
\end{array}
\end{smathpar}
\end{minipage}
%
\begin{minipage}{1.5in}
\begin{smathpar}
\begin{array}{c}
\renewcommand*{\arraystretch}{1.2}
\RULE
  { 
    \\
    \\
    \fgjtywf{\A.\aenv}{T}
  }
  {
    \tywf{\A}{\RgnZ\inang{T}\inang{\toprgn}}
  }
\end{array}
\end{smathpar}
\end{minipage}
%
\begin{minipage}{1in}
\begin{smathpar}
\begin{array}{c}
\renewcommand*{\arraystretch}{1.2}
\RULE
  {
    \\
    \\
    \rgn_0,\rgn_1 \in \rhoenv
%   \mem(\rgn_0) = \LIVE \spc
%   \mem(\rgn_1) = \LIVE
  }
  {
    \tywf{\rhoenv}{\rgn_0 \outlives \rgn_1}
  }
\end{array}
\end{smathpar}
\end{minipage}
%

%
\begin{minipage}{3.5in}
\begin{smathpar}
\begin{array}{c}
\renewcommand*{\arraystretch}{1.2}
\RULE
  {
    CT(B) = \headerOf{B}\{...\}\\
    \rbar \in \rhoenv \spc
%   \mem(\rbar) = \overline{\LIVE} \spc
    \fgjtywf{\aenv}{B\inang{\tbar}}\spc
%   \substFn = [\rbar/\rhobar, \tbar/\bar{\tyvar}] \spc
    \isvalid{\phicx}{[\rbar/\rhobar](\phi)}
  }
  {
    \tywf{(\rhoenv,\aenv,\phicx)}{B\inang{\rbar}\inang{\tbar}}
  }
\end{array}
\end{smathpar}
\end{minipage}
%
\begin{minipage}{1.65in}
\begin{smathpar}
\begin{array}{c}
\renewcommand*{\arraystretch}{1.2}
\RULE
  {
    \fgjtywf{\aenv}{T}\spc
%   \rgn \in \A.\rhoenv\\
    \rgn \in \rhoenv\\
    \fgjsubtyp{\aenv}{T}{\ObjZ}\spc
  }
  {
    \tywf{(\rhoenv,\aenv,\phicx)}{T@\rgn}
  }
\end{array}
\end{smathpar}
\end{minipage}
%
\begin{minipage}{0.75in}
\begin{smathpar}
\begin{array}{c}
\renewcommand*{\arraystretch}{1.2}
\RULE
  {
    \tywf{\rhoenv}{\phi_0} \\ \tywf{\rhoenv}{\phi_1}
  }
  {
    \tywf{\rhoenv}{\phi_0 \wedge \phi_1}
  }
\end{array}
\end{smathpar}
\end{minipage}

%
\bigskip

/Users/gowtham/git/broom/fullversion/broom/paper/fb-morewfrules.tex
%

\caption{\fbname: Subtyping and well-formedness rules}
\label{fig:fb-staticsem-1}
\end{figure*}

\begin{figure*}[!ht]
\textbf{Expression Typing}  \; \fbox
  {\(\hastyp{\exptycx{\env}{\rgn}}{e}{\tau}\)}\\
%
\begin{minipage}{1.2in}
\begin{smathpar}
\begin{array}{l}
\renewcommand*{\arraystretch}{1.2}
\hastyp{\exptycx{\env}}{\unitval}{\unitZ}\\
\hastyp{\exptycx{\env}}{x}{\env(\tau)}
\end{array}
\end{smathpar}
\end{minipage}
%
\begin{minipage}{1.8in}
\begin{smathpar}
\begin{array}{c}
\renewcommand*{\arraystretch}{1.2}
\RULE
  {
    \hastyp{\exptycx{\env}}{e}{\tau'}\\
    \bar{f}:\taubar \,=\, \fields(\bound_{\A.\aenv}(\tau'))
  }
  {
    \hastyp{\exptycx{\env}}{e.f_i}{\tau_i}
  }
\end{array}
\end{smathpar}
\end{minipage}
%
\begin{minipage}{1.2in}
\begin{smathpar}
\begin{array}{c}
\renewcommand*{\arraystretch}{1.2}
\RULE
  {
    \hastyp{\exptycx{\env}}{\bar{e}}{\taubar} \\
    \tywf{\A}{\fbN} \spc
    \fields(\fbN) = \bar{f} : \taubar \\
  }
  {
    \hastyp{\exptycx{\env}}{\C{new}\; \fbN(\bar{e})}{\fbN}
  }
\end{array}
\end{smathpar}
\end{minipage}
%

\begin{minipage}{1.8in}
\begin{smathpar}
\begin{array}{c}
\renewcommand*{\arraystretch}{1.2}
\RULE
  {
    \hastyp{\exptycx{\env}}{e_1}{\tau_1}\\
    \hastyp{\exptycx{\env[x\mapsto\tau_1]}}{e_2}{\tau_2}\\
  }
  {
    \hastyp{\exptycx{\env}}{\letexp{x}{e_1}{e_2}}{\tau_2}
  }
\end{array}
\end{smathpar}
\end{minipage}
%
%
\begin{minipage}{3.2in}
\begin{smathpar}
\begin{array}{c}
\renewcommand*{\arraystretch}{1.2}
\RULE
  {
%   \A = (\mem,\aenv,\phicx)\spc
%   \fgjtywf{\aenv}{T} \\
%   \rgn \in \A.\rhoenv \spc
%   \mem(\rgn) = \LIVE \spc
%   \hastyp{(\rhoenv\cup\{\rho\},\aenv,\phicx),{\env}}{e}
%     {T@\rho}
%   We don't need the following condition because of the
%   guarantee provided by the lambda type rule.
%   \A = (\subtypcx) \spc
%   \rgn \in \rhoenv \\
    \\
    \hastyp{\exptycx{\env}}{e} {\inang{\rho} \unitZ \xrightarrow{\rgn} T@\rho} \spc
  }
  {
    \hastyp{\exptycx{\env}}{\C{new}\;
    \RgnZ\inang{T}\inang{\toprgn}(e)} {\RgnZ\inang{T}\inang{\toprgn}}
  }
\end{array}
\end{smathpar}
\end{minipage}

%
\begin{minipage}{5in}
\begin{smathpar}
\begin{array}{c}
\renewcommand*{\arraystretch}{1.2}
\RULE
  {
    \A = (\subtypcx)\spc
    \A' = (\rhoenv\cup\{\loc\}, \aenv,\phicx) \spc
%   \mem(\loc) = \USED \spc
    \tywf{\A'}{T@\loc} \spc
%   \fgjN = \RgnZ\inang{T}\spc
    \hastyp{\A', {\env}}{e}{T@\loc}
  }
  {
    \hastyp{\exptycx{\env}}{\C{new}\;
    \RgnZT{\toploc\loc}(e)} {\RgnZT{\toprgn}}
  }
\end{array}
\end{smathpar}
\end{minipage}
%

%
\begin{minipage}{2.5in}
\begin{smathpar}
\begin{array}{c}
\renewcommand*{\arraystretch}{1.2}
\RULE
  {
    \A = (\subtypcx) \spc
    \hastyp{\exptycx{\env}}{e_0}{\tau} \\
%   \rbar \in \A.\rhoenv \\
    \mtype(m,\bound_{\aenv}(\tau)) = \inang{\rhobar \,|\, 
        \phi}\bar{\tau^1}\rightarrow{\tau^2} \\
    \rbar \in \rhoenv \spc
    \tywf{\A}{\inang{\rhobar \,|\,
    \phi}\bar{\tau^1}\rightarrow{\tau^2}}\\
    \hastyp{\exptycx{\env}}{\bar{e}}{[\rbar/\rhobar](\bar{\tau^1})} \spc
    \isvalid{\phicx}{[\rbar/\rhobar](\phi)}
  }
  {
    \hastyp{\exptycx{\env}}{e_0.m\inang{\rbar}(\bar{e})} 
           {[\rbar/\rhobar](\tau^2)}
  }
\end{array}
\end{smathpar}
\end{minipage}
%
\begin{minipage}{2.5in}
\begin{smathpar}
\begin{array}{c}
\renewcommand*{\arraystretch}{1.2}
\RULE
  {
    \\
    \A = (\subtypcx) \spc
    \rgn \notin \rhoenv \spc
%   \Delta = \{\rgn \,|\, \rgn \in \rhoenv\} \spc
    \phi = \Delta \outlives \rgn \\
    \A' = (\rhoenv\cup\{\rgn\}, \aenv, \phicx \conj \phi)\\
    \hastyp{\A',\env}{e}{\tau} \spc
    \tywf{\A}{\tau}
  }
  {
    \hastyp{\exptycx{\env}}{\letregion{\rgn}{e}}{\tau}
  }
\end{array}
\end{smathpar}
\end{minipage}
%

%
\begin{minipage}{2.5in}
\begin{smathpar}
\begin{array}{c}
\renewcommand*{\arraystretch}{1.2}
\RULE
  {
    \A = (\subtypcx) \spc
%   \rgn \in \A.\rhoenv \spc
    \rgn \in \rhoenv \\
%   \rhobar \notin dom(\mem)\\
    \rhobar \notin \rhoenv \spc
%   \forall (\rgn'\in\Delta).~(\isvalid{\phicx}{\rgn\outlives\rgn'})

%           ~\Rightarrow~ \rgn = \rgn'\\
    \forall (\rgn'\in\frv(e)\setminus\{\rhobar\}).~\isvalid{\phicx}{\rgn'\outlives\rgn}\\
    \A' = (\rhoenv \cup \{\rhobar\}, \aenv, 
    \A' = (\rhoenv \cup \{\rhobar\}, \aenv, 
          \phicx \conj \phi)\\
    \tywf{\rhoenv \cup \{\rhobar\}}{\phi}\spc
    \tywf{\A'}{\bar{\tau^1}}\\
    \tywf{\A'}{\tau^2}\spc
    \hastyp{\A',\env[\xbar \mapsto \bar{\tau^1}]}{e}{\tau^2}
  }
  {
    \hastyp{\exptycx{\env}}
           {\lambdaexp{\rgn}{\rhobar \,|\, \phi}
                      {\xbar:\bar{\tau^1}}{e}}
           {\inang{\rhobar \,|\, \phi}
            \bar{\tau^1} \xrightarrow{\rgn} \tau^2}
  }
\end{array}
\end{smathpar}
\end{minipage}
%
\begin{minipage}{2.5in}
\begin{smathpar}
\begin{array}{c}
\renewcommand*{\arraystretch}{1.2}
\RULE
  {
    \A = (\subtypcx) \spc
    \rgn \notin \rhoenv \\
    \hastyp{\exptycx{\env}}{e_a}
            {\RgnZ\inang{T}\inang{\toprgn}}\\
%   \rgn \notin dom(\mem) \spc
    \A' = (\rhoenv\cup\{\rgn\},\aenv,\phicx) \spc
    \tywf{\A}{\tau} \\
    \env' =  \env[y\mapsto T@\rgn]\spc
    \hastyp{\A',\env'}{e_b}{\tau} \spc
  }
  {
    \hastyp{\exptycx{\env}}{\open{e_a}{\rgn}{y}{e_b}}
            {\tau}
  }
\end{array}
\end{smathpar}
\end{minipage}
%

%
\begin{minipage}{2.5in}
\begin{smathpar}
\begin{array}{c}
\renewcommand*{\arraystretch}{1.2}
\RULE
  {
    \A = (\subtypcx) \spc
    \rbar \in \rhoenv \\
    \hastyp{\exptycx{\env}}{e}
        {\inang{\rhobar \,|\, \phi}
            \bar{\tau^1} \xrightarrow{\rgn} \tau^2}\\
%   \substFn = \subst{\rbar}{\rhobar} \spc
    \isvalid{\phicx}{\subst{\rbar}{\rhobar}\phi} \spc
    \hastyp{\exptycx{\env}}{\bar{e}}
        {\subst{\rbar}{\rhobar}\bar{\tau^1}}\spc
%   \subtyp{\A}{\taubar}{\bar{\tau^1}}
  }
  {
    \hastyp{\A,\env}{e\inang{\rbar}(\bar{e})}
           {\subst{\rbar}{\rhobar}\tau^2}
  }
\end{array}
\end{smathpar}
\end{minipage}
%
\begin{minipage}{2.5in}
\begin{smathpar}
\begin{array}{c}
\renewcommand*{\arraystretch}{1.2}
\RULE
  {
%   \A = (\subtypcx)\\
%   \loc \in \rhoenv \spc
%   \hastyp{\exptycx{\env}}{e}{\tau}
    \A = (\subtypcx) \spc
    \loc \notin \rhoenv \\
    \A' = (\rhoenv\cup\{\loc\}, \aenv, \phicx \conj \Delta \outlives
    \loc)\\
    \hastyp{\A',\env}{e}{\tau} \spc
    \tywf{\A}{\tau}
  }
  {
    \hastyp{\exptycx{\env}}{\letd{\loc}{e}}{\tau}
  }
\end{array}
\end{smathpar}
\end{minipage}
%

%
\begin{minipage}{1.95in}
\begin{smathpar}
\begin{array}{c}
\renewcommand*{\arraystretch}{1.2}
\RULE
  {
%   \A = (\subtypcx)\\
%   \loc \in \rhoenv \spc
%   \hastyp{\exptycx{\env}}{e}{\tau}
    \A = (\subtypcx) \spc
    \loc \notin \rhoenv \\
%   \loc \notin dom(\mem) \spc
    \A' = (\rhoenv\cup\{\loc\},\aenv,\phicx) \\
    \hastyp{\A',\env}{e}{\tau} \spc
    \tywf{\A}{\tau}
  }
  {
    \hastyp{\exptycx{\env}}{\opened{\loc}{s}{e}}{\tau}
  }
\end{array}
\end{smathpar}
\end{minipage}
%
%
\begin{minipage}{1.6in}
\begin{smathpar}
\begin{array}{c}
\renewcommand*{\arraystretch}{1.2}
\RULE
  {
    \\
    \\
    \hastyp{\exptycx{\env}}{e}{\RgnZ\inang{T}\inang{\toprgn}}
  }
  {
    \hastyp{\exptycx{\env}}{e.\transfer()}{\unitZ}
  }
\end{array}
\end{smathpar}
\end{minipage}
%
\begin{minipage}{1in}
\begin{smathpar}
\begin{array}{c}
\renewcommand*{\arraystretch}{1.2}
\RULE
  {
    \\
    \hastyp{\exptycx{\env}}{e}{\tau} \\
    \subtyp{\A}{\tau}{\tau'}
  }
  {
    \hastyp{\exptycx{\env}}{e}{\tau'}
  }
\end{array}
\end{smathpar}
\end{minipage}
%

\caption{\fbname: Expression typing}
\label{fig:fb-staticsem-2}
\end{figure*}


Figs.~\ref{fig:fb-staticsem-1} and~\ref{fig:fb-staticsem-2} show full
static semantics of \FB. Fig.~\ref{fig:fb-staticsem-1} contains subtyping
rules, and rules to check well-formedness of \FB types, type
constraints, methods, and class definitions. Method well-formedness
rule makes use of the expression typing judgment defined in
Fig.~\ref{fig:fb-staticsem-2}. Auxiliary definitions used in
Figs.~\ref{fig:fb-staticsem-1} and~\ref{fig:fb-staticsem-2} are
defined in Fig.~\ref{fig:fb-auxdef}. As described in
\S~\ref{sec:type-system}, all judgments are parameterized over the
class table ($CT$). The premise $CT(B) = \headerOf{B}\{...\}$
used in some of the rules means that the definition of class $B$ is
present in $CT$ and that the definition is well-formed.

\subsection{Operational Semantics}

% \newcommand{\opsemrule}[3]{%
% \begin{minipage}{#1}\begin{smathpar}\begin{array}{c}%
% \renewcommand*{\arraystretch}{1.2}%
% \RULE {#2} {#3}%
% \end{array}\end{smathpar}\end{minipage}%
% }
% \newcommand{\lopsemrule}[4]{%
% \begin{minipage}{#1}\begin{smathpar}\begin{array}{c}%
% \renewcommand*{\arraystretch}{1.2}%
% [\rulelabel{#4}] \spc \RULE {#2} {#3}%
% \end{array}\end{smathpar}\end{minipage}%
% }
% %
% \newcommand{\anobjty}[0]{B\inang{\tbar}\inang{\ralloc\locbar}}
% \newcommand{\anobj}[0]{\C{new} \; \anobjty(\bar{v})}
% %

\begin{figure*}[t!]

%
\fbox {\(\redstoo{\Delta}{(e,\mem)}{(e',\mem')}\)}\\

\lopsemrule{1.5in}{
    \redstoo{\Delta}{(e,\mem)}{(e',\mem')}
}{
    \redstoo{\Delta}{(E\lbrack e \rbrack, \mem)}
            {(E\lbrack e' \rbrack, \mem')}
	  }{EvalOrder}

\lopsemrule{1.2in}{
    \redstoo{\Delta}{(e,\mem)}{\invalidexn}
  }{
    \redstoo{\Delta}{(E\lbrack e \rbrack, \mem)}
            {\invalidexn}
	  }{Exception}
%

\lopsemrule{2.7in}{
%   \allocRgn(A\inang{\rgn\locbar}\inang{\tbar}) & = & \rgn
%   \allocRgn(\fbN) \in \rhoenv \spc
    \mem(\loc) = \LIVE \spc
    \fields(A\inang{\tbar}\inang{\loc\locbar}) = \bar{f}:\taubar
}{
    \redstoo{\Delta}{((\C{new} \; A\inang{\tbar}\inang{\loc\locbar}(\vbar)).f_i,\mem)}{(v_i,\mem)}
}{FieldAccess}

\lopsemrule{3.85in}{
%   \allocRgn(\fbN) \in \rhoenv \spc
    \mem(\loc) = \LIVE \spc
    \mbody(m,A\inang{\tbar}\inang{\loc\locbar}) = \rho\rhobar.\bar{x}.e 
%   \redstoo{\Delta}{(, \mem)}{(e',\mem')}
}{
    \redstoo{\Delta}{((\C{new}\;A\inang{\tbar}\inang{\loc\locbar}(\bar{v})).m\inang{\loc'\overline{\loc'}}
                      (\bar{v'}),\mem)}
            {([\loc'\overline{\loc'}/\rho\rhobar][\bar{v'}/\xbar]
                [\C{new} \; A\inang{\tbar}\inang{\loc\locbar}(\bar{v})/\thisZ]\,e,\mem)}
	  }{MethodInv}

\lopsemrule{2.85in}{
    v_a = \lambdaexp{\loc'}{\rho\rhobar}
                        {\taubar \; \xbar}{e} \spc
    \mem(\loc') = \LIVE  \spc 
}{
    \redstoo{\Delta}{(v_a\inang{\loc\locbar}(\bar{v}) ,\mem)}
            {([\bar{v}/\xbar][\loc\locbar/\rho\rhobar]\,e,\mem)}
	  }{FnApply}

%
% \lopsemrule{2in}{
% %   \rgn \notin \rhoenv \spc
% %   \fresh(\rgn') \spc
%     \redstoo{\Delta}{(e,\mem)}{\invalidexn}
% }{
%     \redstoo{\Delta}{(\letregion{\rgn}{e},\mem)}{\invalidexn}
%   }{EXCEPTION}

\lopsemrule{2in}{
%   \fgjN = \RgnZ\inang{T} \spc
%   \rgn \in dom(\mem) \spc
    \mem(\loc) = \USED \spc
    \redstocup{\loc}{(e,\mem[\loc \mapsto \LIVE])}{\invalidexn}
}{
    \redstoo{\Delta}{(\C{new} \; \RgnZT{\toploc\loc} (e),\mem)}
            {\invalidexn}
	  }{Exception}

\lopsemrule{2in}{
%   \fgjN = \RgnZ\inang{T} \spc
%   \rgn \in dom(\mem) \spc
}{
    \redstoo{\Delta}{(\letexp{x}{v}{e},\mem)}
            {([v/x]e,\mem)}
	  }{LetExp}

% \lopsemrule{3.2in}{
%     \not\exists \loc.~\mem(\loc) = \FREE
% }{
%     \redstoo{\Delta}{(\letregion{\rgn}{e},\mem)}{\invalidexn}
%   }{EXCEPTION}


% \lopsemrule{3.3in}{
%     \not\exists \loc.~\mem(\loc) = \FREE
% }{
%     \redstoo{\Delta}{(\C{new} \; \RgnZ\inang{T}\inang{\toprgn}
%                 (\lambdaexp{\loc'}{\rho}{}{e}),\mem)}
%             {\invalidexn}
%         }{EXCEPTION}
% \lopsemrule{2.5in}{
%     \mem(\rgn_r) \neq \XFERRED \spc
% %   \fresh(\rgn_1) \spc
% %  \rgn_0 \notin \rhoenv \\
%     \redsto{\Delta \cup \{\rgn\}}{([[\rgn/\rgn_r]v_r/x]e_b,
%       \mem[\rgn_r \mapsto \OPEN])}{\invalidexn}
% }{
%     \redstoo{\Delta}{(\open{(\C{new} \; \RgnZ\inang{T}\inang{\rgn_r}(v_r))}
%                    {\rgn}{x}{e_b},\mem)} 
%             {\invalidexn}
%     }{EXCEPTION}

%
\bigskip

\textbf{Evaluation Context} \fbox {\(E\)}\\
\begin{smathpar}
\begin{array}{lcl}
E & \coloneqq & \bullet \ALT (\bullet).f \ALT \bullet.m\inang{\locbar}(\ebar) \ALT
      v.m\inang{\locbar}(...,\bullet,...) \ALT \C{new}\; \fbN(...,\bullet,...) \ALT
      \C{new} \; \RgnZ\inang{T}\inang{\toprgn}(\bullet) \ALT 
      \bullet\inang{\locbar}(\ebar) \\
  &  & \ALT v\inang{\locbar}(...,\bullet,...) \ALT
       \letexp{x}{\bullet}{e} \ALT \open{\bullet}{\rgn}{y}{e} 
%      \ALT \opened{\loc}{\status}{\bullet} \ALT \letd{\loc}{\bullet}
%     The following should be forbidden. See NEW-REGION rule 2.
%      \ALT \C{new} \; \RgnZ\inang{T}\inang{\toploc\loc}(\bullet)
\end{array}
\end{smathpar}

\caption{\fbname: operational semantics (part 1)}
\label{fig:fb-opsem-1}
\end{figure*}

\begin{figure*}[t!]

\lopsemrule{3.2in}{
%   \fresh(\rgn')\spc
%   \mem(\loc) = \FREE \spc
    \loc \not\in dom(\mem) \spc
    \mem' = \mem[\loc \mapsto \SLIVE]
%   \redstoo{\Delta}{(e,\mem)}{(e',\mem')}
}{
    \redstoo{\Delta}{(\letregion{\rgn}{e},\mem)}{(\letd{\loc}{[\loc/\rgn]e},\mem')}
  }{LetRegionBegin}

\lopsemrule{3in}{
    \redstocup{\loc}{(e,\mem)}{(e',\mem')}
}{
    \redstoo{\Delta}{(\letd{\loc}{e},\mem)}{(\letd{\loc}{e'},\mem')}
  }{LetRegion}

\lopsemrule{3in}{
%   \rgn \notin \rhoenv
%   \mem' = \mem[\loc \mapsto \FREE]
    \mem' = \mem[\loc \mapsto \XFERRED]
}{
    \redstoo{\Delta}{(\letd{\loc}{v},\mem)}{(v,\mem')}
  }{LetRegionEnd}

\lopsemrule{3.3in}{
%   \fgjN = \RgnZ\inang{T} \spc
%   \rgn \in \rhoenv \spc
%   We need Delta here to ensure Delta U {pi_r} in next rule is sound. 
%   \rgn_r \notin \Delta \cup dom(\Sigma) \spc
%   \rgn \notin dom(\mem) \cup \rhoenv \spc
%   \mem' = \mem[\rgn_r \mapsto \CLOSED]
%   \mem(\loc) = \LIVE \spc
%   \mem(\loc') = \FREE \spc
    \loc \not\in dom(\mem) \spc
    \mem' = \mem[\loc \mapsto \USED]
}{
    \redstoo{\Delta}{(\C{new} \; \RgnZ\inang{T}\inang{\toprgn}
                (v),\mem)}
            {(\C{new} \; \RgnZ\inang{T}\inang{\toploc\loc}
                (v\inang{\loc}()),\mem')}
	      }{NewRegion}

\lopsemrule{3.3in}{
%   \mem(\loc) = \USED \spc % Type system cannot guarantee this.
    \redstocup{\loc}{(e,\mem[\loc \mapsto \LIVE]) }{(e',\mem')}
}{
    \redstoo{\Delta}{(\C{new} \; \RgnZ\inang{T}\inang{\toploc\loc}
                (e),\mem)}
            {(\C{new} \; \RgnZ\inang{T}\inang{\toploc\loc}
                (e'),\mem')} %[\loc\mapsto\USED])}
	      }{NewRegion}

% \lopsemrule{2.8in}{
%     \redsto{\Delta \cup \{\rgn_r\}}{(e,\mem)}{(e',\mem')} \spc
% %   \fgjN = \RgnZ\inang{T} \spc
%     \rgn_r \in dom(\mem)
% }{
%     \redstoo{\Delta}{(\C{new} \; \RgnZ\inang{T}\inang{\rgn_r}
%                 (e),\mem)}
%             {(\C{new} \; \RgnZ\inang{T}\inang{\rgn_r}
%                 (e'),\mem')}
%         }{NEW-REGION}

\lopsemrule{4in}{
    v_a = \C{new} \; \RgnZ\inang{T}\inang{\toploc\loc}(v) \spc
    \mem(\loc) = \USED ~\texttt{or}~ \mem(\loc) = \TLIVE \spc
%   \fresh(\rgn_1)\\
%   \rgn_0 \notin \rhoenv \\
    \mem' = \mem[\loc \mapsto \TLIVE]
%   \mem'' = \mem'[\rgn_r \mapsto \mem(\rgn_r)]
}{
    \redstoo{\Delta}{(\open{v_a}{\rgn}{x}{e_b},\mem)} 
            {(\opened{\loc}{\mem(\loc)}{[v/x][\loc/\rgn]e_b},\mem')}
	  }{Open}

\lopsemrule{2.7in}{
    \redstocup{\loc}{(e,\mem)}{(e',\mem')}
}{
    \redstoo{\Delta}{(\opened{\loc}{\status}{e},\mem)}
    {(\opened{\loc}{\status}{e'},\mem')}
  }{Opened}

\lopsemrule{2.7in}{
%   v_a = \C{new} \; \RgnZ\inang{T}\inang{\}(v_r) \spc
%   \mem(\rgn_r) \neq \XFERRED \spc
%   \rgn_0 \notin \rhoenv \spc
    \mem' = \mem[\loc \mapsto \status]
}{
    \redstoo{\Delta}{(\opened{\loc}{\status}{v},\mem)} {(v,\mem')}
  }{OpenEnd}


\lopsemrule{3in}{
    v_a = \C{new} \; \RgnZ\inang{T}\inang{\toploc\loc}(v) \spc
    \mem(\loc) \neq \USED ~\texttt{and}~ \mem(\loc) \neq \LIVE
%   \rgn_0 \notin \rhoenv \\
%   \redstocup{\rgn_0}{([[\rgn_0/\rgn]v_r/x]e_b,
%     \mem[\rgn \mapsto \OPEN])}{(e_b',\mem')}
}{
    \redstoo{\Delta}{(\open{v_a}{\rgn}{x}{e_b},\mem)} 
            {\invalidexn}
	  }{OpenTransferred}


\lopsemrule{3.25in}{
%   \ralloc \in \rhoenv \spc
%   \fbN = \RgnZ\inang{T}\inang{\rgn_r}\spc
%   \redstoo{\Delta}{(, \mem)}{(e',\mem')}
    \mem(\loc) = \USED \spc
    \mem' = \mem[\loc \mapsto \XFERRED]
}{
    \redstoo{\Delta}{((\C{new}\;\RgnZ\inang{T}\inang{\toploc\loc}(v)).\transfer(),\mem)}
            {(\unitval,\mem')}
	  }{Transfer}

\lopsemrule{2.5in}{
%   \fbN = \RgnZ\inang{T}\inang{\rgn}\spc
%   \ralloc \in \rhoenv \spc
%   \redstoo{\Delta}{(, \mem)}{(e',\mem')}
    \mem(\loc) = \TLIVE \spc
}{
    \redstoo{\Delta}{((\C{new}\;\RgnZ\inang{T}\inang{\toploc\loc}(v)).\transfer(),\mem)}
            {\invalidexn}
	  }{TransferOpened}

% \lopsemrule{2.5in}{
%     \not\exists \loc.~\mem(\loc) = \SLIVE \spc
%     \mem(\loc') = \XFERRED
% }{
%     \redstoo{\Delta}{(e,\mem)}{(e,\mem[\loc'\mapsto\FREE])}
%     }{GARBAGE-COLLECT}


\caption{\fbname: operational semantics (part 2)}
\label{fig:fb-opsem-2}
\end{figure*}

Figs.~\ref{fig:fb-opsem-1} and~\ref{fig:fb-opsem-2} show the
operational semantics of \fbname. The semantics defines a five-place
reduction relation:
\begin{smathpar}
  \redstoo{\Delta}{(e,\mem)}{(e',\mem')}
\end{smathpar}
Where $\Delta$ is the set of currently-live region locations, and
$\mem$ is a map from memory locations ($\loc$) to type states ($s$).
Evaluating a $\C{letregion}$ or a $\C{new}\;\C{Region}$ expression
results in the addition of a new binding to $\mem$. For a new static
region, the memory location ($\loc$) is mapped to $\LIVE$ (live), and
for a new transferable region, it is mapped to $\USED$. The binding is
updated when the transferable region is opened, closed, or
transferred. Likewise, when a $\C{letregion}$ expression is evaluated
to a value, the binding for the corresponding location is set to
$\XFERRED$, effectively deallocating the region. Opening an already
transferred region, or transferring a currently-open region results in
an exception ($\bot$). The semantics gets stuck while evaluating a
field access, a method call, or a lambda application, if the region
containing the target object is not live. The type safety result
establishes that this can never happen while evaluating a well-typed
expression. The set ($\Delta$) of live region is the same set used by
the expression typing judgment (Fig.~\ref{fig:fb-staticsem-2}). 
