%\section{Operational Semantics}
%
%Fig.~\ref{fig:fb-opsem} contains a definition of the operational semantics for \fbname.
%
\section{\fbname}

\subsection{Full Syntax}
\input{fb-syntax-full}

The full syntax of \FB, including expressions that manifest only at
runtime, is shown in Fig.~\ref{fig:fb-syntax-full}. Such expressions
include:
\begin{itemize}

\item Memory locations ($\loc$) corresponding to the memory regions,

\item A special expression $\RgnZT{\toploc\loc}(e)$ that evaluates to
a region handler object (the syntax of this expression is captured by
the $\C{new}\;\fbN(\overline{e})$ construct). $\toploc$ is the
location of the special $\toprgn$ region where region handlers are
stored. $\loc$ is the location of the corresponding transferable
region.

\item A $\letd{\loc}{e}$ expression that results when
$\letregion{\pi}{e}$ expression takes a step (see
Fig.~\ref{fig:fb-opsem-2}).  $\loc$ is the location of the newly
allocated region.  

\item An $\opened{\loc}{s}{e}$ expression that
results when $\C{open}\;{\RgnZT{\toploc\loc}(v)} ...$ expression takes
a step. $\loc$ is the location of the newly open transferable region.
The symbol $s$ denotes the typestate of the
transferable region before it is opened\footnote{Operational semantics
lets a transferable region to be opened while it is already open. This
allows methods to safely open a transferable region argument
regardless of the calling context.}. The typestate can assume the
values of $\USED$ (allocated and closed), $\LIVE$ (allocated and
live), and $\XFERRED$ (transferred or freed).

\end{itemize}

\subsection{Full Static Semantics}

\begin{figure*}[t]
%
\begin{minipage}{2.25in}
\begin{smathpar}
\begin{array}{lcl}
  allocRgn(A\inang{\ralloc\rbar}\inang{\tbar}) & = & \ralloc\\
  allocRgn(\inang{\rhoalloc\rhobar \,|\, \phi}\bar{\tau^1}
      \xrightarrow{\ralloc} \tau^2) & = & \ralloc\\
  shape(A\inang{\rhoalloc\rhobar}\inang{\tbar}) & = & A\inang{\tbar}\\
  bound_{\aenv}(\tyvar@\rgn) & = & \aenv(\tyvar)@\rgn\\
  bound_{\aenv}(\fbN) & = & \fbN\\
\end{array}
\end{smathpar}
\end{minipage}
%
\begin{minipage}{1.8in}
\begin{smathpar}
\begin{array}{c}
\renewcommand*{\arraystretch}{1.2}
\RULE
  {
    \\
    B \in \{\ObjZ,\RgnZ\}
  }
  {
    fields(B\inang{\ralloc\rbar}\inang{\tbar}) \;=\; \bullet
  }
\end{array}
\end{smathpar}
\end{minipage}
%
\begin{minipage}{3in}
\begin{smathpar}
\begin{array}{c}
\renewcommand*{\arraystretch}{1.2}
\RULE
  {
    CT(B) = \headerOf{B}\{\bar{\tau^f}\;\bar{f};\,...\}\\
    \substFn = [\rbar/\rhobar, \ralloc/\rhoalloc, \tbar/\bar{\tyvar}] \qquad 
    fields(\substFn(\fbN)) = \bar{g}:\bar{\tau^g}
  }
  {
    fields(B\inang{\ralloc\rbar}\inang{\tbar}) \;=\;
      \bar{g}:\bar{\tau^g},\,\bar{f}:\substFn(\bar{\tau^f})
  }
\end{array}
\end{smathpar}
\end{minipage}
%
\bigskip

\begin{minipage}{3.5in}
\begin{smathpar}
\begin{array}{lcl}
  ctype(\ObjZ\inang{\rgn}) & = & \bullet \\
% ctype(\RgnZ\inang{\rgn}\inang{T}) & = & \inang{\rhoalloc}
%   {\unitZ}\rightarrow{T@\rhoalloc}\\
  ctype(B\inang{\ralloc\rbar}\inang{\tbar}) & = & 
    fields(B\inang{\ralloc\rbar}\inang{\tbar})\\
  mtype(\C{transfer}, \exists\rho.\RgnZ\inang{\rho}\inang{T}) & = & 
    \inang{\rhoalloc} {\unitZ}\rightarrow{\unitZ}\\
  mtype(\C{free}, \exists\rho.\RgnZ\inang{\rho}\inang{T}) & = & 
    \inang{\rhoalloc} {\unitZ}\rightarrow{\unitZ}\\
\end{array}
\end{smathpar}
\end{minipage}
%
\begin{minipage}{3in}
\begin{smathpar}
\begin{array}{c}
\renewcommand*{\arraystretch}{1.2}
\RULE
  {
    CT(B) = \headerOf{B}\{\bar{\tau^f}\;\bar{f};\,k\;\bar{d}\}\\
    m \notin \bar{d} \qquad 
    \substFn = [\rbar/\rhobar, \ralloc/\rhoalloc, \tbar/\bar{\tyvar}]
  }
  {
    mtype (m,B\inang{\ralloc\rbar}\inang{\tbar}) \;=\;
    mtype (m, \substFn(\fbN))
  }
\end{array}
\end{smathpar}
\end{minipage}
%
\bigskip

\begin{minipage}{3.25in}
\begin{smathpar}
\begin{array}{c}
\renewcommand*{\arraystretch}{1.2}
\RULE
  {
    CT(B) = \headerOf{B}\{\bar{\tau^f}\;\bar{f};\,k\;\bar{d}\}\\
    \tau^2 \; m\mang (\bar{\tau^1}\;\bar{x})\{...\} \in \bar{d} \qquad
    \substFn = [\rbar/\rhobar, \ralloc/\rhoalloc, \tbar/\bar{\tyvar}]
  }
  {
    mtype (m,B\inang{\ralloc\rbar}\inang{\tbar}) \;=\;
    \substFn(\mang\bar{\tau^1} \rightarrow \tau^2)
  }
\end{array}
\end{smathpar}
\end{minipage}
%
\begin{minipage}{3.5in}
\begin{smathpar}
\begin{array}{c}
\renewcommand*{\arraystretch}{1.2}
\RULE
  {

    \substFn = \subst{\bar{\rho_2}}{\bar{\rho_1}}
               \subst{\rhoalloc_2}{\rhoalloc_1} \spc
    mtype(m,\fbN) = \inang{\rhoalloc_1\bar{\rho_1},|\, \phi_1}\bar{\tau^{11}} 
                      \rightarrow \tau^{12} \spc \texttt{implies}\\
    \isvalid{\A.\phicx}{\phi_2 \Leftrightarrow \substFn(\phi_1)} 
        \spc \texttt{and} \spc
    \bar{\tau^{21}} = \substFn(\bar{\tau^{11}}) \spc \texttt{and} \spc
    \subtyp{\A}{\tau^{22}} {\substFn(\tau^{12})}
    %\substFn = [\rbar/\rhobar, \ralloc/\rhoalloc, \tbar/\bar{\tyvar}]
  }
  {
    override(\A,\fbN,\inang{\rhoalloc_2\bar{\rho_2},|\, \phi_1}
              \bar{\tau^{21}} \rightarrow \tau^{22})
  }
\end{array}
\end{smathpar}
\end{minipage}
%
\bigskip

\begin{minipage}{5in}
\begin{smathpar}
\begin{array}{c}
  \rhoset,\rhoenv \in 2^{\rho} \qquad
  \aenv \in \tyvar \rightarrow \fgjN \qquad
  \A = (\subtypcx)\\
\end{array}
\end{smathpar}
\end{minipage}
%

\caption{\fbname: Auxiliary Definitions}
\label{fig:fb-auxdef}
\end{figure*}

\renewcommand{\rgn}{r}
\renewcommand{\rbar}{\overline{r}}

\begin{figure*}[!h]
%
\textbf{Subtyping}  \; \fbox
  {\(\subtyp{\A}{\tau_1}{\tau_2}\)}\\
%
\begin{minipage}{1.8in}
\begin{smathpar}
\begin{array}{c}
\renewcommand*{\arraystretch}{1.2}
  \subtyp{\A}{\tau}{\tau} \\
  \subtyp{(\Delta,\aenv,\phicx)}{\tyvar @\rho}{\aenv(\tyvar) @\rho}\qquad
% \subtyp{\A}{\RgnZ\inang{\rgn}}{\RgnZ\inang{\toprgn}}\qquad
% \subtyp{\A}{\RgnZ\inang{\toprgn}}{\RgnZ\inang{\rgn}}
\end{array}
\end{smathpar}
\end{minipage}
%
\begin{minipage}{2.7in}
\begin{smathpar}
\begin{array}{c}
\renewcommand*{\arraystretch}{1.2}
\RULE
  {
    \\
    CT(B) = \headerOf{B}\{...\}
  }
  {
    \subtyp{\A}{B\inang{\tbar}\inang{\rbar}}
        {[\rbar/\rhobar, \tbar/\bar{\tyvar}](\fbN)}
  }
\end{array}
\end{smathpar}
\end{minipage}
%

\begin{minipage}{1.5in}
\begin{smathpar}
\begin{array}{c}
\renewcommand*{\arraystretch}{1.2}
\RULE
  {
    \subtyp{\A}{\tau_1}{\tau_2}\\
    \subtyp{\A}{\tau_2}{\tau_3}
  }
  {
    \subtyp{\A}{\tau_1}{\tau_3}
  }
\end{array}
\end{smathpar}
\end{minipage}
%
\begin{minipage}{2.75in}
\begin{smathpar}
\begin{array}{c}
\renewcommand*{\arraystretch}{1.2}
\RULE
  {
    \isvalid{\A.\phicx}{\phi_1 \Rightarrow \phi_2} \\
    \subtyp{\A}{\bar{\tau^{11}}}{\bar{\tau^{21}}} \spc
    \subtyp{\A}{\tau^{22}}{\tau^{12}}
  }
  {
    \subtyp{\A}
      {\inang{\rhobar \,|\, \phi_2}\bar{\tau^{21}}
          \xrightarrow{\rgn} \tau^{22}}
      {\inang{\rhobar \,|\, \phi_1}\bar{\tau^{11}}
          \xrightarrow{\rgn} \tau^{12}}
  }
\end{array}
\end{smathpar}
\end{minipage}


%
\bigskip

\textbf{Type, and Type Constraint Well-formedness}  \; \fbox
  {\(\tywf{\A}{\tau}, \spc 
     \tywf{\mem}{\phi}\)}\\
/Users/gowtham/git/broom/fullversion/broom/paper/fb-tywfrules-full.tex
%
\bigskip

\textbf{Expression Typing}  \; \fbox
  {\(\hastyp{\exptycx{\env}}{e}{\tau}\)}\\
/Users/gowtham/git/broom/fullversion/broom/paper/fb-exptyprules-full.tex
%
\bigskip

\caption{\fbname: Static Semantics}
\label{fig:fb-staticsem}
\end{figure*}

\renewcommand{\rgn}{\pi}
\renewcommand{\rbar}{\overline{\pi}}


Figs.~\ref{fig:fb-staticsem-1} and~\ref{fig:fb-staticsem-2} show full
static semantics of \FB. Fig.~\ref{fig:fb-staticsem-1} contains subtyping
rules, and rules to check well-formedness of \FB types, type
constraints, methods, and class definitions. Method well-formedness
rule makes use of the expression typing judgment defined in
Fig.~\ref{fig:fb-staticsem-2}. Auxiliary definitions used in
Figs.~\ref{fig:fb-staticsem-1} and~\ref{fig:fb-staticsem-2} are
defined in Fig.~\ref{fig:fb-auxdef}. As described in
\S~\ref{sec:type-system}, all judgments are parameterized over the
class table ($CT$). The premise $CT(B) = \headerOf{B}\{...\}$
used in some of the rules means that the definition of class $B$ is
present in $CT$ and that the definition is well-formed.

\subsection{Operational Semantics}

\newcommand{\redstoo}[2]{\redsto{\rhoenv}{#1}{#2}}
\newcommand{\redstocup}[3]{\redsto{\rhoenv \cup \{#1\}}{#2}{#3}}
\newcommand{\anobjty}[0]{B\inang{\tbar}\inang{\ralloc\rbar}}
\newcommand{\anobj}[0]{\C{new} \; \anobjty(\bar{v})}
\begin{figure*}[t!]

%
\fbox {\(\redstoo{(e,\rhomap)}{(e',\rhomap')}\)}\\

%
\begin{minipage}{2.7in}
\begin{smathpar}
\begin{array}{c}
\renewcommand*{\arraystretch}{1.2}
\RULE
  {
    \allocRgn(\fbN) \in \rhoenv \spc
    \fields(\fbN) = \taubar\;\bar{f}
  }
  {
    \redstoo{((\C{new} \; \fbN(\vbar)).f_i,\rhomap)}{(v_i,\rhomap)}
  }
\end{array}
\end{smathpar}
\end{minipage}
%
% \begin{minipage}{3in}
% \begin{smathpar}
% \begin{array}{c}
% \renewcommand*{\arraystretch}{1.2}
% \RULE
%   {
%     \ralloc \in \rhoenv \spc
%     \redstoo{(e_i,\rhomap)}{(e_i',\rhomap')}
%   }
%   {
%     \redstoo{(\C{new} \; \fbN(...,e_i,...),\rhomap)}
%             {(\C{new} \; \fbN(...,e_i',...),\rhomap')}
%   }
% \end{array}
% \end{smathpar}
% \end{minipage}
%
\begin{minipage}{3.2in}
\begin{smathpar}
\begin{array}{c}
\renewcommand*{\arraystretch}{1.2}
\RULE
  {
    \rgn \notin \rhoenv \spc
    \redsto{\rhoenv \cup \{\rgn\}}{(e,\rhomap)}{(e',\rhomap')}
  }
  {
    \redstoo{(\letregion{\rgn}{e},\rhomap)}{(\letregion{\rgn}{e'},\rhomap')}
  }
\end{array}
\end{smathpar}
\end{minipage}
%

%
\begin{minipage}{3in}
\begin{smathpar}
\begin{array}{c}
\renewcommand*{\arraystretch}{1.2}
\RULE
  {
    \rgn \notin \rhoenv
  }
  {
    \redstoo{(\letregion{\rgn}{v},\rhomap)}{(v,\rhomap)}
  }
\end{array}
\end{smathpar}
\end{minipage}
%
\begin{minipage}{3.3in}
\begin{smathpar}
\begin{array}{c}
\renewcommand*{\arraystretch}{1.2}
\RULE
  {
    \fgjN = \RgnZ\inang{T} \spc
    \ralloc \in \rhoenv \spc
    \rgn \notin dom(\rhomap) \cup \rhoenv \spc
    \rhomap' = \rhomap[\rho \mapsto \CLOSED]
  }
  {
    \redstoo{(\C{new} \; \fgjN\inang{\toprgn}
                (\lambdaexp{\ralloc}{\rhoalloc}{}{e}),\rhomap)}
            {(\C{new} \; \fgjN\inang{\rgn}
                ([\rgn/\rhoalloc]e),\rhomap')}
  }
\end{array}
\end{smathpar}
\end{minipage}
%

%
\begin{minipage}{3.6in}
\begin{smathpar}
\begin{array}{c}
\renewcommand*{\arraystretch}{1.2}
\RULE
  {
    \fgjN = \RgnZ\inang{T} \spc
    \rgn \in dom(\rhomap) \spc
    \redstocup{\rgn}{(e,\rhomap)}{(e',\rhomap')}
  }
  {
    \redstoo{(\C{new} \; \fgjN\inang{\rgn}
                (e),\rhomap)}
            {(\C{new} \; \fgjN\inang{\rgn}
                (e'),\rhomap')}
  }
\end{array}
\end{smathpar}
\end{minipage}
%
\begin{minipage}{2.5in}
\begin{smathpar}
\begin{array}{c}
\renewcommand*{\arraystretch}{1.2}
\RULE
  {
    v_a = \C{new} \; \RgnZ\inang{T}\inang{\rgn}(v_r) \spc
    \rhomap(\rgn) \neq \XFERRED \spc
    \rgn_0 \notin \rhoenv \spc
  }
  {
    \redstoo{(\open{v_a}{\rgn_0}{x}{v_b},\rhomap)} {(v_b,\rhomap)}
  }
\end{array}
\end{smathpar}
\end{minipage}
%

%
\begin{minipage}{4in}
\begin{smathpar}
\begin{array}{c}
\renewcommand*{\arraystretch}{1.2}
\RULE
  {
    v_a = \C{new} \; \RgnZ\inang{T}\inang{\rgn}(v_r) \spc
    \rhomap(\rgn) \neq \XFERRED \spc
    \rgn_0 \notin \rhoenv \\
    \redstocup{\rgn_0}{([[\rgn_0/\rgn]v_r/x]e_b,
      \rhomap[\rgn \mapsto \OPEN])}{(e_b',\rhomap')} \spc
    \rhomap'' = \rhomap'[\rgn \mapsto \rhomap(\rgn)]
  }
  {
    \redstoo{(\open{v_a}{\rgn_0}{x}{e_b},\rhomap)} 
            {(\open{v_a}{\rgn_0}{x}{e_b'},\rhomap'')}
  }
\end{array}
\end{smathpar}
\end{minipage}
%
%
\begin{minipage}{3in}
\begin{smathpar}
\begin{array}{c}
\renewcommand*{\arraystretch}{1.2}
\RULE
  {
    \\
    v_a = \C{new} \; \RgnZ\inang{T}\inang{\rgn}(v_r) \spc
    \rhomap(\rgn) = \XFERRED \spc
%   \rgn_0 \notin \rhoenv \\
%   \redstocup{\rgn_0}{([[\rgn_0/\rgn]v_r/x]e_b,
%     \rhomap[\rgn \mapsto \OPEN])}{(e_b',\rhomap')}
  }
  {
    \redstoo{(\open{v_a}{\rgn_0}{x}{e_b},\rhomap)} 
            {\invalidexn}
  }
\end{array}
\end{smathpar}
\end{minipage}
%

%
\begin{minipage}{3.85in}
\begin{smathpar}
\begin{array}{c}
\renewcommand*{\arraystretch}{1.2}
\RULE
  {
    \allocRgn(\fbN),\ralloc \in \rhoenv \spc
    \mbody(m\inang{\ralloc \rbar},\fbN) = \bar{x}.e 
%   \redstoo{(, \rhomap)}{(e',\rhomap')}
  }
  {
    \redstoo{((\C{new}\;\fbN(\bar{v})).m\inang{\ralloc \rbar}
                      (\bar{v'}),\rhomap)}
            {([\bar{v'}/\xbar][\C{new} \; \fbN(\bar{v})/\thisZ]\,e,\rhomap)}
  }
\end{array}
\end{smathpar}
\end{minipage}
%
\begin{minipage}{3.25in}
\begin{smathpar}
\begin{array}{c}
\renewcommand*{\arraystretch}{1.2}
\RULE
  {
%   \ralloc \in \rhoenv \spc
    \fbN = \RgnZ\inang{T}\inang{\rgn}\spc
%   \redstoo{(, \rhomap)}{(e',\rhomap')}
    \rhomap(\rgn) \neq \OPEN \spc
    \rhomap' = \rhomap[\rgn \mapsto \XFERRED]
  }
  {
    \redstoo{((\C{new}\;\fbN(v)).\transfer\inang{\ralloc}
                      (),\rhomap)}
            {(\unitval,\rhomap')}
  }
\end{array}
\end{smathpar}
\end{minipage}
%

%
\begin{minipage}{2.5in}
\begin{smathpar}
\begin{array}{c}
\renewcommand*{\arraystretch}{1.2}
\RULE
  {
    \fbN = \RgnZ\inang{T}\inang{\rgn}\spc
%   \ralloc \in \rhoenv \spc
%   \redstoo{(, \rhomap)}{(e',\rhomap')}
    \rhomap(\rgn) = \OPEN \spc
  }
  {
    \redstoo{((\C{new}\;\fbN(v)).\transfer\inang{\ralloc}
                      (),\rhomap)}
            {\invalidexn}
  }
\end{array}
\end{smathpar}
\end{minipage}
%
\begin{minipage}{2.5in}
\begin{smathpar}
\begin{array}{c}
\renewcommand*{\arraystretch}{1.2}
\RULE
  {
    v_a = \lambdaexp{\rgn_a}{\rhoalloc\rhobar}
                        {\taubar \; \xbar}{e} \spc
    \rgn_a,\ralloc \in \rhoenv \spc 
  }
  {
    \redstoo{(v_a\inang{\ralloc\rbar}(\bar{v}) ,\rhomap)}
            {([\bar{v}/\xbar][\rbar/\rhobar][\ralloc/\rhoalloc]\,e,\rhomap)}
  }
\end{array}
\end{smathpar}
\end{minipage}
%
% %
% \begin{minipage}{1.8in}
% \begin{smathpar}
% \begin{array}{c}
% \renewcommand*{\arraystretch}{1.2}
% \RULE
%   {
%     \redstoo{(e_1,\rhomap)}{(e_1',\rhomap')}
%   }
%   {
%     \redstoo{(e_1;\,e_2,\rhomap)}
%             {(e_1';\,e_2,\rhomap')}
%   }
% \end{array}
% \end{smathpar}
% \end{minipage}
% %

\begin{minipage}{1.8in}
\begin{smathpar}
\begin{array}{c}
\renewcommand*{\arraystretch}{1.2}
\RULE
  {
    
  }
  {
    \redstoo{(\unitval;\,e_2,\rhomap)}
            {(e_2,\rhomap)}
  }
\end{array}
\end{smathpar}
\end{minipage}
% 
\begin{minipage}{1.8in}
\begin{smathpar}
\begin{array}{c}
\renewcommand*{\arraystretch}{1.2}
\RULE
  {
    \redstoo{(e,\rhomap)}{(e',\rhomap')}
  }
  {
    \redstoo{(E\lbrack e \rbrack, \rhomap)}
            {(E\lbrack e' \rbrack, \rhomap')}
  }
\end{array}
\end{smathpar}
\end{minipage}
%
\begin{minipage}{1.2in}
\begin{smathpar}
\begin{array}{c}
\renewcommand*{\arraystretch}{1.2}
\RULE
  {
    \redstoo{(e,\rhomap)}{\invalidexn}
  }
  {
    \redstoo{(E\lbrack e \rbrack, \rhomap)}
            {\invalidexn}
  }
\end{array}
\end{smathpar}
\end{minipage}

%
\begin{minipage}{2in}
\begin{smathpar}
\begin{array}{c}
\renewcommand*{\arraystretch}{1.2}
\RULE
  {
    \\
    \rgn \notin \rhoenv \spc
    \redsto{\rhoenv \cup \{\rgn\}}{(e,\rhomap)}{\invalidexn}
  }
  {
    \redstoo{(\letregion{\rgn}{e},\rhomap)}{\invalidexn}
  }
\end{array}
\end{smathpar}
\end{minipage}
%
\begin{minipage}{2in}
\begin{smathpar}
\begin{array}{c}
\renewcommand*{\arraystretch}{1.2}
\RULE
  {
    \fgjN = \RgnZ\inang{T} \spc
    \rgn \in dom(\rhomap) \\
    \redstocup{\rgn}{(e,\rhomap)}{\invalidexn}
  }
  {
    \redstoo{(\C{new} \; \fgjN\inang{\rgn}
                (e),\rhomap)}
            {\invalidexn}
  }
\end{array}
\end{smathpar}
\end{minipage}
%
\begin{minipage}{2.5in}
\begin{smathpar}
\begin{array}{c}
\renewcommand*{\arraystretch}{1.2}
\RULE
  {
    v_a = \C{new} \; \RgnZ\inang{T}\inang{\rgn}(v_r) \spc
    \rhomap(\rgn) \neq \XFERRED \spc
    \rgn_0 \notin \rhoenv \\
    \redstocup{\rgn_0}{([[\rgn_0/\rgn]v_r/x]e_b,
      \rhomap[\rgn \mapsto \OPEN])}{\invalidexn}
  }
  {
    \redstoo{(\open{v_a}{\rgn_0}{x}{e_b},\rhomap)} 
            {\invalidexn}
  }
\end{array}
\end{smathpar}
\end{minipage}
%

%
\bigskip

\textbf{Evaluation Context} \fbox {\(E\)}\\
\begin{smathpar}
\begin{array}{lcl}
E & \coloneqq & \bullet \ALT (\bullet).f \ALT \bullet.m\inang{\ralloc\rbar}(\ebar) \ALT
      v.m\inang{\ralloc\rbar}(...,\bullet,...) \ALT \C{new}\; \fbN(...,\bullet,...) \ALT
      \C{new} \; \RgnZ\inang{T}\inang{\toprgn}(\bullet) \ALT \bullet\inang{\ralloc\rbar}(\ebar) \\
  &  & \ALT v\inang{\ralloc\rbar}(...,\bullet,...) \ALT \bullet;\,e \ALT \open{\bullet}{\rgn}{y}{e}
\end{array}
\end{smathpar}

\caption{\fbname: Operational Semantics}
\label{fig:fb-opsem}
\end{figure*}

Figs.~\ref{fig:fb-opsem-1} and~\ref{fig:fb-opsem-2} show the
operational semantics of \fbname. The semantics defines a five-place
reduction relation:
\begin{smathpar}
  \redstoo{\Delta}{(e,\mem)}{(e',\mem')}
\end{smathpar}
Where $\Delta$ is the set of currently-live region locations, and
$\mem$ is a map from memory locations ($\loc$) to type states ($s$).
Evaluating a $\C{letregion}$ or a $\C{new}\;\C{Region}$ expression
results in the addition of a new binding to $\mem$. For a new static
region, the memory location ($\loc$) is mapped to $\LIVE$ (live), and
for a new transferable region, it is mapped to $\USED$. The binding is
updated when the transferable region is opened, closed, or
transferred. Likewise, when a $\C{letregion}$ expression is evaluated
to a value, the binding for the corresponding location is set to
$\XFERRED$, effectively deallocating the region. Opening an already
transferred region, or transferring a currently-open region results in
an exception ($\bot$). The semantics gets stuck while evaluating a
field access, a method call, or a lambda application, if the region
containing the target object is not live. The type safety result
establishes that this can never happen while evaluating a well-typed
expression. The set ($\Delta$) of live region is the same set used by
the expression typing judgment (Fig.~\ref{fig:fb-staticsem-2}). 

\section{Type System: Proofs}

We first define the $\consistent$ relation between $\Delta$ and
$\mem$:

\begin{definition}[\consistent($\Delta$,$\mem$)]
A set $\Delta \in 2^{\rgn}$ of region annotations is said to be
consistent with a map $\mem \in \loc \rightarrow s$ from region
locations to typestates if and only if forall $\loc\in\Delta$,
$\mem(\loc) = \LIVE$.
\end{definition}

Consistency between $\Delta$ and $\mem$ is preserved by the reduction
relation:

\begin{lemma}[\textbf{consistency preservation}]
\label{lem:consistency}
$\forall(e,\Delta,\mem,\phicx,\tau)$. if $\consistent(\Delta,\mem)$
and $\tywf{\Delta}{\phicx}$ and
$\hastyp{(\Delta,\cdot,\phicx),\cdot}{e}{\tau}$ and
$\redstoo{\Delta}{(e,\mem)}{(e',\mem')}$, then
$\consistent(\Delta,\mem')$.
\end{lemma}
\begin{proof}
Proof is by induction on $\redstoo{\Delta}{(e,\mem)}{(e',\mem')}$.
For the rules where $\mem'=\mem$, proof is trivial. For the rules
where $\mem'$ is a result of executing a subexpression under
$\Delta$, proof follows from the inductive hypothesis. Remaining rules
are discussed below:
% \begin{smathpar}
% \begin{array}{cl}
%   \forall(\Delta,\mem,\phicx,\tau).~\consistent(\Delta,\mem)
%   ~\conj ~\tywf{\Delta}{\phicx}
%   ~\conj ~\hastyp{(\Delta,\cdot,\phicx),\cdot}{e_0}{\tau} & \\
%   \hspace*{1in}
%   ~\redstoo{\Delta}{(e_0,\mem)}{(e_0',\mem')}
%   ~\Rightarrow~ \consistent(\Delta,\mem') & IH\\
% \end{array}
% \end{smathpar}
\begin{itemize}
  \item Rule $\rulelabel{LetRegionBegin}$: All live regions in $\mem$
  are also live in $\mem'$. Proof follows
  \item Rule $\rulelabel{LetRegion}$: By inductive hypothesis, $\mem'$
  is consistent with $\Delta \cup \{\loc\}$. Hence, it is consistent
  with $\Delta$.
  \item Rule $\rulelabel{LetRegionEnd}$: Here, $\loc$ is set to
  $\XFERRED$ in $\mem'$. However, since $e$ is the \C{letd}
  expression, and the type rule for the \C{letd} expression gives us
  $\loc \not\in \Delta$. Hence $\Delta$ is consisten with $\mem'$.
  \item Rule $\rulelabel{NewRegion}$: $\mem'$ extends $\mem$ with a
  new binding. Consistency is trivially preserved.
  \item Rule \#2 of $\rulelabel{NewRegion}$ : Since $\mem(\loc)=\USED$
  and $\consistent(\Delta,\mem)$, $\loc \not\in \Delta$. The proof
  follows from inductive hypothesis.
  \item Rule $\rulelabel{Open}$: As $\mem'=\mem[\loc\mapsto\LIVE]$,
  hence $\Delta$ remains consistent with $\mem'$. Here, we also take
  note of the fact that if the result expression (\C{opened}) is
  tagged with a typestate of $\USED$, then $\loc\notin \Delta$.
  \item Rule $\rulelabel{Opened}$: Inductive hypothesis guarantees
  that $\Delta\cup\{\loc\}$ is consistent with $\mem'$. Hence,
  $\Delta$ is consistent with $\mem'$. We also take note of the fact
  that the typestate tagged with the $\C{opened}$ expression remains
  invariant during reduction.
  \item Rule $\rulelabel{OpenEnd}$: $\mem'=\mem[\loc\mapsto s]$, where
  $s$ is the typestate tagged with the \C{opened} expression. As clear
  from the $\rulelabel{Open}$ rule, $s$ can be either $\USED$ or
  $\LIVE$, and if $s$ is $\USED$ then $\loc\notin\Delta$. Hence, in
  either case $\Delta$ remains consistent with $\mem$.
  \item Rules $\rulelabel{Transfer}$: changes binding for a non-live
  location $\loc$. Hence, consistency is preserved.
\end{itemize} 
\qed
\end{proof}

\begin{lemma}[value substitution preserves typing]
\label{lem:substitution}
$\forall(e,z,\tau_1,\tau_2,\Delta,\phicx)$, if $\hastyp{(\Delta,\cdot,
\phicx),\cdot[x\mapsto\tau_1]}{e}{\tau_2}$ and $\hastyp{(\Delta,\cdot,
\phicx),\cdot}{v}{\tau_1}$, then $\hastyp{(\Delta,\cdot,
\phicx),\cdot} {[v/x]e}{\tau_2}$.
\end{lemma}
\begin{proof}
The proof is by induction on typing derivation and follows on the
lines of similar proof for FGJ.
\qed
\end{proof}


% \begin{lemma}[weakening]
% $\forall(v,\tau,\Delta,\Delta_0,\phicx,\phicx_0)$, 
% if 
% $\hastyp{(\Delta \cup \Delta_0,\cdot, \phicx \conj
% \phicx_0),\cdot}{v}{\tau}$
% and
% $\tywf{(\Delta,\cdot,\phicx)}{\tau}$
% then 
% $\hastyp{(\Delta,\cdot, \phicx),\cdot} {v}{\tau}$.
% \end{lemma}
% \begin{proof}
% The proof is by induction on $\hastyp{(\Delta \cup \Delta_0,\cdot, \phicx \conj
% \phicx_0),\cdot}{v}{\tau}$. Since $v$ is a value, we have few cases:

% \begin{itemize}
%   \item Case ($v = \C{new}\; \fbN_0(\vbar)$ and $\tau = \fbN_0$): By
%   inverting 
% \end{itemize}

% \end{proof}

\begin{lemma}[progress]
\label{lem:progress}
$\forall e, \tau, \mem, \rhomap, \phicx$, if $\consistent(\Delta,\mem)$ and 
$\tywf{\Delta}{\phicx}$ and
$\hastyp{\emptyA,\cdot}{e}{\tau}$, then one of the following holds:\\
  \begin{smathpar}
  \begin{array}{rl}
    (i) & \exists (e',\rhomap').\;\redstoo{\Delta}{(e,\rhomap)}{(e',\rhomap')}\\
    (ii) & \valuee(e)\\
    (iii) & \redstoo{\Delta}{(e,\rhomap)}{\invalidexn}\\
  \end{array}
  \end{smathpar}
\end{lemma}
\begin{proof}
Proof is by induction on the typing derivation of $e:\tau$. Most cases
follow from the inductive hypothesis (IH), which claims that if a
subexpression has a typing derivation, then it can make progress. We
will consider cases where all subexpressions are values, but the
expression itself is not a value.
\begin{itemize}
  \item $B\inang{\tbar}{\locbar}(\vbar).f_i$ case: This expression has
  a type only if $\tywf{\emptyA}{B\inang{\tbar}\inang{\locbar}}$, which is
  possible only if $\locbar \in \Delta$. Hence $e$ can make progress.

  \item $\C{let} \; x = v \; \C{in} \; e$ case: Always takes step via
  \rulelabel{LetExp}.

  \item $\C{new}\; \RgnZT{\toprgn}(v)$ case: takes a step by
  \rulelabel{NewRegion} rule to $\C{new}\; \RgnZT{\toploc\loc}
  (v\inang{\loc}())$.

  \item Method call case: Since $\mtype$ is defined, $\mbody$ is also
  defined, and the execution takes a step by $\rulelabel{MethodInv}$.

  \item Lambda application case: takes step by \rulelabel{FnApply}

  \item \C{letregion} case: takes a step by \rulelabel{LetRegionBegin}
  rule.

  \item \C{open} case: takes a step by \rulee{Open} rule, or throws an
  exception by \rulee{OpenTransferred} rule, depending on the typestate
  of the region location.

  \item \C{letd} case: The subexpression is typed under $\Delta \cup
  \{\loc\}$, hence the subexpression takes a step under $\Delta \cup
  \{\loc\}$. This allows \C{letd} expression take a step via
  \rulee{LetRegion} rule. If subexpression is a value, then \C{letd}
  takes a step via \rulee{LetRegionEnd} rule.

  \item \C{opened} case: similar to \C{letd} case.
\end{itemize}
\qed
\end{proof}

\begin{lemma}[preservation]
\label{lem:preservation}
$\forall e, \tau, \Delta, \mem$, such that $\consistent(\Delta,\mem)$
and $\tywf{\rhoenv}{\phicx}$, if $\hastyp{\emptyA,
\cdot}{e}{\tau}$, and $\redstoo{\Delta}{(e,\mem)}{(e',\mem')}$, then 
$\hastyp{\emptyASigp,\cdot}{e'}{\tau}$.
\end{lemma}
\begin{proof}
  Proof is by induction on the reduction step. Most cases follow from
  the inductive hypothesis, which asserts that if a subexpression
  takes a step, it preserves its type under the same $\Delta$ and
  $\phicx$. We consider interesting cases below:
  \begin{itemize}
    \item \rulee{FieldAccess} case: $\C{new}\; A\inang{\tbar}
    \inang{\loc\locbar}(\vbar).f_i$ takes a step to $v_i$. The field
    access expression has a type of $i^{th}$ field returned by
    $\fields$ definition, and so does $v_i$, if ${new}\;
    A\inang{\tbar} \inang{\loc\locbar}(\vbar)$ has to be typable.
    Hence the type is preserved.

    \item \rulee{MethodInv} case: From the method invocation type rule
    and method well-formedness condition, method body ($e'$) has a type
    $\tau^2$ under $(\{\loc,\locbar,\rhobar\},\cdot,\phi),\cdot[\xbar
    \mapsto \overline{\tau^1}]$. Also,
    $\tywf{\loc,\locbar,\rhobar}{\phi}$. Since $\overline{\loc'} \neq \rhobar$
    (locations are never equal to region variables), we have:
    \begin{center}
    $\hastyp{(\{\loc,\locbar,\overline{\loc'}\},\cdot,[\overline{\loc'}/\rhobar]\phi),\cdot[\xbar
    \mapsto \overline{\tau^1}]}{e'}{\tau^2}$ and
    $\tywf{\{\loc,\locbar,\overline{\loc'}\}}{[\overline{\loc'}/\rhobar]\phi}$
    \end{center}
    Since $\{\{\loc,\locbar,\overline{\loc'}\} \subseteq \Delta$, and
    $\phicx \vdash [\overline{\loc'}/\rhobar]\phi$, we can strengthen
    the context and derive:
    \begin{center}
      $\hastyp{(\Delta,\cdot,\phicx),\cdot[\xbar \mapsto \overline{\tau^1}]}{e'}{\tau^2}$
    \end{center}
    Applying the substitution lemma (Lemma~\ref{lem:substitution}) and
    inductive hypothesis gives the proof.

    \item \rulee{FnApply} case: proceeds on the similar lines as
    \rulee{MethodInv} case, except that no strengthening is needed;
    function body is typed under a context that includes $\Delta$.

    \item \rulee{LetRegionBegin} case: Since $\loc \notin dom(\mem)$,
    it follows that $\loc \notin \Delta$. Rest of the premises
    required to apply the \C{letd} type rule on result expression are
    obtained from the \C{letregion} type rule of the initial
    expression ($e$), by substituing $\loc$ for $\pi$.

    \item \rulee{LetRegionEnd} case: $e$ is a \C{letd} expression
    which reduces to a value $v$. Invering the typing derivation of
    $e$ yeilds the premise that $v$ is well-typed under the context
    $(\Delta \cup \{\loc\},\cdot,\phicx \conj \Delta \outlives \loc)$.
    However, we have to prove that $v$ is well-typed under the smaller
    context $(\Delta,\cdot,\phicx)$. We carry out this proof by
    induction on the structure of value $v$:
    \begin{itemize}
      \item If $v$ is an object value (e.g.,
      $B\inang{\tbar}\inang{\locbar}(\vbar)$), IH yeilds the
      well-typedness of $\vbar$ under $(\Delta,\cdot,\phicx)$, and the
      premise 
      $\tywf{(\Delta,\cdot,\phicx)} {B\inang{\tbar}\inang{\locbar}}$,
      obtained by inverting the type judgment for \C{letd}, yeilds the
      well-typedness of whole value.

      \item A region handler value ($\RgnZT{\toploc\loc'}(v)$) is
      well-typed under any context.

      \item If $v$ is a function closure, then it is well-tuped under
      $(\Delta,\cdot,\phicx)$ only if it doesn't trap any references
      to $\loc$, the \C{letd} location. Since closure's type needs to
      be well-formed under $(\Delta,\cdot,\phicx)$, its allocation
      region belongs to $\Delta$, hence outlives $\loc$. Since the
      closure typing rule requires all free region variables of the
      closure body outlive the allocation region of closure, it
      follows that all free region variables strictly outlive $\loc$,
      hence they belong to $\Delta$. Hence the function closure is
      well-typed outside \C{letd}.
    \end{itemize}

  \item \rulee{OpenEnd} case: The proof is similar to the \C{letd}
  case.
  \qed
  \end{itemize}
\end{proof}

\begin{proof}[\textbf{Theorem~\ref{thm:fb-type-safety}}]
Follows from Lemmas~\ref{lem:consistency},~\ref{lem:progress},
and~\ref{lem:preservation}.
\qed
\end{proof}


\begin{figure*}[t!]

\beginrules

%%%%%%%%%%% LAMBDA %%%%%%%%%%%

\lgcrule{LAMBDA}
  {
    \rgn \in \A.\rhoenv \spc
    \rhobar \notin \A.\rhoenv
    \spc
%   \rhoenv' = \rhoenv \cup \{\rhoalloc,\rhobar\}\spc
    \A' = (\A.\rhoenv \cup \{\rhobar\}, \A.\aenv, 
          \A.\phicx \conj \phi)\spc
    \tywf{\A'.\rhoenv}{\phi}
    \\
    \typeok {\A'} {\bar{\tau^1}} {C} \spc
    \typeok {\A'} {\tau^2} {C} \spc
    \exprok {\A',\env[\xbar \mapsto \bar{\tau^1}]} {e} {\tau^2} {C}
  }
  {
    \exprok {\stdcontext}
           {\lambdaexp{\rgn}{\rhobar \,|\, \phi} {\xbar:\bar{\tau^1}}{e}}
           {\inang{\rhobar \,|\, \phi} \bar{\tau^1} \xrightarrow{\rgn} \tau^2}
	   {C}
  }

%%%%%%%%%%% OPEN-REGION %%%%%%%%%%%
\lgcrule{OPEN}{
   \exprok {\stdcontext} {e_a} {\RgnZ\inang{T}\inang{\rho}} {C_1} \spc
   \A = (\rhoenv,\aenv,\phicx) \spc
   \rgn \notin \rhoenv
   \\
   % (\A',\env') = ((\rhoenv \cup \{\rgn\},\aenv,\phicx),\env[y\mapsto T@\rgn] \spc
   \exprok {(\rhoenv \cup \{\rgn\},\aenv,\phicx),\env[y\mapsto T@\rgn]} {e_b} {\tau} {C_2}
}{
   \exprok {\stdcontext} {\open{e_a}{\rgn}{y}{e_b}} {\tau} {(C_1 \cup C_2)}
}


%%%%%%%%%%% FUNCTION INVOCATION %%%%%%%%%%%
\lgcrule{FUN-APPLY}
{
\exprok {\stdcontext} {e_a} {\inang{\rhobar\,|\,\phi}\taubar \xrightarrow{\rgn} \tau} {C_1} \spc
\exprok {\stdcontext} {\bar{e}} {\bar{\tau'}} {C_2} \spc
\substFn = [\bar{\rho'}/\rhobar]
\\
C_3 = \{\bar{\rho'} \in \A.\rhoenv\} \spc
C_4 = \{\isvalid{\A.\phicx}{\substFn(\phi)}\} \spc
\subtypeok {\A} {\bar{\tau'}} {\substFn(\bar{\tau})} {C_5}
}{
\exprok {\stdcontext} {e_a\inang{\bar{\rho'}}(\bar{e})} {\tau} {\cup_{i=1}^5 C_i}
}

%%%%%%%%%%% METHOD %%%%%%%%%%%
\lgcrule{METHOD}{
CT(B) = \hdOf{B}{\varphi}\{\bar{\tau^f}\,\xbar;\;\bar{d}\} \\
\A = (\rhoenv,\aenv,\phicx) = (\{\rhobar,\rhobarm\},\bar{\tyvar} \extends \bar{\fgjN}, \varphi_m) \spc\spc
C_1 = \{ \tywf{\rhoenv}{\varphi_m} \} \\
\env = \cdot[\thisZ \mapsto B\inang{\bar{\tyvar}}\inang{\rhobar}][\xbar \mapsto \taubar] \spc\spc
\exprok {\stdcontext}{e} {\tau'} {C_2} \spc\spc
\subtypeok {\A} {\tau'} {\tau} {C_3}
}{
\typeok{} {(B, \tau \; m\inang{\rhobarm \,|\, \varphi_m} (\taubar \;  \xbar)\{\C{return} e;\})} {(C_1 \cup C_2 \cup C_3)}
}

%%%%%%%%%%% CLASS %%%%%%%%%%%
\lgcrule{CLASS}{
\A = (\rhoenv, \aenv, \phicx) = (\{\rhoalloc,\rhobar\},\bar{\tyvar} \extends \bar{\fgjN},\varphi) \\
C_1 = \{ \tywf{\rhoenv}{\varphi} \} \spc\spc
\typeok {\A} {\fbN} {C_2} \spc\spc
\typeok {\A} {\bar{\tau^f}} {C_3} \\
C_4 = \{\isvalid{\phicx}{\allocRgn(\bar{\tau^f}) \outlives \rhoalloc \conj \allocRgn(\fbN) = \rhoalloc}\} \\
\typeok {} {\bar{d}} {C_5}
}{
\typeok {} {\hdOf{B}{\varphi}\{\bar{\tau^f}\,\xbar;\;\bar{d}\}} {\bigcup_{i=1}^5 C_i}
}

\myendrules

\caption{Constraint generation}
\label{fig:constraint-gen-1}
\end{figure*}

\begin{figure*}[t!]

\beginrules

%%%%%%%%%%% Header Box %%%%%%%%%%%
\fbox{  \( \typeok{\A}{\tau}{C} \)}
\\

%%%%%%%%%%% TYPE WELL-FORMEDNESS %%%%%%%%%%%

%%%%%%%%%%% OBJECT TYPE %%%%%%%%%%%
\lgcrule{TWF}
  {
    C = \{ \rgn \in \A.\rhoenv \}
  }
  {
    \typeok {\A} {\ObjZ\inang{\rgn}} {C}
  }

%%%%%%%%%%% CLASS TYPE %%%%%%%%%%%
  \lgcrule{TWF}
  {
    CT(B) = \headerOf{B}\{...\}
    \spc
    \fgjtywf{\aenv}{B\inang{\tbar}}
    \\
    C = \{ \rbar \in \rhoenv, \isvalid{\phicx}{[\rbar/\rhobar, \tbar/\bar{\tyvar}](\phi)} \}
  }
  {
    \typeok {(\rhoenv,\aenv,\phicx)} {B\inang{\rbar}\inang{\tbar}} {C}
  }

%%%%%%%%%%% GENERIC TYPE PARAMETER %%%%%%%%%%%
  \lgcrule{TWF}
  {
    \fgjtywf{\A.\aenv}{T} \spc
    \fgjsubtyp{\A.\aenv}{T}{\ObjZ} \spc
    \\
    C = \{ \rgn \in \A.\rhoenv \}
  }
  {
    \typeok {\A}{T@\rgn} {C}
  }

%%%%%%%%%%% FUNCTION TYPE %%%%%%%%%%%
  \lgcrule{TWF}
  {
    C_1 = \{ \rgn \in \rhoenv \}
    \\
    \rhobar \notin \A.\rhoenv \spc
    \rhoenv' = \rhoenv \cup \{\rhobar\} \spc
    \A' = (\rhoenv', \aenv, \phicx \conj \phi)
    \\
    \tywf{\rhoenv'}{\phi}\spc 
    \typeok{\A'}{\bar{\tau^1}} {C_2} \spc
    \typeok{\A'}{\tau^2} {C_3}
  }
  {
    \typeok{(\rhoenv,\aenv,\phicx)} {\inang{\rhobar \,|\, \phi} \bar{\tau^1} \xrightarrow{\rgn} \tau^2} 
       {C_1 \cup C_2 \cup C_3}
  }

%%%%%%%%%%% REGION TYPE %%%%%%%%%%%
  \lgcrule{TWF}
  { 
    \fgjtywf{\A.\aenv}{T}
  }
  {
    \typeok {\A} {\RgnZ\inang{T}\inang{\toprgn}} {\{\}}
  }

%%%%%%%%%%% FUNCTION SUBTYPING %%%%%%%%%%%
  \lgcrule{FUN-SUBTYPING}
  {
    C_1 = \{ \isvalid{\A.\phicx}{\phi_1 \Rightarrow \phi_2} \}
    \\
    \subtypeok {\A} {\bar{\tau^{11}}} {\bar{\tau^{21}}} {C_2}
    \\
    \subtypeok {\A} {\tau^{22}} {\tau^{12}} {C_3}
  }
  {
    \subtypeok {\A}
      {\inang{\rhobar \,|\, \phi_2}\bar{\tau^{21}} \xrightarrow{\rgn} \tau^{22}}
      {\inang{\rhobar \,|\, \phi_1}\bar{\tau^{11}} \xrightarrow{\rgn} \tau^{12}}
      {C_1 \cup C_2 \cup C_3}
  }

\myendrules

\caption{Type well-formedness constraint generation}
\label{fig:constraint-gen-2}
\end{figure*}


\begin{lemma}
The antecedent of any generated validity constraint is of the form
$\varphi \conj \phictxt$ where $\varphi$ is a predicate variable and
$\phictxt$ is a conjunction of zero or more outlives-constraints.
Let $\pi_1 \outlives \pi_2$ be a conjunct in $\phictxt$.
Then, $\pi_2 \notin \predDeltaMap(\varphi)$.
Furthermore, if $\pi_1 \in \predDeltaMap(\varphi)$, then for
every $\pi_f \in \predDeltaMap(\varphi)$, $\pi_f \outlives \pi_2$ is
a conjunct in $\phictxt$.
\end{lemma}

\begin{proof}
  By induction over the constraint-generation rules.
  Any context $\A = (\rhoenv,\aenv,\phicx)$ generated by the constraint-generation
  process satisfies the invariant that $\phicx$ is of the form $\varphi \conj \phictxt$
  where $\rhoenv \supseteq \predDeltaMap(\varphi)$.
  The only rule that modifies $\phicx$ is the rule for \C{letregion}
  that adds the set of constraints $\pi_f \outlives \pi$ for every $\pi_f \in \rhoenv$
  as conjuncts to $\phicx$.
\end{proof}


\begin{theorem}
Let $p$, $q$, $r$ and $C$ denote the values of the corresponding
variables in an execution of the type inference algorithm.
\begin{enumerate}
\item $\absof{q}$ = $p$
\item $\absof{q[\sigma]}$ = $p$ for any substitution $\sigma$.
\item If $\sigma$ is any solution to $C$, then $q[\sigma]$ is well-typed.
\item If there is any substitution $\sigma$ such that $q[\sigma]$ is well-typed, then
$C$ has a solution.
\item If SolveConstraints($C$) returns Some($s$), then $s$ is a solution to $C$.
\item If $C$ has a  solution, then SolveConstraints($C$) will return some solution.
\end{enumerate}
\end{theorem}

\section{Other Aspects}

\paragraph{Modularity Aspects of Type Inference.}
The type inference algorithm, as presented, traverses the entire program to
generate the set of constraints, which are solved en masse, using an iterative
fixed point computation. However, the type inference can be realized in a
modular and compositional fashion, subject only to the restrictions imposed
by recursion.

In the elaboration phase, we can process a class \C{C} only after any class
\C{B} that \C{C} depends on has been processed: class \C{C} depends on
class \C{B} if \C{B} is either \C{C}'s base class or the type of any field
of \C{C} depends on \C{B}. In effect, this means that any collection of
mutually recursive classes must be processed together. Non-recursive
dependences can be handled in a compositional fashion: if class \C{C}
depends on \C{B} non-recursively, then the elaboration can be done for
\C{B} first, and then \C{C} can be processed.

The same idea applies to the constraint-solving phase as well.
Given a set of constraints, we say that a predicate variable $\varphi_1$
\emph{directly-depends} on another predicate variable $\varphi_2$ if the set of
constraints includes a constraint $\isvalid{\varphi_1 \conj \phictxt}{F(\varphi_2)}$.
We say that $\varphi_1$ \emph{depends} on $\varphi_2$ if $\varphi_1$ transitively
depends on $\varphi_2$.
The constraint solver needs to process any collection of mutually dependent
predicate variables together.
In effect, this requires the type inference to process any collection of
mutually recursive methods together.
However, methods that are not mutually recursive can be processed separately.

