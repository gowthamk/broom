%\section{Operational Semantics}
%
%Fig.~\ref{fig:fb-opsem} contains a definition of the operational semantics for \fbname.
%
\section{\fbname}

% \subsection{Full Syntax}
% \input{fb-syntax-full}

% The full syntax of \FB, including expressions that manifest only at
% runtime, is shown in Fig.~\ref{fig:fb-syntax-full}. Such expressions
% include:
% \begin{itemize}

% \item Memory locations ($\loc$) corresponding to the memory regions,

% \item A special expression $\RgnZT{\toploc\loc}(e)$ that evaluates to
% a region handler object (the syntax of this expression is captured by
% the $\C{new}\;\fbN(\overline{e})$ construct). $\toploc$ is the
% location of the special $\toprgn$ region where region handlers are
% stored. $\loc$ is the location of the corresponding transferable
% region.

% \item A $\letd{\loc}{e}$ expression that results when
% $\letregion{\pi}{e}$ expression takes a step (see
% Fig.~\ref{fig:fb-opsem-2}).  $\loc$ is the location of the newly
% allocated region.  

% \item An $\opened{\loc}{s}{e}$ expression that
% results when $\C{open}\;{\RgnZT{\toploc\loc}(v)} ...$ expression takes
% a step. $\loc$ is the location of the newly open transferable region.
% The symbol $s$ denotes the typestate of the
% transferable region before it is opened\footnote{Operational semantics
% lets a transferable region to be opened while it is already open. This
% allows methods to safely open a transferable region argument
% regardless of the calling context.}. The typestate can assume the
% values of $\USED$ (allocated and closed), $\LIVE$ (allocated and
% live), and $\XFERRED$ (transferred or freed).

% \end{itemize}

\subsection{Full Static Semantics}

\begin{figure*}[t]
%
\begin{minipage}{2.25in}
\begin{smathpar}
\begin{array}{lcl}
  allocRgn(A\inang{\ralloc\rbar}\inang{\tbar}) & = & \ralloc\\
  allocRgn(\inang{\rhoalloc\rhobar \,|\, \phi}\bar{\tau^1}
      \xrightarrow{\ralloc} \tau^2) & = & \ralloc\\
  shape(A\inang{\rhoalloc\rhobar}\inang{\tbar}) & = & A\inang{\tbar}\\
  bound_{\aenv}(\tyvar@\rgn) & = & \aenv(\tyvar)@\rgn\\
  bound_{\aenv}(\fbN) & = & \fbN\\
\end{array}
\end{smathpar}
\end{minipage}
%
\begin{minipage}{1.8in}
\begin{smathpar}
\begin{array}{c}
\renewcommand*{\arraystretch}{1.2}
\RULE
  {
    \\
    B \in \{\ObjZ,\RgnZ\}
  }
  {
    fields(B\inang{\ralloc\rbar}\inang{\tbar}) \;=\; \bullet
  }
\end{array}
\end{smathpar}
\end{minipage}
%
\begin{minipage}{3in}
\begin{smathpar}
\begin{array}{c}
\renewcommand*{\arraystretch}{1.2}
\RULE
  {
    CT(B) = \headerOf{B}\{\bar{\tau^f}\;\bar{f};\,...\}\\
    \substFn = [\rbar/\rhobar, \ralloc/\rhoalloc, \tbar/\bar{\tyvar}] \qquad 
    fields(\substFn(\fbN)) = \bar{g}:\bar{\tau^g}
  }
  {
    fields(B\inang{\ralloc\rbar}\inang{\tbar}) \;=\;
      \bar{g}:\bar{\tau^g},\,\bar{f}:\substFn(\bar{\tau^f})
  }
\end{array}
\end{smathpar}
\end{minipage}
%
\bigskip

\begin{minipage}{3.5in}
\begin{smathpar}
\begin{array}{lcl}
  ctype(\ObjZ\inang{\rgn}) & = & \bullet \\
% ctype(\RgnZ\inang{\rgn}\inang{T}) & = & \inang{\rhoalloc}
%   {\unitZ}\rightarrow{T@\rhoalloc}\\
  ctype(B\inang{\ralloc\rbar}\inang{\tbar}) & = & 
    fields(B\inang{\ralloc\rbar}\inang{\tbar})\\
  mtype(\C{transfer}, \exists\rho.\RgnZ\inang{\rho}\inang{T}) & = & 
    \inang{\rhoalloc} {\unitZ}\rightarrow{\unitZ}\\
  mtype(\C{free}, \exists\rho.\RgnZ\inang{\rho}\inang{T}) & = & 
    \inang{\rhoalloc} {\unitZ}\rightarrow{\unitZ}\\
\end{array}
\end{smathpar}
\end{minipage}
%
\begin{minipage}{3in}
\begin{smathpar}
\begin{array}{c}
\renewcommand*{\arraystretch}{1.2}
\RULE
  {
    CT(B) = \headerOf{B}\{\bar{\tau^f}\;\bar{f};\,k\;\bar{d}\}\\
    m \notin \bar{d} \qquad 
    \substFn = [\rbar/\rhobar, \ralloc/\rhoalloc, \tbar/\bar{\tyvar}]
  }
  {
    mtype (m,B\inang{\ralloc\rbar}\inang{\tbar}) \;=\;
    mtype (m, \substFn(\fbN))
  }
\end{array}
\end{smathpar}
\end{minipage}
%
\bigskip

\begin{minipage}{3.25in}
\begin{smathpar}
\begin{array}{c}
\renewcommand*{\arraystretch}{1.2}
\RULE
  {
    CT(B) = \headerOf{B}\{\bar{\tau^f}\;\bar{f};\,k\;\bar{d}\}\\
    \tau^2 \; m\mang (\bar{\tau^1}\;\bar{x})\{...\} \in \bar{d} \qquad
    \substFn = [\rbar/\rhobar, \ralloc/\rhoalloc, \tbar/\bar{\tyvar}]
  }
  {
    mtype (m,B\inang{\ralloc\rbar}\inang{\tbar}) \;=\;
    \substFn(\mang\bar{\tau^1} \rightarrow \tau^2)
  }
\end{array}
\end{smathpar}
\end{minipage}
%
\begin{minipage}{3.5in}
\begin{smathpar}
\begin{array}{c}
\renewcommand*{\arraystretch}{1.2}
\RULE
  {

    \substFn = \subst{\bar{\rho_2}}{\bar{\rho_1}}
               \subst{\rhoalloc_2}{\rhoalloc_1} \spc
    mtype(m,\fbN) = \inang{\rhoalloc_1\bar{\rho_1},|\, \phi_1}\bar{\tau^{11}} 
                      \rightarrow \tau^{12} \spc \texttt{implies}\\
    \isvalid{\A.\phicx}{\phi_2 \Leftrightarrow \substFn(\phi_1)} 
        \spc \texttt{and} \spc
    \bar{\tau^{21}} = \substFn(\bar{\tau^{11}}) \spc \texttt{and} \spc
    \subtyp{\A}{\tau^{22}} {\substFn(\tau^{12})}
    %\substFn = [\rbar/\rhobar, \ralloc/\rhoalloc, \tbar/\bar{\tyvar}]
  }
  {
    override(\A,\fbN,\inang{\rhoalloc_2\bar{\rho_2},|\, \phi_1}
              \bar{\tau^{21}} \rightarrow \tau^{22})
  }
\end{array}
\end{smathpar}
\end{minipage}
%
\bigskip

\begin{minipage}{5in}
\begin{smathpar}
\begin{array}{c}
  \rhoset,\rhoenv \in 2^{\rho} \qquad
  \aenv \in \tyvar \rightarrow \fgjN \qquad
  \A = (\subtypcx)\\
\end{array}
\end{smathpar}
\end{minipage}
%

\caption{\fbname: Auxiliary Definitions}
\label{fig:fb-auxdef}
\end{figure*}

\renewcommand{\rgn}{r}
\renewcommand{\rbar}{\overline{r}}

\begin{figure*}[!h]
%
\textbf{Subtyping}  \; \fbox
  {\(\subtyp{\A}{\tau_1}{\tau_2}\)}\\
%
\begin{minipage}{1.8in}
\begin{smathpar}
\begin{array}{c}
\renewcommand*{\arraystretch}{1.2}
  \subtyp{\A}{\tau}{\tau} \\
  \subtyp{(\Delta,\aenv,\phicx)}{\tyvar @\rho}{\aenv(\tyvar) @\rho}\qquad
% \subtyp{\A}{\RgnZ\inang{\rgn}}{\RgnZ\inang{\toprgn}}\qquad
% \subtyp{\A}{\RgnZ\inang{\toprgn}}{\RgnZ\inang{\rgn}}
\end{array}
\end{smathpar}
\end{minipage}
%
\begin{minipage}{2.7in}
\begin{smathpar}
\begin{array}{c}
\renewcommand*{\arraystretch}{1.2}
\RULE
  {
    \\
    CT(B) = \headerOf{B}\{...\}
  }
  {
    \subtyp{\A}{B\inang{\tbar}\inang{\rbar}}
        {[\rbar/\rhobar, \tbar/\bar{\tyvar}](\fbN)}
  }
\end{array}
\end{smathpar}
\end{minipage}
%

\begin{minipage}{1.5in}
\begin{smathpar}
\begin{array}{c}
\renewcommand*{\arraystretch}{1.2}
\RULE
  {
    \subtyp{\A}{\tau_1}{\tau_2}\\
    \subtyp{\A}{\tau_2}{\tau_3}
  }
  {
    \subtyp{\A}{\tau_1}{\tau_3}
  }
\end{array}
\end{smathpar}
\end{minipage}
%
\begin{minipage}{2.75in}
\begin{smathpar}
\begin{array}{c}
\renewcommand*{\arraystretch}{1.2}
\RULE
  {
    \isvalid{\A.\phicx}{\phi_1 \Rightarrow \phi_2} \\
    \subtyp{\A}{\bar{\tau^{11}}}{\bar{\tau^{21}}} \spc
    \subtyp{\A}{\tau^{22}}{\tau^{12}}
  }
  {
    \subtyp{\A}
      {\inang{\rhobar \,|\, \phi_2}\bar{\tau^{21}}
          \xrightarrow{\rgn} \tau^{22}}
      {\inang{\rhobar \,|\, \phi_1}\bar{\tau^{11}}
          \xrightarrow{\rgn} \tau^{12}}
  }
\end{array}
\end{smathpar}
\end{minipage}


%
\bigskip

\textbf{Type, and Type Constraint Well-formedness}  \; \fbox
  {\(\tywf{\A}{\tau}, \spc 
     \tywf{\mem}{\phi}\)}\\
/Users/gowtham/git/broom/fullversion/broom/paper/fb-tywfrules-full.tex
%
\bigskip

\textbf{Expression Typing}  \; \fbox
  {\(\hastyp{\exptycx{\env}}{e}{\tau}\)}\\
/Users/gowtham/git/broom/fullversion/broom/paper/fb-exptyprules-full.tex
%
\bigskip

\caption{\fbname: Static Semantics}
\label{fig:fb-staticsem}
\end{figure*}

\renewcommand{\rgn}{\pi}
\renewcommand{\rbar}{\overline{\pi}}


Figs.~\ref{fig:fb-staticsem-1} and~\ref{fig:fb-staticsem-2} show full
static semantics of \FB. Fig.~\ref{fig:fb-staticsem-1} contains subtyping
rules, and rules to check well-formedness of \FB types, type
constraints, methods, and class definitions. Method well-formedness
rule makes use of the expression typing judgment defined in
Fig.~\ref{fig:fb-staticsem-2}. Auxiliary definitions used in
Figs.~\ref{fig:fb-staticsem-1} and~\ref{fig:fb-staticsem-2} are
defined in Fig.~\ref{fig:fb-auxdef}. As described in
\S~\ref{sec:type-system}, all judgments are parameterized over the
class table ($CT$). The premise $CT(B) = \headerOf{B}\{...\}$
used in some of the rules means that the definition of class $B$ is
present in $CT$ and that the definition is well-formed.

\subsection{Operational Semantics}

% \newcommand{\opsemrule}[3]{%
% \begin{minipage}{#1}\begin{smathpar}\begin{array}{c}%
% \renewcommand*{\arraystretch}{1.2}%
% \RULE {#2} {#3}%
% \end{array}\end{smathpar}\end{minipage}%
% }
% \newcommand{\lopsemrule}[4]{%
% \begin{minipage}{#1}\begin{smathpar}\begin{array}{c}%
% \renewcommand*{\arraystretch}{1.2}%
% [\rulelabel{#4}] \spc \RULE {#2} {#3}%
% \end{array}\end{smathpar}\end{minipage}%
% }
% %
% \newcommand{\anobjty}[0]{B\inang{\tbar}\inang{\ralloc\locbar}}
% \newcommand{\anobj}[0]{\C{new} \; \anobjty(\bar{v})}
% %

\begin{figure*}[t!]

%
\fbox {\(\redstoo{\Delta}{(e,\mem)}{(e',\mem')}\)}\\

\lopsemrule{1.5in}{
    \redstoo{\Delta}{(e,\mem)}{(e',\mem')}
}{
    \redstoo{\Delta}{(E\lbrack e \rbrack, \mem)}
            {(E\lbrack e' \rbrack, \mem')}
	  }{EvalOrder}

\lopsemrule{1.2in}{
    \redstoo{\Delta}{(e,\mem)}{\invalidexn}
  }{
    \redstoo{\Delta}{(E\lbrack e \rbrack, \mem)}
            {\invalidexn}
	  }{Exception}
%

\lopsemrule{2.7in}{
%   \allocRgn(A\inang{\rgn\locbar}\inang{\tbar}) & = & \rgn
%   \allocRgn(\fbN) \in \rhoenv \spc
    \mem(\loc) = \LIVE \spc
    \fields(A\inang{\tbar}\inang{\loc\locbar}) = \bar{f}:\taubar
}{
    \redstoo{\Delta}{((\C{new} \; A\inang{\tbar}\inang{\loc\locbar}(\vbar)).f_i,\mem)}{(v_i,\mem)}
}{FieldAccess}

\lopsemrule{3.85in}{
%   \allocRgn(\fbN) \in \rhoenv \spc
    \mem(\loc) = \LIVE \spc
    \mbody(m,A\inang{\tbar}\inang{\loc\locbar}) = \rho\rhobar.\bar{x}.e 
%   \redstoo{\Delta}{(, \mem)}{(e',\mem')}
}{
    \redstoo{\Delta}{((\C{new}\;A\inang{\tbar}\inang{\loc\locbar}(\bar{v})).m\inang{\loc'\overline{\loc'}}
                      (\bar{v'}),\mem)}
            {([\loc'\overline{\loc'}/\rho\rhobar][\bar{v'}/\xbar]
                [\C{new} \; A\inang{\tbar}\inang{\loc\locbar}(\bar{v})/\thisZ]\,e,\mem)}
	  }{MethodInv}

\lopsemrule{2.85in}{
    v_a = \lambdaexp{\loc'}{\rho\rhobar}
                        {\taubar \; \xbar}{e} \spc
    \mem(\loc') = \LIVE  \spc 
}{
    \redstoo{\Delta}{(v_a\inang{\loc\locbar}(\bar{v}) ,\mem)}
            {([\bar{v}/\xbar][\loc\locbar/\rho\rhobar]\,e,\mem)}
	  }{FnApply}

%
% \lopsemrule{2in}{
% %   \rgn \notin \rhoenv \spc
% %   \fresh(\rgn') \spc
%     \redstoo{\Delta}{(e,\mem)}{\invalidexn}
% }{
%     \redstoo{\Delta}{(\letregion{\rgn}{e},\mem)}{\invalidexn}
%   }{EXCEPTION}

\lopsemrule{2in}{
%   \fgjN = \RgnZ\inang{T} \spc
%   \rgn \in dom(\mem) \spc
    \mem(\loc) = \USED \spc
    \redstocup{\loc}{(e,\mem[\loc \mapsto \LIVE])}{\invalidexn}
}{
    \redstoo{\Delta}{(\C{new} \; \RgnZT{\toploc\loc} (e),\mem)}
            {\invalidexn}
	  }{Exception}

\lopsemrule{2in}{
%   \fgjN = \RgnZ\inang{T} \spc
%   \rgn \in dom(\mem) \spc
}{
    \redstoo{\Delta}{(\letexp{x}{v}{e},\mem)}
            {([v/x]e,\mem)}
	  }{LetExp}

% \lopsemrule{3.2in}{
%     \not\exists \loc.~\mem(\loc) = \FREE
% }{
%     \redstoo{\Delta}{(\letregion{\rgn}{e},\mem)}{\invalidexn}
%   }{EXCEPTION}


% \lopsemrule{3.3in}{
%     \not\exists \loc.~\mem(\loc) = \FREE
% }{
%     \redstoo{\Delta}{(\C{new} \; \RgnZ\inang{T}\inang{\toprgn}
%                 (\lambdaexp{\loc'}{\rho}{}{e}),\mem)}
%             {\invalidexn}
%         }{EXCEPTION}
% \lopsemrule{2.5in}{
%     \mem(\rgn_r) \neq \XFERRED \spc
% %   \fresh(\rgn_1) \spc
% %  \rgn_0 \notin \rhoenv \\
%     \redsto{\Delta \cup \{\rgn\}}{([[\rgn/\rgn_r]v_r/x]e_b,
%       \mem[\rgn_r \mapsto \OPEN])}{\invalidexn}
% }{
%     \redstoo{\Delta}{(\open{(\C{new} \; \RgnZ\inang{T}\inang{\rgn_r}(v_r))}
%                    {\rgn}{x}{e_b},\mem)} 
%             {\invalidexn}
%     }{EXCEPTION}

%
\bigskip

\textbf{Evaluation Context} \fbox {\(E\)}\\
\begin{smathpar}
\begin{array}{lcl}
E & \coloneqq & \bullet \ALT (\bullet).f \ALT \bullet.m\inang{\locbar}(\ebar) \ALT
      v.m\inang{\locbar}(...,\bullet,...) \ALT \C{new}\; \fbN(...,\bullet,...) \ALT
      \C{new} \; \RgnZ\inang{T}\inang{\toprgn}(\bullet) \ALT 
      \bullet\inang{\locbar}(\ebar) \\
  &  & \ALT v\inang{\locbar}(...,\bullet,...) \ALT
       \letexp{x}{\bullet}{e} \ALT \open{\bullet}{\rgn}{y}{e} 
%      \ALT \opened{\loc}{\status}{\bullet} \ALT \letd{\loc}{\bullet}
%     The following should be forbidden. See NEW-REGION rule 2.
%      \ALT \C{new} \; \RgnZ\inang{T}\inang{\toploc\loc}(\bullet)
\end{array}
\end{smathpar}

\caption{\fbname: Operational semantics (part 1)}
\label{fig:fb-opsem-1}
\end{figure*}

\begin{figure*}[t!]

\lopsemrule{3.2in}{
%   \fresh(\rgn')\spc
%   \mem(\loc) = \FREE \spc
    \loc \not\in dom(\mem) \spc
    \mem' = \mem[\loc \mapsto \SLIVE]
%   \redstoo{\Delta}{(e,\mem)}{(e',\mem')}
}{
    \redstoo{\Delta}{(\letregion{\rgn}{e},\mem)}{(\letd{\loc}{[\loc/\rgn]e},\mem')}
  }{LetRegionBegin}

\lopsemrule{3in}{
    \redstocup{\loc}{(e,\mem)}{(e',\mem')}
}{
    \redstoo{\Delta}{(\letd{\loc}{e},\mem)}{(\letd{\loc}{e'},\mem')}
  }{LetRegion}

\lopsemrule{3in}{
%   \rgn \notin \rhoenv
%   \mem' = \mem[\loc \mapsto \FREE]
    \mem' = \mem[\loc \mapsto \XFERRED]
}{
    \redstoo{\Delta}{(\letd{\loc}{v},\mem)}{(v,\mem)}
  }{LetRegionEnd}

\lopsemrule{3.3in}{
%   \fgjN = \RgnZ\inang{T} \spc
%   \rgn \in \rhoenv \spc
%   We need Delta here to ensure Delta U {pi_r} in next rule is sound. 
%   \rgn_r \notin \Delta \cup dom(\Sigma) \spc
%   \rgn \notin dom(\mem) \cup \rhoenv \spc
%   \mem' = \mem[\rgn_r \mapsto \CLOSED]
%   \mem(\loc) = \LIVE \spc
%   \mem(\loc') = \FREE \spc
    \loc \not\in dom(\mem) \spc
    \mem' = \mem[\loc \mapsto \USED]
}{
    \redstoo{\Delta}{(\C{new} \; \RgnZ\inang{T}\inang{\toprgn}
                (v),\mem)}
            {(\C{new} \; \RgnZ\inang{T}\inang{\toploc\loc}
                (v\inang{\loc}()),\mem')}
	      }{NewRegion}

\lopsemrule{3.3in}{
%   \mem(\loc) = \USED \spc % Type system cannot guarantee this.
    \redstocup{\loc}{(e,\mem[\loc \mapsto \LIVE]) }{(e',\mem')}
}{
    \redstoo{\Delta}{(\C{new} \; \RgnZ\inang{T}\inang{\toploc\loc}
                (e),\mem)}
            {(\C{new} \; \RgnZ\inang{T}\inang{\toploc\loc}
                (e'),\mem')} %[\loc\mapsto\USED])}
	      }{NewRegion}

% \lopsemrule{2.8in}{
%     \redsto{\Delta \cup \{\rgn_r\}}{(e,\mem)}{(e',\mem')} \spc
% %   \fgjN = \RgnZ\inang{T} \spc
%     \rgn_r \in dom(\mem)
% }{
%     \redstoo{\Delta}{(\C{new} \; \RgnZ\inang{T}\inang{\rgn_r}
%                 (e),\mem)}
%             {(\C{new} \; \RgnZ\inang{T}\inang{\rgn_r}
%                 (e'),\mem')}
%         }{NEW-REGION}

\lopsemrule{4in}{
    v_a = \C{new} \; \RgnZ\inang{T}\inang{\toploc\loc}(v) \spc
    \mem(\loc) = \USED ~\texttt{or}~ \mem(\loc) = \TLIVE \spc
%   \fresh(\rgn_1)\\
%   \rgn_0 \notin \rhoenv \\
    \mem' = \mem[\loc \mapsto \TLIVE]
%   \mem'' = \mem'[\rgn_r \mapsto \mem(\rgn_r)]
}{
    \redstoo{\Delta}{(\open{v_a}{\rgn}{x}{e_b},\mem)} 
            {(\opened{\loc}{\mem(\loc)}{[v/x][\loc/\rgn]e_b},\mem')}
	  }{Open}

\lopsemrule{2.7in}{
    \redstocup{\loc}{(e,\mem)}{(e',\mem')}
}{
    \redstoo{\Delta}{(\opened{\loc}{\status}{e},\mem)}
    {(\opened{\loc}{\status}{e'},\mem')}
  }{Opened}

\lopsemrule{2.7in}{
%   v_a = \C{new} \; \RgnZ\inang{T}\inang{\}(v_r) \spc
%   \mem(\rgn_r) \neq \XFERRED \spc
%   \rgn_0 \notin \rhoenv \spc
    \mem' = \mem[\loc \mapsto \status]
}{
    \redstoo{\Delta}{(\opened{\loc}{\status}{v},\mem)} {(v,\mem')}
  }{OpenEnd}


\lopsemrule{3in}{
    v_a = \C{new} \; \RgnZ\inang{T}\inang{\toploc\loc}(v) \spc
    \mem(\loc) \neq \USED ~\texttt{and}~ \mem(\loc) \neq \LIVE
%   \rgn_0 \notin \rhoenv \\
%   \redstocup{\rgn_0}{([[\rgn_0/\rgn]v_r/x]e_b,
%     \mem[\rgn \mapsto \OPEN])}{(e_b',\mem')}
}{
    \redstoo{\Delta}{(\open{v_a}{\rgn}{x}{e_b},\mem)} 
            {\invalidexn}
	  }{OpenTransferred}


\lopsemrule{3.25in}{
%   \ralloc \in \rhoenv \spc
%   \fbN = \RgnZ\inang{T}\inang{\rgn_r}\spc
%   \redstoo{\Delta}{(, \mem)}{(e',\mem')}
    \mem(\loc) = \USED \spc
    \mem' = \mem[\loc \mapsto \XFERRED]
}{
    \redstoo{\Delta}{((\C{new}\;\RgnZ\inang{T}\inang{\toploc\loc}(v)).\transfer(),\mem)}
            {(\unitval,\mem')}
	  }{Transfer}

\lopsemrule{2.5in}{
%   \fbN = \RgnZ\inang{T}\inang{\rgn}\spc
%   \ralloc \in \rhoenv \spc
%   \redstoo{\Delta}{(, \mem)}{(e',\mem')}
    \mem(\loc) = \TLIVE \spc
}{
    \redstoo{\Delta}{((\C{new}\;\RgnZ\inang{T}\inang{\toploc\loc}(v)).\transfer(),\mem)}
            {\invalidexn}
	  }{TransferOpened}

% \lopsemrule{2.5in}{
%     \not\exists \loc.~\mem(\loc) = \SLIVE \spc
%     \mem(\loc') = \XFERRED
% }{
%     \redstoo{\Delta}{(e,\mem)}{(e,\mem[\loc'\mapsto\FREE])}
%     }{GARBAGE-COLLECT}


\caption{\fbname: Operational semantics (part 2)}
\label{fig:fb-opsem-2}
\end{figure*}

Figs.~\ref{fig:fb-opsem-1} and~\ref{fig:fb-opsem-2} show the
operational semantics of \fbname (Fig.~\ref{fig:fb-opsem-2} is also
contained in the main paper as Fig.~\ref{fig:fb-opsem}. It is included
here for the sake of completeness). 
