%\section{Operational Semantics}
%
%Fig.~\ref{fig:fb-opsem} contains a definition of the operational semantics for \fbname.
%
\section{\fbname}

\subsection{Full Syntax}
\renewcommand{\rgn}{r}
\renewcommand{\rbar}{\overline{r}}
\begin{figure*}[t!]
%
\begin{smathpar}
\renewcommand{\arraystretch}{1.2}
\begin{array}{lclcl} 
\multicolumn{5}{c}{
  {\pi} \in \mathtt{Static \; region \; ids} \qquad
  {\rho} \in \mathtt{Region \; variables} \qquad
  {\loc} \in \mathtt{Memory\; locations} }\\
\multicolumn{5}{c}{
  {\tyvar, \tyvarb} \in \mathtt{Type \; variables} \qquad
  {m} \in \mathtt{Method \; names} \qquad
  {x,y,f} \in \mathtt{Variables \; and \; fields} }\\
cn & \in & \M{Class \; names} & \coloneqq & \ObjZ \ALT \RgnZ \ALT A \ALT B\\
     \fgjN & \in & \M{FGJ \; class \; types} & \coloneqq & cn\inang{\tbar}\\
T  & \in & \M{FGJ \; types} & \coloneqq & \tyvar \ALT  \fgjN \ALT \unitZ
     \ALT \bar{T} \rightarrow T \\
r  & \in & \M{Region\; annotations} & \coloneqq & \rho \ALT \pi \ALT \loc\\
\status & \in & \mathtt{Region\; Typestate} & \coloneqq & \FREE \ALT
                  \USED \ALT \LIVE \ALT \XFERRED\\
\fbN  & \in & \M{Region-annotated \; class \; types} & \coloneqq & 
     cn\inang{\tbar}\inang{\rbar} \\
\tau &\in& \M{types} & \coloneqq & T@\rgn  
      \ALT \fbN \ALT \unitZ 
      \ALT \inang{\rhobar \,|\, \phi}\bar{\tau}
      \xrightarrow{\rgn} \tau \\
C  & \in & \M{Class \; definitions} & \coloneqq & 
     \C{class} \; cn\inang{\bar{\tyvar} \extends \bar{\fgjN}} 
                    \inang{\rhobar \,|\, \phi}\extends \fbN 
                    \{\bar{\tau} \; \bar{f};\; \bar{d}\}\;\\
%k  & \in & \M{Constructors} & \coloneqq & 
%     cn(\bar{\tau} \; \bar{x})\{\C{super}(\bar{x}); \;
%                                \C{this}.\bar{f}\,=\,\bar{x};\}\\
d  & \in & \M{Methods} & \coloneqq & 
     \tau \; m\inang{\rhobar \,|\, \phi} (\taubar \; \xbar)
     \{\C{return}\;e;\}\\
\phi,\phicx &\in& \M{Region\;constraints} & \coloneqq & true 
      \ALT \rgn \outlives \rgn \ALT \rgn = \rgn \ALT \phi \conj \phi\\
e  & \in & \M{Expressions} & \coloneqq & \unitval \ALT x \ALT e.f 
     \ALT e.m\inang{\rbar}(\ebar) \ALT \C{new}\;\fbN(\ebar)
     \ALT \lambdaexp{\rgn}{\rhobar \,|\, \phi}
                    {\xbar:\taubar} {e}
           \ALT e\inang{\rbar}(\bar{e})\\
   & & & & \ALT \letexp{x}{e}{e} \ALT \letregion{\rgn}{e} 
           \ALT \open{e}{\rgn}{y}{e}\\
   & & & & \ALT \letd{\loc}{e} \ALT \opened{\loc}{\status}{e}\\
\end{array}
\end{smathpar}

\caption{\fbname: Syntax}
\label{fig:fb-syntax}
\end{figure*}

\renewcommand{\rgn}{\pi}
\renewcommand{\rbar}{\overline{\pi}}


The full syntax of \FB, including expressions that manifest only at
runtime, is shown in Fig.~\ref{fig:fb-syntax-full}. Such expressions
include:
\begin{itemize}

\item Memory locations ($\loc$) corresponding to the memory regions,

\item A special expression $\RgnZT{\toploc\loc}(e)$ that evaluates to
a region handler object (the syntax of this expression is captured by
the $\C{new}\;\fbN(\overline{e})$ construct). $\toploc$ is the
location of the special $\toprgn$ region where region handlers are
stored. $\loc$ is the location of the corresponding transferable
region.

\item A $\letd{\loc}{e}$ expression that results when
$\letregion{\pi}{e}$ expression takes a step (see
Fig.~\ref{fig:fb-opsem-2}).  $\loc$ is the location of the newly
allocated region.  

\item An $\opened{\loc}{s}{e}$ expression that
results when $\C{open}\;{\RgnZT{\toploc\loc}(v)} ...$ expression takes
a step. $\loc$ is the location of the newly open transferable region.
The symbol $s$ denotes the typestate of the
transferable region before it is opened\footnote{Operational semantics
lets a transferable region to be opened while it is already open. This
allows methods to safely open a transferable region argument
regardless of the calling context.}. The typestate can assume the
values of $\USED$ (allocated and closed), $\LIVE$ (allocated and
live), and $\XFERRED$ (transferred or freed).

\end{itemize}

\subsection{Full Static Semantics}

\begin{figure*}[t]
%
\begin{minipage}{2.25in}
\begin{smathpar}
\begin{array}{lcl}
  allocRgn(A\inang{\ralloc\rbar}\inang{\tbar}) & = & \ralloc\\
  allocRgn(\inang{\rhoalloc\rhobar \,|\, \phi}\bar{\tau^1}
      \xrightarrow{\ralloc} \tau^2) & = & \ralloc\\
  shape(A\inang{\rhoalloc\rhobar}\inang{\tbar}) & = & A\inang{\tbar}\\
  bound_{\aenv}(\tyvar@\ralloc) & = & \aenv(\tyvar)@\ralloc\\
  bound_{\aenv}(\fbN) & = & \fbN\\
\end{array}
\end{smathpar}
\end{minipage}
%
\begin{minipage}{1.8in}
\begin{smathpar}
\begin{array}{c}
\renewcommand*{\arraystretch}{1.2}
\RULE
  {
    \\
    B \in \{\ObjZ,\RgnZ\}
  }
  {
    fields(B\inang{\ralloc\rbar}\inang{\tbar}) \;=\; \bullet
  }
\end{array}
\end{smathpar}
\end{minipage}
%
\begin{minipage}{3in}
\begin{smathpar}
\begin{array}{c}
\renewcommand*{\arraystretch}{1.2}
\RULE
  {
    CT(B) = \headerOf{B}\{\bar{\tau^f}\;\bar{f};\,...\}\\
    \substFn = [\rbar/\rhobar, \ralloc/\rhoalloc, \tbar/\bar{\tyvar}] \qquad 
    fields(\substFn(\fbN)) = \bar{g}:\bar{\tau^g}
  }
  {
    fields(B\inang{\ralloc\rbar}\inang{\tbar}) \;=\;
      \bar{g}:\bar{\tau^g},\,\bar{f}:\substFn(\bar{\tau^f})
  }
\end{array}
\end{smathpar}
\end{minipage}
%
\bigskip

\begin{minipage}{3.5in}
\begin{smathpar}
\begin{array}{lcl}
  ctype(\ObjZ\inang{\rgn}) & = & \bullet \\
% ctype(\RgnZ\inang{\rgn}\inang{T}) & = & \inang{\rhoalloc}
%   {\unitZ}\rightarrow{T@\rhoalloc}\\
  ctype(B\inang{\ralloc\rbar}\inang{\tbar}) & = & 
    fields(B\inang{\ralloc\rbar}\inang{\tbar})\\
  mtype(\C{transfer}, \RgnZ\inang{\rgn}\inang{T}) & = & 
    \inang{\rhoalloc} {\unitZ}\rightarrow{\unitZ}\\
  mtype(\C{free}, \RgnZ\inang{\rgn}\inang{T}) & = & 
    \inang{\rhoalloc} {\unitZ}\rightarrow{\unitZ}\\
\end{array}
\end{smathpar}
\end{minipage}
%
\begin{minipage}{3in}
\begin{smathpar}
\begin{array}{c}
\renewcommand*{\arraystretch}{1.2}
\RULE
  {
    CT(B) = \headerOf{B}\{\bar{\tau^f}\;\bar{f};\,k\;\bar{d}\}\\
    m \notin \bar{d} \qquad 
    \substFn = [\rbar/\rhobar, \ralloc/\rhoalloc, \tbar/\bar{\tyvar}]
  }
  {
    mtype (m,B\inang{\ralloc\rbar}\inang{\tbar}) \;=\;
    mtype (m, \substFn(\fbN))
  }
\end{array}
\end{smathpar}
\end{minipage}
%
\bigskip

\begin{minipage}{3.25in}
\begin{smathpar}
\begin{array}{c}
\renewcommand*{\arraystretch}{1.2}
\RULE
  {
    CT(B) = \headerOf{B}\{\bar{\tau^f}\;\bar{f};\,k\;\bar{d}\}\\
    \tau^2 \; m\mang (\bar{\tau^1}\;\bar{x})\{...\} \in \bar{d} \qquad
    \substFn = [\rbar/\rhobar, \ralloc/\rhoalloc, \tbar/\bar{\tyvar}]
  }
  {
    mtype (m,B\inang{\ralloc\rbar}\inang{\tbar}) \;=\;
    \substFn(\mang\bar{\tau^1} \rightarrow \tau^2)
  }
\end{array}
\end{smathpar}
\end{minipage}
%
\begin{minipage}{3.5in}
\begin{smathpar}
\begin{array}{c}
\renewcommand*{\arraystretch}{1.2}
\RULE
  {

    \substFn = \subst{\bar{\rho_2}}{\bar{\rho_1}}
               \subst{\rhoalloc_2}{\rhoalloc_1} \spc
    mtype(m,\fbN) = \inang{\rhoalloc_1\bar{\rho_1},|\, \phi_1}\bar{\tau^{11}} 
                      \rightarrow \tau^{12} \spc \texttt{implies}\\
    \isvalid{\A.\phicx}{\phi_2 \Leftrightarrow \substFn(\phi_1)} 
        \spc \texttt{and} \spc
    \bar{\tau^{21}} = \substFn(\bar{\tau^{11}}) \spc \texttt{and} \spc
    \subtyp{\A}{\tau^{22}} {\substFn(\tau^{12})}
    %\substFn = [\rbar/\rhobar, \ralloc/\rhoalloc, \tbar/\bar{\tyvar}]
  }
  {
    override(\A,\fbN,\inang{\rhoalloc_2\bar{\rho_2},|\, \phi_1}
              \bar{\tau^{21}} \rightarrow \tau^{22})
  }
\end{array}
\end{smathpar}
\end{minipage}
%
\bigskip

\begin{minipage}{5in}
\begin{smathpar}
\begin{array}{c}
  \rhoset,\rhoenv \in 2^{\rho} \qquad
  \aenv \in \tyvar \rightarrow \fgjN \qquad
  \A = (\subtypcx)\\
\end{array}
\end{smathpar}
\end{minipage}
%

\caption{\fbname: Auxiliary Definitions}
\label{fig:fb-auxdef}
\end{figure*}

\begin{figure*}[!ht]
%
\textbf{Subtyping}  \; \fbox
  {\(\subtyp{\A}{\tau_1}{\tau_2}\)}\\
%
\begin{minipage}{1.8in}
\begin{smathpar}
\begin{array}{c}
\renewcommand*{\arraystretch}{1.2}
  \subtyp{\A}{\tau}{\tau} \\
  \subtyp{(\Delta,\aenv,\phicx)}{\tyvar @\rho}{\aenv(\tyvar) @\rho}\qquad
% \subtyp{\A}{\RgnZ\inang{\rgn}}{\RgnZ\inang{\toprgn}}\qquad
% \subtyp{\A}{\RgnZ\inang{\toprgn}}{\RgnZ\inang{\rgn}}
\end{array}
\end{smathpar}
\end{minipage}
%
\begin{minipage}{2.55in}
\begin{smathpar}
\begin{array}{c}
\renewcommand*{\arraystretch}{1.2}
\RULE
  {
    \\
    CT(B) = \headerOf{B}\{...\}
  }
  {
    \subtyp{\A}{B\inang{\tbar}\inang{\rbar}}
        {[\rbar/\rhobar, \tbar/\bar{\tyvar}](\fbN)}
  }
\end{array}
\end{smathpar}
\end{minipage}
%

\begin{minipage}{1.5in}
\begin{smathpar}
\begin{array}{c}
\renewcommand*{\arraystretch}{1.2}
\RULE
  {
    \subtyp{\A}{\tau_1}{\tau_2}\\
    \subtyp{\A}{\tau_2}{\tau_3}
  }
  {
    \subtyp{\A}{\tau_1}{\tau_3}
  }
\end{array}
\end{smathpar}
\end{minipage}
%
\begin{minipage}{2.75in}
\begin{smathpar}
\begin{array}{c}
\renewcommand*{\arraystretch}{1.2}
\RULE
  {
    \isvalid{\A.\phicx}{\phi_1 \Rightarrow \phi_2} \\
    \subtyp{\A}{\bar{\tau^{11}}}{\bar{\tau^{21}}} \spc
    \subtyp{\A}{\tau^{22}}{\tau^{12}}
  }
  {
    \subtyp{\A}
      {\inang{\rhobar \,|\, \phi_2}\bar{\tau^{21}}
          \xrightarrow{\rgn} \tau^{22}}
      {\inang{\rhobar \,|\, \phi_1}\bar{\tau^{11}}
          \xrightarrow{\rgn} \tau^{12}}
  }
\end{array}
\end{smathpar}
\end{minipage}


%
\bigskip

\textbf{Type, and Type Constraint Well-formedness}  \; \fbox
  {\(\tywf{\A}{\tau}, \spc 
     \tywf{\rhoenv}{\phi}\)}\\
%
\begin{minipage}{1.25in}
\begin{smathpar}
\begin{array}{c}
\renewcommand*{\arraystretch}{1.2}
\RULE
  {
    \\
    \\
    \rgn \in \rhoenv
  }
  {
    \tywf{(\rhoenv,\aenv,\phicx)}{\ObjZ\inang{\rgn}}
  }
\end{array}
\end{smathpar}
\end{minipage}
% 
\begin{minipage}{2.75in}
\begin{smathpar}
\begin{array}{c}
\renewcommand*{\arraystretch}{1.2}
\RULE
  {
    \rgn \in \rhoenv \spc
    \rhobar \not\in \rhoenv\\
%   \rhobar \notin dom(\mem) \\
%   \mem' = \mem[\rhobar \mapsto \overline{\LIVE}] \spc
    \rhoenv' = \rhoenv \cup \{\rhobar\} \spc
    \A' = (\rhoenv', \aenv, \phicx \conj \phi) \\
    \tywf{\rhoenv'}{\phi}\spc 
    \tywf{\A'}{\bar{\tau^1}} \spc
    \tywf{\A'}{\tau^2}
  }
  {
    \tywf{(\rhoenv,\aenv,\phicx)}{\inang{\rhobar \,|\, \phi}
              \bar{\tau^1} \xrightarrow{\rgn} \tau^2}
  }
\end{array}
\end{smathpar}
\end{minipage}
%
\begin{minipage}{1.5in}
\begin{smathpar}
\begin{array}{c}
\renewcommand*{\arraystretch}{1.2}
\RULE
  { 
    \\
    \\
    \fgjtywf{\A.\aenv}{T}
  }
  {
    \tywf{\A}{\RgnZ\inang{T}\inang{\toprgn}}
  }
\end{array}
\end{smathpar}
\end{minipage}
%
\begin{minipage}{1in}
\begin{smathpar}
\begin{array}{c}
\renewcommand*{\arraystretch}{1.2}
\RULE
  {
    \\
    \\
    \rgn_0,\rgn_1 \in \rhoenv
%   \mem(\rgn_0) = \LIVE \spc
%   \mem(\rgn_1) = \LIVE
  }
  {
    \tywf{\rhoenv}{\rgn_0 \outlives \rgn_1}
  }
\end{array}
\end{smathpar}
\end{minipage}
%

%
\begin{minipage}{3.5in}
\begin{smathpar}
\begin{array}{c}
\renewcommand*{\arraystretch}{1.2}
\RULE
  {
    CT(B) = \headerOf{B}\{...\}\\
    \rbar \in \rhoenv \spc
%   \mem(\rbar) = \overline{\LIVE} \spc
    \fgjtywf{\aenv}{B\inang{\tbar}}\spc
%   \substFn = [\rbar/\rhobar, \tbar/\bar{\tyvar}] \spc
    \isvalid{\phicx}{[\rbar/\rhobar](\phi)}
  }
  {
    \tywf{(\rhoenv,\aenv,\phicx)}{B\inang{\rbar}\inang{\tbar}}
  }
\end{array}
\end{smathpar}
\end{minipage}
%
\begin{minipage}{1.65in}
\begin{smathpar}
\begin{array}{c}
\renewcommand*{\arraystretch}{1.2}
\RULE
  {
    \fgjtywf{\aenv}{T}\spc
%   \rgn \in \A.\rhoenv\\
    \rgn \in \rhoenv\\
    \fgjsubtyp{\aenv}{T}{\ObjZ}\spc
  }
  {
    \tywf{(\rhoenv,\aenv,\phicx)}{T@\rgn}
  }
\end{array}
\end{smathpar}
\end{minipage}
%
\begin{minipage}{0.75in}
\begin{smathpar}
\begin{array}{c}
\renewcommand*{\arraystretch}{1.2}
\RULE
  {
    \tywf{\rhoenv}{\phi_0} \\ \tywf{\rhoenv}{\phi_1}
  }
  {
    \tywf{\rhoenv}{\phi_0 \wedge \phi_1}
  }
\end{array}
\end{smathpar}
\end{minipage}

%
\bigskip

/Users/gowtham/git/broom/fullversion/broom/paper/fb-morewfrules.tex
%

\caption{\fbname: Subtyping and well-formedness rules}
\label{fig:fb-staticsem-1}
\end{figure*}

\begin{figure*}[!ht]
\textbf{Expression Typing}  \; \fbox
  {\(\hastyp{\exptycx{\env}{\rgn}}{e}{\tau}\)}\\
%
\begin{minipage}{1.2in}
\begin{smathpar}
\begin{array}{l}
\renewcommand*{\arraystretch}{1.2}
\hastyp{\exptycx{\env}}{\unitval}{\unitZ}\\
\hastyp{\exptycx{\env}}{x}{\env(\tau)}
\end{array}
\end{smathpar}
\end{minipage}
%
\begin{minipage}{1.8in}
\begin{smathpar}
\begin{array}{c}
\renewcommand*{\arraystretch}{1.2}
\RULE
  {
    \hastyp{\exptycx{\env}}{e}{\tau'}\\
    \bar{f}:\taubar \,=\, \fields(\bound_{\A.\aenv}(\tau'))
  }
  {
    \hastyp{\exptycx{\env}}{e.f_i}{\tau_i}
  }
\end{array}
\end{smathpar}
\end{minipage}
%
\begin{minipage}{1.2in}
\begin{smathpar}
\begin{array}{c}
\renewcommand*{\arraystretch}{1.2}
\RULE
  {
    \hastyp{\exptycx{\env}}{\bar{e}}{\taubar} \\
    \tywf{\A}{\fbN} \spc
    \fields(\fbN) = \bar{f} : \taubar \\
  }
  {
    \hastyp{\exptycx{\env}}{\C{new}\; \fbN(\bar{e})}{\fbN}
  }
\end{array}
\end{smathpar}
\end{minipage}
%

\begin{minipage}{1.8in}
\begin{smathpar}
\begin{array}{c}
\renewcommand*{\arraystretch}{1.2}
\RULE
  {
    \hastyp{\exptycx{\env}}{e_1}{\tau_1}\\
    \hastyp{\exptycx{\env[x\mapsto\tau_1]}}{e_2}{\tau_2}\\
  }
  {
    \hastyp{\exptycx{\env}}{\letexp{x}{e_1}{e_2}}{\tau_2}
  }
\end{array}
\end{smathpar}
\end{minipage}
%
%
\begin{minipage}{3.2in}
\begin{smathpar}
\begin{array}{c}
\renewcommand*{\arraystretch}{1.2}
\RULE
  {
%   \A = (\mem,\aenv,\phicx)\spc
%   \fgjtywf{\aenv}{T} \\
%   \rgn \in \A.\rhoenv \spc
%   \mem(\rgn) = \LIVE \spc
%   \hastyp{(\rhoenv\cup\{\rho\},\aenv,\phicx),{\env}}{e}
%     {T@\rho}
%   We don't need the following condition because of the
%   guarantee provided by the lambda type rule.
%   \A = (\subtypcx) \spc
%   \rgn \in \rhoenv \\
    \\
    \hastyp{\exptycx{\env}}{e} {\inang{\rho} \unitZ \xrightarrow{\rgn} T@\rho} \spc
  }
  {
    \hastyp{\exptycx{\env}}{\C{new}\;
    \RgnZ\inang{T}\inang{\toprgn}(e)} {\RgnZ\inang{T}\inang{\toprgn}}
  }
\end{array}
\end{smathpar}
\end{minipage}

%
\begin{minipage}{5in}
\begin{smathpar}
\begin{array}{c}
\renewcommand*{\arraystretch}{1.2}
\RULE
  {
    \A = (\subtypcx)\spc
    \A' = (\rhoenv\cup\{\loc\}, \aenv,\phicx) \spc
%   \mem(\loc) = \USED \spc
    \tywf{\A'}{T@\loc} \spc
%   \fgjN = \RgnZ\inang{T}\spc
    \hastyp{\A', {\env}}{e}{T@\loc}
  }
  {
    \hastyp{\exptycx{\env}}{\C{new}\;
    \RgnZT{\toploc\loc}(e)} {\RgnZT{\toprgn}}
  }
\end{array}
\end{smathpar}
\end{minipage}
%

%
\begin{minipage}{2.5in}
\begin{smathpar}
\begin{array}{c}
\renewcommand*{\arraystretch}{1.2}
\RULE
  {
    \A = (\subtypcx) \spc
    \hastyp{\exptycx{\env}}{e_0}{\tau} \\
%   \rbar \in \A.\rhoenv \\
    \mtype(m,\bound_{\aenv}(\tau)) = \inang{\rhobar \,|\, 
        \phi}\bar{\tau^1}\rightarrow{\tau^2} \\
    \rbar \in \rhoenv \spc
    \tywf{\A}{\inang{\rhobar \,|\,
    \phi}\bar{\tau^1}\rightarrow{\tau^2}}\\
    \hastyp{\exptycx{\env}}{\bar{e}}{[\rbar/\rhobar](\bar{\tau^1})} \spc
    \isvalid{\phicx}{[\rbar/\rhobar](\phi)}
  }
  {
    \hastyp{\exptycx{\env}}{e_0.m\inang{\rbar}(\bar{e})} 
           {[\rbar/\rhobar](\tau^2)}
  }
\end{array}
\end{smathpar}
\end{minipage}
%
\begin{minipage}{2.5in}
\begin{smathpar}
\begin{array}{c}
\renewcommand*{\arraystretch}{1.2}
\RULE
  {
    \\
    \A = (\subtypcx) \spc
    \rgn \notin \rhoenv \spc
%   \Delta = \{\rgn \,|\, \rgn \in \rhoenv\} \spc
    \phi = \Delta \outlives \rgn \\
    \A' = (\rhoenv\cup\{\rgn\}, \aenv, \phicx \conj \phi)\\
    \hastyp{\A',\env}{e}{\tau} \spc
    \tywf{\A}{\tau}
  }
  {
    \hastyp{\exptycx{\env}}{\letregion{\rgn}{e}}{\tau}
  }
\end{array}
\end{smathpar}
\end{minipage}
%

%
\begin{minipage}{2.5in}
\begin{smathpar}
\begin{array}{c}
\renewcommand*{\arraystretch}{1.2}
\RULE
  {
    \A = (\subtypcx) \spc
%   \rgn \in \A.\rhoenv \spc
    \rgn \in \rhoenv \\
%   \rhobar \notin dom(\mem)\\
    \rhobar \notin \rhoenv \spc
%   \forall (\rgn'\in\Delta).~(\isvalid{\phicx}{\rgn\outlives\rgn'})

%           ~\Rightarrow~ \rgn = \rgn'\\
    \forall (\rgn'\in\frv(e)\setminus\{\rhobar\}).~\isvalid{\phicx}{\rgn'\outlives\rgn}\\
    \A' = (\rhoenv \cup \{\rhobar\}, \aenv, 
    \A' = (\rhoenv \cup \{\rhobar\}, \aenv, 
          \phicx \conj \phi)\\
    \tywf{\rhoenv \cup \{\rhobar\}}{\phi}\spc
    \tywf{\A'}{\bar{\tau^1}}\\
    \tywf{\A'}{\tau^2}\spc
    \hastyp{\A',\env[\xbar \mapsto \bar{\tau^1}]}{e}{\tau^2}
  }
  {
    \hastyp{\exptycx{\env}}
           {\lambdaexp{\rgn}{\rhobar \,|\, \phi}
                      {\xbar:\bar{\tau^1}}{e}}
           {\inang{\rhobar \,|\, \phi}
            \bar{\tau^1} \xrightarrow{\rgn} \tau^2}
  }
\end{array}
\end{smathpar}
\end{minipage}
%
\begin{minipage}{2.5in}
\begin{smathpar}
\begin{array}{c}
\renewcommand*{\arraystretch}{1.2}
\RULE
  {
    \A = (\subtypcx) \spc
    \rgn \notin \rhoenv \\
    \hastyp{\exptycx{\env}}{e_a}
            {\RgnZ\inang{T}\inang{\toprgn}}\\
%   \rgn \notin dom(\mem) \spc
    \A' = (\rhoenv\cup\{\rgn\},\aenv,\phicx) \spc
    \tywf{\A}{\tau} \\
    \env' =  \env[y\mapsto T@\rgn]\spc
    \hastyp{\A',\env'}{e_b}{\tau} \spc
  }
  {
    \hastyp{\exptycx{\env}}{\open{e_a}{\rgn}{y}{e_b}}
            {\tau}
  }
\end{array}
\end{smathpar}
\end{minipage}
%

%
\begin{minipage}{2.5in}
\begin{smathpar}
\begin{array}{c}
\renewcommand*{\arraystretch}{1.2}
\RULE
  {
    \A = (\subtypcx) \spc
    \rbar \in \rhoenv \\
    \hastyp{\exptycx{\env}}{e}
        {\inang{\rhobar \,|\, \phi}
            \bar{\tau^1} \xrightarrow{\rgn} \tau^2}\\
%   \substFn = \subst{\rbar}{\rhobar} \spc
    \isvalid{\phicx}{\subst{\rbar}{\rhobar}\phi} \spc
    \hastyp{\exptycx{\env}}{\bar{e}}
        {\subst{\rbar}{\rhobar}\bar{\tau^1}}\spc
%   \subtyp{\A}{\taubar}{\bar{\tau^1}}
  }
  {
    \hastyp{\A,\env}{e\inang{\rbar}(\bar{e})}
           {\subst{\rbar}{\rhobar}\tau^2}
  }
\end{array}
\end{smathpar}
\end{minipage}
%
\begin{minipage}{2.5in}
\begin{smathpar}
\begin{array}{c}
\renewcommand*{\arraystretch}{1.2}
\RULE
  {
%   \A = (\subtypcx)\\
%   \loc \in \rhoenv \spc
%   \hastyp{\exptycx{\env}}{e}{\tau}
    \A = (\subtypcx) \spc
    \loc \notin \rhoenv \\
    \A' = (\rhoenv\cup\{\loc\}, \aenv, \phicx \conj \Delta \outlives
    \loc)\\
    \hastyp{\A',\env}{e}{\tau} \spc
    \tywf{\A}{\tau}
  }
  {
    \hastyp{\exptycx{\env}}{\letd{\loc}{e}}{\tau}
  }
\end{array}
\end{smathpar}
\end{minipage}
%

%
\begin{minipage}{1.95in}
\begin{smathpar}
\begin{array}{c}
\renewcommand*{\arraystretch}{1.2}
\RULE
  {
%   \A = (\subtypcx)\\
%   \loc \in \rhoenv \spc
%   \hastyp{\exptycx{\env}}{e}{\tau}
    \A = (\subtypcx) \spc
    \loc \notin \rhoenv \\
%   \loc \notin dom(\mem) \spc
    \A' = (\rhoenv\cup\{\loc\},\aenv,\phicx) \\
    \hastyp{\A',\env}{e}{\tau} \spc
    \tywf{\A}{\tau}
  }
  {
    \hastyp{\exptycx{\env}}{\opened{\loc}{s}{e}}{\tau}
  }
\end{array}
\end{smathpar}
\end{minipage}
%
%
\begin{minipage}{1.6in}
\begin{smathpar}
\begin{array}{c}
\renewcommand*{\arraystretch}{1.2}
\RULE
  {
    \\
    \\
    \hastyp{\exptycx{\env}}{e}{\RgnZ\inang{T}\inang{\toprgn}}
  }
  {
    \hastyp{\exptycx{\env}}{e.\transfer()}{\unitZ}
  }
\end{array}
\end{smathpar}
\end{minipage}
%
\begin{minipage}{1in}
\begin{smathpar}
\begin{array}{c}
\renewcommand*{\arraystretch}{1.2}
\RULE
  {
    \\
    \hastyp{\exptycx{\env}}{e}{\tau} \\
    \subtyp{\A}{\tau}{\tau'}
  }
  {
    \hastyp{\exptycx{\env}}{e}{\tau'}
  }
\end{array}
\end{smathpar}
\end{minipage}
%

\caption{\fbname: Expression typing}
\label{fig:fb-staticsem-2}
\end{figure*}


Figs.~\ref{fig:fb-staticsem-1} and~\ref{fig:fb-staticsem-2} show full
static semantics of \FB. Fig.~\ref{fig:fb-staticsem-1} contains subtyping
rules, and rules to check well-formedness of \FB types, type
constraints, methods, and class definitions. Method well-formedness
rule makes use of the expression typing judgment defined in
Fig.~\ref{fig:fb-staticsem-2}. Auxiliary definitions used in
Figs.~\ref{fig:fb-staticsem-1} and~\ref{fig:fb-staticsem-2} are
defined in Fig.~\ref{fig:fb-auxdef}. As described in
\S~\ref{sec:type-system}, all judgments are parameterized over the
class table ($CT$). The premise $CT(B) = \headerOf{B}\{...\}$
used in some of the rules means that the definition of class $B$ is
present in $CT$ and that the definition is well-formed.

\subsection{Operational Semantics}

\newcommand{\opsemrule}[3]{%
\begin{minipage}{#1}\begin{smathpar}\begin{array}{c}%
\renewcommand*{\arraystretch}{1.2}%
\RULE {#2} {#3}%
\end{array}\end{smathpar}\end{minipage}%
}
\newcommand{\lopsemrule}[4]{%
\begin{minipage}{#1}\begin{smathpar}\begin{array}{c}%
\renewcommand*{\arraystretch}{1.2}%
[\rulelabel{#4}] \spc \RULE {#2} {#3}%
\end{array}\end{smathpar}\end{minipage}%
}
%
\newcommand{\anobjty}[0]{B\inang{\tbar}\inang{\ralloc\locbar}}
\newcommand{\anobj}[0]{\C{new} \; \anobjty(\bar{v})}
%

\begin{figure*}[t!]

%
\fbox {\(\redstoo{\Delta}{(e,\mem)}{(e',\mem')}\)}\\

\lopsemrule{1.5in}{
    \redstoo{\Delta}{(e,\mem)}{(e',\mem')}
}{
    \redstoo{\Delta}{(E\lbrack e \rbrack, \mem)}
            {(E\lbrack e' \rbrack, \mem')}
	  }{EVAL-ORDER}

\lopsemrule{1.2in}{
    \redstoo{\Delta}{(e,\mem)}{\invalidexn}
  }{
    \redstoo{\Delta}{(E\lbrack e \rbrack, \mem)}
            {\invalidexn}
	  }{EXCEPTION}
%

\lopsemrule{2.7in}{
%   \allocRgn(A\inang{\rgn\locbar}\inang{\tbar}) & = & \rgn
%   \allocRgn(\fbN) \in \rhoenv \spc
    \mem(\loc) = \LIVE \spc
    \fields(A\inang{\tbar}\inang{\loc\locbar}) = \bar{f}:\taubar
}{
    \redstoo{\Delta}{((\C{new} \; A\inang{\tbar}\inang{\loc\locbar}(\vbar)).f_i,\mem)}{(v_i,\mem)}
}{FIELD-ACCESS}

\lopsemrule{3.85in}{
%   \allocRgn(\fbN) \in \rhoenv \spc
    \mem(\loc) = \LIVE \spc
    \mbody(m,A\inang{\tbar}\inang{\loc\locbar}) = \rhobar.\bar{x}.e 
%   \redstoo{\Delta}{(, \mem)}{(e',\mem')}
}{
    \redstoo{\Delta}{((\C{new}\;A\inang{\tbar}\inang{\loc\locbar}(\bar{v})).m\inang{\overline{\loc'}}
                      (\bar{v'}),\mem)}
            {([\overline{\loc'}/\rhobar][\bar{v'}/\xbar]
                [\C{new} \; A\inang{\tbar}\inang{\loc\locbar}(\bar{v})/\thisZ]\,e,\mem)}
	  }{METHOD-INV}

\lopsemrule{2.85in}{
    v_a = \lambdaexp{\loc}{\rhobar}
                        {\taubar \; \xbar}{e} \spc
    \mem(\loc) = \LIVE  \spc 
}{
    \redstoo{\Delta}{(v_a\inang{\locbar}(\bar{v}) ,\mem)}
            {([\bar{v}/\xbar][\locbar/\rhobar]\,e,\mem)}
	  }{FUN-APPLY}

%
% \lopsemrule{2in}{
% %   \rgn \notin \rhoenv \spc
% %   \fresh(\rgn') \spc
%     \redstoo{\Delta}{(e,\mem)}{\invalidexn}
% }{
%     \redstoo{\Delta}{(\letregion{\rgn}{e},\mem)}{\invalidexn}
%   }{EXCEPTION}

\lopsemrule{2in}{
%   \fgjN = \RgnZ\inang{T} \spc
%   \rgn \in dom(\mem) \spc
    \mem(\loc) = \USED \spc
    \redstocup{\loc}{(e,\mem[\loc \mapsto \LIVE])}{\invalidexn}
}{
    \redstoo{\Delta}{(\C{new} \; \RgnZT{\toploc\loc} (e),\mem)}
            {\invalidexn}
	  }{EXCEPTION}

% \lopsemrule{3.2in}{
%     \not\exists \loc.~\mem(\loc) = \FREE
% }{
%     \redstoo{\Delta}{(\letregion{\rgn}{e},\mem)}{\invalidexn}
%   }{EXCEPTION}


% \lopsemrule{3.3in}{
%     \not\exists \loc.~\mem(\loc) = \FREE
% }{
%     \redstoo{\Delta}{(\C{new} \; \RgnZ\inang{T}\inang{\toprgn}
%                 (\lambdaexp{\loc'}{\rho}{}{e}),\mem)}
%             {\invalidexn}
%         }{EXCEPTION}
% \lopsemrule{2.5in}{
%     \mem(\rgn_r) \neq \XFERRED \spc
% %   \fresh(\rgn_1) \spc
% %  \rgn_0 \notin \rhoenv \\
%     \redsto{\Delta \cup \{\rgn\}}{([[\rgn/\rgn_r]v_r/x]e_b,
%       \mem[\rgn_r \mapsto \OPEN])}{\invalidexn}
% }{
%     \redstoo{\Delta}{(\open{(\C{new} \; \RgnZ\inang{T}\inang{\rgn_r}(v_r))}
%                    {\rgn}{x}{e_b},\mem)} 
%             {\invalidexn}
%     }{EXCEPTION}

%
\bigskip

\textbf{Evaluation Context} \fbox {\(E\)}\\
\begin{smathpar}
\begin{array}{lcl}
E & \coloneqq & \bullet \ALT (\bullet).f \ALT \bullet.m\inang{\locbar}(\ebar) \ALT
      v.m\inang{\locbar}(...,\bullet,...) \ALT \C{new}\; \fbN(...,\bullet,...) \ALT
      \C{new} \; \RgnZ\inang{T}\inang{\toprgn}(\bullet) \ALT 
      \bullet\inang{\locbar}(\ebar) \\
  &  & \ALT v\inang{\locbar}(...,\bullet,...) \ALT
       \letexp{x}{\bullet}{e} \ALT \open{\bullet}{\rgn}{y}{e} 
%      \ALT \opened{\loc}{\status}{\bullet} \ALT \letd{\loc}{\bullet}
%     The following should be forbidden. See NEW-REGION rule 2.
%      \ALT \C{new} \; \RgnZ\inang{T}\inang{\toploc\loc}(\bullet)
\end{array}
\end{smathpar}

\caption{\fbname: Operational Semantics (Part 1)}
\label{fig:fb-opsem-1}
\end{figure*}

\begin{figure*}[t!]

\lopsemrule{3.2in}{
%   \fresh(\rgn')\spc
%   \mem(\loc) = \FREE \spc
    \loc \not\in dom(\mem) \spc
    \mem' = \mem[\loc \mapsto \SLIVE]
%   \redstoo{\Delta}{(e,\mem)}{(e',\mem')}
}{
    \redstoo{\Delta}{(\letregion{\rgn}{e},\mem)}{(\letd{\loc}{[\loc/\rgn]e},\mem')}
  }{LET-REGION-BEGIN}

\lopsemrule{3in}{
    \redstocup{\loc}{(e,\mem)}{(e',\mem')}
}{
    \redstoo{\Delta}{(\letd{\loc}{e},\mem)}{(\letd{\loc}{e'},\mem')}
  }{LET-REGION}


\lopsemrule{3in}{
%   \rgn \notin \rhoenv
    \mem' = \mem[\loc \mapsto \FREE]
}{
    \redstoo{\Delta}{(\letd{\loc}{v},\mem)}{(v,\mem)}
  }{LET-REGION-END}

\lopsemrule{3.3in}{
%   \fgjN = \RgnZ\inang{T} \spc
%   \rgn \in \rhoenv \spc
%   We need Delta here to ensure Delta U {pi_r} in next rule is sound. 
%   \rgn_r \notin \Delta \cup dom(\Sigma) \spc
%   \rgn \notin dom(\mem) \cup \rhoenv \spc
%   \mem' = \mem[\rgn_r \mapsto \CLOSED]
%   \mem(\loc) = \LIVE \spc
%   \mem(\loc') = \FREE \spc
    \loc \not\in dom(\mem) \spc
    \mem' = \mem[\loc \mapsto \USED]
}{
    \redstoo{\Delta}{(\C{new} \; \RgnZ\inang{T}\inang{\toprgn}
                (v),\mem)}
            {(\C{new} \; \RgnZ\inang{T}\inang{\toploc\loc}
                (v\inang{\loc}()),\mem')}
	      }{NEW-REGION}

\lopsemrule{3.3in}{
    \mem(\loc) = \USED \spc
    \redstocup{\loc}{(e,\mem[\loc \mapsto \LIVE]) }{(e',\mem')}
}{
    \redstoo{\Delta}{(\C{new} \; \RgnZ\inang{T}\inang{\toploc\loc}
                (e),\mem)}
            {(\C{new} \; \RgnZ\inang{T}\inang{\toploc\loc}
                (e'),\mem'[\loc\mapsto\USED])}
	      }{NEW-REGION}

% \lopsemrule{2.8in}{
%     \redsto{\Delta \cup \{\rgn_r\}}{(e,\mem)}{(e',\mem')} \spc
% %   \fgjN = \RgnZ\inang{T} \spc
%     \rgn_r \in dom(\mem)
% }{
%     \redstoo{\Delta}{(\C{new} \; \RgnZ\inang{T}\inang{\rgn_r}
%                 (e),\mem)}
%             {(\C{new} \; \RgnZ\inang{T}\inang{\rgn_r}
%                 (e'),\mem')}
%         }{NEW-REGION}

\lopsemrule{4in}{
    v_a = \C{new} \; \RgnZ\inang{T}\inang{\toploc\loc}(v) \spc
    \mem(\loc) = \USED ~\texttt{or}~ \mem(\loc) = \TLIVE \spc
%   \fresh(\rgn_1)\\
%   \rgn_0 \notin \rhoenv \\
    \mem' = \mem[\loc \mapsto \TLIVE]
%   \mem'' = \mem'[\rgn_r \mapsto \mem(\rgn_r)]
}{
    \redstoo{\Delta}{(\open{v_a}{\rgn}{x}{e_b},\mem)} 
            {(\opened{\loc}{\mem(\loc)}{[v/x][\loc/\rgn]e_b},\mem')}
	  }{OPEN}

\lopsemrule{2.7in}{
    \redstocup{\loc}{(e,\mem)}{(e',\mem')}
}{
    \redstoo{\Delta}{(\opened{\loc}{\status}{e},\mem)}
    {(\opened{\loc}{\status}{e'},\mem')}
  }{OPENED}

\lopsemrule{2.7in}{
%   v_a = \C{new} \; \RgnZ\inang{T}\inang{\}(v_r) \spc
%   \mem(\rgn_r) \neq \XFERRED \spc
%   \rgn_0 \notin \rhoenv \spc
    \mem' = \mem[\loc \mapsto \status]
}{
    \redstoo{\Delta}{(\opened{\loc}{\status}{v},\mem)} {(v,\mem')}
  }{OPEN-END}


\lopsemrule{3in}{
    v_a = \C{new} \; \RgnZ\inang{T}\inang{\toploc\loc}(v) \spc
    \mem(\loc) \neq \USED ~\texttt{and}~ \mem(\loc) \neq \LIVE
%   \rgn_0 \notin \rhoenv \\
%   \redstocup{\rgn_0}{([[\rgn_0/\rgn]v_r/x]e_b,
%     \mem[\rgn \mapsto \OPEN])}{(e_b',\mem')}
}{
    \redstoo{\Delta}{(\open{v_a}{\rgn}{x}{e_b},\mem)} 
            {\invalidexn}
	  }{OPEN-TRANSFERRED}


\lopsemrule{3.25in}{
%   \ralloc \in \rhoenv \spc
%   \fbN = \RgnZ\inang{T}\inang{\rgn_r}\spc
%   \redstoo{\Delta}{(, \mem)}{(e',\mem')}
    \mem(\loc) = \USED \spc
    \mem' = \mem[\loc \mapsto \XFERRED]
}{
    \redstoo{\Delta}{((\C{new}\;\RgnZ\inang{T}\inang{\toploc\loc}(v)).\transfer(),\mem)}
            {(\unitval,\mem')}
	  }{TRANSFER}

\lopsemrule{2.5in}{
%   \fbN = \RgnZ\inang{T}\inang{\rgn}\spc
%   \ralloc \in \rhoenv \spc
%   \redstoo{\Delta}{(, \mem)}{(e',\mem')}
    \mem(\loc) = \TLIVE \spc
}{
    \redstoo{\Delta}{((\C{new}\;\RgnZ\inang{T}\inang{\toploc\loc}(v)).\transfer(),\mem)}
            {\invalidexn}
	  }{TRANSFER-OPENED}

% \lopsemrule{2.5in}{
%     \not\exists \loc.~\mem(\loc) = \SLIVE \spc
%     \mem(\loc') = \XFERRED
% }{
%     \redstoo{\Delta}{(e,\mem)}{(e,\mem[\loc'\mapsto\FREE])}
%     }{GARBAGE-COLLECT}


\caption{\fbname: Operational Semantics (Part 2)}
\label{fig:fb-opsem-2}
\end{figure*}

Figs.~\ref{fig:fb-opsem-1} and~\ref{fig:fb-opsem-2} show the
operational semantics of \fbname. The semantics defines a five-place
reduction relation:
\begin{smathpar}
  \redstoo{\Delta}{(e,\mem)}{(e',\mem')}
\end{smathpar}
Where $\Delta$ is the set of currently-live region locations, and
$\mem$ is a map from memory locations ($\loc$) to type states ($s$).
Evaluating a $\C{letregion}$ or a $\C{new}\;\C{Region}$ expression
results in the addition of a new binding to $\mem$. For a new static
region, the memory location ($\loc$) is mapped to $\LIVE$ (live), and
for a new transferable region, it is mapped to $\USED$. The binding is
updated when the transferable region is opened, closed, or
transferred. Likewise, when a $\C{letregion}$ expression is evaluated
to a value, the binding for the corresponding location is set to
$\XFERRED$, effectively dellocating the region. Opening an already
transferred region, or transferring a currently-open region results in
an exception ($\bot$). The semantics gets stuck while evaluating a
field access, a method call, or a lambda application, if the region
containing the target object is not live. The type safety result
establishes that this can never happen while evaluating a well-typed
expression. The set ($\Delta$) of live region is the same set used by
the expression typing judgment (Fig.~\ref{fig:fb-staticsem-2}). 

\newcommand{\localokin}[2]{#1 \; \texttt{ok} \; \texttt{in} \; #2}
\newcommand{\exptycxFix}[1]{\A,#1,r}

\begin{figure*}[t!]

\beginrules

%%%%%%%%%%% Header Box %%%%%%%%%%%
\multicolumn{2}{l}{
  \textbf{Expression Typing Constraint Generation} \; \fbox{  \( \exprok{\stdcontext}{e}{\tau}{C} \)}
}
\\[0.3cm]

%%%%%%%%%%% () and x %%%%%%%%%%%
\lgcfact{Unit}{\exprok{\stdcontext}{\unitval}{\unitZ}{\{\}}}

\lgcfact{Var}{\exprok{\stdcontext}{x}{\env(\tau)}{\{\}}}

% \begin{minipage}{1.2in}
% \begin{smathpar}
% \begin{array}{l}
% \renewcommand*{\arraystretch}{1.2}
% \exprok{\stdcontext}{\unitval}{\unitZ}{\{\}} \\
% \exprok{\stdcontext}{x}{\env(\tau)}{\{\}}
% \end{array}
% \end{smathpar}
% \end{minipage}

%%%%%%%%%%% FIELD-ACCESS: e.f %%%%%%%%%%%
\lgcrule{FieldAccess}
  {
    \exprok{\stdcontext}{e}{\tau'}{C} \spc
    \bar{f}:\taubar \,=\, \fields(\bound_{\A.\aenv}(\tau'))
  }
  {
    \exprok{\stdcontext}{e.f_i}{\tau_i}{C}
  }

%%%%%%%%%%% LET %%%%%%%%%%%
  \lgcrule{Let}
  {
    \exprok{\stdcontext}{e_1}{\tau_1}{C_1} \spc
    \exprok{\A,{\env[x\mapsto\tau_1]},r}{e_2}{\tau_2}{C_2} \\
  }
  {
    \exprok{\stdcontext}{\letexp{x}{e_1}{e_2}}{\tau_2}{C_1 \cup C_2}
  }

%%%%%%%%%%% METHOD-INV %%%%%%%%%%%
  \lgcrule{MethodInv}
  {
    \exprok {\stdcontext} {e_0} {\tau} {C_1} \spc C_2 = \{ r \rbar \in \A.\rhoenv \}
    \nl
    \mtype(m,\bound_{\A.\aenv}(\tau)) = \inang{\rho \rhobar \,|\, \phi}\bar{\tau^1}\rightarrow{\tau^2}
    \nl
%   \substFn = [\rbar/\rhobar] \\
    \typeok {\A} {\inang{\rho\rhobar \,|\,\phi}\bar{\tau^1}\rightarrow{\tau^2}} {C_3}
       \spc
       \exprok {\stdcontext} {\bar{e}} {[r\rbar/\rho\rhobar](\bar{\tau^1})} {C_4}
    \nl
%   \subtyp{\A}{\bar{\tau'}}{\substFn(\bar{\tau^1})} \spc
    C_5 = \{ \isvalid{\A.\phicx}{[r\rbar/\rho\rhobar](\phi)} \}
  }
  {
    \exprok {\stdcontext} {e_0.m\inang{r \rbar}(\bar{e})} 
       {[r \rbar / \rho \rhobar](\tau^2)} {C_1 \cup C_2 \cup C_3 \cup C_4 \cup C_5}
  }

%%%%%%%%%%% LAMBDA %%%%%%%%%%%

\lgcrule{Lambda}
  {
    \rgn \in \A.\rhoenv \spc
    \rho\rhobar \notin \A.\rhoenv
    \spc
%   \rhoenv' = \rhoenv \cup \{\rhoalloc,\rhobar\}\spc
    \A' = (\A.\rhoenv \cup \{\rho\rhobar\}, \A.\aenv, 
          \A.\phicx \conj \phi)
    \\
    \tywf{\A'.\rhoenv}{\phi}\spc
    \typeok {\A'} {\bar{\tau^1}} {C_1} \spc
    \typeok {\A'} {\tau^2} {C_2}
    \\
    \exprok {\A',\env[\xbar \mapsto \bar{\tau^1}],\rho} {e} {\tau^2} {C_3}
    % \spc
    % C_4 = \bigcup\limits_{\pi'\in \frv(e)\setminus\{\rhobar\}}\{\pi'\outlives\pi\}
  }
  {
    \exprok {\stdcontext}
           {\lambdaexp{\rgn}{\rho\rhobar \,|\, \phi} {\xbar:\bar{\tau^1}}{e}}
           {\inang{\rho\rhobar \,|\, \phi} \bar{\tau^1} \xrightarrow{\rgn} \tau^2}
	   {\cup_{i=1}^{4} C_i}
  }


%%%%%%%%%%% SUBTYPING %%%%%%%%%%%
\lgcrule{SubTyping}{
    \exprok {\exptycxFix{\env}} {e} {\tau} {C_1} \spc  \subtypeok {\A} {\tau} {\tau'} {C_2}
}{
    \exprok {\exptycxFix{\env}} {e} {\tau'} {C_1 \cup C_2}
}

%%%%%%%%%%% METHOD %%%%%%%%%%%
\multicolumn{2}{l}{
   \textbf{Method Well-formedness Constraint Generation}  \; \fbox{$\okinok{d}{B}{C}$}
}
\\[0.3cm]

\lgcrule{Method}{
CT(B) = \C{class}\; B \angAlpha \inang{\rhobar \,|\, \varphi} \extends \fbN
   \{ \cdots \}
   % \{ \bar{\tau^f}\,\xbar;\;\bar{d} \}
\nl
\A = (\rhoenv,\aenv,\phicx) = (\{\rhobar,\rho_m,\rhobarm\}, [\bar{\tyvar} \mapsto \bar{\fgjN}], \varphi_m) \spc\spc
C_1 = \{ \tywf{\rhoenv}{\varphi_m} \}
\nl
\env = \cdot[\thisZ \mapsto B\inang{\bar{\tyvar}}\inang{\rhobar}][\xbar \mapsto \bar{\tau^1}]
\spc
\mtype(m,\fbN) = \inang{\rhobarm \,|\, \phi_m}\bar{\tau^1} \rightarrow \tau^2
\nl
\exprok {\A,\env,\rho_m}{e} {\tau^2} {C_2}
\spc \typeok {\A} {\bar{\tau^1}} {C_3}
\spc \typeok {\A} {\tau^2} {C_4}
% \subtypeok {\A} {\tau'} {\tau} {C_3}
}{
\okinok {\tau^2 \; m\inang{\rho_m \rhobarm \,|\, \varphi_m} (\bar{\tau^1} \;  \xbar)\{\C{return} e;\}}
   {B}
   {(C_1 \cup C_2 \cup C_3 \cup C_4)}
}

%%%%%%%%%%% CLASS %%%%%%%%%%%
\multicolumn{2}{l}{
   \textbf{Class Well-formedness Constraint Generation}  \; \fbox{$\classok{B}{C}$}
}
\\[0.3cm]

\lgcrule{Class}{
  \A = (\rhoenv, \aenv, \phicx) = (\{\rho,\rhobar\}, [\bar{\tyvar} \mapsto \bar{\fgjN}],\varphi) \\
C_1 = \{ \tywf{\rhoenv}{\varphi} \} \spc\spc
\fgjtywf{\aenv}{\bar{\fgjN}} \spc\spc
\typeok {\A} {\fbN} {C_2} \spc\spc
\typeok {\A} {\bar{\tau^f}} {C_3} \\
C_4 = \{\isvalid{\phicx}{\allocRgn(\bar{\tau^f}) \outlives \rho \conj \allocRgn(\fbN) = \rho}\} \\
\okinok {\bar{d}} {B} {C_5}
}{
% \typeok {} {\hdOf{B}{\varphi}\{\bar{\tau^f}\,\xbar;\;\bar{d}\}} {\bigcup_{i=1}^5 C_i}
\classok
  {\C{class}\; B\angAlpha \inang{\rho \rhobar \,|\, \varphi} \extends \fbN
      \{\bar{\tau^f}\,\xbar;\;\bar{d}\}}
  {\bigcup_{i=1}^5 C_i}
}


\myendrules

\caption{Constraint generation rules: Part 2}
\label{fig:constraint-gen-1}
\end{figure*}

\begin{figure*}[t!]

\beginrules

%%%%%%%%%%% Header Box %%%%%%%%%%%
\multicolumn{2}{l}{
  \textbf{Subtyping Constraint Generation} \;
\fbox{  \( \subtypeok{\A}{\tau_1}{\tau_2}{C} \)}
}
\\[0.3cm]

%%%%%%%%%%% SUBTYPING: REFLEXIVE %%%%%%%%%%%

  \lgcfact{Reflexivity}{
    \subtypeok{\A}{\tau}{\tau}{\{\}}
  }

%%%%%%%%%%% SUBTYPING: UNIFICATION %%%%%%%%%%%
  \lgcfact{Unify}{
    \subtypeok{\A}{\tau}{[\pi/\rho](\tau)}{ \{ \pi \outlives \rho, \rho \outlives \pi \} }
  }

%%%%%%%%%%% SUBTYPING: TRANSITIVITY %%%%%%%%%%%

  \lgcrule{Transitivity}{
    \subtypeok{\A}{\tau_1}{\tau_2}{C_1} \spc
    \subtypeok{\A}{\tau_2}{\tau_3}{C_2}
  }{
    \subtypeok{\A}{\tau_1}{\tau_3}{C_1 \cup C_2}
  }

%%%%%%%%%%% FUNCTION SUBTYPING %%%%%%%%%%%
  \lgcrule{FnSubtyping}
  {
    C_1 = \{ \isvalid{\A.\phicx}{\phi_1 \Rightarrow \phi_2} \}
    \\
    \subtypeok {\A} {\bar{\tau^{11}}} {\bar{\tau^{21}}} {C_2}
    \spc
    \subtypeok {\A} {\tau^{22}} {\tau^{12}} {C_3}
  }
  {
    \subtypeok {\A}
      {\inang{\rhobar \,|\, \phi_2}\bar{\tau^{21}} \xrightarrow{\rgn} \tau^{22}}
      {\inang{\rhobar \,|\, \phi_1}\bar{\tau^{11}} \xrightarrow{\rgn} \tau^{12}}
      {C_1 \cup C_2 \cup C_3}
  }

%%%%%%%%%%% Header Box %%%%%%%%%%%
\multicolumn{2}{l}{
  \textbf{Type Well-formedness Constraint Generation} \;
  \fbox{  \( \typeok{\A}{\tau}{C} \)}
}
\\[0.3cm]

%%%%%%%%%%% TYPE WELL-FORMEDNESS %%%%%%%%%%%

%%%%%%%%%%% OBJECT TYPE %%%%%%%%%%%
\lgcrule{ObjectType}
  {
    C = \{ \rgn \in \rhoenv \}
  }
  {
    \typeok {(\rhoenv,\aenv,\phicx)} {\ObjZ\inang{\rgn}} {C}
  }

%%%%%%%%%%% CLASS TYPE %%%%%%%%%%%
  \lgcrule{ClassType}
  {
    CT(B) = \headerOf{B}\{...\}
    \spc
    \fgjtywf{\aenv}{B\inang{\tbar}}
    \\
    C = \{ \rbar \in \rhoenv, \isvalid{\phicx}{[\rbar/\rhobar](\phi)} \}
  }
  {
    \typeok {(\rhoenv,\aenv,\phicx)} {B\inang{\tbar}\inang{\rbar}} {C}
  }

%%%%%%%%%%% GENERIC TYPE PARAMETER %%%%%%%%%%%
  \lgcrule{TypeParam}
  {
    \fgjtywf{\aenv}{T} \spc
    \fgjsubtyp{\aenv}{T}{\ObjZ} \spc
    \spc
    C = \{ \rgn \in \rhoenv \}
  }
  {
    \typeok {(\rhoenv,\aenv,\phicx)}{T@\rgn} {C}
  }

%%%%%%%%%%% FUNCTION TYPE %%%%%%%%%%%
  \lgcrule{FnType}
  {
    C_1 = \{ \rgn \in \rhoenv \}
    \\
    \rhobar \notin \rhoenv \spc
    \rhoenv' = \rhoenv \cup \{\rhobar\} \spc
    \A' = (\rhoenv', \aenv, \phicx \conj \phi)
    \\
    \typeconsok{\rhoenv'}{\phi} {C_2} \spc 
    \typeok{\A'}{\bar{\tau^1}} {C_3} \spc
    \typeok{\A'}{\tau^2} {C_4}
  }
  {
    \typeok{(\rhoenv,\aenv,\phicx)} {\inang{\rhobar \,|\, \phi} \bar{\tau^1} \xrightarrow{\rgn} \tau^2} 
       {C_1 \cup C_2 \cup C_3 \cup C_4}
  }

%%%%%%%%%%% REGION TYPE %%%%%%%%%%%
  \lgcrule{RegionType}
  { 
    \fgjtywf{\aenv}{T}
  }
  {
    \typeok {(\rhoenv,\aenv,\phicx)} {\RgnZ\inang{T}\inang{\toprgn}} {\{\}}
  }

\myendrules

\caption{Constraint generation rules: Part 3}
\label{fig:constraint-gen-2}
\end{figure*}


\newcommand{\deltaPC}{\Delta_P^C}

\section{Type Inference: Proofs}

%\begin{theorem}
%\label{thm:constraint-generation-sc}
%Let $C = \consOf{q}$.
%An assignment $\soln$ (for the region and predicate variables in $q$)
%satisfies $C$ iff $q[\soln]$ is well-typed.
%\end{theorem}

\begin{proof}[\textbf{Theorem~\ref{thm:constraint-generation-sc}}]
The correspondence between the static semantics rules and the constraint generation
rules induces a correspondence between derivation trees produced by the
two systems.
% We do not present a formal proof of the theorem.
We can establish the following lemma inductively:
given a derivation tree $\zeta_1$ for
$\exprok {\stdcontext}{e}{\tau}{C}$
and any substitution $\soln$,
we can construct a corresponding \emph{candidate} derivation
tree $\zeta_2$ of $\hastyp {\A[\soln],\env[\soln]}{e[\soln]}{\tau[\soln]}$.
We can show that $\zeta_2$ is a valid derivation tree iff $\soln$ satisfies $C$.
(We make use of analogous lemmas for type well-formedness rules as well.)
\qed
\end{proof}

%\begin{lemma}
%  \label{lemma:gc-is-decomposable}
%  Let $C = \consOf{q}$. Every abduction constraint in $C$ is decomposable.
%\end{lemma}

\begin{proof}[\textbf{Lemma~\ref{lemma:gc-is-decomposable}}]
  By induction over the constraint-generation rules.
  Any context $\A = (\rhoenv,\aenv,\phicx)$ generated by the constraint-generation
  process satisfies the invariant that $\phicx$ is of the form $\varphi \conj \phictxt$
  where $\rhoenv \supseteq \predDeltaMap(\varphi)$.
  The only rule that modifies $\phicx$ is the rule for \C{letregion}
  that adds the set of constraints $\pi_f \outlives \pi$ for every $\pi_f \in \rhoenv$
  as conjuncts to $\phicx$.
\qed
\end{proof}

We recall the definition of $\predDeltaMap(\varphi)$ first. We work with sets of constraints $C$ that
contain exactly only constraint of the form  $\tywf{\Delta}{\varphi}$ for any predicate variable $\varphi$
and we refer to this $\Delta$ as $\predDeltaMap(\varphi)$. In the sequel, we use the notation
$\deltaPC(\varphi)$ to clarify the dependence on $C$, and will omit the superscript if no confusion
is likely.

%\begin{lemma}
%  \label{lemma:decomposition}
%Consider any decomposable constraint  $\isvalid{\varphi \conj \phi}{\pi_i \outlives \pi_j}$
%where both $\pi_i$ and $\pi_j$ are region constants.
%(a) If $\{ \pi_i, \pi_j \} \subseteq \deltaPC(\varphi)$,
%then $\soln$ satisfies $C \cup \{ \isvalid{\varphi \conj \phi}{\pi_i \outlives \pi_j} \}$
%iff
%$\soln$ satisfies $C \cup \{ \isvalid{\varphi}{\pi_i \outlives \pi_j} \}$
%  (b) If $\{ \pi_i, \pi_j \} \not \subseteq \deltaPC(\varphi)$,
%$\soln$ satisfies $C \cup \{ \isvalid{\varphi \conj \phi}{\pi_i \outlives \pi_j} \}$
%iff
%$\soln$ satisfies $C \cup \{ \isvalid{\phi}{\pi_i \outlives \pi_j} \}$
%\end{lemma}

\begin{proof}[\textbf{Lemma~\ref{lemma:decomposition}}]
  Let $\soln$ be any assignment that satisfies $C$.
  Let $\phi'$ denote $\solnP(\varphi)$.
  The lemma follows once we show that
  $\isvalid{\phi' \conj \phi}{\pi_i \outlives \pi_j}$ iff
  $\isvalid{\phi'}{\pi_i \outlives \pi_j}$ or
  $\isvalid{\phi}{\pi_i \outlives \pi_j}$.
  The reverse implication is trivial, and we establish the forward implication below.

  Assume that $\isvalid{\phi' \conj \phi}{\pi_i \outlives \pi_j}$.
  Recall that we can represent $\phi' \conj \phi$ as a graph, with each vertex
  representing a region identifier, and each outlives-constraint $\pi \outlives \pi'$
  represented as an edge from $\pi$ to $\pi'$. Further,
  $\isvalid{\phi' \conj \phi}{\pi_i \outlives \pi_j}$ holds iff there exists a path
  from $\pi_i$ to $\pi_j$ in this graph.

  Let $\Delta$ denote $\deltaPC(\varphi)$.
  Clearly, $\phi'$ can include an outlives-constraint $\pi \outlives \pi'$ only if
  $\{ \pi, \pi' \} \subseteq \Delta$.
  Thus, $\phi'$ represents a set of edges between vertices in $\Delta$.
  On the other hand, $\phi$ represents a set of edges whose targets are outside $\Delta$
  (from the definition of a decomposable constraint).
  Furthermore, it follows from the definition of a decomposable constraint that
  if $\phi$ includes an outlives-constraint $\pi \outlives \pi'$ for some $\pi \in \Delta$,
  then $\phi$ includes the outlives-constraint $\pi'' \outlives \pi'$ for every $\pi'' \in \Delta$.

  The result follows.
\qed
\end{proof}

%\begin{theorem}
%  \label{thm:closure}
%Let $C = \consOf{q}$.
%An assignment $\soln$ satisfies $C$ iff $\soln$ satisfies $\satC$.
%\end{theorem}

\begin{proof}[\textbf{Theorem~\ref{thm:closure}}]
  For the forward implication, we use induction to show that every constraint
  added to $\satC$ preserves satisfiability (with respect to an assignment $\soln$).
  The Transitivity and Substitution steps are straightforward. The Abduction Decomposition
  step is justified by Lemmas~\ref{lemma:gc-is-decomposable} and~\ref{lemma:decomposition}.
  The reverse implication is trivial.
\qed
\end{proof}

\begin{lemma}
  \label{lemma:completely-bound}
Let $C = \consOf{q}$. If $\satC$ is satisfiable, then
every region variable occurring in $\satC$ is bound to some region constant in $\satC$.
\end{lemma}

\begin{proof}
The lemma formalizes the intuition that every region variable will be ``unified'' with
some region constant (by the generated constraints).
%
Region variables are introduced at call-sites and represent the formal region parameters for the call.
Some correspond to region parameters of types of input parameters, and these will be bound by the
actual input parameter expressions. Other region variables correspond to region parameters of the
return type. These are determined by the actual return-expression in the called function/method.
It can be shown that these will be bound to one of ``input region parameters'', provided the called
function/method is typable (\ie, if $\satC$ is satisfiable).

There is one special case where this does not hold: a recursive function
that calls itself in a non-terminating fashion. Such a function never
returns a value, and so the return-value can be typed as anything.
We assume that the return value of such a function is typed to be \C{unit},
which will not introduce any region variables at the call-site of such a function.

% This property follows from several aspects of $\FB$, e.g., it does not admit uninitialized variables.
Note that the region \C{R} where a new object is allocated (by the \C{new} construct) is indicated
by the user (as \C{new@R}). If the user does not specify this, it defaults to the allocation
context region. Thus, this aspect is handled by the frontend, and does not require or use
the constraint solver.

Formally, we can prove via induction that if $\exprok{\stdcontext}{e}{\tau}{C}$ then
every region variable in the result type $\tau$ will be bound to one of the ``live''
region constants (\ie, region constants in $\A$).
\qed

\end{proof}

\begin{lemma}
  \label{lemma:two-bindings}
Let $C = \consOf{q}$. Assume that $\groundC$ is valid.
Suppose a region variable $\rho$ is bound to two different region constants $\pi_1$
and $\pi_2$ in a context $\varphi \conj \phi$. % in $\satC$.
% Let $\soln$ satisfy $\satC$.
% For any context $\varphi \conj \phi$ that contains $\rho$, we must have
Then, 
(a) $\{ \pi_1, \pi_2 \} \subseteq \predDeltaMap(\varphi)$,
(b) $\isvalid{\thesolnP(\varphi)}{\pi_1 = \pi_2}$.
(c) For any $\soln$ that satisfies $\satC$, we must have $\isvalid{\solnP(\varphi)}{\pi_1 = \pi_2}$.
\end{lemma}

\begin{proof}
  Assume that a region variable $\rho$ is bound to two different
  region constants $\pi_1$ and $\pi_2$ in a context $\varphi \conj \phi$.
  By transitivity, $\satC$ is guaranteed to have the constraints
  $\isvalid{\varphi \conj \phi}{\pi_1 \outlives \pi_2}$
  and $\isvalid{\varphi \conj \phi}{\pi_2 \outlives \pi_1}$.
  By abduction decomposition, one of the following conditions must hold:
  (a) $\{ \pi_1, \pi_2 \} \subseteq \predDeltaMap(\varphi)$ and
  both $\isvalid{\varphi}{\pi_1 \outlives \pi_2}$ and
  $\isvalid{\varphi}{\pi_2 \outlives \pi_1}$ are in $\satC$, or
  (b) $\{ \pi_1, \pi_2 \} \not\subseteq \predDeltaMap(\varphi)$ and
  both $\isvalid{\phi}{\pi_1 \outlives \pi_2}$ and
  $\isvalid{\phi}{\pi_2 \outlives \pi_1}$ are in $\satC$.
  In case (b), $\satC$ will not be satisfiable, so we can ignore this case.
  In case (a), $\varphi$ must be bound to a conjunction that includes both
  $\pi_1 \outlives \pi_2$ and $\pi_2 \outlives \pi_1$ in any satisfying assignment
  for $\satC$.
\qed
\end{proof}

% \begin{lemma}
% Let $C = \consOf{q}$.
% Let $\satC_1$ denote the subset of $\satC$ obtained by removing all constraints
% that contain a region variable.
% $\satC$ is satisfiable iff $\satC_1$ is satisfiable.
% \end{lemma}
% 
% \begin{proof}
%   The forward implication is trivial.
%   The reverse implication follows from Lemma~\label{lemma:completely-bound}, as shown below.
% \end{proof}

\begin{theorem}
\label{thm:constraint-solver-soundness}
Let $C = \consOf{q}$.
If $\solveCon{C} = \textsc{Some}(\soln)$,
then $\soln$ satisfies $\satC$.
\end{theorem}

\begin{proof}
Assume $\solveCon{C} = \textsc{Some}(\soln)$.

By definition (of $\solveCon{C}$), $\groundC$ (the set of all variable-free constraints in $\satC$)
is satisfiable. Hence, $\soln$ trivially satisfies all constraints in $\groundC$.
%
The definition also ensures that $\soln$ satisfies the well-formedness constraints
for any region variable $\rho$.

Consider any constraint of the form $\isvalid{\varphi}{\pi_i \outlives \pi_j}$ in $\satC$,
where $\pi_i$ and $\pi_j$ are distinct region constants. The definition of $\solnP(\varphi)$
ensures that $\soln$ satisfies this constraint.
Furthermore, $\soln$ must also satisfy the well-formedness constraint
$\tywf{\predDeltaMap(\varphi)}{\varphi}$.
(Otherwise, the application of Abduction Decomposition to this constraint will add
$\isvalid{}{\pi_i \outlives \pi_j}$ to $\satC$. This constraint is invalid, which
contradicts the fact that $\groundC$ is valid.)

Consider any constraint of the form $\isvalid{\varphi \conj \phi}{\pi_i \outlives \pi_j}$ in $\satC$,
It follows from abduction decomposition that either
$\isvalid{\varphi}{\pi_i \outlives \pi_j}$
or
$\isvalid{\phi}{\pi_i \outlives \pi_j}$ must be in $\satC$, which we know is satisfied by $\soln$.
Hence, $\soln$ must satisfy
$\isvalid{\varphi \conj \phi}{\pi_i \outlives \pi_j}$ as well.

This shows that $\soln$ satisfies all outlives-constraints in $\satC$ that have no variables in the consequent.

Consider any constraint of the form $\isvalid{\ell}{F_j(\varphi_j)}$.
Substituting the value $\solnP(\varphi_j)$ reduces this constraint to one of
the form $\isvalid{\ell}{\conj_i \pi_i \outlives \pi_i'}$ where the substitution rule
guarantees $\isvalid{\ell}{\pi_i \outlives \pi_i'}$ is already in $\satC$ and, hence,
satisfied by $\soln$.
Hence, $\soln$ satisfies $\isvalid{\ell}{F_j(\varphi_j)}$ as well.

Consider any constraint of the form $\isvalid{\ell}{r}$ where $r$ contains a single occurrence
of a region variable, say $\rho$.
It follows from Lemma~\ref{lemma:completely-bound} that $\rho$ is bound to some region
constant $\pi$.
Let $r'$ denote $r$ with $\rho$ replaced by $\pi$.
It follows from the transitivity rule that $\satC$ must contain the constraint $\isvalid{\ell}{r'}$.
Since this constraint has no variables on the consequent, we have already shown that $\soln$
must satisfy this constraint ($\isvalid{\ell}{r'}$).
Let $\pi' = \solnR(\rho)$.
We can show (using Lemma~\ref{lemma:two-bindings}) that $\soln$ satisfies $\isvalid{\ell}{\pi = \pi'}$.
Hence, $\soln$ satisfies $\isvalid{\ell}{r}$ as well.

The same idea works for constraints with occurrences of two region variables in the consequent.
\qed
\end{proof}

\begin{theorem}
\label{thm:constraint-solver-completeness}
Let $C = \consOf{q}$.
If $\solveCon{C} = \textsc{None}$, then $C$ is unsatisfiable.
\end{theorem}

\begin{proof}
We prove the theorem by contradiction.
Assume that $\satC$ is satisfiable.
We will show that the precondition for the first case in the definition of $\textsc{Solve}$
holds and, hence, $\solveCon{C}$ cannot be $\textsc{None}$.

Since $\satC$ is satisfiable, $\groundC$ is trivially valid.
We just need to show that $\thesolnR$ satisfies $\rhoC$.
Recall that $\rhoC$ is the subset of $C$ of well-formedness constraints
on region variables.
So, we just need to show for any constraint $\rho \in \rhoenv$ in $C$ that
$\thesolnR(\rho) \in \rhoenv$.
Since we have assumed that $\satC$ is satisfiable, there exists some $\soln$ that satisfies $\satC$
and, in particular, the constraint $\rho \in \rhoenv$.
Let $\pi$ denote $\soln(\rho)$. Thus, we have $\pi \in \rhoenv$.

Let $\hat{\pi}$ denote $\thesolnR(\rho)$. If $\pi$ and $\hat{\pi}$ are the same region
constant, then $\hat{\pi}\ \in \rhoenv$ and the proof is complete.
%
Otherwise, it follows from the definition of $\thesoln$, that we have some constraint
$\isvalid{\varphi \conj \phi}{\rho = \hat{\pi}}$ in $\satC$.
Since $\soln$ satisfies this constraint, we have
$\isvalid{\solnP(\varphi) \conj \phi}{\pi = \hat{\pi}}$.
We can show that $\{ \pi, \hat{\pi} \} \subseteq \predDeltaMap(\varphi)$
(as in the proof of Lemma~\ref{lemma:two-bindings}).
We can show, by induction over the constraint-generation rules, that if $\Delta$
includes some element of $\predDeltaMap(\varphi)$, then it includes all elements of $\predDeltaMap(\varphi)$.
Hence, it follows that $\hat{\pi} \in \Delta$.
\qed
% for some predicate variable $\varphi$.
%
% If $\groundC$ is satisfiable, the solver will fail to return an assignment only if the set $S_\rho$ = 
% $\{ \pi \in \regionDeltaMap(\rho) \; | \; \isvalid{\ell}{\rho = \pi} \in \satC \}$ is empty for some region variable $\rho$.
% Consider any such $\rho$.
% Let $\soln$ be a satisfying assignment for $\satC$.
% Let $\pi' = \solnP(\rho)$. Then, we must have $\pi' \in \regionDeltaMap(\rho)$.
% From Lemma~\ref{lemma:completely-bound}, $\rho$ must be bound to some $\pi$ in some context $\ell$.
% Thus, $\isvalid{\ell}{\rho = \pi} \in \satC$.
% Since $\soln$ satisfies this constraint, we have $\isvalid{\ell[\soln]}{\pi' = \pi}$.
% If $\pi$ and $\pi'$ are the same region identifier, we have a contradiction, since
% $\pi \in S_\rho$ and $S_\rho$ cannot be empty.
% If $\pi$ and $\pi'$ are distinct region identifiers, then it follows that the precondition of case (a) must hold
% in Lemma~\ref{lemma:decomposition}, and we must have $\{ \pi, \pi' \} \subseteq \predDeltaMap(\varphi)$.
% We can show, by induction over the constraint-generation rules, that if $\regionDeltaMap(\rho)$ includes
% some element of $\predDeltaMap(\varphi)$, then it includes all elements of $\predDeltaMap(\varphi)$.
% It follows that $\pi \in \regionDeltaMap(\rho)$. Hence, the set $S_\rho$ is non-empty (as it contains $\pi$.
% The result follows.
\end{proof}

%\begin{theorem}
%\label{thm:constraint-solver-sc}
%Let $C = \consOf{q}$.\\
%(Soundness) If $\solveCon{C} = \textsc{Some}(\soln)$, then $\soln$ satisfies $C$.\\
%(Completeness) If $\solveCon{C} = \textsc{None}$, then $C$ is unsatisfiable.
%\end{theorem}

\begin{proof}[\textbf{Theorem~\ref{thm:constraint-solver-sc}}]

Follows from Theorems~\ref{thm:constraint-solver-soundness} and~\ref{thm:constraint-solver-completeness}.
\qed
\end{proof}


% \subsection{Other Aspects}

\subsection{Modularity Aspects of Type Inference}

The type inference algorithm, as presented, traverses the entire program to
generate the set of constraints, which are solved en masse, using an iterative
fixed point computation. However, the type inference can be realized in a
modular and compositional fashion, subject only to the restrictions imposed
by recursion.

In the elaboration phase, we can process a class \C{C} only after any class
\C{B} that \C{C} depends on has been processed: class \C{C} depends on
class \C{B} if \C{B} is either \C{C}'s base class or the type of any field
of \C{C} depends on \C{B}. In effect, this means that any collection of
mutually recursive classes must be processed together. Non-recursive
dependences can be handled in a compositional fashion: if class \C{C}
depends on \C{B} non-recursively, then the elaboration can be done for
\C{B} first, and then \C{C} can be processed.

The same idea applies to the constraint-solving phase as well.
Given a set of constraints, we say that a predicate variable $\varphi_1$
\emph{directly-depends} on another predicate variable $\varphi_2$ if the set of
constraints includes a constraint $\isvalid{\varphi_1 \conj \phictxt}{F(\varphi_2)}$.
We say that $\varphi_1$ \emph{depends} on $\varphi_2$ if $\varphi_1$ transitively
depends on $\varphi_2$.
The constraint solver needs to process any collection of mutually dependent
predicate variables together.
In effect, this requires the type inference to process any collection of
mutually recursive methods together.
However, methods that are not mutually recursive can be processed separately.

