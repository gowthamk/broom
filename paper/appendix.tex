%\section{Operational Semantics}
%
%Fig.~\ref{fig:fb-opsem} contains a definition of the operational semantics for \fbname.
%
\section{\fbname}

\subsection{Full Syntax}
\input{fb-syntax-full}

The full syntax of \FB, including expressions that manifest only at
runtime, is shown in Fig.~\ref{fig:fb-syntax-full}. Such expressions
include:
\begin{itemize}
\item Memory locations ($\loc$) corresponding to the memory regions,
\item A special expression $\RgnZT{\toploc\loc}(e)$ that evaluates to
a region handler object (the syntax of this expression is captured by
the $\C{new}\;\fbN(\overline{e})$ construct). $\toploc$ is the
location of the special $\toprgn$ region where region handlers are
stored. $\loc$ is the location of the corresponding transferable
region.
\item A $\letd{\loc}{e}$ expression that results when
$\letregion{\pi}{e}$ takes a step (see Fig.~\ref{fig:fb-opsem-2}).
$\loc$ is the location of the newly allocated region.
\item An $\opened{\loc}{s}{e}$ expression that results when
$\open{\RgnZT{\toploc\loc}(v)}{\pi}{x}{e}$ takes a step. $\loc$ is the
location of the newly open transferable region.
\end{itemize}

\subsection{Full Static Semantics}

\renewcommand{\rgn}{r}
\renewcommand{\rbar}{\overline{r}}

\begin{figure*}[!h]
%
\textbf{Subtyping}  \; \fbox
  {\(\subtyp{\A}{\tau_1}{\tau_2}\)}\\
%
\begin{minipage}{1.8in}
\begin{smathpar}
\begin{array}{c}
\renewcommand*{\arraystretch}{1.2}
  \subtyp{\A}{\tau}{\tau} \\
  \subtyp{(\Delta,\aenv,\phicx)}{\tyvar @\rho}{\aenv(\tyvar) @\rho}\qquad
% \subtyp{\A}{\RgnZ\inang{\rgn}}{\RgnZ\inang{\toprgn}}\qquad
% \subtyp{\A}{\RgnZ\inang{\toprgn}}{\RgnZ\inang{\rgn}}
\end{array}
\end{smathpar}
\end{minipage}
%
\begin{minipage}{2.7in}
\begin{smathpar}
\begin{array}{c}
\renewcommand*{\arraystretch}{1.2}
\RULE
  {
    \\
    CT(B) = \headerOf{B}\{...\}
  }
  {
    \subtyp{\A}{B\inang{\tbar}\inang{\rbar}}
        {[\rbar/\rhobar, \tbar/\bar{\tyvar}](\fbN)}
  }
\end{array}
\end{smathpar}
\end{minipage}
%

\begin{minipage}{1.5in}
\begin{smathpar}
\begin{array}{c}
\renewcommand*{\arraystretch}{1.2}
\RULE
  {
    \subtyp{\A}{\tau_1}{\tau_2}\\
    \subtyp{\A}{\tau_2}{\tau_3}
  }
  {
    \subtyp{\A}{\tau_1}{\tau_3}
  }
\end{array}
\end{smathpar}
\end{minipage}
%
\begin{minipage}{2.75in}
\begin{smathpar}
\begin{array}{c}
\renewcommand*{\arraystretch}{1.2}
\RULE
  {
    \isvalid{\A.\phicx}{\phi_1 \Rightarrow \phi_2} \\
    \subtyp{\A}{\bar{\tau^{11}}}{\bar{\tau^{21}}} \spc
    \subtyp{\A}{\tau^{22}}{\tau^{12}}
  }
  {
    \subtyp{\A}
      {\inang{\rhobar \,|\, \phi_2}\bar{\tau^{21}}
          \xrightarrow{\rgn} \tau^{22}}
      {\inang{\rhobar \,|\, \phi_1}\bar{\tau^{11}}
          \xrightarrow{\rgn} \tau^{12}}
  }
\end{array}
\end{smathpar}
\end{minipage}


%
\bigskip

\textbf{Type, and Type Constraint Well-formedness}  \; \fbox
  {\(\tywf{\A}{\tau}, \spc 
     \tywf{\mem}{\phi}\)}\\
/Users/gowtham/git/broom/fullversion/broom/paper/fb-tywfrules-full.tex
%
\bigskip

\textbf{Expression Typing}  \; \fbox
  {\(\hastyp{\exptycx{\env}}{e}{\tau}\)}\\
/Users/gowtham/git/broom/fullversion/broom/paper/fb-exptyprules-full.tex
%
\bigskip

\caption{\fbname: Static Semantics}
\label{fig:fb-staticsem}
\end{figure*}

\renewcommand{\rgn}{\pi}
\renewcommand{\rbar}{\overline{\pi}}


\subsection{Operational Semantics}

\newcommand{\redstoo}[2]{\redsto{\rhoenv}{#1}{#2}}
\newcommand{\redstocup}[3]{\redsto{\rhoenv \cup \{#1\}}{#2}{#3}}
\newcommand{\anobjty}[0]{B\inang{\tbar}\inang{\ralloc\rbar}}
\newcommand{\anobj}[0]{\C{new} \; \anobjty(\bar{v})}
\begin{figure*}[t!]

%
\fbox {\(\redstoo{(e,\rhomap)}{(e',\rhomap')}\)}\\

%
\begin{minipage}{2.7in}
\begin{smathpar}
\begin{array}{c}
\renewcommand*{\arraystretch}{1.2}
\RULE
  {
    \allocRgn(\fbN) \in \rhoenv \spc
    \fields(\fbN) = \taubar\;\bar{f}
  }
  {
    \redstoo{((\C{new} \; \fbN(\vbar)).f_i,\rhomap)}{(v_i,\rhomap)}
  }
\end{array}
\end{smathpar}
\end{minipage}
%
% \begin{minipage}{3in}
% \begin{smathpar}
% \begin{array}{c}
% \renewcommand*{\arraystretch}{1.2}
% \RULE
%   {
%     \ralloc \in \rhoenv \spc
%     \redstoo{(e_i,\rhomap)}{(e_i',\rhomap')}
%   }
%   {
%     \redstoo{(\C{new} \; \fbN(...,e_i,...),\rhomap)}
%             {(\C{new} \; \fbN(...,e_i',...),\rhomap')}
%   }
% \end{array}
% \end{smathpar}
% \end{minipage}
%
\begin{minipage}{3.2in}
\begin{smathpar}
\begin{array}{c}
\renewcommand*{\arraystretch}{1.2}
\RULE
  {
    \rgn \notin \rhoenv \spc
    \redsto{\rhoenv \cup \{\rgn\}}{(e,\rhomap)}{(e',\rhomap')}
  }
  {
    \redstoo{(\letregion{\rgn}{e},\rhomap)}{(\letregion{\rgn}{e'},\rhomap')}
  }
\end{array}
\end{smathpar}
\end{minipage}
%

%
\begin{minipage}{3in}
\begin{smathpar}
\begin{array}{c}
\renewcommand*{\arraystretch}{1.2}
\RULE
  {
    \rgn \notin \rhoenv
  }
  {
    \redstoo{(\letregion{\rgn}{v},\rhomap)}{(v,\rhomap)}
  }
\end{array}
\end{smathpar}
\end{minipage}
%
\begin{minipage}{3.3in}
\begin{smathpar}
\begin{array}{c}
\renewcommand*{\arraystretch}{1.2}
\RULE
  {
    \fgjN = \RgnZ\inang{T} \spc
    \ralloc \in \rhoenv \spc
    \rgn \notin dom(\rhomap) \cup \rhoenv \spc
    \rhomap' = \rhomap[\rho \mapsto \CLOSED]
  }
  {
    \redstoo{(\C{new} \; \fgjN\inang{\toprgn}
                (\lambdaexp{\ralloc}{\rhoalloc}{}{e}),\rhomap)}
            {(\C{new} \; \fgjN\inang{\rgn}
                ([\rgn/\rhoalloc]e),\rhomap')}
  }
\end{array}
\end{smathpar}
\end{minipage}
%

%
\begin{minipage}{3.6in}
\begin{smathpar}
\begin{array}{c}
\renewcommand*{\arraystretch}{1.2}
\RULE
  {
    \fgjN = \RgnZ\inang{T} \spc
    \rgn \in dom(\rhomap) \spc
    \redstocup{\rgn}{(e,\rhomap)}{(e',\rhomap')}
  }
  {
    \redstoo{(\C{new} \; \fgjN\inang{\rgn}
                (e),\rhomap)}
            {(\C{new} \; \fgjN\inang{\rgn}
                (e'),\rhomap')}
  }
\end{array}
\end{smathpar}
\end{minipage}
%
\begin{minipage}{2.5in}
\begin{smathpar}
\begin{array}{c}
\renewcommand*{\arraystretch}{1.2}
\RULE
  {
    v_a = \C{new} \; \RgnZ\inang{T}\inang{\rgn}(v_r) \spc
    \rhomap(\rgn) \neq \XFERRED \spc
    \rgn_0 \notin \rhoenv \spc
  }
  {
    \redstoo{(\open{v_a}{\rgn_0}{x}{v_b},\rhomap)} {(v_b,\rhomap)}
  }
\end{array}
\end{smathpar}
\end{minipage}
%

%
\begin{minipage}{4in}
\begin{smathpar}
\begin{array}{c}
\renewcommand*{\arraystretch}{1.2}
\RULE
  {
    v_a = \C{new} \; \RgnZ\inang{T}\inang{\rgn}(v_r) \spc
    \rhomap(\rgn) \neq \XFERRED \spc
    \rgn_0 \notin \rhoenv \\
    \redstocup{\rgn_0}{([[\rgn_0/\rgn]v_r/x]e_b,
      \rhomap[\rgn \mapsto \OPEN])}{(e_b',\rhomap')} \spc
    \rhomap'' = \rhomap'[\rgn \mapsto \rhomap(\rgn)]
  }
  {
    \redstoo{(\open{v_a}{\rgn_0}{x}{e_b},\rhomap)} 
            {(\open{v_a}{\rgn_0}{x}{e_b'},\rhomap'')}
  }
\end{array}
\end{smathpar}
\end{minipage}
%
%
\begin{minipage}{3in}
\begin{smathpar}
\begin{array}{c}
\renewcommand*{\arraystretch}{1.2}
\RULE
  {
    \\
    v_a = \C{new} \; \RgnZ\inang{T}\inang{\rgn}(v_r) \spc
    \rhomap(\rgn) = \XFERRED \spc
%   \rgn_0 \notin \rhoenv \\
%   \redstocup{\rgn_0}{([[\rgn_0/\rgn]v_r/x]e_b,
%     \rhomap[\rgn \mapsto \OPEN])}{(e_b',\rhomap')}
  }
  {
    \redstoo{(\open{v_a}{\rgn_0}{x}{e_b},\rhomap)} 
            {\invalidexn}
  }
\end{array}
\end{smathpar}
\end{minipage}
%

%
\begin{minipage}{3.85in}
\begin{smathpar}
\begin{array}{c}
\renewcommand*{\arraystretch}{1.2}
\RULE
  {
    \allocRgn(\fbN),\ralloc \in \rhoenv \spc
    \mbody(m\inang{\ralloc \rbar},\fbN) = \bar{x}.e 
%   \redstoo{(, \rhomap)}{(e',\rhomap')}
  }
  {
    \redstoo{((\C{new}\;\fbN(\bar{v})).m\inang{\ralloc \rbar}
                      (\bar{v'}),\rhomap)}
            {([\bar{v'}/\xbar][\C{new} \; \fbN(\bar{v})/\thisZ]\,e,\rhomap)}
  }
\end{array}
\end{smathpar}
\end{minipage}
%
\begin{minipage}{3.25in}
\begin{smathpar}
\begin{array}{c}
\renewcommand*{\arraystretch}{1.2}
\RULE
  {
%   \ralloc \in \rhoenv \spc
    \fbN = \RgnZ\inang{T}\inang{\rgn}\spc
%   \redstoo{(, \rhomap)}{(e',\rhomap')}
    \rhomap(\rgn) \neq \OPEN \spc
    \rhomap' = \rhomap[\rgn \mapsto \XFERRED]
  }
  {
    \redstoo{((\C{new}\;\fbN(v)).\transfer\inang{\ralloc}
                      (),\rhomap)}
            {(\unitval,\rhomap')}
  }
\end{array}
\end{smathpar}
\end{minipage}
%

%
\begin{minipage}{2.5in}
\begin{smathpar}
\begin{array}{c}
\renewcommand*{\arraystretch}{1.2}
\RULE
  {
    \fbN = \RgnZ\inang{T}\inang{\rgn}\spc
%   \ralloc \in \rhoenv \spc
%   \redstoo{(, \rhomap)}{(e',\rhomap')}
    \rhomap(\rgn) = \OPEN \spc
  }
  {
    \redstoo{((\C{new}\;\fbN(v)).\transfer\inang{\ralloc}
                      (),\rhomap)}
            {\invalidexn}
  }
\end{array}
\end{smathpar}
\end{minipage}
%
\begin{minipage}{2.5in}
\begin{smathpar}
\begin{array}{c}
\renewcommand*{\arraystretch}{1.2}
\RULE
  {
    v_a = \lambdaexp{\rgn_a}{\rhoalloc\rhobar}
                        {\taubar \; \xbar}{e} \spc
    \rgn_a,\ralloc \in \rhoenv \spc 
  }
  {
    \redstoo{(v_a\inang{\ralloc\rbar}(\bar{v}) ,\rhomap)}
            {([\bar{v}/\xbar][\rbar/\rhobar][\ralloc/\rhoalloc]\,e,\rhomap)}
  }
\end{array}
\end{smathpar}
\end{minipage}
%
% %
% \begin{minipage}{1.8in}
% \begin{smathpar}
% \begin{array}{c}
% \renewcommand*{\arraystretch}{1.2}
% \RULE
%   {
%     \redstoo{(e_1,\rhomap)}{(e_1',\rhomap')}
%   }
%   {
%     \redstoo{(e_1;\,e_2,\rhomap)}
%             {(e_1';\,e_2,\rhomap')}
%   }
% \end{array}
% \end{smathpar}
% \end{minipage}
% %

\begin{minipage}{1.8in}
\begin{smathpar}
\begin{array}{c}
\renewcommand*{\arraystretch}{1.2}
\RULE
  {
    
  }
  {
    \redstoo{(\unitval;\,e_2,\rhomap)}
            {(e_2,\rhomap)}
  }
\end{array}
\end{smathpar}
\end{minipage}
% 
\begin{minipage}{1.8in}
\begin{smathpar}
\begin{array}{c}
\renewcommand*{\arraystretch}{1.2}
\RULE
  {
    \redstoo{(e,\rhomap)}{(e',\rhomap')}
  }
  {
    \redstoo{(E\lbrack e \rbrack, \rhomap)}
            {(E\lbrack e' \rbrack, \rhomap')}
  }
\end{array}
\end{smathpar}
\end{minipage}
%
\begin{minipage}{1.2in}
\begin{smathpar}
\begin{array}{c}
\renewcommand*{\arraystretch}{1.2}
\RULE
  {
    \redstoo{(e,\rhomap)}{\invalidexn}
  }
  {
    \redstoo{(E\lbrack e \rbrack, \rhomap)}
            {\invalidexn}
  }
\end{array}
\end{smathpar}
\end{minipage}

%
\begin{minipage}{2in}
\begin{smathpar}
\begin{array}{c}
\renewcommand*{\arraystretch}{1.2}
\RULE
  {
    \\
    \rgn \notin \rhoenv \spc
    \redsto{\rhoenv \cup \{\rgn\}}{(e,\rhomap)}{\invalidexn}
  }
  {
    \redstoo{(\letregion{\rgn}{e},\rhomap)}{\invalidexn}
  }
\end{array}
\end{smathpar}
\end{minipage}
%
\begin{minipage}{2in}
\begin{smathpar}
\begin{array}{c}
\renewcommand*{\arraystretch}{1.2}
\RULE
  {
    \fgjN = \RgnZ\inang{T} \spc
    \rgn \in dom(\rhomap) \\
    \redstocup{\rgn}{(e,\rhomap)}{\invalidexn}
  }
  {
    \redstoo{(\C{new} \; \fgjN\inang{\rgn}
                (e),\rhomap)}
            {\invalidexn}
  }
\end{array}
\end{smathpar}
\end{minipage}
%
\begin{minipage}{2.5in}
\begin{smathpar}
\begin{array}{c}
\renewcommand*{\arraystretch}{1.2}
\RULE
  {
    v_a = \C{new} \; \RgnZ\inang{T}\inang{\rgn}(v_r) \spc
    \rhomap(\rgn) \neq \XFERRED \spc
    \rgn_0 \notin \rhoenv \\
    \redstocup{\rgn_0}{([[\rgn_0/\rgn]v_r/x]e_b,
      \rhomap[\rgn \mapsto \OPEN])}{\invalidexn}
  }
  {
    \redstoo{(\open{v_a}{\rgn_0}{x}{e_b},\rhomap)} 
            {\invalidexn}
  }
\end{array}
\end{smathpar}
\end{minipage}
%

%
\bigskip

\textbf{Evaluation Context} \fbox {\(E\)}\\
\begin{smathpar}
\begin{array}{lcl}
E & \coloneqq & \bullet \ALT (\bullet).f \ALT \bullet.m\inang{\ralloc\rbar}(\ebar) \ALT
      v.m\inang{\ralloc\rbar}(...,\bullet,...) \ALT \C{new}\; \fbN(...,\bullet,...) \ALT
      \C{new} \; \RgnZ\inang{T}\inang{\toprgn}(\bullet) \ALT \bullet\inang{\ralloc\rbar}(\ebar) \\
  &  & \ALT v\inang{\ralloc\rbar}(...,\bullet,...) \ALT \bullet;\,e \ALT \open{\bullet}{\rgn}{y}{e}
\end{array}
\end{smathpar}

\caption{\fbname: Operational Semantics}
\label{fig:fb-opsem}
\end{figure*}

Figs.~\ref{fig:fb-opsem-1} and~\ref{fig:fb-opsem-2} show the
operational semantics of \fbname. \emph{TODO: 1. Some more text needs
to be added, 2. To represent typestate, color coding itself is not
enough; change in symbols needed.}.


\begin{figure*}[t!]

\beginrules

%%%%%%%%%%% LAMBDA %%%%%%%%%%%

\lgcrule{LAMBDA}
  {
    \rgn \in \A.\rhoenv \spc
    \rhobar \notin \A.\rhoenv
    \spc
%   \rhoenv' = \rhoenv \cup \{\rhoalloc,\rhobar\}\spc
    \A' = (\A.\rhoenv \cup \{\rhobar\}, \A.\aenv, 
          \A.\phicx \conj \phi)\spc
    \tywf{\A'.\rhoenv}{\phi}
    \\
    \typeok {\A'} {\bar{\tau^1}} {C} \spc
    \typeok {\A'} {\tau^2} {C} \spc
    \exprok {\A',\env[\xbar \mapsto \bar{\tau^1}]} {e} {\tau^2} {C}
  }
  {
    \exprok {\stdcontext}
           {\lambdaexp{\rgn}{\rhobar \,|\, \phi} {\xbar:\bar{\tau^1}}{e}}
           {\inang{\rhobar \,|\, \phi} \bar{\tau^1} \xrightarrow{\rgn} \tau^2}
	   {C}
  }

%%%%%%%%%%% OPEN-REGION %%%%%%%%%%%
\lgcrule{OPEN}{
   \exprok {\stdcontext} {e_a} {\RgnZ\inang{T}\inang{\rho}} {C_1} \spc
   \A = (\rhoenv,\aenv,\phicx) \spc
   \rgn \notin \rhoenv
   \\
   % (\A',\env') = ((\rhoenv \cup \{\rgn\},\aenv,\phicx),\env[y\mapsto T@\rgn] \spc
   \exprok {(\rhoenv \cup \{\rgn\},\aenv,\phicx),\env[y\mapsto T@\rgn]} {e_b} {\tau} {C_2}
}{
   \exprok {\stdcontext} {\open{e_a}{\rgn}{y}{e_b}} {\tau} {(C_1 \cup C_2)}
}


%%%%%%%%%%% FUNCTION INVOCATION %%%%%%%%%%%
\lgcrule{FUN-APPLY}
{
\exprok {\stdcontext} {e_a} {\inang{\rhobar\,|\,\phi}\taubar \xrightarrow{\rgn} \tau} {C_1} \spc
\exprok {\stdcontext} {\bar{e}} {\bar{\tau'}} {C_2} \spc
\substFn = [\bar{\rho'}/\rhobar]
\\
C_3 = \{\bar{\rho'} \in \A.\rhoenv\} \spc
C_4 = \{\isvalid{\A.\phicx}{\substFn(\phi)}\} \spc
\subtypeok {\A} {\bar{\tau'}} {\substFn(\bar{\tau})} {C_5}
}{
\exprok {\stdcontext} {e_a\inang{\bar{\rho'}}(\bar{e})} {\tau} {\cup_{i=1}^5 C_i}
}

%%%%%%%%%%% METHOD %%%%%%%%%%%
\lgcrule{METHOD}{
CT(B) = \hdOf{B}{\varphi}\{\bar{\tau^f}\,\xbar;\;\bar{d}\} \\
\A = (\rhoenv,\aenv,\phicx) = (\{\rhobar,\rhobarm\},\bar{\tyvar} \extends \bar{\fgjN}, \varphi_m) \spc\spc
C_1 = \{ \tywf{\rhoenv}{\varphi_m} \} \\
\env = \cdot[\thisZ \mapsto B\inang{\bar{\tyvar}}\inang{\rhobar}][\xbar \mapsto \taubar] \spc\spc
\exprok {\stdcontext}{e} {\tau'} {C_2} \spc\spc
\subtypeok {\A} {\tau'} {\tau} {C_3}
}{
\typeok{} {(B, \tau \; m\inang{\rhobarm \,|\, \varphi_m} (\taubar \;  \xbar)\{\C{return} e;\})} {(C_1 \cup C_2 \cup C_3)}
}

%%%%%%%%%%% CLASS %%%%%%%%%%%
\lgcrule{CLASS}{
\A = (\rhoenv, \aenv, \phicx) = (\{\rhoalloc,\rhobar\},\bar{\tyvar} \extends \bar{\fgjN},\varphi) \\
C_1 = \{ \tywf{\rhoenv}{\varphi} \} \spc\spc
\typeok {\A} {\fbN} {C_2} \spc\spc
\typeok {\A} {\bar{\tau^f}} {C_3} \\
C_4 = \{\isvalid{\phicx}{\allocRgn(\bar{\tau^f}) \outlives \rhoalloc \conj \allocRgn(\fbN) = \rhoalloc}\} \\
\typeok {} {\bar{d}} {C_5}
}{
\typeok {} {\hdOf{B}{\varphi}\{\bar{\tau^f}\,\xbar;\;\bar{d}\}} {\bigcup_{i=1}^5 C_i}
}

\myendrules

\caption{Constraint generation}
\label{fig:constraint-gen-1}
\end{figure*}

\begin{figure*}[t!]

\beginrules

%%%%%%%%%%% Header Box %%%%%%%%%%%
\fbox{  \( \typeok{\A}{\tau}{C} \)}
\\

%%%%%%%%%%% TYPE WELL-FORMEDNESS %%%%%%%%%%%

%%%%%%%%%%% OBJECT TYPE %%%%%%%%%%%
\lgcrule{TWF}
  {
    C = \{ \rgn \in \A.\rhoenv \}
  }
  {
    \typeok {\A} {\ObjZ\inang{\rgn}} {C}
  }

%%%%%%%%%%% CLASS TYPE %%%%%%%%%%%
  \lgcrule{TWF}
  {
    CT(B) = \headerOf{B}\{...\}
    \spc
    \fgjtywf{\aenv}{B\inang{\tbar}}
    \\
    C = \{ \rbar \in \rhoenv, \isvalid{\phicx}{[\rbar/\rhobar, \tbar/\bar{\tyvar}](\phi)} \}
  }
  {
    \typeok {(\rhoenv,\aenv,\phicx)} {B\inang{\rbar}\inang{\tbar}} {C}
  }

%%%%%%%%%%% GENERIC TYPE PARAMETER %%%%%%%%%%%
  \lgcrule{TWF}
  {
    \fgjtywf{\A.\aenv}{T} \spc
    \fgjsubtyp{\A.\aenv}{T}{\ObjZ} \spc
    \\
    C = \{ \rgn \in \A.\rhoenv \}
  }
  {
    \typeok {\A}{T@\rgn} {C}
  }

%%%%%%%%%%% FUNCTION TYPE %%%%%%%%%%%
  \lgcrule{TWF}
  {
    C_1 = \{ \rgn \in \rhoenv \}
    \\
    \rhobar \notin \A.\rhoenv \spc
    \rhoenv' = \rhoenv \cup \{\rhobar\} \spc
    \A' = (\rhoenv', \aenv, \phicx \conj \phi)
    \\
    \tywf{\rhoenv'}{\phi}\spc 
    \typeok{\A'}{\bar{\tau^1}} {C_2} \spc
    \typeok{\A'}{\tau^2} {C_3}
  }
  {
    \typeok{(\rhoenv,\aenv,\phicx)} {\inang{\rhobar \,|\, \phi} \bar{\tau^1} \xrightarrow{\rgn} \tau^2} 
       {C_1 \cup C_2 \cup C_3}
  }

%%%%%%%%%%% REGION TYPE %%%%%%%%%%%
  \lgcrule{TWF}
  { 
    \fgjtywf{\A.\aenv}{T}
  }
  {
    \typeok {\A} {\RgnZ\inang{T}\inang{\toprgn}} {\{\}}
  }

%%%%%%%%%%% FUNCTION SUBTYPING %%%%%%%%%%%
  \lgcrule{FUN-SUBTYPING}
  {
    C_1 = \{ \isvalid{\A.\phicx}{\phi_1 \Rightarrow \phi_2} \}
    \\
    \subtypeok {\A} {\bar{\tau^{11}}} {\bar{\tau^{21}}} {C_2}
    \\
    \subtypeok {\A} {\tau^{22}} {\tau^{12}} {C_3}
  }
  {
    \subtypeok {\A}
      {\inang{\rhobar \,|\, \phi_2}\bar{\tau^{21}} \xrightarrow{\rgn} \tau^{22}}
      {\inang{\rhobar \,|\, \phi_1}\bar{\tau^{11}} \xrightarrow{\rgn} \tau^{12}}
      {C_1 \cup C_2 \cup C_3}
  }

\myendrules

\caption{Type well-formedness constraint generation}
\label{fig:constraint-gen-2}
\end{figure*}


\begin{lemma}
The antecedent of any generated validity constraint is of the form
$\varphi \conj \phictxt$ where $\varphi$ is a predicate variable and
$\phictxt$ is a conjunction of zero or more outlives-constraints.
Let $\pi_1 \outlives \pi_2$ be a conjunct in $\phictxt$.
Then, $\pi_2 \notin \predDeltaMap(\varphi)$.
Furthermore, if $\pi_1 \in \predDeltaMap(\varphi)$, then for
every $\pi_f \in \predDeltaMap(\varphi)$, $\pi_f \outlives \pi_2$ is
a conjunct in $\phictxt$.
\end{lemma}

\begin{proof}
  By induction over the constraint-generation rules.
  Any context $\A = (\rhoenv,\aenv,\phicx)$ generated by the constraint-generation
  process satisfies the invariant that $\phicx$ is of the form $\varphi \conj \phictxt$
  where $\rhoenv \supseteq \predDeltaMap(\varphi)$.
  The only rule that modifies $\phicx$ is the rule for \C{letregion}
  that adds the set of constraints $\pi_f \outlives \pi$ for every $\pi_f \in \rhoenv$
  as conjuncts to $\phicx$.
\end{proof}


\begin{theorem}
Let $p$, $q$, $r$ and $C$ denote the values of the corresponding
variables in an execution of the type inference algorithm.
\begin{enumerate}
\item $\absof{q}$ = $p$
\item $\absof{q[\sigma]}$ = $p$ for any substitution $\sigma$.
\item If $\sigma$ is any solution to $C$, then $q[\sigma]$ is well-typed.
\item If there is any substitution $\sigma$ such that $q[\sigma]$ is well-typed, then
$C$ has a solution.
\item If SolveConstraints($C$) returns Some($s$), then $s$ is a solution to $C$.
\item If $C$ has a  solution, then SolveConstraints($C$) will return some solution.
\end{enumerate}
\end{theorem}

\section{Other Aspects}

\paragraph{Modularity Aspects of Type Inference.}
The type inference algorithm, as presented, traverses the entire program to
generate the set of constraints, which are solved en masse, using an iterative
fixed point computation. However, the type inference can be realized in a
modular and compositional fashion, subject only to the restrictions imposed
by recursion.

In the elaboration phase, we can process a class \C{C} only after any class
\C{B} that \C{C} depends on has been processed: class \C{C} depends on
class \C{B} if \C{B} is either \C{C}'s base class or the type of any field
of \C{C} depends on \C{B}. In effect, this means that any collection of
mutually recursive classes must be processed together. Non-recursive
dependences can be handled in a compositional fashion: if class \C{C}
depends on \C{B} non-recursively, then the elaboration can be done for
\C{B} first, and then \C{C} can be processed.

The same idea applies to the constraint-solving phase as well.
Given a set of constraints, we say that a predicate variable $\varphi_1$
\emph{directly-depends} on another predicate variable $\varphi_2$ if the set of
constraints includes a constraint $\isvalid{\varphi_1 \conj \phictxt}{F(\varphi_2)}$.
We say that $\varphi_1$ \emph{depends} on $\varphi_2$ if $\varphi_1$ transitively
depends on $\varphi_2$.
The constraint solver needs to process any collection of mutually dependent
predicate variables together.
In effect, this requires the type inference to process any collection of
mutually recursive methods together.
However, methods that are not mutually recursive can be processed separately.

