\begin{lemma}
\emph{(\textbf{Weakening})}
\label{thm:fb-tywf}
$\forall v, \tau, \rhoenv, \rhoenv_0, \rhoset, \phicx, \rgn, \rgn_0$, such that $\valuee(v)$, $\rgn
\in \rhoenv$, $\rgn_0 \notin \rhoenv$, and $\rhoenv_0 \subseteq \rhoenv$, if
$\hastyp{\emptyADelcupPhicapp{\rgn_0}{\rhoenv_0 \outlives \rgn_0},\rgn_0,\cdot}{v}{\tau}$ and
$\tywf{\emptyA}{\tau}$, then $\hastyp{\emptyA,\rgn,\cdot}{v}{\tau}$.
\end{lemma}
\begin{proof}
Intros. Hypotheses:
\begin{smathpar}
\begin{array}{cr}
  \rgn \in \rhoenv & H1\\
  \rgn_0 \notin \rhoenv & H2\\
  \rhoenv_0 \subseteq \rhoenv & H4\\
  \hastyp{\emptyADelcupPhicapp{\rgn_0}{\rhoenv_0 \outlives \rgn_0},\rgn_0, \cdot}{v}{\tau} & H6\\
  \tywf{\emptyA}{\tau} & H8\\
\end{array}
\end{smathpar}
Proof by induction on $H6$. Cases:
\begin{itemize}
  \item Case ($v = \C{new}\; \fbN_0(\vbar)$ and $\tau = \fbN_0$): Inversion on $H6$:
  \begin{smathpar}
  \begin{array}{cr}
    \fields(\fbN_0) = \bar{f}:\bar{\tau} & H10\\
    \tywf{\emptyA}{\fbN_0} & H11\\
    \hastyp{\emptyADelcupPhicapp{\rgn_0}{\rhoenv_0 \outlives \rgn_0},\rgn_0,\cdot}{\vbar}{\taubar} & H12\\
  \end{array}
  \end{smathpar}
  Inductive hypothesis on $\vbar$:
  \begin{smathpar}
  \begin{array}{cr}
    \tywf{\emptyA}{\taubar}\; \Rightarrow\; \hastyp{\emptyA,\rgn,\cdot}{\vbar}{\taubar} & IH1\\
  \end{array}
  \end{smathpar}
  Inversion on $H8$ tells us that $\fbN_0$ is of the form $\BZT{\rgn\rbar}$, where,
  $\headerOf{B}\{\bar{\tau^h} \; \bar{h},...\}$ is a well-formed class definition. Furthermore, we get:
  \begin{smathpar}
  \begin{array}{cr}
    \rgn,\rbar \in \rhoenv & H13\\
    \fgjtywf{\cdot}{\bar{T}} & H14\\
    \isvalid{\phicx}[\rgn/\rhoalloc][\rbar/\rhobar]\phi & H16\\
  \end{array}
  \end{smathpar}
  Let $\fields(\fbN) = \bar{g}:\bar{\tau^g}$. Inverting the well-formedness of class $B$, we get
  the following:
  \begin{smathpar}
  \begin{array}{cr}
    \tywf{(\emptyset,\{\rhoalloc,\rhobar\},[\bar{a} \mapsto
    \bar{\fgjN}],\phi)}{\bar{\tau^h}\bar{\tau^g}} & H18\\
  \end{array}
  \end{smathpar}
  Since $\fgjtywf{\cdot}{\BZ}$, $H18$ gives:
  \begin{smathpar}
  \begin{array}{cr}
    \tywf{(\emptyset,\{\rhoalloc,\rhobar\},\cdot,
    \phi)}{[\tbar/\bar{a}](\bar{\tau^h}\bar{\tau^g})} & H20\\
  \end{array}
  \end{smathpar}
  And, since $\rgn \neq \rhoalloc$ and $\{\rbar\} \cap \{\rhobar\} = \emptyset$:
  \begin{smathpar}
  \begin{array}{cr}
    \tywf{(\emptyset,\{\rgn,\rbar\},\cdot, [\rbar/\rhobar][\rgn/\rhoalloc]\phi)}
      {[\rbar/\rhobar][\rgn/\rhoalloc][\tbar/\bar{a}](\bar{\tau^h}\bar{\tau^g})} & H22\\
  \end{array}
  \end{smathpar}
  From the definition of $\fields$, we have:
  \begin{smathpar}
  \begin{array}{cr}
    \fields(\BZT{\rgn\rbar}) = \bar{h}:[\rbar/\rhobar][\rgn/\rhoalloc][\tbar/\bar{a}]\bar{\tau^h},\,
                      \bar{g}:[\rbar/\rhobar][\rgn/\rhoalloc][\tbar/\bar{a}]\bar{\tau^g} & H24\\
  \end{array}
  \end{smathpar}
  From $H10$ and  $H24$, we know that $\taubar =
  [\rbar/\rhobar][\rgn/\rhoalloc][\tbar/\bar{a}](\bar{\tau^h}\bar{\tau^g)}$. Substituting this in
  H22:
  \begin{smathpar}
  \begin{array}{cr}
    \tywf{(\emptyset,\{\rgn,\rbar\},\cdot, [\rbar/\rhobar][\rgn/\rhoalloc]\phi)} {\taubar} & H26\\
  \end{array}
  \end{smathpar}
  Using $H13$ and $H16$, from $H26$, we derive:
  \begin{smathpar}
  \begin{array}{cr}
    \tywf{\emptyA} {\taubar} & H28\\
  \end{array}
  \end{smathpar}
  From $H28$ and $IH1$, we get:
  \begin{smathpar}
  \begin{array}{cr}
    \hastyp{\emptyA,\rgn,\cdot}{\vbar}{\taubar} & H30\\
  \end{array}
  \end{smathpar}
  From $H10$, $H11$, and $H30$, we prove the required goal:
  \begin{smathpar}
  \begin{array}{cr}
    \hastyp{\emptyA,\rgn,\cdot}{\C{new}\; \fbN_0(\vbar)}{\fbN_0} & \\
  \end{array}
  \end{smathpar}

  \item Case ($v = \C{new}\; \RgnZT{\rgn_i}(v_0)$ and $\tau = \RgnZT{\toprgn}$): Inversion on $H6$:
  \begin{smathpar}
  \begin{array}{cr}
    \rgn_i \in dom(\rhomap) & H10\\
    \rgn_i \notin \rhoenv \cup \{\rgn_0,\toprgn\} & H12\\
    \tywf{\emptyADelcupPhicapp{\rgn_i,\rgn_0}{\rhoenv_0 \outlives \rgn_0}}{T@\rgn_i} & H14\\
    \hastyp{\emptyADelcupPhicapp{\rgn_i,\rgn_0}{\rhoenv_0 \outlives \rgn_0},\rgn,\cdot}{v_0}{T@\rgn_i} & H16\\
  \end{array}
  \end{smathpar}
  Inductive hypothesis on $v_0$:
  \begin{smathpar}
  \begin{array}{cr}
    \tywf{\emptyADelcup{\rgn_i}}{T@\rgn_i}\; \Rightarrow\; \hastyp{\emptyADelcup{\rgn_i},\rgn,\cdot}{v_0}{T@\rgn_i} & IH1\\
  \end{array}
  \end{smathpar}
  Inverting $H14$, we get:
  \begin{smathpar}
  \begin{array}{cr}
    \fgjsubtyp{\cdot}{T}{\ObjZ} & H18\\
    \fgjtywf{\cdot}{T} & H19\\
    \rgn_i \in \rhoenv \cup \{\rgn_i,\rgn_0\} & H20\\
  \end{array}
  \end{smathpar}
  $H20$ implies $\rgn_i \in \rhoenv \cup \{\rgn_i\}$. Using this, $H18$, and $H19$, and applying the
  type well-formedness rule for $T@\rgn_i$, we derive the following:
  \begin{smathpar}
  \begin{array}{cr}
    \tywf{\emptyADelcup{\rgn_i}}{T@\rgn_i} & H22\\
  \end{array}
  \end{smathpar}
  $H22$ and $IH1$ gives:
  \begin{smathpar}
  \begin{array}{cr}
    \hastyp{\emptyADelcup{\rgn_i},\rgn,\cdot}{v_0}{T@\rgn_i} & H24\\
  \end{array}
  \end{smathpar}
  $H12$ implies $\rgn_i \notin \rhoenv \cup \{\toprgn\}$. Using this, and $H10$, $H22$ and $H24$, we
  conclude:
  \begin{smathpar}
  \begin{array}{cr}
    \hastyp{\emptyA,\rgn,\cdot}{\RgnZT{\rgn_i}}{\RgnZT{\toprgn}} & \\
  \end{array}
  \end{smathpar}

  \item Case ($v = \lambdaexp{\ralloc}{\rhoalloc\rhobar \,|\, \phi}{\bar{\tau^1} \; \xbar}{e}$ and 
              $\tau = \inang{\rhoalloc\rhobar\,|\,\phi}\taubar \xrightarrow{\ralloc} \tau^2$):
  By inversion on $H6$:
  \begin{smathpar}
  \begin{array}{cr}
    \ralloc = \rgn_0 & H10\\
    \hastyp{\emptyADelcupPhicapp{\rgn_0}{\rhoenv_0 \outlives \rgn_0},\rgn_0, \cdot}
      {\lambdaexp{\rgn_0}{\rhoalloc\rhobar \,|\, \phi}{\bar{\tau^1} \; \xbar}{e}}
      {\inang{\rhoalloc\rhobar\,|\,\phi}\taubar \xrightarrow{\rgn_0} \tau^2} & H12\\
  \end{array}
  \end{smathpar}
  From $H8$:
  \begin{smathpar}
  \begin{array}{cr}
    \tywf{\emptyA}{\inang{\rhoalloc\rhobar\,|\,\phi}\taubar \xrightarrow{\rgn_0} \tau^2} & H14\\
  \end{array}
  \end{smathpar}
  Inverting $H14$ tells us that $\rgn_0 \in \rhoenv$. But $H2$ tells us that $\rgn_0 \notin
  \rhoenv$. This is a contradiction, telling us that it cannot be the case that the type of lambda
  is well-formed under $\emptyA$. Intuitively, this means that a $\C{letregion}$ expression or an
  $\C{open}$ expression can never return a closure, thereby ensure that a function application never
  dereferences an invalid reference.
% End of proof cases.
\end{itemize}
\qed
\end{proof}
