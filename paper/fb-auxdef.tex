\begin{figure*}[t]
%
\begin{minipage}{2.25in}
\begin{smathpar}
\begin{array}{lcl}
  \allocRgn(A\inang{\ralloc\rbar}\inang{\tbar}) & = & \ralloc\\
  \allocRgn(\inang{\rhoalloc\rhobar \,|\, \phi}\bar{\tau^1}
      \xrightarrow{\ralloc} \tau^2) & = & \ralloc\\
  \shape(A\inang{\rhoalloc\rhobar}\inang{\tbar}) & = & A\inang{\tbar}\\
\end{array}
\end{smathpar}
\end{minipage}
%
\begin{minipage}{2.25in}
\begin{smathpar}
\begin{array}{lcl}
  \bound_{\aenv}(\tyvar@\rgn) & = & \aenv(\tyvar)@\rgn\\
  \bound_{\aenv}(\fbN) & = & \fbN\\
  \fields(\ObjZ\inang{\rgn}) & = & \bullet \\
\end{array}
\end{smathpar}
\end{minipage}
%
\begin{minipage}{2.25in}
\begin{smathpar}
\begin{array}{lcl}
  \ctype(\ObjZ\inang{\rgn}) & = & \bullet \\
% ctype(\RgnZ\inang{\rgn}\inang{T}) & = & \inang{\rhoalloc}
%   {\unitZ}\rightarrow{T@\rhoalloc}\\
  \ctype(B\inang{\tbar}\inang{\ralloc\rbar}) & = & 
    \fields(B\inang{\tbar}\inang{\ralloc\rbar})\\
%% ---- The following are no longer needed since
%% ---- \exists\rho.\RgnZ<\rho> = \RgnZ<\rgn_\top>
% mtype(\C{transfer}, \exists\rho.\RgnZ\inang{\rho}\inang{T}) & = & 
%   \inang{\rhoalloc} {\unitZ}\rightarrow{\unitZ}\\
% mtype(\C{free}, \exists\rho.\RgnZ\inang{\rho}\inang{T}) & = & 
%   \inang{\rhoalloc} {\unitZ}\rightarrow{\unitZ}\\
\end{array}
\end{smathpar}
\end{minipage}
%


% \begin{minipage}{1.8in}
% \begin{smathpar}
% \begin{array}{c}
% \renewcommand*{\arraystretch}{1.2}
% \RULE
%   {
%     \\
%     B \in \{\ObjZ,\RgnZ\}
%   }
%   {
%     fields(B\inang{\tbar}\inang{\ralloc\rbar}) \;=\; \bullet
%   }
% \end{array}
% \end{smathpar}
% \end{minipage}
% %
\begin{minipage}{3in}
\begin{smathpar}
\begin{array}{c}
\renewcommand*{\arraystretch}{1.2}
\RULE
  {
    CT(B) = \headerOf{B}\{\bar{\tau^f}\;\bar{f};\,...\}\\
    \substFn = [\rbar/\rhobar, \ralloc/\rhoalloc, \tbar/\bar{\tyvar}] \qquad 
    \fields(\substFn(\fbN)) = \bar{g}:\bar{\tau^g}
  }
  {
    \fields(B\inang{\tbar}\inang{\ralloc\rbar}) \;=\;
      \bar{g}:\bar{\tau^g},\,\bar{f}:\substFn(\bar{\tau^f})
  }
\end{array}
\end{smathpar}
\end{minipage}
%
\begin{minipage}{3in}
\begin{smathpar}
\begin{array}{c}
\renewcommand*{\arraystretch}{1.2}
\RULE
  {
    CT(B) = \headerOf{B}\{\bar{\tau^f}\;\bar{f};\,\bar{d}\}\\
    m \notin \bar{d} \qquad 
    \substFn = [\rbar/\rhobar, \ralloc/\rhoalloc, \tbar/\bar{\tyvar}]
  }
  {
    \mtype (m,B\inang{\tbar}\inang{\ralloc\rbar}) \;=\;
    \mtype (m, \substFn(\fbN))
  }
\end{array}
\end{smathpar}
\end{minipage}
%


\begin{minipage}{3.4in}
\begin{smathpar}
\begin{array}{c}
\renewcommand*{\arraystretch}{1.2}
\RULE
  {
    \\
    CT(B) = \headerOf{B}\{\bar{\tau^f}\;\bar{f};\,\bar{d}\}\\
    \tau^2 \; m\mang (\bar{\tau^1}\;\bar{x})\{...\} \in \bar{d} \qquad
    \substFn = [\rbar/\rhobar, \ralloc/\rhoalloc, \tbar/\bar{\tyvar}]
  }
  {
    \mtype (m,B\inang{\tbar}\inang{\ralloc\rbar}) \;=\;
    \substFn(\mang\bar{\tau^1} \rightarrow \tau^2)
  }
\end{array}
\end{smathpar}
\end{minipage}
%
\begin{minipage}{3.5in}
\begin{smathpar}
\begin{array}{c}
\renewcommand*{\arraystretch}{1.2}
\RULE
  {

    \mtype(m,\fbN) = \inang{\rhoalloc_1\bar{\rho_1},|\, \phi_1}\bar{\tau^{11}} 
                      \rightarrow \tau^{12} \\ 
    \texttt{implies} \spc \isvalid{\A.\phicx}
                  {\phi_2 \Leftrightarrow \substFn(\phi_1)} 
                  \spc \texttt{and} \spc
    \bar{\tau^{21}} = \substFn(\bar{\tau^{11}}) \\ \texttt{and} \spc
    \subtyp{\A}{\tau^{22}} {\substFn(\tau^{12})} \spc
    \substFn = \subst{\bar{\rho_2}}{\bar{\rho_1}}
               \subst{\rhoalloc_2}{\rhoalloc_1}
    %\substFn = [\rbar/\rhobar, \ralloc/\rhoalloc, \tbar/\bar{\tyvar}]
  }
  {
    \A \vdash \override(m,\fbN,\inang{\rhoalloc_2\bar{\rho_2},|\, \phi_1}
              \bar{\tau^{21}} \rightarrow \tau^{22})
  }
\end{array}
\end{smathpar}
\end{minipage}
%


% \begin{minipage}{5in}
% \begin{smathpar}
% \begin{array}{c}
%   \rhoset,\rhoenv \in 2^{\rho} \qquad
%   \aenv \in \tyvar \rightarrow \fgjN \qquad
%   \A = (\subtypcx)\\
% \end{array}
% \end{smathpar}
% \end{minipage}
%

\caption{\fbname: Auxiliary Definitions}
\label{fig:fb-auxdef}
\end{figure*}
